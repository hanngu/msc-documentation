\chapter{Protocol}
\label{appendixA}

%TODO: The final protocol with the final search term and justifications.
%Ref: https://raw.githubusercontent.com/kenborge/slr-scbw/master/sections/protocol.tex
\noindent
\begin{table}[h]
\resizebox{12.5cm}{!} {

    \hspace*{-2.25cm}
    \begin{tabular}[c]{| m{1.25cm} | m{1.5cm} | m{1.8cm} | m{1.5cm} | m{1.5cm} | m{1.5cm} | m{1.3cm} | m{1.5cm} | m{1.5cm} |}
    \hline
    & \textbf{Group 1} & \textbf{Group 2} & \textbf{Group 3} & \textbf{Group 4} & \textbf{Group 5} & \textbf{Group 6} & \textbf{Group 7} & \textbf{Group 8} \\ \hline   
    \textbf{Term1} & Train & Path optimization & Bee colony optimization & Transit & Artificial intelligence & Multi agent & Routing & Neo4j \\ \hline
    \textbf{Term2} & Plane & \hspace{0pt}Scheduling optimization & Particle swarm optimization & \hspace{0pt}Transportation & AI & & & Graph database \\ \hline
    \textbf{Term3} & Bus & Route optimization & Swarm intelligence & Traffic & Machine Learning & & & \\ \hline
    \textbf{Term4} & Delivery & Planning & Ant colony optimization & Vehicle & & & & \\ \hline
    \textbf{Term5} & & Multimodal & BCO & & & & & \\ \hline
    \textbf{Term6} & & & PSO & & & & & \\ \hline
    \textbf{Term7} & & & ACO & & & & & \\ \hline
    \end{tabular}\hspace*{-2.25cm}
    
    }
    \caption{Matrix of search terms}
    \label{table:searchterms}
\end{table}

\section{Search Terms}

\begin{itemize}

\item Group 1: Train, plane, bus, delivery
\item Group 2: Path optimization, Scheduling Optimization, Route Optimization, Planning, Multimodal
\item Group 3: Bee colony optimization, Particle swarm optimization, Swarm intelligence, Ant colony optimization, BCO, PSO, ACO
\item Group 4: Transit, Transportation, Traffic, Vehicle
\item Group 5: Artificial Intelligence, ai, Machine Learning
\item Group 6: Multi-agent
\item Group 7: Routing
\item Group 8: Neo4j, Graph database

\end{itemize}

\section{Complete Search Term}
\label{searchterm}

\textit{(train OR plane OR bus OR delivery) AND (``path optimization'' OR ``scheduling optimization'' OR ``route optimization'' OR planning OR multimodal) AND (``bee colony optimization'' OR ``particle swarm optimization'' OR ``swarm intelligence'' OR ``ant colony optimization'' OR bco OR pso OR aco) AND (transit OR transportation OR traffic OR vehicle) AND (``artificial intelligence'' OR ai OR ``machine learning'') AND ``multi-agent'' AND routing)}

% Har ditcha gruppe 8 i den "komplette søketermen", fordi vi ikke bruker gruppe 8 som en del av literatursøket, men kun til å "prove a point"

 %Lol & Group 1 & Group 2 & Group 3 & Group 4 & Group 5 & Group 6 & Group 7 & Group 8 \\ \hline
 %   Term 1 & Train & Path optimization & Bee colony optimization & Transit & Artificial intelligence & Multi agent & Routing & Neo4j \\ \hline


\section{Research Questions}
To conduct a structured literature review it is vital to decide the problem to be solved, referred to as P, and the constraints used to guide the search, referred to as C.
\newline
\newline
One of the goals for the environment package for transportation in Trondheim, ``Miljøpakken'', is to reduce percentage of people travelling with cars from 58 \% to 50 \% by 2018 \citep{website:miljopakken}. If this goal is reached, it will be an increased need for public transportation in Trondheim. There has never been done any optimization of the bus routes in Trondheim, the existing solution is purely based on experience. The problem formulation for this thesis was therefore based on the idea to improve todays solution by optimizing the bus routes using AI-methods. (And as a result of this satisfy the same amount of users today with less resources.)

\begin{itemize}
\item \textbf{P:} “Optimizing the bus routes in Trondheim using AI-methods. “ This problem can be characterized as a \textit{General Pickup and Delivery Problem (GPDP)} \citep[p.22-25]{vehiclerouting}.  
\item \textbf{C:} 
    \begin{enumerate}
        \item To optimize the bus routes in Trondheim we wanted to explore the possibility using methods from swarm intelligence. This idea came from an initial, non-structured literature review were we did a broad search among different artificial intelligence methods and route optimizing. %Todo: Kanskje skrive litt mer om dette? Evt. sitere noe literatureshit
        \item We believe that a part of solving the problem, P, is how we choose to represent the network of the bus routes in Trondheim. The chosen algorithms to optimize the routes with respect to minimize the number of resources used will use this representation. We have some experience with the graph database Neo4j. Neo4j has several benefits that we believe we can take advantage of when solving P, including a natural node-edge-structure and the possibility of saving information to both the nodes and edges. We envision that the nodes will represent bus stops, and the edges will represent the connectivity between the stops. 
    \end{enumerate}
\end{itemize}

\textbf{This gives us the following research questions:}
\begin{enumerate}
\item What are the existing solutions to this problem?
\item Which swarm intelligence methods is best suited for optimizing? 
\item Is it practicle to represent and work with this route network as a graph database for this kind of methods?
\item Does this solution help optimize the bus routes? 
\end{enumerate}



%Todo: Skrive om Zotero

\section{Inclusion Criteria}
To exclude irrelevant literature, some inclusion criterias were decided to ensure a level of relevance to the very first pool. First of all, duplicate literature, book of chapters, book of abstracts, book of references, literature not written in english, books, and literature with cleary irrelevant titles (for example literature from different research areas) were removed based on title. After that, we decided to filter out relevant literature based on the abstracts. Because we had relatively many sources to related literature after the initial filtering (367), we decided that we wanted the abstracts (or the keyword section) to explicitly mention swarm intelligence or algorithms associated with swarm intelligence, while it also described a problem connected to vehicle routing. For our literature review we decided to use the inclusion criterias solely on the title, abstract and keywords. After a discussion and reading a few abstracts we landed on the following inclusion criterias:
\begin{itemize}
\item The main concern is route optimization focusing on vehicles. 
\item The study focuses on the use of swarm intelligence.
\item The literature must contain an abstract. 
\item The literature must still excist (some literature were removed from its original source).
\item The literature must be free of charge.
\end{itemize}

After the inclusion criteria filtering, we had 42 sources to related literature, including scientific papers and master theses. 

\section{Quality Criteria} 

\begin{enumerate}
\item How relevant is it?
\begin{enumerate}
\item Is the problem of the research a vehicle routing problem?
\item Is swarm intelligence the main optimization method? 
\end{enumerate}
\item Is there is a clear statement of the aim of the research?
\item Is the study put into context of other studies and research?
\item Are system of algoritmic design decisions justified?
\item Is the test data set reproducible?
\item Is the study algorithm reproducible?
\item Is the experimental procedure thoroughly explained and reproducible?
\item Is it cleary stated in the study which other algorithms the study`s algorithm(s) have been compared with?
\item Are the performance metrics used in the study explained and justified?
\item Are the test results thorougly analysed?
\item Does the test evidence support the findings presented?
\item Has the architecture been implemented (and published)?
\item Is the amount/quality of citation satisfactory? ($<$ $\frac{1}{3}$  self-citation and $>$ 10 citations)
\end{enumerate}

\subsection{Scoring}
Point 1-13 was given a score, with the granularity of 0 (no), $\frac{1}{2}$ (partly), and 1 (yes). For this structured literature review we wanted to emphasize on the quality criteria that covered the relevance. We read some literature that were quite good regarding to the example structure and composition, but not relevant for our thesis. Therefore, we chose to multiply the 1a and 1b quality criteria with 3. Table \ref{table:literature} shows the papers that were read and scored according to the quality criteria: 

\begin{table}[H]
    {
    \tiny
    \hspace*{-2cm}
    \resizebox{16.5cm}{!} {
    \begin{tabular}[c]{|m{6.5cm}|c|c|c|c|c|c|c|c|c|c|c|c|c|c|c|}
        \hline
        \textbf{Title} & \textbf{1a} & \textbf{1b} & \textbf{2} & \textbf{3} & \textbf{4} & \textbf{5} & \textbf{6} & \textbf{7} & \textbf{8} & \textbf{9} & \textbf{10} & \textbf{11} & \textbf{12} & \textbf{13} & \textbf{Total} \\ \hline
        \textit{``A comprehensive review of firefly algorithms''}&0&3&1&1&1&0&1&0&1&0.5&0&0&0&1&9 \\ \hline
        \textit{``Adaptive Comprehensive Learning Bacterial Foraging Optimization and Its Application on Vehicle Routing Problem with Time Windows''}&1.5&1.5&0.5&1&0.5&1&1&1&1&1&1&0&0&1&12 \\ \hline
        \textit{``Adapt-Traf: An adaptive multiagent road traffic management system based on hybrid ant-hierarchical fuzzy model''}&1.5&1.5&1&1&1&0.5&0.5&1&0&1&1&1&0&1&12 \\ \hline
        \textit{``A Design of Intelligent and Autonomous Public Transportation System by Co-evolution''}&3&1.5&0&0&0.5&0&0&0&0&0&0&0&0&0.5&5.5 \\ \hline
        \textit{``A fast solution method for the time-dependent orienteering problem''}&3&1.5&1&1&0.5&1&1&1&1&1&1&1&0&1&15  \\ \hline
        \textit{``Agent-based Simulation for UAV Swarm Mission Planning and Execution''}&0&1.5&1&1&0.5&0.5&1&0.5&1&0.5&0.5&0.5&0&1&9.5 \\ \hline
        \textit{``A hybrid particle swarm optimization approach for the sequential ordering problem''}&1.5&3&1&1&1&1&1&1&1&1&0.5&0.5&0&0.5&14 \\ \hline
        \textit{``An ant based algorithm approach to vehicle navigation''}&3&3&1&0.5&1&0.5&1&1&0.5&1&1&0.5&0&0.5&14.5 \\ \hline
        \textit{``An Ant Based Simulation Optimization for Vehicle Routing Problem with Stochastic Demands''}&1.5&3&1&1&1&1&1&1&1&0.5&1&0.5&0&1&14.5 \\ \hline
        \textit{``An Ant Colony Optimization Approach to Solve Cooperative Transportation Planning Problems''}&1.5&1.5&0&1&0.5&0.5&1&1&1&1&1&1&0&1&12 \\ \hline
        \textit{``An Ant System application to the Bus Network Design Problem: an algorithm and a case study''}&3&3&1&1&1&1&1&1&1&1&1&1&0&1&17 \\ \hline
        \textit{``An exploration of the literature on the use of 'swarm intelligence-based techniques' for public service problems''}&1.5&3&1&0.5&1&1&0.5&1&0.5&1&0.5&0.5&0&1&13 \\ \hline
        textit{``An improved Ant Colony algorithm for Urban Transit Network Optimization''}&3&3&1&1&1&1&1&1&1&1&1&1&0&1&17 \\ \hline
        \textit{``An Inverted Ant Colony Optimization approach to traffic''}&1.5&3&1&1&1&1&1&1&0.5&1&1&0.5&0&1&14.5 \\ \hline
        \textit{``Ant Colony Optimization''} &1.5&3&0.5&0.5&0.5&1&0.5&0.5&0.5&0.5&0&0&0&0&9 \\ \hline
        \textit{``Ant colony optimization for best path planning''}&3&3&1&1&1&1&1&1&0.5&1&1&1&0&0.5&16 \\ \hline
        \textit{``Ant colony optimization techniques for the vehicle routing problem''}&3&3&1&0&0.5&0.5&0.5&0.5&1&1&1&1&0&1&14 \\ \hline
        \textit{``Ant dispersion routing for traffic optimization''}&1.5&1.5&1&1&1&1&1&1&1&1&1&0.5&0&1&13.5 \\ \hline
        \textit{``A parallel ant colony algorithm for bus network optimization''}&3&3&1&1&0.5&1&1&1&0.5&1&1&1&0&1&16 \\ \hline
        \textit{``A review of ant algorithms''}&1.5&3&0.5&1&0.5&1&1&1&1&1&1&0.5&0&1&14 \\ \hline
        \textit{``A simultaneous transit network design and frequency setting: Computing with bees''} &3&3&1&1&1&1&1&1&1&1&1&1&0&1&17 \\ \hline
        \textit{``A Study on Bus Routing Problem: An Ant Colony Optimization Algorithm Approach''} &3&3&1&1&1&0.5&0.5&0.5&1&0.5&0.5&0.5&0&0&13 \\ \hline
        \textit{``A Swarm Based Method for Solving Transit Network Design Problem''}&3&3&0.5&1&0.5&0.5&1&1&1&1&0.5&0.5&0&0.5&14 \\ \hline
        \textit{``Bi-objective bimodal urban road network design using hybrid metaheuristics''}&1.5&1.5&1&1&0.5&1&1&1&1&1&1&1&0&1&13.5 \\ \hline
        \textit{``Bus Transit Service Optimization–The State-of-the-Art. State-of-the-Practice, and Challenges''}&1.5&0&1&0&1&1&0.5&0.5&0.5&0.5&0.5&0.5&0&0.5&8 \\ \hline
        \textit{``Combining new Fast Opposite Gradient Search with Ant Colony Optimization for solving travelling salesman problem''}&0&3&1&1&0.5&1&1&1&1&1&1&0.5&0&1&13 \\ \hline
        \textit{``Computing with bees: Attacking complex transportation engineering problems''}&1.5&3&1&1&1&1&1&1&0&1&1&1&0.5&1&15 \\ \hline
        \textit{``Data mining with various optimization methods''}&0&1.5&1&1&0.5&0&1&0.5&1&1&1&1&0.5&1&11 \\ \hline
        \textit{``Designing a multimodal feeder network by covering stops with different modes''}&1.5&3&1&1&1&0&0.5&1&0&1&1&1&0.5&1&13.5 \\ \hline
        \textit{``Dynamic Fuzzy Logic-Ant Colony System-Based Route Selection System''}&3&3&0.5&1&1&0.5&0.5&0.5&1&1&1&1&0.5&1&15.5 \\ \hline
        \textit{``Multimodal Feeder Network Design Problem: Ant Colony Optimization Approach''}&1.5&3&1&1&0.5&0&1&0.5&0&1&1&1&0.5&1&13 \\ \hline
        \textit{``Optimization of a Transit Services Model with a Feeder Bus and Rail System Using Metaheuristic Algorithms''}&3&1.5&0.5&1&1&0&1&0.5&1&1&1&1&0.5&1&14 \\ \hline
        \textit{``Optimization of Large Transport Networks Using Ant Colony Heuristic''}&0&3&0.5&0.5&0.5&0.5&1&0.5&0&1&0.5&1&0.5&1&10.5 \\ \hline
        \textit{``Optimizing bus transit network with parallel ant colony algorithm''}&1.5&3&1&0.5&0.5&1&1&0&1&1&1&1&0.5&1&14 \\ \hline
        \textit{``Real-time route planning of the public transportation system''}&3&3&1&0&0.5&0.5&1&0.5&0.5&1&0.5&1&0.5&1&14 \\ \hline
        \textit{``Route Optimization for Bus Dispatching Based on Improved Ant Colony Algorithm''} &3&3&0.5&0&0.5&0.5&1&0.5&0&0&0.5&1&0.5&0.5&11.5 \\ \hline
        \textit{``Solving the open vehicle routing problem by a hybrid ant colony optimization''}&3&3&1&1&1&1&1&1&1&1&1&1&0.5&1&17.5 \\ \hline
        \textit{``Solving the Urban Transit Routing Problem using a particle swarm optimization based algorithm''}&3&3&1&1&1&1&1&1&1&1&1&1&0.5&1&17.5 \\ \hline
        \textit{``The Application of Artificial Intelligence Hybrid in Traffic Flow''}&0&3&0.5&0&0&0&0&0&0&0&0&0&0&1&4.5 \\ \hline
        \textit{``Transit network design by Bee Colony Optimization''}&1.5&3&1&1&1&1&1&1&0.5&0&0.5&1&0.5&1&14 \\ \hline
        \textit{``Transportation Modeling: An Artificial Life Approach''} &1.5&3&1&0.5&0.5&0&0.5&0.5&0.5&0.5&0.5&1&0.5&1&11.5 \\ \hline
        \textit{``Transport Modeling by Multi-Agent Systems: A Swarm Intelligence Approach''}&3&3&0.5&0.5&0.5&0&0&0&0.5&0&0&0&0&1&9 \\ \hline

    \end{tabular}
    } 
    }
\caption{Quality Criteria Scoring}\label{table:literature}
\end{table}\hspace*{-2cm}

\section{Selecting the final literature}
When selecting the final literature we decided to do this solely based on the quality criteria scores. The average score of the read literature was 13.01 $\approx$ 13. We decided that literature given a score $\geq{1.5}$ above average were selected. After this sorting we ended up with 14 final literature. These 12 literature are going to create the foundation of our thesis. Table \ref{table:finalliterature} shows the selected literature. 

\begin{table}[!htb]
    {
    \begin{center}
    \small
    \begin{tabular}[c]{| m{6cm} | m{6cm} |}
        \hline
        \textbf{Title} & \textbf{Author} \\ \hline
        \textit{``An ant based algorithm approach to vehicle navigation''} & Salehi-nezhad and Farrahi-Moghaddam \\ \hline
        \textit{``An Ant Based Simulation Optimization for Vehicle Routing Problem with Stochastic Demands''} & Tripathi et al. \\ \hline
        \textit{``An Ant System application to the Bus Network Design Problem: an algorithm and a case study ''} & Poorzahedy and Safari \\ \hline
        \textit{``An improved Ant Colony algorithm for Urban Transit Network Optimization''} & Jiang et al. \\ \hline
        \textit{``An Inverted Ant Colony Optimization approach to traffic''} & Dias et al. \\ \hline
        \textit{``Ant colony optimization for best path planning''} & Hsiao et al. \\ \hline
        \textit{``A parallel ant colony algorithm for bus network optimization''} & Yang et al. \\ \hline
        \textit{``A simultaneous transit network design and frequency setting: Computing with bees''} & Nikolić and Teodorović \\ \hline
        \textit{``Computing with bees: Attacking complex transportation engineering problems''} & Panta and Du San Teodorovi \\ \hline
        \textit{``Dynamic Fuzzy Logic-Ant Colony System-Based Route Selection System''} & Salehinejad and Talebi \\ \hline
        \textit{``Solving the open vehicle routing problem by a hybrid ant colony optimization''} & Sedighpour et al. \\ \hline
        \textit{``Solving the Urban Transit Routing Problem using a particle swarm optimization based algorithm''} & Kechagiopoulos and Beligiannis \\ 
        \hline
    \end{tabular}
    \end{center}
    } 
\caption{Final literature}\label{table:finalliterature}
\end{table}