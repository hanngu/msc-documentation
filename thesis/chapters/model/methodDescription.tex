\section{Super Swarm Optimization}

The baseline of the proposed algorithm, Super Swarm Optimization (SSO), is an implementation of the basic ACO algorithm introduced by \citet{nanda11} and described in chapter \vref{backgroundAndMotivation}. Because the basis of the algorithm is ACO, we have chosen to call the individuals of the SuperSwarm colony for ants. %(if is it efficient to combine attributes from different swarm intelligence algorithms to solve the vehicle routing problem,) 
In the SSO, some additional features are given; the ants are given memory, and can remember whether or not it has visited a node in the creation of the route set. The ants are also given some acknowledged features inspired by the PSO and BSO are added to the SSO algorithm, which are the following:

\begin{itemize}
\item Inertia Weight (IW) from PSO. IW is a decreasing parameter brought to the PSO for balancing local and global search. In the early iterations of the algorithm, the particles tends to explore more, and becoming more organized and coordinated in the late iterations. To do this a generated ant is declared ``crazy'' by a probability $E$. A \textit{Crazy Ant (CA)} does not select next nodes based on the edge value of the connecting edge, but instead selects it randomly. The probability that an ant is declared crazy are given by a predefined start value, and this value is decreased iteratively by an inertia weight (IW). The IW value is high in the early iterations, denoting a higher amount of CA, and when this value decreases (for each iteration), the amount of CA decreases.
\item The awareness of the global best solution is given the ants. In PSO each individual particle knows its best position so far, in addition to the best position achieved among all the particles. To implement this, the best ant's results is retained after each iteration of the algorithm. If an edge is walked by the best ant so far, the probability that this also will be walked by ant $a$ increases.
\item The \textit{following} feature from BSO. Deciding which bee to follow is considered to be a function of the quality of the food source to the recruiter. After the first run, some ants are initialized as ``uncommitted followers'', and they follow the $n$ best ants from the previous iteration unconditionally.
\end{itemize}

%What we hope to achieve with these features is overcoming the ACO limitation of getting stuck at a local optima, as well achieving better results concerning the Performance Criteria, introduced in Section \vref{sec:performanceCriteria}.

%%Artifacts:
%\begin{enumerate}
%\item The ants have memory. It can remember whether or not it has visited a node in the creation of the route set.
%\item The ants have a notion of the global best ant. The ant will favor the edge chosen by the global best ant. (PSO)
%\item The ants can be ``crazy''. A crazy ant chooses next nodes at random given possible nodes, not considering the edge %values. This is to compensate for ACO getting stuck at local optima.
%\item An inertia weight is used. The inertia weight is used to calculate the number of crazy ants. At the beginning the number of crazy ants is larger (exploring). The inertia weight is decreased by each iteration, and therefore the number of crazy ants is decreased (exploiting). PSO.  \emph{\color{red} Check if inertia weight is mentioned in the PSO-section. If not write about it and refer to http://www.softcomputing.net/nabic11_7.pdf}
%\item After the first run, some ants are initialized as ``uncommitted followers''. They follow the $n$ best ants from the previous iteration without doubt. (BCO)
%\end{enumerate}



