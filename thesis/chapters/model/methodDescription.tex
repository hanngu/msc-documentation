\section{Combined Swarm System}
\label{section:methodDescription}

The basis of the proposed system, Combined Swarm System (CSS), is the ACO metaheuristic shown in Algorithm \vref{algo:aco}. As initiated in Section \vref{subsec:problemStatement}, some acknowledged attributes, inspired by PSO and BCO, will be added to the solution. The artificial ants generated, henceforth called ants, will also be given ``memory''. This attribute is given the ants to recall whether a node is visited in an earlier route within the same route set.  This attribute is not linked to any optimization method from swarm intelligence. The feature is added because we observed by giving the ants memory, a higher amount of the generated route networks corresponded to a connected graph. This is one of the system's constraints initiated in Section \vref{sec:algoConstraints}.

One of the supplementary attribute inspired by SI is the \textit{Inertia Weight} from PSO. Inertia Weight is a decreasing parameter added to PSO for balancing local and global search. As illustrated in Figure \vref{fig:psoBeginning} and Figure \vref{fig:psoEnd}, the particle in PSO tends to explore more in early iterations, and becoming more organized and coordinated in the later iterations of the algorithm. In the initialization phase of CSS, an amount of the generated ants will be declared ``crazy''. A so-called ``crazy ant'', will work randomly, and not consider edge values when selecting nodes to be included in a route. The probability of an ant being declared ``crazy'' is given by a predefined start value (CA), decreasing iteratively with the inertia weight (IW).

The \textit{following} attribute inspired by BCO is also added. As explained in \vref{subsec:BCO}, deciding which bee to follow in BCO is considered to be a function of the quality of the food source found by the recruiter. After each iteration will an amount of the evaluated best ants be ``followed'' in the next iteration. The ``following ants'' in the next iteration will follow the same path as the best ants from the previous iteration unconditionally. 



%The Crazy Ants are implemented in order to compensate for the original ACO algorithm's weakness of frequently getting stuck at a local optimum. %The IW value, as in PSO, is high in the early iterations, denoting a higher amount of CA, and decreases in pace with the IW. 

 %The probability that ant $a$ is declared crazy will never reach zero, so there always is a probability that $a$ is declared crazy.
%Further, the individual's awareness of the global best solution from PSO is added. Each individual particle in PSO knows the best position achieved among all the particles so far. To implement this feature to CSS, the route set of the global best ant is retained and updated if an ant performs better than the current best ant. If an edge is included in the route set of the global best ant, this increases the probability that this edge will be chosen by other ants. 
 %CA combined with the inertia weight is inspired by the way the inertia weight in PSO favors exploration in early iterations and exploitation in the later.

%The Following Ants and the awareness of the global best solution is implemented in order to boost the algorithm's performance, by favoring good solutions.

%What we hope to achieve with these features is overcoming the ACO limitation of getting stuck at a local optima, as well achieving better results concerning the Performance Criteria, introduced in Section \vref{sec:performanceCriteria}.

%%Artifacts:
%\begin{enumerate}
%\item The ants have memory. It can remember whether or not it has visited a node in the creation of the route set.
%\item The ants have a notion of the global best ant. The ant will favor the edge chosen by the global best ant. (PSO)
%\item The ants can be ``crazy''. A crazy ant chooses next nodes at random given possible nodes, not considering the edge %values. This is to compensate for ACO getting stuck at local optima.
%\item An inertia weight is used. The inertia weight is used to calculate the number of crazy ants. At the beginning the number of crazy ants is larger (exploring). The inertia weight is decreased by each iteration, and therefore the number of crazy ants is decreased (exploiting). PSO.  \emph{\color{red} Check if inertia weight is mentioned in the PSO-section. If not write about it and refer to http://www.softcomputing.net/nabic11_7.pdf}
%\item After the first run, some ants are initialized as ``uncommitted followers''. They follow the $n$ best ants from the previous iteration without doubt. (BCO)
%\end{enumerate}



