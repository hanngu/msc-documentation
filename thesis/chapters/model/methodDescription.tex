\section{The SuperSwarm Algorithm}

The baseline of the proposed SuperSwarm algorithm is an implementation of the basic ACO algorithm introduced by \citet{nanda11} and described in chapter \ref{backgroundAndMotivation}. The basic ACO algorithm have some limitations[ref]. To overcome some of these known limitations, and to answer our research question 2a) (if is it efficient to combine attributes from different swarm intelligence algorithms to solve the vehicle routing problem,) we have added some features from PSO and BSO to the ACO implementation.

First, Inertia Weight (IW) from PSO is added to the algorithm. Inertia weight is a decreasing parameted brought to the PSO for balancing local and global search. In the early iterations of the algorithm, the particles tends to explore more, and becoming more organized and coordinated in the late iterations.  To do this in our algorithm we have implemented a feature called CrazyAnts (CA). A given amount of the ants to be generated become crazy; A CA does not select edges based on the pheromone value, but instead it selects a random node for the next edge. The IW value is high in the start, denoting a higher amount of crazy ants, and when this value decreases (for each iteration), the amount of CA decreases. Second, the awareness of the global best solution is given the ants. In PSO each individual particle knows its best position so far, in addition to the best position achieved among all the particles. To do this, the best ants results is retained after each iteration. The edge of current best ant is added to the probability for this edge to be selected again, and when the ants are selecting the next edge (given that it is not the first ant), the chance for this best edge to be selected becomes greater. What we hope to achieve with these features is overcoming the ACO limitation of getting stuck at a local optima.

Third, the so called \textit{following} feature from BSO is added to the algorithm. Deciding which bee to follow is considered to be a function of the quality of the food source to the recruiter. To implement this feature to our algorithm, an amount of the ants will not generate new route sets, but instead select the same routes as the final best ants. This feature is added to reward the edges in the final best route sets with a higher pheromone value, and with this, hopefully achieve better results concerning the performance criteria, introduced in section \ref{sec:performanceCriteria} on page \pageref{sec:performanceCriteria}. 




