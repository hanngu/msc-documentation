\section{Initialization}
\label{sec:algoInitialization}
\subsection{Parameters}
Initial values for the algorithm parameters are set to a numeric value. The parameters to be set for the algorithm are:
\begin{enumerate}
\item The SuperSwarm Colony Size, $s$. 
\item The numbers of iterations (which is the stop criteria), $i$.
\item The percentage of pheromones to evaporate at each iteration, $E$.
\item The pheromone constant, $p_v$, to determine how much pheromone to be added to each edge as it is visited by ants.
\item The pheromone constant, $p_b$, to determine how much extra pheromone to be granted to edges included in the \textit{n} best route sets.
\item The percentage of complete route sets to be granted extra pheromone, $BR$.
\item The percentage of ants to be followed, $FA$.
\item The probability that a given ant is declared ``crazy'', $CA$.
\item Number of routes in a complete route set, $RS_{max}$. 
\item The maximum number of nodes in a route, $R_{max}$.
\item The minimum number of nodes in a route, $R_{min}$.
\item Initial value for the inertia Weight, $IW$.
\end{enumerate}
The value of parameters 1-8 is determined by a conducted experiment described in sections \vref{subsec:parameterSettings_plan}, \vref{subsec:parameterSettings_setup}, and \vref{subsec:parameterSettings_results}. The value of parameters 9-11 will be the same as the corresponding parameters in \citet{mandl79}, \citet{kechagiopoulos14}, and \citet{nikolic14} for comparison reasons. The value of parameter 12 is constant.   

\subsection{Input Data}
The input data used, described in Section \vref{sec:inputData}, includes:
\begin{itemize}
\item data concerning the structure of the road network (nodes with coordinates)
\item data concerning the travel times between the nodes 
\item data concerning the travel demand between each node couple
\end{itemize}

\subsection{Network Generation}
A network is generated consisting of nodes from the data in the MandlCoordinates file shown in Table \vref{table:MandlCoords} and edges from the data in the MandlDemand file shown in Table \vref{table:MandlDemand}. Travel times are added to the generated edges based on the data from the MandlTravelTimes file shown in Table \vref{table:MandlTravelTimes}. Some of the travel times are described as ``Inf'', meaning there is no direct path between the two nodes in question even though there is a (possibly) high demand between them. These edges are excluded for the most part of the algorithm, because there is in fact not possible to travel between the nodes connected by the edge. However, when the route sets are evaluated, the algorithm awards sets that satisfy node couples with high demand directly. 

The generated network is used to create a Neo4j graph database that holds the created network and all routes created by all ants. As stated in Research Question \vref{itm:3a}, we want to explore the possible advantages and disadvantages of using Neo4j in our implementation. We discovered that Neo4j contains a built-in Dijkstra algorithm, which is used to determine the shortest possible path between every node couple in the graph. 

Dijkstra's algorithm \citet[p.658-662]{cormen09} maintains a set $S$ of vertices's whose final shortest-path weights from the source $s$ have already been determined. The algorithm repeatedly selects the vertex $u = V - S$ with the minimum shortest path estimate, adds $u$ to $S$, and relaxes\footnote{Making a change that reduces constraints.} all edges leaving $u$. The running time of Dijkstra's algorithm is $O((V + E)lg V)$ and it is guaranteed to find the shortest path\citep[p.~661]{cormen09}. The shortest possible paths are used in the evaluation phase is described in Section \vref{sec:algoEvaluation}. 
