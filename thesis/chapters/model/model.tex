Here you will present the architecture or model that you have chosen and that is (or will be) implemented in your work. Note that putting algorithms in your report is not desirable but in certain cases these might be placed in the appendix. Code further be avoided in the report itself but may be delivered in the fashion requested by the supervisor or, in the case of masters delivery, submitted as additional documents. 

In Mandl's demand matrix we changed the numbers in the main diagonal (from to left corner to bottom right corner) from all zero to all 'a's. 

\textit{This is the main structure of what you built.
- Not at code level, but you can include pseudocode.
Explain the system in a way the reader understands it.
Include diagrams and the algorithms used.}

\subsection{Research question 2}
Using Mandl, the constraints and evaluation criteria defined under, we will manage to answer research question number 2: Is it efficient to combine attributes from different swarm intelligence methods to optimize a transit network? This is because we facilitate the comparison with other algorithms results mentioned in the literature. 

\textbf{Initialization}
\begin{itemize}
\item The route size is predefined. The algorithm / routes shall not exceed the minimum or maximum limit of nodes.
\end{itemize}

\textbf{Constraints}
\begin{itemize}
\item A passenger should be able to travel between any two nodes in the route network. In other words, it should be a connected graph. 
\item No cycles (or backtracks) in the graph is allowed.

\end{itemize}

\textbf{Evaluation criteria}
\begin{itemize}
\item Total travel time (should be as low as possible)
\item Number of transfers
\begin{itemize}
\item Number of direct travelers (should be as high as possible)
\item Travelers with one transfer
\item Travelers with two transfers
\end{itemize}
\item Number of unsatisfied customers (should be as low as possible)
\item The nodes with high demand should be as high as possible (have higher priority)
\end{itemize}


\textbf{Algorithm}
\begin{itemize}
\item Stop criteria: a number of iterations. 
\item Start node: random? (find out what is done and what is best)
\item For all ants in A until every ant has a solution
\item for all edges in B, give higher pheromone based on evaluation criteria. 
\item for all edges in E, evaporation over time. 
\item 
\end{itemize}


