\section{Development Environment}

The proposed algorithm is programmed in Java programming language. Java is chosen because we are most familiar with that language and therefor believe the development time is reduced. Further, Java is one of the most used programming languages world wide, and the possibility of further contributions is therefore enhanced by using Java. An additional benefit is that Neo4j offers a rich set of integration possibilities for Java and that these possibilities are well documented. Nevertheless, compared to languages like C++, Java is often considered slow and memory-intensive \citep{alnaser12}. However, \citet{sestoft10} states that managed languages like Java are easier and safer to use than traditional languages like C++. \citet{sestoft10} concludes, based on his conducted experiments, that ``there is no obvious relation between the execution speeds of different software platforms, even for the very simple programs studied here: the C, C\# and Java platforms are variously fastest and slowest''.

An embedded version of the Neo4j database is used in order to represent the network including nodes, edges and the created routes. An embedded database in preferred to lower the latency of the many reads and writes executed when running the algorithm compared to a stand-alone version. 

In order to handle project dependencies, such as Neo4j, and configuration details, such as how much memory that is allocated to the program, Apache Maven\citep{website:maven} is used. Maven is a tool that can be used for building and managing any Java-based project. 