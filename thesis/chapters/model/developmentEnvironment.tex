\section{Development environment}

The proposed system is implemented using the Java programming language. In addition to being the programming language we are the most familiar with, Java is one of the most used programming languages worldwide. Henceforth, the possibility of further contributions will enhance. Further, Neo4j offers a rich set of integration possibilities for Java which are well documented. Nevertheless, compared to languages like C++, Java is often considered slow and memory-intensive \citep{alnaser12}. However, \citet{sestoft10} states that managed languages like Java are easier and safer to use than traditional languages like C++. \citet{sestoft10} concludes, based on his conducted experiments, that ``there is no obvious relation between the execution speeds of different software platforms, even for the very simple programs studied here: the C, C\#, and Java platforms are variously fastest and slowest''.

An embedded version of the Neo4j database is used to represent the network including nodes, edges and the created routes. An embedded database is preferred to lower the latency of the many reads and writes executed when running the system compared to a stand-alone version. 

To handle project dependencies, such as Neo4j, and configuration details, such as how much memory that is allocated to the system, Apache Maven\citep{website:maven} is used. Maven is a tool that can be used for building and managing any Java-based project. 