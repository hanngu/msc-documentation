\section{Creating the Route Set}
\label{sec:algoCreatingRouteSet}

When a node is selected as the next node, it is added to the route, and declared as the current node. If the current route has reached its maximum number, $R_{max}$, of nodes the current route is added to the route set and a new route is initialized. As Constraint \vref{itm:constraintRouteSetSize} specifies, the route set size is predefined, and if the route set has reached its limit, $RS$, no new route will be created. If $RS$ is reached, $a$ has finished its tour, the route set is added to the Neo4j database and $a$ is ready for evaluation. 

For each route created by each ant in each iteration a RelationshipType is created in Neo4j. A Relationship of the given RelationshipType is created for each edge used in each route. Neo4j's built-in Dijkstra Algorithm is used to find both the shortest possible path in the network and the shortest possible path using a given route set. Dijkstra's algorithm \cite[p.658-662]{cormen09} maintains a set $S$ of vertices's whose final shortest-path weights from the source $s$ have already been determined. The algorithm repeatedly selects the vertex $u = V - S$ with the minimum shortest path estimate, adds $u$ to $S$, and relaxes\footnote{Making a change that reduces constraints.} all edges leaving $u$. The running time of Dijkstra's algorithm is $O((V + E)lg V)$ and it is guaranteed to find the shortest path\cite[p.~661]{cormen09}. The built-in Dijkstra in Neo4j takes one or more RelationshipTypes as input, and finds the shortest path using only the given RelationshipTypes, if such a path exists. The shortest possible paths are used in the evaluation phase described in Section \vref{sec:algoEvaluation}.