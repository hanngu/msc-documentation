\section{Objective function?}
\textbf{Reserach Question 2: }
Using Mandl, the constraints and evaluation criteria defined under, we will manage to answer research question number 2: Is it efficient to combine attributes from different swarm intelligence methods to optimize a transit network? This is because we facilitate the comparison with other algorithms results mentioned in the literature. 

\textbf{Constraints}
\begin{enumerate}
\item \label{itm:constraintCycles} No cycles (or backtracks) in the graph is allowed.
\item \label{itm:constraintRouteSize} The route size is predefined. The algorithm / routes shall not exceed the minimum or maximum limit of nodes.
\item \label{itm:constraintRouteSet} To simplify the problem, there are exactly r routes in the route set.
\end{enumerate}

\textbf{Evaluation criteria}
\par
These arcs/edges will be given higher pheromone
\begin{enumerate}
\item \label{itm:criteriaConnectedGraph} It should be a connected graph. (A passenger should be able to travel between any two nodes in the route network. In other words, it should be a connected graph.) 
\item \label{itm:criteriaTotalTravelTime} Total travel time (should be as low as possible)
\item \label{itm:TODO} Number of transfers
\begin{itemize}
\item Number of direct travelers (should be as high as possible)
\item Travelers with one transfer
\item Travelers with two transfers
\end{itemize}
\item \label{itm:TODO} Number of unsatisfied customers (should be as low as possible)
\item \label{itm:TODO} The nodes with high demand should have higher priority for direct routes.
\end{enumerate}

