\section{Pheromone Evaporation}

To simulate how pheromone on paths in nature evaporate, a percentage of the pheromone values, $E$, on each edge, $e$ in the network is removed after each iteration. In \citet{nanda11}'s implementation of the ACO algorithm, a constant is used for the pheromone evaporation. We experienced, however, that the difference of the pheromone level on the edges was huge. A small constant would hardly affect the edges with the most pheromone and a large constant would affect the edges with the least pheromone more than desired. Instead, a percentage is used to determine the evaporation.  
%In the early iterations the ants acts more randomly, and the edges chosen is not always the best. Among others, evaporation ensures that the pheromone value of edges used in early iterations, but not in the latter, decreases. 
The formula for updating the pheromone value $e_p$ on edge $e$ according to evaporation is as follows: 
\newline
$$e_p \minuseq e_p*E$$
\newline
The percentage is, similar to many other parameters used, selected after excessive testing. 

% and not a constant. This is done because we experienced that the difference of the pheromone level on the edges was huge. A small constant would hardly affect the edges with the most pheromone and a large constant would affect the edges with the least pheromone more than desired. 

\emph{\color{blue}Her må vi ha ei litta avslutning.}