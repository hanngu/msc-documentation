\section{Route Set Representation}
For the route set representation, a two dimensional array is used. In the route set representation presented in Table \vref{table:routeSetRepr}, Route 1 starts from node 14, continues to node 10, and so on. The maximal numbers of nodes in a route, $R_{max}$, is sat to 8. Route 3 and Route 4 consists of less than 8 nodes. This happens because the ant has reached a node were there are no edges to walk without violating Constraint \vref{itm:constraintCycles}.
\begin{table}[H]
    \begin{center}
        \begin{tabular}{|l| l l l l l l l l|}
      \hline
        Route 1: & 14 & 10 & 13 & 11 & 12 & 4 & 6 & 15 \\
        Route 2: & 9 & 15 & 7 & 10 & 8 & 6 & 4 & 5 \\
        Route 3: & 6 & 4 & 2 & 1 & - & - & - & - \\
        Route 4: & 4 & 6 & 3 & 2 & 1 & - & - & - \\
      \hline
        \end{tabular}
    \end{center}
    \caption {Route Set Representation}
    \label{table:routeSetRepr}
\end{table}

In Neo4j, each route, R, produced by each ant, A, in each iteration, I, is given a RelationshipType, $RT$.

$$RT_j = I_{in}A_{an}R_{rn}$$

where in is the given iteration number, an is the ant number, and rn is the route number.
%TODO fortsett her