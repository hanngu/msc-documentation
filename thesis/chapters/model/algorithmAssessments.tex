\section{Algorithm Assessments}
The aim of designing a route network is to optimize specific criteria that define its efficiency and quality. In addition, some real world constraints should be satisfied. The evaluation criteria and constraints are adopted by many researchers in the literature, including \citet{kechagiopoulos14}.

\subsection{Constraints}
\label{sec:algoConstraints}
During the route set generation, described in Section \vref{sec:algoCreatingRouteSet}, the algorithm constraints are the following:
\begin{enumerate}
\item \label{itm:constraintCycles} No cycles (or backtracks) in the graph is allowed. I.e. a node visited once in a route should not be visited again in the same route. 
\item \label{itm:constraintRouteSize} The route size is predefined. The routes shall not exceed the minimum or maximum limit of nodes.
\item \label{itm:constraintRouteSetSize} The route set size is predefined. When ant $a$ reaches the number of allowed routes, $a$ is finished and ready for evaluation. This constraint is usually sat by the service provided due to cost limitations.
\item \label{itm:criteriaConnectedGraph} It must be a connected graph, because a passenger should be able to travel between any two nodes in the route network.
\end{enumerate}

\subsection{Evaluation Criteria} 
This evaluation of the ant's performance is done after the route set generation, and is based on the following evaluation criteria:
\begin{enumerate}
\item \label{itm:criteriaTotalTravelTime} Total travel time, referred to as $F_1(r)$.
\item \label{itm:f2} Number of transfers, referred to as $F_2(r)$.
\item Number of unsatisfied customers, referred to as $F_3(r)$. 
\end{enumerate}
These evaluation criteria are used to calculate the total fit of route set $r$ ($TOTFIT(r)$). This process is explained in Section \vref{sec:totfit}.
%(The nodes with high demand should have higher priority for direct routes.)



