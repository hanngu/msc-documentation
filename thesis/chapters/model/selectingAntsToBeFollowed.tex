\section{Selecting Ants to be Followed}
\label{sec:selctingAntsToBeFollowed}

After the $TOTFIT(r)$ value of each ant is calculated, the ants are added to a list sorted in descending order, where the ant with the best $TOTFIT(r)$ value is placed in the first position. As initiated in Section \vref{sec:algoGeneratingSuperSwarm}, becomes a percentage of the colony followers, and follows the same path as one of the best ants from the previous iteration. To determine the number of ants to be followed for the next iterations, a value $NAF$, is calculated:

$$NAF = ants_{size} * AF$$
 
where $ants_{size}$ is the number of ants that satisfied the constraints, described in Section \vref{sec:algoRemoval}, and the value of $AF$ is selected after excessive testing. Because the list is sorted with respect to the $TOTFIT(r)$ value, the $NAF$ first ants in the list will be the $NAF$ best ants, and these are chosen to get a follower in the next iteration. %The number of followers will also be the same amount of ants to be followed, where $FA_1$ will follow $ba_1$, $FA_2$ will follow $ba_2$, and so on.  

To boost the algorithm's performance, the edges in the $NAF$ are granted more pheromone. To do this, a pheromone constant, $p_b$, is added to the edges in these routes. Further, the formula for updating the pheromone value, $e_p$, of edge $e$, is as follows:

$$e_p \pluseq \tau*(\frac{p_b}{T_e})$$ 

where $T_e$ is the travel time of edge $e$, and $\tau$ is the number of route sets the edge appeared in. As mentioned in Section \vref{sec:selectingNextNode}, is the pheromone value divided on the travel time to benefit edges with smaller travel time.


 