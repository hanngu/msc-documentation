\section{Selecting Start Node}

At the creation of each route in each ant's route set a start node is chosen. This start node is chosen at random to allow for a variation in the routes created. We experienced, however, that if the start node selected was connected to another node that only connected to one edge, the ant would often get stuck with only two nodes in the route set. Because we do not allow backtracking, a node connected to only one edge will always be a start or an end node in a route. To achieve better results by preventing routes containing only two nodes, we therefor said that nodes connected to a node with only one connecting edge could not be start nodes. Examples of such tabu nodes are node 2 and node 15 in Mandl's Transit Network (Figure \ref{fig:MandlNetwork} on page \pageref{fig:MandlNetwork}). They are respectively connected node 1 and node 9, which are again only connected to one edge. 