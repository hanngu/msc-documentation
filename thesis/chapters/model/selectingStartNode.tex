\section{Selecting Start Node}

At the creation of each route in each ant's route set, a start node is chosen. This start node should ideally be selected randomly to allow for a variation in the routes created. We experienced, however, that if the start node is connected to another node that is only connected by one edge, the ant would often get stuck with only two nodes in the route set. Constraint \vref{itm:constraintCycles} specifies that no backtracking is allowed, and a node connected to only one edge will always be a start or an end node in a route. To prevent routes from containing only two nodes, nodes connected to a node with only one connecting edge are not selected as a start node. Examples of such tabu nodes are node 2 and node 15 found in Figure \vref{fig:MandlNetwork_problemstatement}. 