\section{The Proposed Algorithm}

We started out with implementing the basic ACO algorithm by \citet{nanda11} as described in chapter \ref{backgroundAndMotivation}.

To overcome some of the known limitations of ACO, and to answer our research question: is it efficient to combine attributes from different swarm intelligence algorithms to solve the vehicle routing problem, we have added some features to the basic ACO

\begin{itemize}
\item \textbf{Inertia weight(IW) from PSO}. CrazyAnts: picking a random edge, not based on the pheromone value on the edge. IW his high in the start; more CrazyAnts, but decreases over time (each iteration) resulting in less and less CrazyAnts. This is done to overcome the known limitation of ACO: getting stuck on local optima.
\item \textbf{Global best solution from PSO}. Each iteration, the best ant is retained - and when the next ant is selecting the next edge, the edge of current best ant is added to the probability for this edge to be selected again, resulting in an even greater chance for this edge to be selected.
\item \textbf{Followers from BSO}. 10\% of the ants selects the same routes as the final best ants, and the pheromone value for the edges in these route sets decreases.
\end{itemize}

: SuperSwarm

\subsection{Initialization}
Initial values for algorithm parameters are set to a numeric value. The parameters to be set for this algorithm are:
\begin{enumerate}
\item The ant colony size. 
\item The numbers of iterations (which is the stop criteria).
\item The evaporation constant, $pheromone_{evaporation}$, to determine how much pheromone to be removed at each edge at each iteration.
\item The pheromone constant, $pheromone_{visited}$, to determine how much pheromone to be added to each edge as it is visited by ants.
\item The pheromone constant, $pheromone_{bestRoutes}$, to determine how much extra pheromone to be granted to edges included in the \textit{n} best route sets.
\item The percentage of complete route sets to be granted extra pheromone.
\item Number of routes in a complete route set. 
\item The maximum and minimum number of nodes in the routes.
\end{enumerate}
The value of parameters 1-6 are determined by a conducted experiment described in \ref{subsec:parameterSettings_plan}, \ref{subsec:parameterSettings_setup}, and \ref{subsec:parameterSettings_results} on the pages \pageref{subsec:parameterSettings_plan}, \pageref{subsec:parameterSettings_setup}, and \pageref{subsec:parameterSettings_results}. The value of parameters 7-8 will be the same as the corresponding parameters in \citet{kechagiopoulos14} and \citet{nikolic14} for comparison reasons. 

Further, a network is generated consisting of nodes from the data in the MandlCoordinates file shown in table \ref{table:MandlCoords} on page \pageref{table:MandlCoords} and edges from the data in the MandlDemand file shown in table \ref{table:MandlDemand} on page \pageref{table:MandlDemand}. Travel times are added to the generated edges based on the data in the MandlTravelTimes file shown in table \ref{table:MandlTravelTimes} on page \pageref{table:MandlTravelTimes}. As one can see in table \ref{table:MandlTravelTimes}, some of the travel times are described as ``Inf'', meaning there is no direct path between the two nodes in question even though there is a (possibly) high demand between them. Edges with an ``Inf'' travel time are excluded for the most part of the algorithm, because there is in fact not possible to travel between the nodes connected by the edge. However, when the route sets are evaluated, the algorithm awards sets that satisfy node couples with high demand directly. After this, the generated network is used to create a Neo4j graph database. As stated in research question \ref{itm:3a}, we want to explore the possible advantages and/or disadvantages of using Neo4j in our implementation.



\subsection{Selecting the start node}
The specific steps executed to select the start node of each route are the following:
%Selecting the first node: 
\begin{itemize}
\item[Step 1] The demand value for each node is estimated, which is the sum of each line in the demand table \ref{table:MandlDemand} on page \pageref{table:MandlDemand}. 
\item[Step 2] The nodes is sorted in descending order based on the demand value of each node.
\item[Step 3] The first k nodes from the list is selected, which comprise the initial node set (INS). 
\item[Step 4] Based on the demand value belonging to INS, a probability is assigned to each node, which reflects the probability of each node to be selected as the first node of the route. 
\item[Step 5] A random node is selected based on the values of probabilities. (To prevent a route from having less than the minimum number of nodes, constraint \ref{itm:constraintRouteSize}, it is not possible to select a node with a connected node that only has one edge connected to it.)
\end{itemize}
%Start node for the next ants: Possible hot spots may be detected, and that this can be used as a start node for the next ants. 

\subsection{Selecting the next nodes}
The next nodes is selected based on the demand value, the pheromone value and the visited status value for the edges. \emph{\color{red} A good route network will ensure that routes having the most traveling demands are satisfied with short paths and few vehicle transfers as stated in the problem statement.} The visited status value checks if the node is visited in an earlier route within the same route set. Its value is 0 if it is unvisited, else 1, giving the unvisited nodes a higher probability of being selected. It is only possible to select edges connected by travel time when this indicates whether it is a direct link between the two nodes, but the travel time value is not taken into account before the evaluation phase \emph{\color{red} because..}.  %TODO: give higher score if there is direct routes between node couples with high demand. 
The specific steps executed to select the next nodes of each route are the following:
\begin{itemize}
\item[Step 1] The probability, $P_k$, for the next node to be selected is calculated by adding the pheromone(P), demand(D) and an visited status value (V), and dividing it by the total demand (TD), total pheromone (TP) and the total visited status value for all the connected edges to the node (TV):
$$ P_{k} = \frac{D}{TD} + \frac{P}{TP} + \frac{V}{TV}$$ 

\item[Step 2] After each edge is given a probability, these are added in a range starting from 3, decreasing with the probability value for each edge. 
\item[Step 3] A value between 3 and 0 is then random selected, giving the edges with highest probability a larger range than the others, resulting in a larger probability for this edge to be selected. 
\item[Step 4] After an node is selected, the pheromone value for this edge increases 
\end{itemize}

\subsection{Creating the route set}
The node selecting phase stops if it exceeds $k$ nodes, as constraint \ref{itm:constraintRouteSize} specifies; the route size is predefined, so the routes shall not exceed the maximum limit of nodes. It also stops if it is ``stuck'', because it is should not be possible to select the same node twice, according to constraint \ref{itm:constraintCycles}; no cycles (or backtracks) in the graph is allowed. The ant then begins the next route exploring. The route exploring stops when it exceeds the maximum number of routes, as specified in constraint \ref{itm:constraintRouteSet}; there are exactly $r$ routes in the route set. The routes are then added to the ant's route set for evaluation.

\subsection{Route set evaluation}


\begin{itemize}
\item[Step 1] Only the ants with a connected route set is selected for evaluation. This is important because based on constraint \ref{itm:criteriaConnectedGraph}, a passenger should be allowed to travel from every node to every other node within the route set. In an undirected graph $G$ two vertices $u$ and $v$ are called connected if G contains a path from $u$ to $v$. An ants route set can have included all nodes, but still be disconnected. The ants with a route set that is disconnected is therefore removed from the best ants. The complexity of evaluating this during the route generation stage is the reason this is evaluated after. 

\item[Step 2] This ants are added to the ``best route set'' for evaluation. (Evaluation a route itself as no sense, since its path depends on the rest of the members of the same route set. As a result, all routes of a route set should be evaluated as a whole. This evaluation is based on the evaluation criteria stated in previous section.)

\item [Step 3] The route sets are added to the Neo4J database, and evaluated by a fitness function, TOTFIT(r), where $r$ is the route set, inspired by \citep{kechagiopoulos14}.
This fitness value is calculated as follows:
$$ TOTFIT(r) = F_{1}(r) + F_{2}(r) + F_{3}(r), $$

where $ F_{1}(r)$ is the score obtained by evaluating the route set by the total in-vehicle time, $ F_{2}(r)$ is the score obtained by evaluating the route set using evaluation criteria direct traveler, one transfer, two transfers, and $ F_{3}(r)$ is the score obtained by evaluating the route set using the criteria unsatisfied customers (more than two transfers). 
%$ w_{1}, w_{2}$, and$ w_{3}$ are user specified weights for scores $ F_{1}(r), F_{2}(r),$ and $ F_{3}(r)$, respectively. 
%To find w, we will calculate an average with ratio, and give weights from that, and theese weights sum is 1.  
\end{itemize}

\subsubsection{Calculating $F_{1}(r)$}

This score is obtained by evaluating how good the path between two nodes, $k_1$ and $k_2$, in the route set is. The score is calculated by finding the difference ($DIF$) in travel time between the edges included in the route set, and the actual minimum travel time (shortest path):

$$ DIF = TT(k_{1},k_{2}) - MTT(k_{1},k_{2})$$

where $ TT(k_{1},k_{2})$ is the travel time using the edges included in the route set, and $ MTT(k_{1},k_{2})$ is the actual minimum travel time (shortest path) between the two nodes, regardless of the edges included in the route set. Calculating the minimum travel time (shortest path) is done by Neo4j, and the selection of which algorithm used is described under.

\paragraph{Calculating the Minimum Travel Time}\mbox{}\\ 

Neo4j includes two built-in algorithms for finding the shortest path, including A* search\citep{russel10}  and Dijkstra's algorithm\citep{cormen09}. 

A* search\cite[p.93-94]{russel10} evaluates nodes by combining $g(n)$, the cost to reach the node, and $h(n)$, the cost to get from the node to the goal.

$$ f(n) = g(n) + h(n).$$
Since g(n) gives the path cost from the start node to node $n$, and $h(n)$ is the estimated cost of the cheapest path from $n$ to the goal, we have
$$f(n) = \text{estimated cost of the cheapest solution through $n$} $$ 

A* search is both optimal and complete, provided that $h(n)$ is admissible\footnote{It never overestimates the cost to reach the goal.} or consistent\footnote{When estimating the distance of a given state to a goal state, it is always at most equal to the estimated distance from any neighboring vertex plus the step cost of reaching that neighbor.}. 

$$ \text{A* search running time: } O(E) = O(b^d)$$
\textit{where $b$ is the branching factor (average number of successors per state)}

Dijkstra's algorithm \cite[p.658-662]{cormen09} maintains a set $S$ of vertices's whose final shortest-path weights from the source $s$ have already been determined. The algorithm repeatedly selects the vertex $u = V - S$ with the minimum shortest path estimate, adds $u$ to $S$, and relaxes\footnote{Making a change that reduces constraints.} all edges leaving $u$.

Dijkstra's algorithm is guaranteed to find the shortest path\cite[p.~661]{cormen09}.
$$\text{Dijkstra's algorithm running time: } O((V + E)lg V)$$

\begin{table}[H]
    \begin{center}
        \begin{tabular}{|l|l|l|}
      \hline
      ~ & A* search & Dijkstra's algorithm\\
      \hline
        MIN & 291 & 569 \\
        AVG & 328 & 589 \\
        MAX & 301 & 610 \\
      \hline
        \end{tabular}
    \end{center}
    \caption {A* Search vs. Dijkstra's algorithm}
    \label{table:astarvsdijkstras}
\end{table}

Table \ref{table:astarvsdijkstras} presents the running times for the different algorithms, in seconds, after 20 runs, with 100 iterations, 100 ants, 4 routes,  and max 8 nodes in each route. Observations shows that A* search is faster, \emph{\color{red} because blabla}, and this method is therefore chosen for finding the shortest path in $F_1(r)$.
 
% In other words, it reflects the average time spent by each passenger when traveling along a specific route set. Its value is small if the respective average traveling time is big and big is traveling time is small. In order to estimate F1(r), not only the average traveling time has to be calculated, but one also determine whether this value should be considered big or small. 

\subsubsection{Calculating $F_{2}(r)$}
%TODO: skrive om dette
$F_{2}(r)$ score reflects the percentage of passengers traveling from their origin to their destination either directly, making a single transfer, or transferring twice. Calculating $F_{2}$ is done using the following equation \emph{\color{red} -  kechapocholus har gjort noe mer her, sjekk ut dette..}:
$$F_2(r) = 3xd_0(r)+ 2xd_1(r)+ xd_2(r) $$

where  $d_0(r)$ is the percentage of passengers traveling directly, $d_1(r)$ is the percentage of passengers traveling from their origin to their destination making a single transfer, and $d_2(r)$ is the percentage of passengers traveling from their origin to their destination transferring twice. \emph{\color{red}$x=-1$, and represents? Her må det forklares hvorfor $d_0$ ganges med et høyere tall osv.}. The reason the number is negative is \emph{\color{red}coding reasons. } 

%\begin{itemize}
%\item it checks for the node couple in a route, if the route contains both the nodes, then it is a direct traveler
%\item it checks if the node couple is withing 2 routes
%\item it checks if the the node couple is withing 3 routes
%\end{itemize}

%$f_2(r) = (-3) * directCouples + (-2) * oneTransferCouples + (-1) * twoTransferCouples $

\subsubsection{Calculating $F_3(r)$}
%TODO: skrive om dette
$F_3(r)$ is a score which reflects the percentage of unsatisfied passengers, meaning passengers traveling from their origin to their destination where the number of transfers is more than 3, using route set $r$. It is calculated using the following formula:
$$\emph{\color{red}hannee?}$$

\begin{itemize}
\item[Step 5] 50\% of the best ants are added to the final best ants set.
\item[Step 6] The algorithm awards sets that satisfy node couples with high demand directly. Demand value to consideration.
\item[Step 7] Reward the edges in these routes by giving them more pheromone, and is given by:

$$ \tau_{ij} = \sum_{k=1}^{m} \Delta \tau^k_{ij}$$

where $ \Delta \tau^k_{ij} $ is the amount of pheromone laid on route (i,j) by the $k^{th}$ ant and is given by

$$
\Delta \tau^k_{ij} = \Bigg\{
\begin{array}{l l}
\underline{Q} &  \quad \text{if route (i,j) be traversed by}\\
f_k, &  \quad \text{the $k^{th}$ ant (at the current cycle) , }\\
0 &  \quad \text{otherwise}
\end{array}
$$

Q is a number, tested in parameter setting experiments, and $f_k$: the travel time on the edge.

\end{itemize}

\subsection{Pheromone Evaporation}

The pheromone on each edge decrease by $p\%$ after each iteration. 


\subsection{Route set representation}
For the route set representation, a two dimensional array is selected. In the route set representation presented in table \ref{table:routeSetRepr}, route 1 starts from node 14, continues to node 13, and so on. The routes comprise of 8 nodes, as stated before, the total number of routes is specific, set by the bus company due to cost limitations, while the maximum length is usually defines by the user before the execution of the algorithm \citep{kechagiopoulos14}.
\begin{table}[H]
    \begin{center}
        \begin{tabular}{|l| l l l l l l l l|}
      \hline
        Route 1: & 14 & 13 & 11 & 12 & 4 & 6 & 15 & 9 \\
        Route 2: & 1 & 2 & 5 & 4 & 12 & 11 & 10 & 8 \\
        Route 3: & 9 & 15 & 7 & 10 & 8 & 6 & 3 & 2 \\
        Route 4: & 7 & 10 & 14 & 13 & 11 & 12 & 4 & 5 \\
      \hline
        \end{tabular}
    \end{center}
    \caption {Route set representation}
    \label{table:routeSetRepr}
\end{table}


%TODO: skrive om, og gjøre?
%The number of buses is restricted by some number M. Then, a feasible set of edges is to be found, such that the total transportation time according to some descriptive assignment (with waiting times defined by the number of buses M) is minimized. Since the problem is so complex, only a heuristic algorithm seems appropriate. 
%\begin{itemize}
 %\item Each node will not be served by each bus, but each stop must be served by at least one route ( each node must belong to at least one route. ) Because a passenger must be able to reach a stop from any other stop by using a sequence of routes. But even though all nodes is reachable, many cases people will have to change lines to reach their destination, and this means they have to wait for the next vehicle. Therefore the travel time is the sum of transportation time along the lines plus the sum of waiting. (To include waiting time, means that a node is split into as many nodes as there are lines passing through this node, and if the original node belongs to these new nodes, they are connected by arcs denoting the average travel time. )
 %\item Because changing lines is not only time consuming but also inconvenient for the passengers, many people want to find the one with the least changes necessary. This minimum change path can be found in exactly the same way as the shortest path; instead of assigning the real waiting time for each arc denoting a possible change, one assigns a high value to those arcs, such that a is much greater than the transportation time.
%\end{itemize}



