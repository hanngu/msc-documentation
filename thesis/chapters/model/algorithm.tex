\section{Algorithm}

%TODO: skrive om, og gjøre?
The number of buses is restricted by some number M. Then, a feasible set of edges is to be found, such that the total transportation time according to some descriptive assignment (with waiting times defined by the number of buses M) is minimized. Since the problem is so complex, only a heuristic algorithm seems appropriate. 
\begin{itemize}
 \item Each node will not be served by each bus, but each stop must be served by at least one route ( each node must belong to at least one route. ) Because a passenger must be able to reach a stop from any other stop by using a sequence of routes. But even though all nodes is reachable, many cases people will have to change lines to reach their destination, and this means they have to wait for the next vehicle. Therefore the travel time is the sum of transportation time along the lines plus the sum of waiting. (To include waiting time, means that a node is split into as many nodes as there are lines passing through this node, and if the original node belongs to these new nodes, they are connected by arcs denoting the average travel time. )
 \item Because changing lines is not only time consuming but also inconvenient for the passengers, many people want to find the one with the least changes necessary. This minimum change path can be found in exactly the same way as the shortest path; instead of assigning the real waiting time for each arc denoting a possible change, one assigns a high value to those arcs, such that a is much greater than the transportation time.
\end{itemize}

We started out with implementing the basic ACO algorithm by \citet{nanda11} as described in chapter 2.
\begin{itemize}
\item Stop criteria: a number of iterations. 
\item Start node: random? (find out what is done and what is best) Maybe we should give nodes pheromones as well, so that possible hot spots may be detected, and that this can be used as a start node. 
\item For all ants in A until every ant has a solution
\begin{itemize}
\item Select next node based on demand, travel time and pheromones. The ants are smarter than the nature because they have included features about demand and distance(total travel time)
\item Highest pheromones, highest demand and lowest travel time. But also a random function, for them to explore new routes as well.
\item Calculate the probability for each next node to be selected. This probability is calculated as follows:
\item Probability of demand + probability of pheromone - probability of total travel time (because the lower the prob of TT the better the TT is)
\item Pk = (D/TD) + (P/TP) (TTT is not yet included)
\item After we have given each edge a probability, we add these to a range starting from 2.0 increasing with the probability for each edge. The edge with the highest probability will therefore be in a larger range than the others
\item We then use a math.random generator picking a random number between 2 and 0. 
\item The edges with a higher probability will have a higher change of getting picked (because of the bigger range)
\item Example probability range: [2.0, 1.333, 0.99, 0.3]. 
\end{itemize}
\item Select best ants with best route set
\begin{itemize}
\item Pick out a number of best ants with their complete route set (And because we want the complete route set, not just best routes.) based on the constraint: The ants must have included all nodes. 
\item Pick out the ones who has included all the nodes. Easier to evaluate after, because it will be to complex to evaluate during the route generation. (May happen naturally when cycles are not allowed.? )
\end{itemize}
\item for all edges in B, give higher pheromone based on all the evaluation criteria. (Total travel time, number of transits, higher demand: higher priority)
\begin{itemize}
\item After we have selected the best route sets, we reward the edges in the routes by giving these more pheromone.
\end{itemize}
\item for all edges in E, evaporation over time. 
\item 
\end{itemize}

