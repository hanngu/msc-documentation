


%TODO: skrive om, og gjøre?
%The number of buses is restricted by some number M. Then, a feasible set of edges is to be found, such that the total transportation time according to some descriptive assignment (with waiting times defined by the number of buses M) is minimized. Since the problem is so complex, only a heuristic algorithm seems appropriate. 
%\begin{itemize}
 %\item Each node will not be served by each bus, but each stop must be served by at least one route ( each node must belong to at least one route. ) Because a passenger must be able to reach a stop from any other stop by using a sequence of routes. But even though all nodes is reachable, many cases people will have to change lines to reach their destination, and this means they have to wait for the next vehicle. Therefore the travel time is the sum of transportation time along the lines plus the sum of waiting. (To include waiting time, means that a node is split into as many nodes as there are lines passing through this node, and if the original node belongs to these new nodes, they are connected by arcs denoting the average travel time. )
 %\item Because changing lines is not only time consuming but also inconvenient for the passengers, many people want to find the one with the least changes necessary. This minimum change path can be found in exactly the same way as the shortest path; instead of assigning the real waiting time for each arc denoting a possible change, one assigns a high value to those arcs, such that a is much greater than the transportation time.
%\end{itemize}

Artifacts:
\begin{enumerate}
\item The ants have memory. It can remember whether or not it has visited a node in the creation of the route set.
\item The ants have a notion of the global best ant. The ant will favor the edge chosen by the global best ant. (PSO)
\item The ants can be ``crazy''. A crazy ant chooses next nodes at random given possible nodes, not considering the edge values. This is to compensate for ACO getting stuck at local optima.
\item An inertia weight is used. The inertia weight is used to calculate the number of crazy ants. At the beginning the number of crazy ants is larger (exploring). The inertia weight is decreased by each iteration, and therefore the number of crazy ants is decreased (exploiting). PSO.  \emph{\color{red} Check if inertia weight is mentioned in the PSO-section. If not write about it and refer to http://www.softcomputing.net/nabic11_7.pdf}
\item After the first run, some ants are initialized as ``uncommitted followers''. They follow the $n$ best ants from the previous iteration without doubt. (BCO)
\end{enumerate}



