\section{Evaluation}
\label{sec:algoEvaluation}
After each ant in the SuperSwarm colony has created a complete route set, the ants are evaluated. The results of the evaluation determines which routes that are granted extra pheromone and which ants to be followed in the next iteration. In addition the best ant so far is updated, if the given iteration has produced an ant with a better total fit. In this evaluation, the route sets are evaluated as a whole, because the connectivity and the paths chosen are dependent of the entire route set. 

\subsection{Removing Ants that did not Fulfill the Constraints}
\label{sec:algoRemoval}
The first step of evaluation is the removal of ants that have generated route sets which corresponds to a disconnected graph. This is because constraint \vref{itm:criteriaConnectedGraph} states that a passenger should be able to travel from every node to every other node within the route set. For an undirected graph $G$ to be classified as connected, there must be a path between every pair of nodes. 

\subsection{Calculating Total Fit}
\label{sec:totfit}
After the removal of ants that did not satisfy the constraint mentioned above, the Total Fit $TOTFIT(r)$ for the route sets of the remaining ants are calculated. The calculation of $TOTFIT(r)$ for route set $r$ is inspired by \citet{kechagiopoulos14} and is described as the sum $F_{1}(r)$, $F_{2}(r)$ and $F_{3}(r)$: 
\newline
$$ TOTFIT(r) = F_{1}(r) + F_{2}(r) + F_{3}(r)$$
\subsubsection*{Calculating $F_{1}(r)$}
\label{sec:f1}
$F_{1}(r)$ is related to the total travel time for each passenger in the network. The demands described in Table \vref{tbl:mandlDemand} corresponds to how many passengers that travels between node $i$ and $j$ each day. $F_{1}(r)$ is the sum of the difference between the travel time of the shortest path possible given the route set, $TT_{spr}$, and the shortest path possible in the network, $TT_{spn}$, for every passenger $p$:
\newline
$$F_{1}(r) = \sum\limits^{p}_{p=1}TT_{spr}-TT_{spn}$$
\newline
Both $TT_{spr}$ and $TT_{spn}$ are calculated using the built-in Dijkstra algorithm in Neo4j, and in the theoretical best case $F_{1}(r)$ should be 0, meaning there is no difference between $TT_{spr}$ and $TT_{spn}$ for any node couple. Dijkstra's algorithm \citet[p.658-662]{cormen09} maintains a set $S$ of vertices's whose final shortest-path weights from the source $s$ have already been determined. The algorithm repeatedly selects the vertex $u = V - S$ with the minimum shortest path estimate, adds $u$ to $S$, and relaxes\footnote{Making a change that reduces constraints.} all edges leaving $u$. The running time of Dijkstra's algorithm is $O((V + E)lg V)$ and it is guaranteed to find the shortest path\citep[p.~661]{cormen09}. In \citet{mandl79}, the author proposes two different methods for calculating $TT_{spr}$ given a specific route set. \textit{Method 1} chooses the path with the shortest traveling time, not considering transfer penalties. \textit{Method 2} chooses the path with shortest traveling time, including the transfer penalties. To achieve the most accurate and realistic results this research uses \textit{Method 2} to choose the path of passenger $p$. This gives us the following equation for calculating $TT_{spr}$ between two nodes: 
\newline
$$TT_{spr} = IVT + (\gamma*NT)$$
\newline
where $IVT$ is the in-vehicle travel time, $\gamma$ is the transfer penalty and $NT$ is the number of transfers. For comparison reasons $\gamma$ is set to 5 minutes. $TT_{spn}$ is, as mentioned, the travel time of the shortest possible path in the entire network. $TT_{spn}$ assumes unlimited routes and $TT_{spn}$ equals therefor $IVT$, because no transfers are needed.  

\subsubsection*{Calculating $F_{2}(r)$}
\label{sec:f2}
$F_{2}(r)$ reflects the percentage of passengers traveling from their origin to their destination either directly, making a single transfer, or transferring twice. Calculating $F_{2}$ is done using the following equation: 
\newline
$$F_2(r) = (-\tau*d_0(r)) + (-\phi*d_1(r)) + (-\omega*d_2(r))$$
\newline
where $d_0(r)$ is the percentage of passengers traveling directly, $d_1(r)$ is the percentage of passengers making a single transfer, and $d_2(r)$ is the percentage of passengers transferring twice. In this research $\tau$ is sat to $3$, $\phi$ is sat to $2$ and $\omega$ is sat to $1$. This is done to favor route sets with many direct travelers over route sets with many one transfer travelers and two transfers travelers, and route sets with many one transfer travelers over route sets with many two transfers travelers. $\tau$, $\phi$ and $\omega$ are all negative values because the smaller the values of both $F_{1}(r)$ described in above and $F_{3}(r)$ described below, the better the route set. The theoretical best value of $F_{2}(r)$ is therefor -300, meaning that 100\% of the passengers travels directly from their origin to their destination. 

\subsubsection*{Calculating $F_{3}(r)$}
\label{sec:f3}
$F_3(r)$ is a score which reflects the percentage of unsatisfied passengers. By unsatisfied passengers we mean passengers traveling from their origin to their destination where the number of transfers is more than or equal to 3, using route set $r$. $F_3(r)$ is calculated using the following formula:
\newline
$$F_3(r) = 100 - d_0(r) - d_1(r) - d_2(r)$$
\newline
where, as described above, $d_0(r)$ is the percentage of passengers traveling directly, $d_1(r)$ is the percentage of passengers making a single transfer, and $d_2(r)$ is the percentage of passengers transferring twice, using route set $r$.The number $100$ represent 100\%. The theoretical best value of $F_3(r)$ is 0, meaning that no passenger is unsatisfied. 

$F_{1}(r)$, $F_{2}(r)$, and $F_{3}(r)$ are, as described above summed and the sum corresponds to the total fit of the route set $r$ of ant $a$. The lower the value of $TOTFIT(r)$, the better the route set $r$. 

\section{Selecting Ants to be Followed}
\label{sec:selctingAntsToBeFollowed}
After the total fit of ant $a$ is calculated as described in Section \vref{sec:totfit}, the list of ants are sorted with respect to the total fit so that the ant with the best total fit is placed on position 0. The number of ants to be followed, $NAF$, is calculated as follows:
\newline
$$NAF = ants_{size} * FA$$
\newline 
where $ants_{size}$ is the number of ants that satisfied the constraints (described in Section \vref{sec:algoRemoval}) and $FA$ is the percentage of ants to be followed. As mentioned in \vref{sec:algoInitialization}, $FA$ is a parameter selected after excessive testing to establish which value that results in the minimum average travel time and best total fit. Because the list is now sorted with respect to total fit, the $n$ first ants in the list are now the $n$ best ants with respect to total fitness. Therefor, the $NAF$ first ants in the list are chosen to get a follower in the next iteration. 

\section{Selecting Route Sets to be Granted More Pheromone}
To boost the performance of our algorithm, we want to grant the edges in the $m$ best route sets with more pheromone. Both the pheromone constant $p_b$ to be added to the edges in $m$ and the the percentage $BR$ to determine the size of $m$ are, like $FA$ described in Section \vref{sec:selctingAntsToBeFollowed}, selected after excessive testing. The formula for calculating the number of ants/route sets that are granted extra pheromone, $NBRS$, are as follows:
\newline
$$NBRS = ants_{size} * BR$$
\newline
Because the list of ants is now sorted, as described in Section \vref{sec:selctingAntsToBeFollowed}, the $NBRS$ first ants are also the $NBRS$ best ants. The reader remembers from Section \vref{sec:algoCreatingRouteSet} that every ant contains exactly one route set. Because the ants are evaluated based on its route sets, the route sets of the $NBRS$ best ants are therefor the $NBRS$ best route sets. Further, the edges of the $NBRS$ best route sets are selected to be granted extra pheromone. The formula for updating pheromone value of edge $e$, $e_p$ is as follows:
\newline
$$e_p \pluseq \frac{p_b}{T_e}$$ 
\newline
where $T_e$ is the travel time of edge $e$. As described in Section \vref{sec:selectingNextNode}

[Step 7] Reward the edges in these routes by giving them more pheromone, and is given by:

$$ \tau_{ij} = \sum_{k=1}^{m} \Delta \tau^k_{ij}$$

where $ \Delta \tau^k_{ij} $ is the amount of pheromone laid on route (i,j) by the $k^{th}$ ant and is given by

$$
\Delta \tau^k_{ij} = \Bigg\{
\begin{array}{l l}
\underline{P_{b}} &  \quad \text{if route (i,j) be traversed by}\\
f_k, &  \quad \text{the $k^{th}$ ant (at the current cycle) , }\\
0 &  \quad \text{otherwise}
\end{array}
$$

$p_b$ is a value determined in the parameter setting experiments, and $f_k$ is the travel time on the edge.