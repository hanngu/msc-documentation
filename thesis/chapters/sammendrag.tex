Målet med denne oppgaven er å utvikle et system for å optimalisere rutenettverk i byer, for å videre gjøre offentlig transport mer praktisk for passasjerer. Et godt rutenettverk kan redusere antall biler på veien ved at passasjerer velger kollektivtransport fremfor egne transportmidler. Dette vil igjen gradvis redusere trafikkøer og miljøutslipp.

``Urban Transit Routing Problems'' (UTRP) omhandler konstruksjon av rutenettverk for kollektivtransport. UTRP er et komplekst og ``multiconstrained'' problem, der konstruksjonen av rutenettene kan være både kompliserte og tidkrevende. Metaheuristiske metoder, som sverm intelligens, har vist seg å være effektive for å finne tilstrekkelige løsninger på denne typen ``NP-hard'' problemer. I dette bidraget, er et sverm inspirert optimaliseringssystem utformet og presentert, med formålet om å skape effektive løsninger på UTRP. Det foreslåtte systemet bruker en ``ant colony'' tilnærming med, i motsetning til tidligere løsninger, tilleggsattributter inspirert av ``bee colony optimalization'' og ``particle swarm optimalization''.

En strukturert litteratur søk er gjennomført for å syntetisere relevante primærstudier. Alle resultater blir presentert og analysert. Videre, fordi metaheuristics krever gode parameterverdier for å løse konkrete problemer optimalt, er en grundig gjennomgang og begrunnelse for hvert valgte parameter dokumentert. Denne dokumentasjonen kan være et utgangspunkt for potensiell fremtidig forskning. En sammenligning med en standard ``ant colony optimalization'' (ACO) implementasjon er gjennomført for å vise om det foreslåtte systemet forbedrer en standard ACO. For å undersøke ytelsen til det foreslåtte systemet, sammenlignes de oppnådde resultatene mot resultater i litteraturen, med Mandls benchmark problem som grunnlag. Det foreslåtte systemet er også testet på større nettverk, mer likt nettverk i ekte byer, for å validere om det foreslåtte systemet støtter større nettverk som input. Denne oppgaven vil også rapportere hvordan bruken av grafdatabasen Neo4j har påvirket utviklingen og ytelsen av den foreslåtte løsningen.

Sammenligningen av oppnådde resultater med den standard ACO implementasjonen og andre publiserte resultater er lovende, spesielt når det gjelder gjennomsnittlig reisetid per reisende.
