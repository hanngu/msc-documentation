Målet med denne oppgaven er å optimalisere rutenettverk i byer for å gjøre offentlig transport mer praktisk for passasjerene. Et godt rutenettverk kan redusere antall biler på veien ved at passasjerer favoriserer kollektivtransport over egne transportmidler, som igjen vil gradvis redusere opphoping av biler og miljøutslipp.

Urban Transit Routing Problemer (UTRP) omhandler konstruksjon av rutenettverk for kollektivtransport. UTRP er et komplekst og multi begrenset problem, der konstruksjonen av rutenettene kan være både komplisert og tidkrevende. Metaheuristiske metoder, som sverm intelligens, har vist seg å være effektive for å finne tilstrekkelige løsninger på slike ``NP-hard'' problemer. I dette bidraget, er et sverm inspirert optimalisering system utformet og presentert, med formål om å skape effektive løsninger til UTRP. Det foreslåtte systemet bruker en ``ant colony'' tilnærming med, i motsetning til tidligere løsninger, tilleggs attributter inspirert av ``bee colony optimalization'' og ``particle swarm optimalization''.

Et strukturert litteratur søk er gjennomført for å samle relevante primærstudier, med en presentasjon og analyse av de mottatte studiene. Fordi metaheuristiske metoder krever gode startparametere for å løse konkrete problemer optimalt, er en grundig gjennomgang og begrunnelse for hvert valgte parameter dokumentert. Denne dokumentasjonen vil bidra med å gi et utgangspunkt for potensiell fremtidig forskning. En sammenligning mot en standard ``ant colony optimalization'' (ACO) implementasjon er utført for å vise om det foreslåtte systemet forbedrer en standard ACO. For å demonstrere ytelsen til det foreslåtte systemet, er oppnådde resultater sammenlignes på grunnlag av Mandl’s benchmark problem, som er et allment akseptert benchmark problem. Det foreslåtte systemet er også testet på større nettverk, mer lik ekte rutenettverk, for å validere om det foreslåtte systemet støtter større nettverk som input. Denne oppgaven vil også rapportere hvordan bruken av grafdatabasen Neo4j har påvirket ytelsen og kvaliteten til den foreslåtte løsningen.

Det foreslåtte systemet yter bedre enn implementasjonen av en standard ACO. Sammenligning av oppnådde resultater med andre publiserte resultater er lovende, spesielt med tanke på gjennomsnittlig reisetid per passasjer.

