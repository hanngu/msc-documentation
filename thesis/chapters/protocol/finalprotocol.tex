\chapter{Protocol}

%TODO: The final protocol with the final search term and justifications.
%Ref: https://raw.githubusercontent.com/kenborge/slr-scbw/master/sections/protocol.tex
\noindent
\begin{table}[h]
\label{table:searchterms}
\begin{center}
	{
	\small
    \hspace*{-2.25cm}\begin{tabular}[c]{| m{1.25cm} | m{1.5cm} | m{1.5cm} | m{1.5cm} | m{1.5cm} | m{1.5cm} | m{1.5cm} | m{1.5cm} | m{1.5cm} |}
    \hline
    & \textbf{Group 1} & \textbf{Group 2} & \textbf{Group 3} & \textbf{Group 4} & \textbf{Group 5} & \textbf{Group 6} & \textbf{Group 7} & \textbf{Group 8} \\ \hline   
    \textbf{Term1} & Train & Path optimization & Bee colony optimization & Transit & Artificial intelligence & Multi agent & Routing & Neo4j \\ \hline
    \textbf{Term2} & Plane & \hspace{0pt}Scheduling optimization & Particle swarm optimization & \hspace{0pt}Transportation & AI & & & Graph database \\ \hline
    \textbf{Term3} & Bus & Route optimization & Swarm intelligence & Traffic & Machine Learning & & & \\ \hline
    \textbf{Term4} & Delivery & Planning & Ant colony optimization & Vehicle & & & & \\ \hline
    \textbf{Term5} & & Multimodal & BCO & & & & & \\ \hline
    \textbf{Term6} & & & PSO & & & & & \\ \hline
    \textbf{Term7} & & & ACO & & & & & \\ \hline
    \end{tabular}\hspace*{-2.25cm}
    }
\end{center}

\caption{Matrix of search terms}

\end{table}

\section{Search Terms}

\begin{itemize}

\item Group 1: Train, plane, bus, delivery
\item Group 2: Path optimization, Scheduling Optimization, Route Optimization, Planning, Multimodal
\item Group 3: Bee colony optimization, Particle swarm optimization, Swarm intelligence, Ant colony optimization, BCO, PSO, ACO
\item Group 4: Transit, Transportation, Traffic, Vehicle
\item Group 5: Artificial Intelligence, ai, Machine Learning
\item Group 6: Multi-agent
\item Group 7: Routing
\item Group 8: Neo4j, Graph database

\end{itemize}

\section{Complete Search Term}
\label{searchterm}

\textit{(train OR plane OR bus OR delivery) AND (``path optimization'' OR ``scheduling optimization'' OR ``route optimization'' OR planning OR multimodal) AND (``bee colony optimization'' OR ``particle swarm optimization'' OR ``swarm intelligence'' OR ``ant colony optimization'' OR bco OR pso OR aco) AND (transit OR transportation OR traffic OR vehicle) AND (``artificial intelligence'' OR ai OR ``machine learning'') AND ``multi-agent'' AND routing)}

% Har ditcha gruppe 8 i den "komplette søketermen", fordi vi ikke bruker gruppe 8 som en del av literatursøket, men kun til å "prove a point"

 %Lol & Group 1 & Group 2 & Group 3 & Group 4 & Group 5 & Group 6 & Group 7 & Group 8 \\ \hline
 %   Term 1 & Train & Path optimization & Bee colony optimization & Transit & Artificial intelligence & Multi agent & Routing & Neo4j \\ \hline


\section{Research Questions}
To conduct a structured literature review it is vital to decide the problem to be solved, referred to as P, and the constraints used to guide the search, referred to as C.
\newline
\newline
One of the goals for the environment package for transportation in Trondheim, ``Miljøpakken'', is to reduce percentage of people travelling with cars from 58 "\%" to 50 "\%" by 2018 \cite{website:miljopakken}. If this goal is reached, it will be an increased need for public transportation in Trondheim. There has never been done any optimization of the bus routes in Trondheim, the existing solution is purely based on experience. The problem formulation for this thesis was therefore based on the idea to improve todays solution by optimizing the bus routes using AI-methods. (And as a result of this satisfy the same amount of users today with less resources.)

\begin{itemize}
\item \textbf{P:} “Optimizing the bus routes in Trondheim using AI-methods. “ This problem can be characterized as a \textit{General Pickup and Delivery Problem (GPDP)} \cite[p.22-25]{vehiclerouting}.  
\item \textbf{C:} 
    \begin{enumerate}
        \item To optimize the bus routes in Trondheim we wanted to explore the possibility using methods from swarm intelligence. This idea came from an initial, non-structured literature review were we did a broad search among different artificial intelligence methods and route optimizing. %Todo: Kanskje skrive litt mer om dette? Evt. sitere noe papershit
        \item We believe that a part of solving P is how we choose to represent the network of the bus routes in Trondheim. The chosen algorithms to optimize the routes with respect to minimize the number of resources used will use this representation. We have some experience with the graph database Neo4j. Neo4j has several benefits that we believe we can take advantage of when solving P, including a natural node-edge-structure and the possibility of saving information to both the nodes and edges. We envision that the nodes will represent bus stops, and the edges will represent the connectivity between the stops. 
    \end{enumerate}
\end{itemize}

\textbf{This gives us the following research questions:}
\begin{enumerate}
\item What are the existing solutions to this problem?
\item Which swarm intelligence methods is best suited for optimizing? 
\item Is it convenient to represent and work with this route network as a graph database for this kind of methods?
\item Does this solution help optimize the bus routes? 
\end{enumerate}

\section{Scoring Criteria}
\subsection{Inclusion Criteria}
To exclude irrelevant papers, some inclusion criteria was decided to ensure a level of relevance to the very first pool. 
\newline
\newline
\textbf{Abstract inclusion criteria}
\begin{enumerate}
\item Get a clear view of the studys main concern, and that this is close related to, given P, route optimizing and AI. 
\item The study is a primary study presenting emperical results
\end{enumerate}
\textbf{Full text inclusion criteria}
\begin{enumerate}
\item The study focuses on the use of swarm intellgence to route optimization. 
\item The study describes an artitecture, framwork or design. 
\end{enumerate}

\subsection{Quality Criteria}
