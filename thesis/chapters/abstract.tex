%The abstract is your sales pitch which encourages people to read your work but unlike sales it should be realistic with respect to the contributions of the work. It should include:
%\begin{itemize}
%\item the field of research
%\item a brief motivation for the work
%\item what the research topic is and
%\item the research approach(es) applied. 
%\item contributions
%\end{itemize}

%The abstract length should be roughly half a page of text --- without lists, tables or figures.  

Urban Transit Routing Problems (UTRP) belongs to a class of NP-hard optimization problems, whose optimal solution may not be feasible to find. UTRP deals with the construction of route networks for public transportation. It is a complex and multi constrained problem, in which the creation of route networks can be both time consuming and challenging. Metaheuristics methods like swarm intelligence have proven to be effective for finding sufficient solutions to NP-hard problems. In this contribution, a swarm inspired optimization system is therefore designed and presented, aiming at creating efficient solutions to UTRP. The proposed system uses an ant colony approach with, unlike previous techniques, additional attributes inspired by bee colony optimization and particle swarm optimization. Results are compared on the basis of Mandl's benchmark problem, which is a widely investigated and accepted benchmark problem in the relevant literature. Comparison of obtained results with results published in the literature is promising, especially regarding the average traveling time per traveler.


\section*{Sammendrag}

Urban Transit Routing Problemer (UTRP) tilhører en klasse av NP-harde optimeringsproblemer, der den optimale løsningen ikke nødvendigvis er mulig å finne. UTRP omhandler konstruksjon av rutenettverk for kollektivtransport. Det er et komplekst og multi begrenset problem, der konstruksjon av rutenett kan være både tidkrevende og utfordrende. Metaheuristiske metoder som sverm intelligens har vist seg å være effektive for å finne tilstrekkelige løsninger på NP-harde problemer. I dette bidraget, er et sverm inspirert optimalisering system derfor utformet og presentert, med sikte på å skape effektive løsninger til UTRP. Det foreslåtte systemet bruker en ant colony tilnærming som, i motsetning til tidligere løsninger, har tilleggsattributter inspirert av bee colony optimization og particle swarm optimization. Resultatene er sammenlignet på basis av Mandl benchmark problem, som er et allment akseptert benchmark problem i  relevant litteratur. Sammenligning av oppnådde resultater med resultater publisert i litteraturen er lovende, spesielt med tanke på gjennomsnittlig reisetid per reisende.
