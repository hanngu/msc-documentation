%The abstract is your sales pitch which encourages people to read your work but unlike sales it should be realistic with respect to the contributions of the work. It should include:
%\begin{itemize}
%\item the field of research
%\item a brief motivation for the work
%\item what the research topic is and
%\item the research approach(es) applied. 
%\item contributions
%\end{itemize}

%The abstract length should be roughly half a page of text --- without lists, tables or figures. 

The goal of this thesis is to optimize urban transit networks to make public transportation more convenient for passengers. A good transit network can reduce the number of vehicles on the road as people will favor public transport over their own transport, which will eventually reduce congestion and environmental emissions.

The Urban Transit Routing Problem (UTRP) deals with the creation of route networks.  UTRP is a complex and multiconstrained problem, in which creation of route networks can both be challenging and time consuming. Metaheuristics like swarm intelligence methods have proven to be effective for finding sufficient solutions to these types of NP-hard problems. In this contribution, a swarm inspired optimization system is designed and presented, aiming to create efficient solutions to the UTRP. The proposed system uses an ant colony approach with, unlike previous techniques, additional attributes inspired by bee colony optimization and particle swarm optimization. 

A structured literature review is conducted to synthesize the relevant primary studies, with a presentation and analysis of all retrieved studies.  Because metaheuristics require good initial parameters to solve concrete problems optimally, a thorough review and justification of each selected parameter is documented. This documentation will contribute in providing a starting point for potential future research. A comparison against a standard ant colony optimization (ACO) implementation is performed to demonstrate whether the proposed system improves the standard ACO. To demonstrate the performance of the proposed system, obtained results are compared on the basis of Mandl's benchmark problem, which is a widely investigated and accepted benchmark problem. The proposed system is also tested on larger networks, more similar to real transit networks, to validate whether the proposed system supports larger networks as input. This thesis will also report how the usage of the graph database Neo4j has affected the performance and quality of the proposed solution. 

The proposed system performs better than the standard ACO method. Comparison of obtained results with other published results is promising, especially regarding the average traveling time per transit users. 

\section*{Sammendrag}

Urban Transit Routing Problemer (UTRP) tilhører en klasse av NP-harde optimeringsproblemer, der den optimale løsningen ikke nødvendigvis er mulig å finne. UTRP omhandler konstruksjon av rutenettverk for kollektivtransport. Det er et komplekst og multi begrenset problem, der konstruksjon av rutenett kan være både tidkrevende og utfordrende. Metaheuristiske metoder som sverm intelligens har vist seg å være effektive for å finne tilstrekkelige løsninger på NP-harde problemer. I dette bidraget, er et sverm inspirert optimalisering system derfor utformet og presentert, med sikte på å skape effektive løsninger til UTRP. Det foreslåtte systemet bruker en ant colony tilnærming som, i motsetning til tidligere løsninger, har tilleggsattributter inspirert av bee colony optimization og particle swarm optimization. Resultatene er sammenlignet på basis av Mandl benchmark problem, som er et allment akseptert benchmark problem i  relevant litteratur. Sammenligning av oppnådde resultater med andre publiserte resultater er lovende, spesielt med tanke på gjennomsnittlig reisetid per reisende.
