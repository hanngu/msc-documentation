%The abstract is your sales pitch which encourages people to read your work but unlike sales it should be realistic with respect to the contributions of the work. It should include:
%\begin{itemize}
%\item the field of research
%\item a brief motivation for the work
%\item what the research topic is and
%\item the research approach(es) applied. 
%\item contributions
%\end{itemize}

%The abstract length should be roughly half a page of text --- without lists, tables or figures. 

The goal of this thesis is to develop a system to improve urban transit networks. A good transit network can reduce the number of vehicles on the road as people will favor public transport over private transportation. This will eventually reduce congestion and environmental emissions.%, which is an extent problem worldwide.

The Urban Transit Routing Problem (UTRP) concerns the creation of route networks. UTRP is a complex and multiconstrained problem, in which creation of route networks can both be challenging and time consuming. Metaheuristics like swarm intelligence methods have proven to be effective of finding sufficient solutions to these types of NP-hard problems. In this contribution, a swarm inspired optimization system is designed and presented, aiming to create efficient solutions to the UTRP. The proposed system uses an ant colony approach with, unlike previous techniques, additional attributes inspired by bee colony optimization and particle swarm optimization. 

A structured literature review is conducted to synthesize the relevant primary studies. All retrieved results are presented and analyzed. Further, because metaheuristics require good parameter values to solve concrete problems optimally, a thorough review and justification of each selected parameter is documented. This documentation will contribute in providing a starting point for potential future research. A comparison of a standard ant colony optimization (ACO) implementation is performed to demonstrate whether the proposed system is an improvement. To investigate the performance of the proposed system, obtained results are also compared against results published in the literature, with Mandl's benchmark problem as a basis.  The proposed system is also tested on larger networks, more similar to real transit networks, to validate whether the proposed system supports larger networks as input. This thesis will also report how the usage of the graph database Neo4j has affected the development and performance of the proposed solution.  

Comparison of obtained results with the standard ACO implementation and other published results are promising, especially regarding the average traveling time per transit user. 



