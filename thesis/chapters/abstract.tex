%The abstract is your sales pitch which encourages people to read your work but unlike sales it should be realistic with respect to the contributions of the work. It should include:
%\begin{itemize}
%\item the field of research
%\item a brief motivation for the work
%\item what the research topic is and
%\item the research approach(es) applied. 
%\item contributions
%\end{itemize}

%The abstract length should be roughly half a page of text --- without lists, tables or figures.  

The Urban Transit Routing Problem (UTRP) belongs to a class of NP-hard optimization problems, whose optimal solution is not easy to find. The UTRP deals with the construction of route networks for public transit networks. It is a complex and multi constrained problem, in which the creation of route networks can be both time consuming and challenging. Because of this, multiple candidate solutions will be considered unfeasible and further be rejected. Due to this difficulty, metaheuristics like swarm intelligence methods can be very effective for finding sufficient solutions. 

A swarm inspired optimization system is designed and presented in this contribution, aiming at the efficient solution to UTRP. The proposed system uses an ant colony approach with, unlike previous techniques, additional attributes inspired by bee colony optimization and particle swarm optimization. Results are compared on the basis of Mandl's benchmark problem, which is a widely investigated and accepted benchmark problem in the relevant literature within UTRP. Comparison of the obtained results published in the literature shows promising results, especially regarding the average traveling time per transit users. 


\section*{Sammendrag}

Urban Transit Routing Problemer (UTRP) tilhører en klasse av NP-harde optimeringsproblemer, hvor de optimale løsningene ikke er lett å finne. UTRP omhandler bygging av rutenettverk for nettverk innen kollektivtransport. Det er et komplekst og multi begrenset problem, der etableringen av rutenett kan være både tidkrevende og utfordrende. Flere kandidat løsninger vil derfor anses som ugjennomførte som videre bli avvist. På grunn av denne vanskeligheten kan metaheuristiske metoder som sverm intelligens være svært effektive for å finne tilstrekkelige løsninger.

Et sverm inspirert optimalisering system er utformet og presentert for å bidra til feltet, med sikte på en effektiv løsning for UTRP. Det foreslåtte systemet bruker en ant colony optimalisering med, i motsetning til tidligere løsninger, tilleggs attributter inspirert av bee colony optimalisering og partcle svarm optimalisering. Resultatene er sammenlignet på basis av Mandl's benchmark problem, som er et allment akseptert benchmark og referanse problem i relevant litteratur innenfor UTRP. Sammenligning av de oppnådde resultatene med resultater publisert i litteraturen er lovende, spesielt med tanke på gjennomsnittlig reisetid per offentlig transport passasjer.
