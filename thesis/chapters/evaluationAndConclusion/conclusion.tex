\section{Conclusion}

\subsection*{RQ 2: Is it efficient to add attributes from other swarm intelligence-methods in order to improve the ant colony optimization algorithm?}

We have demonstrated how the classic ant colony optimization (ACO) converge towards a local optimum when the pheromone trail laid by previous ants become excessively attractive to the following once. We have also explained why the additional attributes inspired by other swarm intelligence-methods improved the performance of ACO. \newline

A following feature, inspired by how BCO operates, was added to the proposed SuperSwarm Optimization system (SSO). This feature rewards the edges in the best route sets found so far with extra pheromone. An additional random feature was added to enable exploration of (possible) better solutions regardless of the pheromone trails laid by the previous ants. A decreasing parameter, inspired by PSO, was implemented to balance the global and local search. \newline

%Because the classic ACO does not include a function for rewarding the best solutions found so far, an additional attribute inspired by BCO was added to the proposed solution. BCO includes a recruitment function where other artificial bees follow the same path as the best paths found so far. A similar following feature was added to the proposed solution. Rewarding the edges in the best route sets found so far with extra pheromone improved the performance of the proposed solution. Because the classic ACO get stuck at local optima it will unable exploration in the late iterations. An additional random feature was added to enable exploration of (possible) better solutions regardless of the pheromone trails laid by the previous ants. A decreasing parameter, which is inspired by PSO, was added to the proposed solution to balance the global and local search.\newline

In order to be determent of the increased performance, the proposed solution (SSO) is implemented as a ACO but with the additional features. It is worth mentioning that the additional ``memory'' attribute, inspired by \citet{dorigo96, sedighpour14, poorzahedy11, salehinejad10}, is also responsible for the performance improvement, However, individual tests on the additional parameters inspired by PSO and BCO was conducted and their results showed further improvement. With this, we can conclude that the additional attributes inspired by other swarm intelligence-methods improve the ant colony optimization algorithm. 

%In ACO the shorter paths will be favored over the short once, simply because the shorter paths take shorter paths. The pheromone value on first best paths will become excessively attractive to the following once. This will unable the next ants to explore better solutions. Because the classic ACO does not contain a rewarding function, there will be no possibility to reward the best route sets. The proposed system 

%We can not unambiguously conclude that the additional attributes inspired by BCO and / or PSO was the only reason the performance of ACO improved. Giving the ants memory and thus avoiding cycles has previously proven to be effective concerning the performance of ACO\citep{dorigo96, sedighpour14, poorzahedy11, salehinejad10}.  However, the additional featured inspired by the other swarm intelligence-methods improved the proposed solution further. This makes it reasonable to conclude that it is effective to add attributes from other swarm intelligence-methods in order to improve the ant colony optimization algorithm.

\subsection*{RQ 3: Is it possible to apply the proposed algorithm to optimize urban transit routes in large urban cities?}

We have run the proposed system on the acknowledged Mandl Network which demonstrated good performance. We have also conducted additional experiments on larger networks, more similar to real transit networks. Based on the results of the latter, the proposed solution as-is will not be possible to use for optimizing transit routes in large urban cities. \newline

The number of RelationshipTypes generated by the proposed system is above the limit of allowed RelatinshipTypes in the graph database Neo4j. As an example consists Trondheim of 1289 bus routes compared to the 15 nodes in the Mandl Network. If we were to use the current solution, the number of RelationshipTypes created would be 8 056 250, whereas the number of allowed RelationshipTypes in Neo4j is 65 536. \newline

We have also shown that the run time and the performance of the proposed system is highly dependent on which method is used for evaluating the route sets. And one would have to choose one method at the expense of the other. However, because generating bus routes is an infrequent task in most urban cities we believe a long run time is acceptable, given that the system creates a better solution than the one that already exists. 

\subsection*{Overall conclusion}

The task of solving vehicle routing problems, more precisely the urban transit routing problem (UTRP), using swarm intelligence (SI) methods is attempted solved by several researchers in the community. We have seen a trend in making change in the original ACO algorithm in order to improve the performance.  The attempt of combining attributes from different SI-method seems to be an innovative approach, and the proposed system was implemented to establish whether adding attributes from other swarm intelligence method improved the performance of the classic ACO. Results are compared on the basis of Mandl's benchmark problem of a Swiss bus network, which is probably the widely investigated and accepted benchmark problem in the relevant literature. The goal in this thesis is to optimize urban transit networks in order to increase the number of public passengers. The proposed system shows great promise for optimizing transit routes in such a way that the average travel time experienced by passengers is reduced. We have also investigate how the usage of a Neo4j database affects our development process and the quality of the solution. The graph database includes several features which is advantegous for these types of routing problems. However, to use the proposed system on larger networks, changes or removal of the use of the graph database Neo4j will have to be conducted. 

%The goal was initially to use the proposed algorithm to optimize the bus network in Trondheim, in order to increase the number of public transportation passengers in the city. However, to determine the proposed algorithm's performance, the results had to be compared to a standard. For the urban transit network routing problem (UTRP), Mandl's benchmark problem is widely used in the literature, whereas a recognized metric is established for evaluating the performance. The performance of the proposed algorithm produced better average travel time than all approaches published in the literature. However, the number of direct travels was smaller. Whether a passenger would travel travel direct with a larger travel time versus transferring and thus decrease the travel time, is a matter of preferences.

%\subsubsection*{Goal: Increase the number of public transportation passengers by optimizing urban transit networks}


%Conclusion stuff


%However, even if the number of RelationshipTypes were reduced, there is an other drawback that makes our solution unpractical to use: The large difference in run time between using Method 1 and Method 2 when evaluating the route sets. 
%The reader recalls that we in Section \vref{subsec:scalabilityExperiments_setup} indicate that Method 2 gives a better Total Fitness than Method 1, but that the runtime is more than ten times larger compared to using Method 1. These results are also supported by the research of \citet{fan09}. Nevertheless, generating bus routes is an infrequent task in most urban cities and a because of this we believe that a long run time could be acceptable, given that the system creates a better solution than the one that already exists. 

%A requirement that must be fulfilled for our system to be applicable is files containing average number of demand between every two nodes/bus stops, travel times between nodes/bus stops and a list of all the nodes/bus stops in the network, similar to the ones provided by \citet{mandl79} and \citet{mumford13}. This data can typically be obtained by the bus companies.


%

%The implementation is made adaptable in such a way that different networks (number of nodes, route sets) and parameters (demand values, travel times) can be tested - and the algorithm was tested on larger networks to establish if the algorithm produce viable results regarding Trondheim's transit network. These test showed that the use of Neo4j in the implementation proved this not being possible. mer her

%---------------- Gammel INTRODUCTION ------------------%
%The task of solving vehicle routing problems, more precisely the urban transit routing problem (UTRP), using swarm intelligence (SI) methods is attempted solved by several researchers in the community, described in Section \vref{sec:relatedWork}. However, the attempt of combining attributes from different SI-method seems to be an innovative approach. 

%The proposed algorithm was implemented to determine if combining attributes from different SI-methods was effective. 

%The goal was initially to optimize bus routes in Trondheim to increase the number of public transportation passengers. However, to determine the algorithm's performance, the results will have to be compared to a standard. For the UTRP, Mandl's benchmark problem is used by several researchers in the literature, and a recognized metric is established for evaluating the performance. Not only are these metric used by other researchers and therefore easy to compare with the algorithm's results, but they also corresponds to our goal, described in Section \vref{itm:goal}, which is increasing the number of public transportation passengers. A good solution for passengers and thus for the algorithm is, as mentioned, one that provides a low average travel time, a high percentage of passengers traveling directly or with one transfer form the origin to their destination and a low percentage of both passengers transferring twice and the amount of \textit{unsatisfied} passengers. An unsatisfied passenger is one that needs to transfer more than two times. 

%Mandl's network is, as mentioned, a small network, and only testing the algorithm's performance on Mandl's network will not determine if the proposed algorithm works for a broad number of transit networks. The implementation is made adaptable in such a way that different networks (number of nodes, route sets) and parameters (demand values, travel times) can be tested - and the algorithm is tested on larger networks to establish if the algorithm produce viable results regarding Trondheim's transit network.

