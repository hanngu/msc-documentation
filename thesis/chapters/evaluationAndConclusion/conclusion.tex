\section{Conclusion}



The task of solving vehicle routing problems, more precisely the urban transit routing problem (UTRP), using swarm intelligence (SI) methods is attempted solved by several researchers in the community. However, the attempt of combining attributes from different SI-method seems to be an innovative approach. The proposed algorithm was implemented to determine if combining attributes from different swarm intelligence-methods improved the standard ant colony optimization (ACO) performance. %We are not unambiguously able in concluding that the additional attributes inspired by BCO and / or PSO was the only reason the performance of ACO improved. Giving the ants memory, inspired by \citet{dorigo96}, has previously proven to be effective concerning the performance of ACO. However, the individual tests on the additional SI features improved the ACO performance further. It therefore makes it reasonable to conclude that adding attributes from other swarm intelligence-methods improve the performance of the ant colony optimization algorithm further.


\subsubsection*{RQ 2: Is it efficient to add attributes from other swarm intelligence-methods in order to improve the ant colony optimization algorithm?}

We can not unambiguously conclude that the additional attributes inspired by BCO and / or PSO is the only reason the performance of ACO improved. Giving the ants memory, inspired by \citet{dorigo96}, has previously proven to be effective concerning the performance of ACO. \citet{sedighpour14}, \citet{poorzahedy11} \citet{salehinejad10} also gave the ant's similar features, by maintaining a list of the nodes that were visited, in order to avoid cycles. This function was also added to the proposed algorithm. And as we saw in Section \vref{subsec:evaluating_PerfomanceComparison}, the proposed algorithm performed better already in the first iteration because of the larger amount of produced route sets.

\subsubsection*{RQ 3: Is it possible to apply the proposed algorithm to optimize urban transit routes in large urban cities?}

Based on the results of the Network Expansion experiments we have shown that the proposed solution as-is will not be possible to use for optimizing transit routes in large urban cities. This is mainly because the number of RelationshipTypes generated by the proposed method is above the limit of allowed RelatinshipTypes in Neo4j. 

As an example consists Trondheim city, which is the forth largest city in Norway, of 33 bus routes \citep{website:atb-linjenett}. If we were to use the current solution, including the selected parameters for the swarm size and iteration number, as described in Section \vref{subsec:parameterSettings_results}, the number of RelationshipTypes created would be $125 (number of iterations) * 50 (the swarm size) * 33 (bus routes) = 206 250$. This is more than three times the number of allowed RelationshipTypes in Neo4j at the current date. 

We have also shown that the run time of the proposed method is highly dependent on whether Method 1 or Method 2 is used for evaluating the route sets. The reader recalls that we in Section \vref{subsec:scalabilityExperiments_setup} indicate that Method 2 gives a better Total Fitness than Method 1, but that the runtime is more than ten times larger compared to using Method 1. However, because generating bus routes is an infrequent task in most urban cities we believe a long run time is acceptable, given that the system creates a better solution than the one that already exists. 

Nevertheless, based on the results achieved using the Mandl Network, we can conclude that the proposed method shows great promise for optimizing transit routes in such a way that the average travel time experienced by passengers is reduced. 

%However, even if the number of RelationshipTypes were reduced, there is an other drawback that makes our solution unpractical to use: The large difference in run time between using Method 1 and Method 2 when evaluating the route sets. 
%The reader recalls that we in Section \vref{subsec:scalabilityExperiments_setup} indicate that Method 2 gives a better Total Fitness than Method 1, but that the runtime is more than ten times larger compared to using Method 1. These results are also supported by the research of \citet{fan09}. Nevertheless, generating bus routes is an infrequent task in most urban cities and a because of this we believe that a long run time could be acceptable, given that the system creates a better solution than the one that already exists. 

%A requirement that must be fulfilled for our system to be applicable is files containing average number of demand between every two nodes/bus stops, travel times between nodes/bus stops and a list of all the nodes/bus stops in the network, similar to the ones provided by \citet{mandl79} and \citet{mumford13}. This data can typically be obtained by the bus companies.


The goal was initially to use the proposed algorithm to optimize the bus network in Trondheim, in order to increase the number of public transportation passengers in the city. However, to determine the proposed algorithm's performance, the results had to be compared to a standard. For the urban transit network routing problem (UTRP), Mandl's benchmark problem is widely used in the literature, whereas a recognized metric is established for evaluating the performance. The performance of the proposed algorithm produced better average travel time than all approaches published in the literature. However, the number of direct travels was smaller. Whether a passenger would travel travel direct with a larger travel time versus transferring and thus decrease the travel time, is a matter of preferences.

The implementation is made adaptable in such a way that different networks (number of nodes, route sets) and parameters (demand values, travel times) can be tested - and the algorithm was tested on larger networks to establish if the algorithm produce viable results regarding Trondheim's transit network. These test showed that the use of Neo4j in the implementation proved this not being possible. mer her

