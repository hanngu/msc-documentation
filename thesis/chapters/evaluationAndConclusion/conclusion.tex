\section{Conclusion}

The task of solving vehicle routing problems, more precisely the urban transit routing problem (UTRP), using swarm intelligence (SI) methods is attempted solved by several researchers in the community. However, the attempt of combining attributes from different SI-method seems to be an innovative approach. The proposed algorithm was implemented to determine if combining attributes from different SI-methods was effective. The goal was initially to optimize bus routes in Trondheim to increase the number of public transportation passengers. However, to determine the algorithm's performance, the results will have to be compared to a standard. For the UTRP, Mandl's benchmark problem is used by several researchers in the literature, and a recognized metric is established for evaluating the performance. 

We can not unambiguously conclude that the additional attributes inspired by BCO and / or PSO was the only reason the performance of ACO improved. Giving the ants memory, inspired by \citet{dorigo96}, has previously proven to be effective concerning the performance of ACO. However, the individual tests on the additional parameters all improved the ACO performance further. It therefore makes it reasonable to conclude that adding attributes from other swarm intelligence-methods improve the performance of the ant colony optimization algorithm further.

The implementation is made adaptable in such a way that different networks (number of nodes, route sets) and parameters (demand values, travel times) can be tested - and the algorithm was tested on larger networks to establish if the algorithm produce viable results regarding Trondheim's transit network. Neo4j

