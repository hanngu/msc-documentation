\subsection{Run time}
\label{subsec:runtime}

The run time for the proposed system is dependent on the $NRS$, the number of ants ($s$), the number of iterations ($i$), the size of the network, and whether Method 1 or Method 2 is used.  Of course, the run time is dependent on both the $s$ and $i$ because both parameters increase the number of route sets to be generated and evaluated. Because a route set must be evaluated on the shortest travel time between every two nodes in the network, the run time increases as the network expands. As one can observe from Table \vref{table:results_mumford0runtime}, the run time is highly dependent on whether Method 1 or Method 2 is used. The run time is, in fact, more than ten times greater when using Method 2 compared to Method 1. In Method 1, Dijkstra only will only need to find the absolute shortest path between two nodes, while in Method 2 Dijkstra will need to find all possible paths between the two nodes to further evaluate them based on the travel time and potential transfer penalties. The difference in using Method 1 and Method 2 will increase as both the network and the $NRS$ increases, due to the enlarged number of possible routes. As one can see from the average run times regarding the experiments using the Mandl Network in Table \vref{tabel:runTimeMandl}, the run time increases drastically from 7 to 8 routes. This is may be because, in the generation of 8 routes, the need for more memory led to using an instance from Google with more memory and less CPU power. 
 

%As one can observe from Table \vref{tabel:runTimeMandl} the average run time increases drastically when the number of routes in the Mandl network increases. 


%This may be because $NRT$ and the number of Relationships in the Neo4j database becomes significantly larger when increasing the number of routes. . The run time is also more than 10 times greater using using Method 2 compared to Method 1 as seen in \vref{table:results_mumford0RunTime}. When using Method 1 it is sufficient to use the built in Dijkstra algorithm in Neo4j to only find the shortest path between two nodes given a route set. When Method 2 is used, Dijkstra must find all possible routes between two nodes given a route set, and further evaluates which route to choose based on the total travel time (including transfer penalties). The difference in the run time, using Method 1 compared to Method 2, also increases as the network becomes bigger. This is because the algorithm finds a path between every two nodes in the network, and when the network increases, so does (usually) the number of nodes and possible paths as well. Table \vref{tabel:runTimeMumford} shows that the average run times of Mumford0 and Mumford1. It is worth mentioning, that the run times are approximately the same as the ones for the experiments who uses the Mandl Network, even though these experiments have a smaller network and fewer $NRS$, due to the decreased number of $i$ and the fact that Method 1 is used instead of Method 2. 

%Independent of whether it is the route set size, network size, the number of iterations or the number of ants in the colony that becomes bigger, the number of reads and writes to the Neo4j database increases, which again increases the run time. %\emph{\color{blue} Kanskje vi må si noe mer her, finne noe data på hvorfor dette øker run timen eller noe. Kanskje vi må si noe om at vi har observert at minnebruken aldri går over 30\%?}

\begin{table}[H]
    \centering
    \hspace*{-1.0cm}
    \begin{tabular}{|c|c|}
        \hline
        \textbf{Network} & \textbf{Run time$^1$ } \\
        \hline
        Mumford0 & 2368.0\\
        
        Mumford1 & 5862.4\\
        \hline
    \end{tabular}
    \caption{Average run time in seconds of 10 runs using the Mumford Networks}
    \label{tabel:runTimeMumford}
    \begin{itemize}[noitemsep]
    \item[$^1$:] Swarm size, $s$ = 50, Number of iterations, $i$ = 50
    \end{itemize} 
\end{table}

\begin{table}[H]
    \centering
    \hspace*{-1.0cm}
    \begin{tabular}{|c|c|}
        \hline
        \textbf{Network} & \textbf{Run time$^1$}\\
        \hline
        Mandl (4 routes) & 673.0\\
        
        Mandl (6 routes) & 2442.1\\
      
        Mandl (7 routes) & 3891.1\\
       
        Mandl (8 routes) & 8515.9\\
        \hline
    \end{tabular}
    \caption{Average run time in seconds of 50 runs for each route set size on the Mandl Network}
    \begin{itemize}[noitemsep]
    \item[$^1$:] Swarm size, $s$ = 50, Number of iterations, $i$ = 125
    \end{itemize} 
    \label{tabel:runTimeMandl}
\end{table}

Generating urban transit routes is not a frequent task. Changing the transit routes often, will result in a whole lot of unsatisfied passengers due to the frequent need to adapt to changes. Moreover, once an optimal urban transit network is created, there is no need for change. Because of this, the run time will not be an issue, given that the system creates a better solution than the one that already exists. 