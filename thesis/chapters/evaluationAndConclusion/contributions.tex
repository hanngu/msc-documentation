\section{Contributions}
\label{sec:contributions}
%What are the main contributions made to the field and how significant are these contribution.

%

In this thesis, we proposed a system for the urban transit routing problem. The proposed Combined Swarm System creates feasible and efficient route networks with Mandl's benchmark problem\citep{mandl79} as a basis. The system shows especially promising results concerning the average travel time per transit user. 

We have demonstrated that the performance of a standard ant colony optimization algorithm improves when adding additional attributes inspired by other swarm intelligence methods. 

We also conducted experiments regarding the parameter setting. The parameter settings have a great influence on the performance of metaheuristic methods, like the proposed system. By describing the experiments conducted and justifying the choices regarding the parameter setting, we contributed with a valuable starting point for future research. 

The implemented system also utilize the use of the graph database Neo4j. Neo4j has several advantages, such as storing the objects as a graph and providing built-in graph algorithms. However, we have demonstrated some shortcomings of Neo4j for the proposed system. Both the advantages and shortcomings of Neo4j are considered as a contribution to the field and future research. 

We contributed with the results of a conducted Structured Literature Review. The retrieved literature can provide important information to researchers who attempt to solve vehicle routing problems with swarm intelligence in the future. 


%siste setning på omformuleres


%The system developed is based on  Swarm Intelligence methods. We showed that adding additional attributes to a classic Ant colony optimization method clearly[hardt å melde?] improved the performance. The proposed SuperSwarm optimization system 
