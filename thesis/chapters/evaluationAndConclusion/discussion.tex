\section{Discussion}

Discuss what you managed, and why you had sucess / not success. Show that you understand. In the discussion it is important to include a discussion of not just the merits of the work conducted but also the limitations. 

\textbf{Goal:}
\begin{itemize}
\item  Increase the number of public transportation passengers by making urban transit networks more efficient.
\end{itemize}
\textbf{Research Questions:}
\begin{enumerate}[label=\textbf{\arabic*})]
\item 
    \begin{enumerate}
    \item  Is swarm intelligence methods suitable for the vehicle routing problem?
    \item  What is the state-of-the-art in solving vehicle routing problems using swarm intelligence methods?
    \item  What changes have been done to the classical swarm intelligence-methods to improve them?
    \item  Have there been any attempts to combine different swarm intelligence-methods?
	\end{enumerate}
\item
    \begin{enumerate}
    \item  Is it efficient to add attributes from other swarm intelligence-methods in order to improve the ant colony optimization algorithm?
    \item  How does the proposed algorithm's computational results compare to results published in the literature?
    \item  Is it possible to apply the proposed algorithm to optimize urban transit routes in real urban cities?
    \end{enumerate}
\item
	\begin{enumerate}
    \item  Have graph databases been employed in combination with the vehicle routing problem and swarm intelligence?
	\item  What are the potential advantages and disadvantages of using a graph database in our implementation?
    \end{enumerate}
\end{enumerate}

\textbf{\color{blue} Til diskusjon:}

The parameters tested and used are specifically tuned for Mandl's Transit Network containing 4 routes, with maximum 8 nodes in one route, but the same parameters are used when testing with 6, 7, and 8 routes as well. This may be considered a weakness, because metaherustic methods, such as the one proposed in this thesis, requires calibration of parameters with respect to the problem at hand \citep{dobslaw09}.

The proposed method requires input on a particular form. Both nodes and their coordinates, travel time between nodes and the demand between nodes needs to be provided on a particular form in an .txt-document. This may be considered as a weakness of the method, but implementing support for retrieving the mentioned data on another form are considered as fairly easy.

