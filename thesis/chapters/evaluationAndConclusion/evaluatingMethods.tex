%1. How is the method an improvement over existing technologies?
%(a) Does it account for more situations? (input)
%(b) Does it produce a wider variety of desired behaviors? (output)
%(c) It the method expected to be more efficient? (space, solution time, development time, etc.)
%(d) Does it hold more promise for further development? (for example, due to the opening of a new paradigm)

%2. Is there a recognized metric for evaluating the performance of your method? (eg. normative, cognitively valid, etc.)

%3. Does it rely on other methods? (Does it require input in a particular form or preprocessed? Does it require to a certain type of knowledge base or routines?)

%4. What are the underlying assumptions? (know limitations, scope of expected input, scope of desired output, expected performance criteria, etc.)

%5. What is the scope of the method?
%(a) How extendible is it? Will it scale up to a large knowledge base?
%(b) Does it address exactly the task? portions of the task? a class of tasks?
%(c) Could it, or parts of it, be applied to other problems?
%(d) Does it transfer to more complicated problems? (perhaps more knowledge intensive or more/less constrained or with more complex interactions)

%6. When it cannot provide a good solution, does it do nothing or does it provide bad solutions or does it provide the best solution given the available resources?

%7. How well is the method understood?
%(a) Why does it work?
%(b) Under what circumstances, won't it work?
%(c) Are the limitations of the method inherent or simply not yet addressed?
%(d) Have the design decisions been justified?

%8. What is the relationship between the problem and the method? Why does it work for this task?
