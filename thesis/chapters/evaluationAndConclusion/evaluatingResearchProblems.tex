\subsection{Evaluating Research Problems}

Refining the research topic to a task, and identifying a view.
Task: what we want a computer to do, view: a rough idea how to do it. Can you justify the research task, and is your view how to solve the task viable? 

\textbf{Is the task worthy of attention? Why?}
\emph{\color{orange} Motivation}
Why is is interesting? AtB, pollution ++ 

\textbf{How is your reformulation an improvement?}
Has it been studied before. Yes, the conducted SLR helped us find these results.\emph{\color{orange} Structured Literature Review}
Why do you expect this new perspective to be an improvement? (Cannot afford to spend months implementing a system unless were pretty sure something interesting will happen.)\emph{\color{orange} Problem Statement}
\begin{itemize}
\item Trondheim has not been computationally optimized.
\item ACO with additional features. ACO limitations.ss
\item Neo4j.
\end{itemize}

\textbf{Is the research representative of a class of tasks?}
\emph{\color{orange} Problem Statement, Evaluation}
Developing a general solution.

\textbf{Have any aspects been abstracted away?}
\emph{\color{orange} Problem Statement}
Accurate estimates of travel demand is an important factor for the algorithm, AtB does not possess accurate data about the travel demand, and detailed investigations into measuring and predicting travel demand is an complex research problem, and beyond the scope of this thesis. 

\textbf{When have you successfully demonstrated a solution?}
\emph{\color{orange} Problem Statement, Experiments and Results, Evaluation}
Mandl's benchmark problem, performance criteria.