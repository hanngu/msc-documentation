\section{Conclusion}

The goal with this thesis was to optimize urban transit routes in order to increase the number of public transportation passengers. The task of solving vehicle routing problems, more precisely the urban transit routing problem (UTRP), has proven to be useful using swarm intelligence (SI) methods. The proposed Combined Swarm System is an implementation of the ant colony optimization with additional features inspired by other swarm intelligence methods. Results are compared on the basis of Mandl's benchmark problem of a Swiss bus network, which is probably the widely investigated and accepted benchmark problem in the relevant literature within UTRP. The proposed system shows promise for optimizing transit routes in such a way that the average travel time experienced by passengers is reduced. 

\subsection*{RQ 2: Is it efficient to add attributes from other swarm intelligence methods to improve the ant colony optimization algorithm?}

We have demonstrated how the implementation of a classic ant colony optimization (ACO) converge towards a local optimum when the pheromone trail laid by previous ants become excessively attractive to the following once. The proposed system was tested towards a standard ACO to establish whether adding attributes from other swarm intelligence methods improved its performance. To be determined whether the additional attributes improved the ACO algorithm, the basis of the proposed system use an ACO. By having an otherwise identical implementation allowed direct comparison of the two. %A following feature, inspired by how BCO operates, was added to the proposed system. This feature rewards edges in the best route sets found so far with additional pheromone. In addition, a random feature was added to enable exploration of (possible) better solutions regardless of the pheromone values on the edges. A decreasing parameter, inspired by PSO, was implemented to balance the global and local search. 

We can not unambiguously conclude that the additional attributes inspired by PSO and BCO was the only reason the performance of the ant colony optimization algorithm improved. The additional ``memory'' attribute, inspired by \citet{dorigo96, sedighpour14, poorzahedy11, salehinejad10}, was also implemented to the system and is partly responsible for the performance improvement. However, results from the individual tests conducted on the additional parameters inspired by PSO and BCO demonstrated further improvement. With this, we can conclude that the additional attributes inspired by other swarm intelligence methods improved the ant colony optimization algorithm. 

\subsection*{RQ 3: Is it possible to apply the proposed algorithm to optimize urban transit routes in large urban cities?}

We have conducted additional experiments on larger networks, more similar to real transit networks. We have also investigated how the usage of a Neo4j database affects our development process and the quality of the solution. 

The graph database includes several features that are advantageous for these types of routing problems. However, the proposed solution as-is will not be possible to use for optimizing transit routes in large urban cities. The number of RelationshipTypes generated by the proposed system is above the limit of allowed RelationshipTypes in the graph database Neo4j. As an example consists Trondheim of 1289 bus routes compared to the 15 nodes in the Mandl Network. If we were to use the current solution, the number of RelationshipTypes created would be 8 056 250 whereas the number of allowed RelationshipTypes in Neo4j is 65 536. To use the proposed system on larger networks, changes or removal of the utilization of the graph database Neo4j will have to be done. 

The run time and performance of the proposed system are highly dependent on the method used for evaluation, and one would have to choose one method at the expense of the other. However, because generating bus routes is an infrequent task, we believe an extended run time is acceptable, given that the system creates a better solution than the one that already exists.  