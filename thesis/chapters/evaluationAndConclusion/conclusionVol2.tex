\section{Conclusion}

The goal with this thesis is to optimize urban transit routes in order to increase the number of public transportation passengers. The task of solving vehicle routing problems, more precisely the urban transit routing problem (UTRP), using swarm intelligence (SI) methods has proven to be useful... The proposed SuperSwarm optimization system is an implementation of the ant colony optimization with additional features from other swarm intelligence methods. To demonstrate the performance, results are compared on the basis of Mandl's benchmark problem of a Swiss bus network, which is probably the widely investigated and accepted benchmark problem in the relevant literature within UTRP. The proposed system shows great promise for optimizing transit routes in such a way that the average travel time experienced by passengers is reduced. 
\newline

We have demonstrated how the classic ant colony optimization (ACO) converge towards a local optimum when the pheromone trail laid by previous ants become excessively attractive to the following once. The proposed system was tested with a classic ACO to establish whether adding attributes from other swarm intelligence methods improved the performance of a classic ACO. A following feature, inspired by how BCO operates, was added to the proposed system. This feature rewards edges in the best route sets found so far with extra pheromone. An additional random feature was added to enable exploration of (possible) better solutions regardless of the pheromone trails laid by the previous ants. A decreasing parameter, inspired by PSO, was implemented to balance the global and local search. 

\subsection*{RQ 2: Is it efficient to add attributes from other swarm intelligence-methods in order to improve the ant colony optimization algorithm?}

In order to be determine whether the additional attributes from SI method improved the solution, the proposed system is implemented as a ACO but with the additional features. The additional ``memory'' attribute, inspired by \citet{dorigo96, sedighpour14, poorzahedy11, salehinejad10}, is partly responsible for the performance improvement. However, results from the individual tests conducted on the additional parameters inspired by PSO and BCO showed further improvement. With this, we can conclude that the additional attributes inspired by other swarm intelligence-methods improve the ant colony optimization algorithm. 

\subsection*{RQ 3: Is it possible to apply the proposed algorithm to optimize urban transit routes in large urban cities?}

We have conducted additional experiments on larger networks than the Mandl Network, more similar to real transit networks. Based on the results of the latter, the proposed solution as-is will not be possible to use for optimizing transit routes in large urban cities. We have also investigate how the usage of a Neo4j database affects our development process and the quality of the solution. The graph database includes several features which is advantegous for these types of routing problems.  The number of RelationshipTypes generated by the proposed system is above the limit of allowed RelatinshipTypes in the graph database Neo4j. As an example consists Trondheim of 1289 bus routes compared to the 15 nodes in the Mandl Network. If we were to use the current solution, the number of RelationshipTypes created would be 8 056 250, whereas the number of allowed RelationshipTypes in Neo4j is 65 536. However, to use the proposed system on larger networks, changes or removal of the use of the graph database Neo4j will have to be conducted. We have also shown that the run time and the performance of the proposed system is highly dependent on which method is used for evaluating the route sets. And one would have to choose one method at the expense of the other. However, because generating bus routes is an infrequent task in most urban cities we believe a long run time is acceptable, given that the system creates a better solution than the one that already exists. 