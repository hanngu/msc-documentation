\section{Conclusion}
\label{sec:conclusion}

We have in this thesis demonstrated how swarm intelligence inspired methods can create sufficient solutions to Urban Transit Routing Problems (UTRP). This is managed by conducting a structured literature review\citep{kofod2014}, as well as designing, implementing and excessively testing the proposed system. These processes are performed in order to establish the formulated Research Questions, introduced in Section \vref{sec:goalAndResearchQuestions}. \ref{itm:RQ1} is answered in Section \vref{subsec:relatedWorkConclusion}, after a thorough analysis of the primary relevant studies. 

The conducted experiments and a discussion of the obtained results helped establish \ref{itm:RQ2} and \ref{itm:RQ3}, which both are answered below. 

%The task of solving vehicle routing problems, more precisely the urban transit routing problem (UTRP), has proven to be useful using swarm intelligence (SI) methods. The proposed Combined Swarm System is an implementation of the ant colony optimization with additional features inspired by other swarm intelligence methods. Results are compared on the basis of Mandl's benchmark problem of a Swiss bus network, which is probably the widely investigated and accepted benchmark problem in the relevant literature within UTRP. The proposed system shows promise for optimizing transit routes in such a way that the average travel time experienced by passengers is reduced. 

\subsection*{RQ 2: Is it efficient to add attributes from other swarm intelligence methods in order to improve a standard ant colony optimization implementation?}

%We have demonstrated how the implementation of a standard ant colony optimization (ACO) converge towards a local optimum when the pheromone trail laid by previous ants become excessively attractive to the following once. 
A Combined Swarm System (CSS) is designed, implemented and applied to the UTRP in order to create feasible and efficient route networks. The proposed system was compared against a standard ACO implementation to establish if the additional attributes added from bee colony optimization and particle swarm optimization were effective. The proposed ACO implementation was identical to the proposed system, but without the additional ``memory'', ``following ants'', and ``crazy ants'' attributes. This resemblance enabled a direct comparison of the two, and a viable comparison basis. The obtained results demonstrate that the proposed system on average performs better than the standard ACO implementation regarding all performance criteria.

The computational results of the proposed system were also compared with eight other methods published in the literature. CSS shows promising and competitive results, especially regarding the average travel time experienced by travelers. The results are all compared on the basis of Mandl's benchmark problem of a Swiss bus network, which is a well-known and accepted benchmark problem. 

Based on the comparison of the standard ACO implementation and CSS we cannot, unambiguously, conclude that the additional attributes inspired by PSO and BCO were the only reason for the improved performance. The additional ``memory'' attribute, inspired by \citet{dorigo96, sedighpour14, poorzahedy11, salehinejad10}, was also implemented in the proposed system and is partly responsible for the performance improvement. However, results from the individual tests conducted on the additional parameters inspired by PSO and BCO demonstrated further improvement. With this we can, therefore, conclude that the additional attributes inspired by other swarm intelligence methods improved the standard implementation of ACO. 

%The proposed system was tested towards a standard ACO implementation to establish whether adding attributes from other swarm intelligence methods improved its performance. To be determined whether the additional attributes improved the ACO algorithm, the basis of the proposed system use an ACO. By having an otherwise identical implementation allowed direct comparison of the two. 

%A following feature, inspired by how BCO operates, was added to the proposed system. This feature rewards edges in the best route sets found so far with additional pheromone. In addition, a random feature was added to enable exploration of (possible) better solutions regardless of the pheromone values on the edges. A decreasing parameter, inspired by PSO, was implemented to balance the global and local search. 



\subsection*{RQ 3: Is it possible to apply the proposed algorithm to optimize urban transit routes in large urban cities?}

We have conducted additional experiments on larger networks, more similar to real transit networks. This is because the Mandl's Network used for comparison is a relatively small network. We have also investigated how the usage of a Neo4j graph database affects our development process and the quality of the solution. 

Neo4j includes several features that are advantageous for these types of routing problems. However, the proposed solution as-is will not be possible to use for optimizing transit routes in large urban cities. This is because the number of RelationshipTypes generated by the proposed system generally will be above the limit of allowed RelationshipTypes in Neo4j. The number of RelationshipTypes generated are dependent on the swarm size, number of iterations and allowed routes. As an example, Trondheim consists of 42 routes compared to the 4-8 routes in the Mandl Network. If we were to use the current solution to optimize the route network in Trondheim, the number of RelationshipTypes created would be $8056250$ whereas the number of allowed RelationshipTypes in Neo4j is $65536$. To use the proposed system on larger networks, changes in the utilization or removal of Neo4j will have to be done. 

The run time of the proposed system is dependent on the method used for evaluation, as well as the size of the network, number of iterations and colony size. However, generating urban transit routes is not a frequent task. Changing the transit routes often, will result in unsatisfied passengers due to the frequent need to adapt to changes. Moreover, once an optimal urban transit network is created, there will be no need for frequent changes. Because of this, the runtime will not be an issue, given that the system creates a better solution than the one that already exists. 


\subsection*{Overall conclusion}
The goal of this thesis has been to develop a system in which improves urban transit networks. The motivation is that these transit routes further can increase the number of public transportation passengers. With a sufficient transit network, public transportation will be more attractive to urban travelers.

A system is developed, and the generated networks' results demonstrates good performance on a relatively small transit network.  However, because the system as-is cannot be run on larger networks, which a majority of real urban transit networks often are, there will be no possibility to conclude whether the proposed solution will increase the number of public transportation passengers. 
%The run time and performance of the proposed system are highly dependent on the method used for evaluation, and one would have to choose one method at the expense of the other. However, because generating bus routes is an infrequent task, we believe an extended run time is acceptable, given that the system creates a better solution than the one that already exists.  