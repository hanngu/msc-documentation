\section{Discussion}
\begin{itemize}
\item[Goal:]  Increase the number of public transportation passengers by making urban transit networks more efficient.
\end{itemize}
Research Question \vref{itm:1} as a whole and Research Question \vref{itm:3a} is answered in Chapter \vref{relatedWork}.

\begin{itemize}
\item[\textbf{(2) a)}] Does adding attributes from other swarm intelligence-methods improve the performance of the standard ant colony optimization algorithm?
\end{itemize}

First of all, it is important to acknowledge that ACO has many advantages in solving VRPs. Solving certain types of NP-hard problems in polynomial time, and being optimal given there is no absolutely correct solution are some of these advantages. As demonstrated in Section \vref{subsec:evaluating_PerfomanceComparison}, before any distinct pheromone trail is laid, the ant's choices are more random and thus they perform a broad search in the environment. This randomness will decrease over time as the pheromone trails become more distinct. Because pheromone evaporate over time, shorter paths will be favored over longer paths simply because shortest paths takes shorter time. The next ants follow the trails with the most pheromone, giving these shorter paths even more pheromone. This solution is good when the problem to be solved is to find the shortest possible route set. However, in the UTRP problem, there are many factors for determining a good route set. As mentioned, a good route set is the one with the best results concerning the performance criteria. The evaluation of the route set as a whole is done after each iteration, and this evaluation will determine if the route set produced actually is a good route set. Because ACO doesn't reward good route sets as a whole, there will be no possibility to ``inform'' the ants if they are walking towards a local optimum. To overcome these weaknesses of ACO and to boost the algorithms performance, attributes inspired by other approaches within swarm intelligence methods was added the proposed algorithm. 

In the bee colony optimization (BCO), the idea is to apply collective intelligence to the optimization process. When the artificial bee has produced a good route set it can ``recruit'' other nest-mates, and thus inform the others that a good route set is found. This process inspired to add the ``following'' feature to the proposed method. After the route sets are evaluated, an amount of the best ants with the best route sets is selected to be followed in the next iteration. The same amount of ants will follow the same routes as the one they are following, and thus create the exact same route set. This will give the edges in the best route sets more pheromone, and thus distinct edges with much pheromone (because they are walked by many ants, given the short path) from edges that are better concerning the performance criteria. The proposed algorithm's performance was improved with this additional attribute. However, the algorithm performed best with a relatively small amount of followers. Rewarding edges in a large amount best route sets will result in over appreciating too many edges, which again will unable to distinct edges in good route sets from edges in the best route sets. Thats why rewarding the edges in the best route sets with extra pheromone improved the performance further.  %By making ants follow new best routes after each iteration will also give a variation of the best produced route sets, and thus avoid over rewarding edges in a local optima. 

% bra setning : and it can converge towards a (less optimal) solution. 

%With this additional attribute, information of the best known solution is still not known to the ants. Best route sets frequently produced and thus given additional pheromone may still not be the best known solution. In the particle swarm optimization (PSO) each particle's movement is influenced by its local best known position, and is guided toward the best known positions in the search-space, which are updated as better positions are found by other particles. An additional attribute, the notion of the global best solution so far is therefore added to the proposed method. This gives edges walked in the best known solution a higher probability of being selected. \emph{\color{blue} And thus. TODO, vente på resultater.}

Like ACO, the particles in PSO also tend to explore more in the the early iterations of the algorithm, and becoming more organized and coordinated at the late iterations. In PSO, this is due to a parameter called  Inertia Weight. This parameter decreases after each iteration, preventing the particles from drastically changing directions. However, the best known solution the particles are drawn against may, similar to ACO and BCO, be a local optimum. An attribute called ``crazy ants'' is added to the proposed method. This additional feature improved the performance of the proposed algorithm. An amount of the colony is given the possibility to explore edges completely random. Making decisions based on random choices will not be influenced by neither good solutions nor the best known solution (which again may be a possible local optima). The attribute enables some of the ants to explore undetected better solutions when pheromone trails becomes too great. Inertia Weight inspired by PSO was added to balance the local and global search. The parameter denotes the amount of crazy ants in the beginning of each run, and the amount of ``crazy ants'' decrease in line with the parameter. This manage the algorithm, as PSO, to act more random at the early iterations and becoming more organized in the later iterations. Decreasing the inertia weight in PSO may suffer from low global search ability at the end of the run, and thus the possibly of getting stuck at a local optimum. However, in the proposed method the crazy ants are not searching towards the best known solution, and will thus prevent the same disadvantage.

As mentioned in Section \vref{subsec:evaluating_PerfomanceComparison}, we can not unambiguously conclude that the additional attributes inspired by BCO and / or PSO was the only reason the performance of ACO improved. Giving the ants memory, inspired by \citet{dorigo96}, has previously proven to be effective concerning the performance of ACO. 

\begin{itemize}
\item[\textbf{(2) b)}] How does the proposed method perform compared to methods published in literature?
%\item[\textbf{(2) b?)}] Does the proposed method demonstrate good performance?
\end{itemize}

As stated in Section \vref{subsec:evaluating_PerfomanceComparison}, the proposed algorithm produce the best average travel time compared to all route sets published in the literature. This also concerns all route set design sizes. Concerning the unsatisfied passengers criteria, there are no unsatisfied passengers in the best produced route set, which is equal to all approaches published in the literature. For the rest of the performance criteria concerning the number of direct travelers, one transfers, and two transfers, the algorithm performs below average compared to the other approaches. In the proposed algorithm, a user-defined parameter is sat to favor a small travel time over a large number of direct transfers, if the ratio between these parameters is small. Again, it is worth mentioning that a direct route is still an important factor when selecting the best route set, which reflects that the number of direct travelers in the best produced route set is relatively high.

As mentioned in the motivation, citizens often prefer private transportation because of the decreased travel time when no detours is needed. Then again, another important issue concerning passenger satisfiability, is a passengers which will not have to change vehicles during a trip. Whether a passenger would travel travel direct with a bigger travel time versus transferring and thus decrease the travel time, is a matter of preferences. As you can see in all the approaches published in the literature including the proposed algorithm, you will have to choose one at the expense of the other. 
%Obviously, a direct route is important when...  

One can argue back and forth on the importance of each criteria. We believe that in the modern urban city, a minimum travel time is an important factor, because people should have the possibility to select a small travel time if is emergent. The algorithm is also designed to favor direct routes. This gives the possibility to select a a direct route if it is desired. And reflecting the average travel time of the best route set produced by the proposed algorithm, the direct travels are not very long. %jeg er litt usikker på om jeg er på villspor her
Then again, this can be discussed back and forth infinity, making it up to the individual passenger what is most convenient for them. 


