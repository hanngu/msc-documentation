\section{Discussion}
Research Question \label{itm:1} has a whole is answered in Chapter \label{relatedWork}.

\textbf{Goal:}
\begin{itemize}
\item  Increase the number of public transportation passengers by making urban transit networks more efficient.
\end{itemize}

\textbf{Research Questions:}
\begin{enumerate}[label=\textbf{\arabic*})]
\item[\textbf{2)}]
    \begin{enumerate}

    \item[(a)]  \textbf{Does adding attributes from other swarm intelligence-methods improve the performance of the standard ant colony optimization algorithm?}

    The experiments conducted in this thesis indicates that adding attributes from respectively bee colony optimization and particle swarm optimization improves the performance of the standard ant colony optimization (ACO) algorithm. We included experiments were the parameters linked to the added attributes of both BCO and PSO were removed. These results, shows that the performance of SSO improved. TODO: ``crazy ant'' ($CA$) inspired by PSO and the ``ants to be followed'' ($AF$) inspired by BCO, compared to the results were these were sat to 0. 

    However, the additional attributes from PSO and BCO was not the main reason the proposed algorithm performed better. A memory feature was added, which is feature not linked to neither BCO, PSO or any other optimization method from swarm intelligence. We have observed that by adding this feature the ants were more capable of creating route sets that together corresponded to a connected graph and by not able passengers to travel between every two nodes in the network using a given route set. The reader recall from Section \vref{sec:algoRemoval} that if the route set generated by a given ant does not fulfill this constraint, the route set will neither be evaluated. Because removal of the memory feature decreases the number of ants to be evaluated, so does the probability of finding the very best route set. This leads to us not being able to unambiguously conclude that the features added from BCO and/or PSO improved the standard ACO algorithm.

    * ACO has previously shown that it manage to find good solutions - [ref]. It manage to find good solutions fast, by following pheromone trails. The disadvantage is that pheromone trails laid can be a local optima.
    
    * As seen in the ACO vs SSO performance comparison one can see that SSO performs overall better than a plain ACO implementation. But as mentioned in the evaluation. ACO does not possess the memory feature. Which enables the ants to remember which node it has visited within the same route set. This feature makes it possible for the ants to create more route sets that are connected, this is important, because a passenger should be able to travel from every node to every other node withing the route network. This results in a lot of ACOs route sets will be discarded, and therefore not taken into evaluation. This means ACO will have less good route sets to be evaluated, and therefore increases the chance of finding the optimal route set. 

    * However. As we observed in the parameter settings experiments. When adding the additional parameters from other swarm inspired methods. These both increased the performance of the algorithm. 

    * The feature added from PSO - where the particles explore more in the early iterations and become more organized in the later iterations. The PSO algorithms have shown to find good solutions[ref], because they manage to get out of a local optima. This feature was added to the proposed. And we observed that adding this additional feature to ACO managed to increase the performance of our algorithm. This feature enables the algorithm to explore new (possible better) routes, regardless of the pheromone value laid on the edges.
    %global best solution of all iterations

    * In BCO - communicate and share knowledge. The once who have found good routes, share this knowledge with other bees, and make more bees will follow the same path. BCO has proven too find good solutions[ref], and this reflects the implementation of this feature in the proposed algorithm. Wee see that rewarding the very best routes with a high amount of pheromone is beneficial for the performance of the algorithm. 

    \item[(b)]  \textbf{How does the proposed method perform compared to methods published in literature?}

    * As stated in Evaluation, the proposed algorithm produce the best average travel time, compared to all the published literature this algorithm is compared to. It performs best concerning the average travel time, regardless of the route set size, on the Mandl network. 
    * The rest of the performance criteria, concerning the number of transfers - the algorithm performs just below average compared to the other approaches. 

    * Again, is it worth mentioning that a direct route is still an important factor when selecting the best route set. But, this approach sat to favor a small average travel time.

    * This is because:

    * Whether a passenger would travel a little longer and travel direct, versus changing a route once and decrease the travel time is a matter of preferences. And as one can see, you have to choose one at the expense of the other. But, as mentioned in the motivation, citizens often prefer private transportation because of the decreased travel time. 

    \end{enumerate}
\end{enumerate}