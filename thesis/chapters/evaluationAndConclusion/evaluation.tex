\section{Evaluation}

When evaluating your results, avoid drawing grand conclusions, beyond that which your results can in fact support. Further, although you may have designed your experiments to answer certain questions, the results may raise other questions in the eyes of the reader. It is important that you study the graphs/tables to look for unusual features/entries and discuss these as well as discussing the main findings in the results. 

\subsection{Parameter Settings}
Remember:
\begin{itemize}
\item $p_v$, $p_b$: giving these values a higher number, results in awarding the local best ants edges by giving them more pheromone, which we observed gave worse results, \emph{\color{red}because get stuck at local optima? blabla}
\item with 100\% crazy ants, the ants did not manage to find a solution \emph{\color{red}because...}. 
\item 0\% crazy ants gave worse results than both 5\% and 10\% CA, where we can conclude that some crazy ants is better than none. 
\item Our start values - 10\% on $p_e$, BR, and CA, was actually the best values. We could have set these to bad values in the beginning, giving increasing results concerning $TOTFIT$ and $ATT$, but \emph{\color{red} we set the start values to be what we though was best, because???}
\item ant colony size pm testing will be optimized for mandl network only.
\end{itemize}

\subsection{Neo4j}

Whether Neo4J is suited. Advantages / disadvantages of using a graph database in our implementation (research question 3a)