\section{Evaluation}

When evaluating your results, avoid drawing grand conclusions, beyond that which your results can in fact support. Further, although you may have designed your experiments to answer certain questions, the results may raise other questions in the eyes of the reader. It is important that you study the graphs/tables to look for unusual features/entries and discuss these as well as discussing the main findings in the results. 

\subsection{Evaluating the Research - COHEN}




%\begin{itemize}
%\item[1] Is the task significant? Why?
%\item[(a)] If the problem has been previously defined, how is your implementation an improvement?
%\end{itemize}
\citep{cohen88}

\subsection{Parameter Settings}

Metaheuristics, like ACO, requires a good initial parameter setting to solve concrete problems optimally. As mentioned in the literature review in Section \vref{}

\emph{\color{blue} TODO:}
Punkter vi må huske på:
\begin{itemize}
\item The most optimal would have been to run the algorithm 100 times. It takes approximately 20 minutes to run \textit{one} test case 10 times on the machine we have been assigned. %Because of a limited amount of time on our thesis, we do not have the available time to test the each one of the instances 100 times.
Some of the results can have been to some extent, random, because the tests only ran 10 times, and not 100, which would have been the optimal amount. But because of a limited amount of time on this thesis...., and because the parameter setting experiments not are the main experiment for our research, \emph{\color{blue} lalalala}. 
\item The parameter settings will be optimized for the Mandl network with 4 routes
\item $p_v$, $p_b$: giving these values a higher number, results in awarding the local best ants edges by giving them more pheromone, which we observed gave worse results, \emph{\color{blue}because get stuck at local optima? blabla}
\item with 100\% crazy ants, the ants did not manage to find a solution 
\item 0\% crazy ants gave worse results than both 5\% and 10\% CA: some crazy ants is better than none. But, as we see in the results, more than 10\% CA gives worse results than 0\% CA, \emph{\color{blue} it does not mean the more crazy ants, the better results regarding the final results. But some CA is good for the colony..}
\item Our start values - 10\% on $p_e$, $BR$, $FA$, and $CA$, was actually the best values all along. We could have set these to have worse values in the beginning of the test experiment, and receiving increasing results concerning $TOTFIT$ and $ATT$, but \emph{\color{blue} we set the start values to be what we though was best, because???, and with this conclude that our instinct regarding these values was the best. (Vi har ikke kjørt testene på nytt med se samme verdiene, bare hentet verdien fra den første som gav så bra resultater.. Selvom det kanskje var litt tilfeldig at den gav så bra resultater, viser det jo bare at med de verdiene kan den gi såpass bra resultater, det er derfor vi ikke har testet de på nytt..) - Og her må vi begrunne HVORFOR vi mener 10\% er de beste resultene, så det er også grunnen til at vi ikke endret de, og gikk for de verdiene.}
\item 100\% and 50\% FA was better than having 0\% following ants; \emph{\color{blue}better to have following ants, than not have following ants ?}
\item Ant colony size ,$s$, parameter testing will be optimized for Mandl network only. As you can see in Table C.3, the values concerning the colony size, the higher the number it did not produce better results regarding the travel time, and we therefore chose to select the size of 50 ants, when this gave the same results regarding ATT. The difference in the TOTFIT results was minimal.
\end{itemize}

\subsection{Neo4j}

\emph{\color{blue} TODO:}

Whether Neo4J is suited. Advantages / disadvantages of using a graph database in our implementation (research question 3a)

Remember:
\begin{itemize}
\item Advantages: built-in-method for finding shortest path.
\item We used A* first, which we also tested and showed that it was a lot more faster than Dijkstra's. But the implementation of A* could not find all paths, only one, so we had to use Dijkstra's for the evaluation.. Disadvantage of using the Neo4j package? Maybe easier to have made our own implementation?
\end{itemize}

