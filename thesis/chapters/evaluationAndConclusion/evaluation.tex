\section{Evaluation}

%When evaluating your results, avoid drawing grand conclusions, beyond that which your results can in fact support. Further, although you may have designed your experiments to answer certain questions, the results may raise other questions in the eyes of the reader. It is important that you study the graphs/tables to look for unusual features/entries and discuss these as well as discussing the main findings in the results. 

\emph{\color{blue} TODO: omformuleres}
To evaluate the proposed method and this project as a whole, some evaluation criteria must be defined. Why we are doing our research, why our views and methods are a step forward, how completely they are implemented by our programs, how these programs work, whether their performance is likely to increase or has reached a limit (and why), and what problems we encounter ate each stage of our research. Evaluation should be a mechanism of progress both within and across AI research. Evaluation provides a driving force to the research cycle. Evaluation provides a basis for congestions of knowledge. Without evaluation: cannot replicate results. Evaluation: convince research community that ideas are worthwhile, that they work and how. 

Evaluation is an issue discussed a lot in the field of artificial intelligence. Where other sciences has specific aspects of research to be presented, empirical AI has not comparable standards, in this effort has \citet{cohen88}, as initiated in Section\vref{sec:structuredLiteratureReview}, introduced a five-stage mode to use for evaluating AI systems: 
\begin{enumerate}
\item Refine the research topic to a task and a define a view of how to accomplish the task
\item Refined the view to a specific method
\item Develop a program to implement the method
\item Design experiments to test the program
\item Run the experiments
\end{enumerate} 
 

The task is UTRP using SI. 
And the goal is to produce better results than other research, regarding the performance criteria.  
The hypothesis is that, implementation from PSO: CA: act more random in beginning - compensate for the original ACO algorithm's weakness of frequently getting stuck at a local optima. Implemented the program to test the method, and run experiments to evaluate against performance criteria. One attempt to combine different methods from swarm intelligence - include features from other swarm inspired methods. Neo4j.



