\section{Future Work}

Consider where you would like to extend this work. These extensions might either be continuing the ongoing direction or taking a side direction that became obvious during the work. Further, possible solutions to limitations in the work conducted, highlighted in Discussion may be presented.

Future work?: Travel demand can be estimated by examining ticket sales, carrying out a survey, or undertaking analysis. This is difficult in practice, because demand is dynamical and highly sensitive to factors such as pricing and quality of service. In addition to satisfying customer demand, design guidelines are determined by many additional factors, including the street environment and management policies by the local government\citep{fan09}.

One of the main handicaps of urban transportation systems is estimating is rather expensive and therefore not not frequently.In reality the demand is different each time of day and to deal with this problem average demand has to be used or the max demand. It is no use finding optimal sets of lines for different hours of the day, because the organization problems would become enormous and also no passenger would be interested in having different lines at different times. In addition to varying travel and waiting times, also change during the day. The transportation times should then include the average transportation times. \citep{mandl79}

Traffic assignment includes finding the number of people traveling along a specific edge. 
To know how many will use a certain edge. Total flow between each pair of nodes is called the traffic assignment problem.
%and supplies the planner of a road network with important information about the flow density. Kleins alg for minimum cost flow)
%It is vital to minimizing changes and some behavior according to a weighted sum of transportation and waiting time. The flow of each edge it computes as the sum of all people using the edge by one of the two paths. The flow on each arc of each line is not completely defined, because if more than one line proceeds along the same arcs, people might use both of them along these arcs, thus the flow assignment to lines is not unique. 

Travel demand is a key variable for the UTRP. Detailed investigations into measuring and predicting travel demand is an complex research problem, and beyond the scope of this thesis. Another practical issue: geographical regions in which people live, work and shop. In some areas city planners stipulate that a bus stop must be positioned where local people reach it within 10 minutes by walking(AtB?). 


