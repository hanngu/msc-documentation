\section{Future work}
\label{sec:futureWork}

\subsection*{Use Neo4j differently or remove from implementation}
The proposed system does not perform sufficiently when the network size is bigger than the Mandl Network. Supporting larger networks is vital, because most real world transit networks are significantly larger than the Mandl Network. A reason for the insufficient performance is that the current solution with Neo4j does not support the excessive amount of RelationshipTypes required when the network size increases. An implementation where the generated amount of RelationshipTypes were significantly smaller would have been an interesting approach. Neo4j's built-in method is used for the traversal of graphs and is dependent on RelationshipTypes to distinguish the generated routes. Adding the ``properties'' function to edges instead of RelationshipTypes would be one approach to change the usage of Neo4j. However, an implementation of Dijkstra's algorithm from scratch will then be required.  


The run time of the proposed system is relatively large. The run time increases fast if either the size of the network, number of iterations, the size of the colony or number of routes in a route set increases. We believe the use of Neo4j is at least partially responsible for this increase in run time. An implementation of the current system without Neo4j would, therefore, be interesting to investigate an eventual change in run time.  
%An implementation without Neo4j would also help to further address the potential advantages/disadvantages of Neo4j because one would have a basis for comparison. \emph{\color{blue} Vi burde si noe om minnebruken i discussion. Jeg tror at løsningen i dag er veldig minneeffektiv.}

\subsection*{Adjust the Total Fitness Function}
The Total Fitness Function used for evaluation in the proposed system is sat to favor a low traveling time over the importance of number of transfers. This resulted in the lowest average travel time of all the compared researches. However, this favoring seems to increase as the network size increases. The influence of number of direct transfers borders thus against zero, meaning the number of direct transfers is not taken into consideration in the evaluation phase. A satisfied passenger emphasizes both a minimum travel time \textit{and} the a minimum number of transfers. However, one would often have to choose one at the expense of the other. An interesting approach would be to create a Total Fitness Function that easily allows a service provider to determine the importance of the two.

\subsection*{Use the proposed system to solve the Urban Transit Scheduling Problem}
The Urban Transit Network Design Problem (UTNDP) consist of two stages. The first stage is to create the physical route network (UTRP), whereas the second stage involves designing the schedules of the developed network (UTSP). This thesis has focused on the UTRP, creating effective urban transit routes. However, deciding when and how frequent a route should be serviced is also a vital part of optimizing a transit network. To increase the number of public transportation passengers, the schedules for the public vehicles must also be optimized. A good starting point could be to investigate the solution proposed by \citet{nikolic14}, who solved both UTRP and UTSP using a BCO approach. Additional data must be provided to solve the UTSP efficiently. This data should include information about the demand between each bus stops different hours of the day.  

\subsection*{Use the proposed system to optimize a transit network in a large city}
The motivation for implementing the proposed system was initially to optimize the transit network in Trondheim. However, AtB (the bus service provider in Trondheim) does not possess the required data, such as the average demand values between each bus stop. An interesting approach would be to test the proposed system on a large transit network where the transit routes are manually designed, and where the service provider possesses the required data. These experiments would allow further investigation of the strength and weaknesses of the proposed system. It would also help determine how and if the system improves the current, manually designed, transit network. As stated above, the proposed system does not transfer well to large networks at this point, and changes must be done with the implementation before this is possible. 

%Consider where you would like to extend this work. These extensions might either be continuing the ongoing direction or taking a side direction that became obvious during the work. Further, possible solutions to limitations in the work conducted, highlighted in Discussion may be presented.
%\newline
%Hva burde blitt gjort annerledes - hva kan man gjøre i fremtiden for å gjøre det bedre?


%---- Neo4j -----

% --- Trondheim ---

%Travel demand:
%Travel demand can be estimated by examining ticket sales, carrying out a survey, or undertaking analysis. This is difficult in practice, because demand is dynamical and highly sensitive to factors such as pricing and quality of service. In addition to satisfying customer demand, design guidelines are determined by many additional factors, including the street environment and management policies by the local government\citep{fan09}. %One of the main handicaps of urban transportation systems is estimating is rather expensive and therefore not not frequently.In reality the demand is different each time of day and to deal with this problem average demand has to be used or the max demand. It is no use finding optimal sets of lines for different hours of the day, because the organization problems would become enormous and also no passenger would be interested in having different lines at different times. In addition to varying travel and waiting times, also change during the day. The transportation times should then include the average transportation times. \citep{mandl79} %Travel demand is a key variable for the UTRP. Detailed investigations into measuring and predicting travel demand is an complex research problem, and beyond the scope of this thesis. Another practical issue: geographical regions in which people live, work and shop. In some areas city planners stipulate that a bus stop must be positioned where local people reach it within 10 minutes by walking(AtB?). 

%--- Scalability ----