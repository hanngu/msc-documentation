\section{Future Work}

\subsection*{Use Neo4j Differently or Not At All}
The method proposed in this solution do not transfer well to larger networks. Being able to support larger networks is important because most real world transit networks are larger than the networks used for experiments in this thesis.  

A reason for the insufficient performance is that Neo4j does not support the current solution very well, due to the excessive amount of RelationshipTypes created. An implementation where the generated amount of RelationshipTypes were significantly smaller would have been an interesting approach. 

The running time of the proposed method is relatively large, and it increases fast if either the size of the network, number of iterations, size of the colony or number of routes in a route set increases. We believe the use of Neo4j is at least partially responsible for this. An implementation of the current algorithm without Neo4j would therefore be interesting in order to investigate an eventual change in run time.  
%An implementation without Neo4j would also help to further address the potential advantages/disadvantages of Neo4j because one would have a basis for comparison. \emph{\color{blue} Vi burde si noe om minnebruken i discussion. Jeg tror at løsningen i dag er veldig minneeffektiv.}

\subsection*{Change The Total Fitness Function}
In the proposed method the Total Fitness Function of a route set is sat to favor a low traveling time, compared to the number of transfers. This results in the lowest average travel time of all the compared researches. However does this favoring seem to increase as the network increases, which leads to that the influence of number of transfers borders zero. For a passenger is both the travel time \textit{and} the number of transfers important, an often a matter of taste. An interesting approach would be to find a Total Fitness Function that allow a service provider to determine the importance of the two easily, independent of the size of the network.

\subsection*{Use The Proposed Method to Solve The Urban Transit Scheduling Problem}
The Urban Transit Network Design Problem (UTNDP) is usually divided into two problems, where one is designing the actual route network and one is designing the schedules given the routes. In this thesis the physical routes are determined, but when and how frequent a route should be serviced is not. In order to achieve the goal of an increased number of public transportation passengers, must the schedules for the public vehicles also be efficient. This problem is described as the Urban Transit Scheduling Problem. A good starting point could be to investigate the solution proposed by \citet{nikolic14}, who solved this problem using a BCO approach. In order to solve this problem the most efficient, must additional data be provided. This includes data about the demand between the bus stops on different hours of the day. 

\subsection*{Use The Proposed Method to To Create A Route Network in a Large City}
A motivation for the creation of the proposed method was initially to use it on a relatively large city, such as Trondheim, Norway. However, as mentioned, does AtB (the bus service provider in Trondheim) not possess the required data, including the average demand between the bus stops. An interesting approach would be to try the proposed method on the transit network of a larger city, where the bus service provider possesses the required data. The results achieved by the proposed method compared to the current route network would be important in order to further investigate the method's strengths and weaknesses. As stated above does the proposed method not transfer well to larger networks at this point, and changes must therefore be done to the implementation before this is possible. 




%Consider where you would like to extend this work. These extensions might either be continuing the ongoing direction or taking a side direction that became obvious during the work. Further, possible solutions to limitations in the work conducted, highlighted in Discussion may be presented.
%\newline
%Hva burde blitt gjort annerledes - hva kan man gjøre i fremtiden for å gjøre det bedre?


%---- Neo4j -----

% --- Trondheim ---

%Travel demand:
%Travel demand can be estimated by examining ticket sales, carrying out a survey, or undertaking analysis. This is difficult in practice, because demand is dynamical and highly sensitive to factors such as pricing and quality of service. In addition to satisfying customer demand, design guidelines are determined by many additional factors, including the street environment and management policies by the local government\citep{fan09}. %One of the main handicaps of urban transportation systems is estimating is rather expensive and therefore not not frequently.In reality the demand is different each time of day and to deal with this problem average demand has to be used or the max demand. It is no use finding optimal sets of lines for different hours of the day, because the organization problems would become enormous and also no passenger would be interested in having different lines at different times. In addition to varying travel and waiting times, also change during the day. The transportation times should then include the average transportation times. \citep{mandl79} %Travel demand is a key variable for the UTRP. Detailed investigations into measuring and predicting travel demand is an complex research problem, and beyond the scope of this thesis. Another practical issue: geographical regions in which people live, work and shop. In some areas city planners stipulate that a bus stop must be positioned where local people reach it within 10 minutes by walking(AtB?). 

%--- Scalability ----