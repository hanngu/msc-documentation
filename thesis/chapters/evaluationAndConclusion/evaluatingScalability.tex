\subsection{Evaluating Scalability Results}
\label{subsec:evalScalability}

Observing the results found in Table \vref{table:mumford0_and1_results}, one can see that the experiments using the  instances Mumford0 and Mumford1 were ran with a value of the parameter $i$ sat to 50. This is done because with original value, neither the experiments on Mumford0 or Mumford1 were able to produce any results. The reason for this is that the number of generated RelationshipTypes becomes higher than the limit of allowed RelationshipTypes sat by Neo4j. Compared to the Mandl network have all of the Mumford-instances an increased number of routes in the route sets ($NRS$). The reader recalls from Section \vref{subsec:networkGeneration} that for each route in each ant's route set for each iteration a new RelationshipType is created. The number of RelationshipTypes ($NRT$) is therefore dependent on both the number ants, the number of iterations and the $NRS$. Neo4j has a limit of $2^{16} \approx 66 000$ RelationshipTypes \emph{\color{blue} Finn ut hvordan vi siterer dette}. As demonstrated in Table \vref{tabel:numberOfRelationshipTypes}, neither of the Mumford-instances is within this limit with the original value of $i$ and the given $NRS$. However, by reducing $i$ to 50, both Mumford0 and Mumford1 will be within the limit. The reader recalls from Section \vref{subsec:parameterSettings_results} that the higher the $i$, the better the result. Due to the reduction of $i$, we therefore believe the results of the experiments run on the Mumford0 and Mumford1 networks have suffered. To be able to run experiments on Mumford2 or Mumford3, either $s$ or $i$ had to be further reduced. Because further reduction of either of these parameters would lead to a further deterioration of the results, this was not done.

Comparing the results produced by the proposed algorithm on the Mumford0 and Mumford1 networks to the results produced on the Mandl network, found in Table \vref{table:performanceComparison_routesets}, one can see that the average number of direct travelers ($d_0$) decreases, and the number of unsatisfied passengers ($d_{unsat}$) increases. 

This could, as mentioned above, be linked to the reduction of $i$. This reduction results in that the experiments ran on the Mumford networks, which are in fact twice or more the size of the Mandl network, are explored by 3750 less ants in total. This leads to a smaller probability of finding the best route sets.

Another reason for this is, as mentioned in Section \vref{subsec:scalabilityExperiments_setup}, that Method 1 is used when deciding which route a passenger should use, where Method 2 is used for the experiments ran on the Mandl Network. In Method 1 the transfer penalties are not considered when choosing a path, which leads to possibly choosing paths with many transfers. This again leads to a higher number of unsatisfied travelers, and a lower number of direct travelers compared to using Method 2. Further it leads to a higher average travel time, due to the fact that transfer penalties is added to the total travel time afterwards. 

%Another reason is that when testing the Mumford Networks, 50 iterations is used compared to 125 when testing the Mandl Network. This results in that the Mumford networks, which are in fact twice or more the size of the Mandl network, are explored by 3750 less ants in total and thus have a smaller probability of finding the best route sets. The reason for this is that each route of each ant in a given iteration has its own relationship type, and the number of routes in a route set ($NRS$) is larger in the Mumford Networks. Due to the fact that Neo4j has a limit of $2^{16} \approx 66 000$ relationship types, as demonstrated in \vref{tabel:numberOfRelationshipTypes}, the number of relationship types, $NRT$, would exceed Neo4j's capacity. 

\begin{table}[H]
    \centering
    \hspace*{-1.0cm}
    \begin{tabular}{|c|c|}
        \hline
        \textbf{$NRS$} & \textbf{$NRT$}\\
        \hline
        4 & 25000\\
        \hline
        6 & 37500\\
        \hline
        7 & 43750\\
        \hline
        8 & 50000\\
        \hline
        12 & 75000\\
        \hline
        15 & 93750\\
        \hline
        56 & 350000\\
        \hline
        60 & 375000\\
        \hline
    \end{tabular}
    \caption{Number of Relationship Types that will be Generated using Different Route Set Sizes, with a swarm size of 50 and 125 iterations.}
    \begin{itemize}[noitemsep]
    \item[$NRT$:] 50 ($s$) * 125 ($i$) * $NRS$
    \end{itemize} 
    \label{tabel:numberOfRelationshipTypes}
\end{table} 

A third reason, and maybe the most important, is that when calculating the $TOTFIT$ as described in Section \vref{sec:algoEvaluation}, the $F_1(r)$ are emphasized more than both $F_2(r)$ and $F_3(r)$. As described in Section \vref{subsec:evaluating_PerfomanceComparison}, $F_1(r)$ is already favored when running the tests on the Mandl Network, and this results in the best average travel time over all the published methods. However, when the networks becomes significantly larger, such as the ones provided by \citet{mumford13}, the value of $F_1(r)$ becomes much larger than the values of both $F_2(r)$ and $F_3(r)$. This results in that the effect of $F_2(r)$ and $F_3(r)$ boarders against zero. In Table \ref{tabel:averageTotfitAllTestCases} the results of the average $TOTFIT$ is shown for each of the test cases ran. As one can observe the average $TOTFIT$ is much larger for the Mumford0 and Mumford1 cases compared to the Mandl-cases. The reader recalls from Section \vref{sec:totfit}, that the value of $F_2(r)$ is between $-300$ and $0$, and that the value of $F_3(r)$ is between $0$ and $1000$. The value of $F_1(r)$ is dependent on network size and the total demand, and as Table \vref{tabel:averageTotfitAllTestCases} shows dividing $F_1(r)$ on the $nodeSize^2$ is not sufficient enough because $F_1(r)$ becomes too great. 

\begin{table}[H]
    \centering
    \hspace*{-1.0cm}
    \begin{tabular}{|c|c|c|}
        \hline
        \textbf{Number} & \textbf{Instance} & \textbf{Average $TOTFIT$}\\
        \hline
        1 & Mandl (4 routes) & 98.0\\
        \hline
        2 & Mandl (6 routes) & 89.5\\
        \hline
        3 & Mandl (7 routes) & 87.1\\
        \hline
        4 & Mandl (8 routes) & 83.4\\
        \hline
        5 & Mumford0 & 3886.4\\
        \hline
        6 & Mumford1 & 14459.8\\
        \hline
    \end{tabular}
    \caption{Average Total Fitness for All Tests Cases. 1-4 are the results of 50 runs, 5-6 are the results of 10 runs}
    \label{tabel:averageTotfitAllTestCases}
\end{table}
 
The run time for the proposed method is dependent on $NRS$, the number of ants ($s$), the number of iterations ($i$), the size of the network, and whether Method 1 or Method 2 is used. As one can observe from Table \vref{tabel:runTimeMandl} the average run time increases drastically when the number of routes in the Mandl network increases. This may be because $NRT$ in the Neo4j database becomes significantly larger when increasing the number of routes. Which again increase the number of read and writes to the Neo4j database. The run time is also more than 10 times greater using using Method 2 compared to Method 1 as seen in \vref{table:results_mumford0RunTime}. When using Method 1 it is sufficient to use the built in Dijkstra algorithm in Neo4j to only find the shortest path between two nodes given a route set. When Method 2 is used, the proposed method finds all possible routes between two nodes given a route set, and further evaluates which route to choose based on the total travel time (including transfer penalties). The difference in the run time, using Method 1 compared to Method 2, also increases as the network becomes bigger. This is because the algorithm finds a path between every two nodes in the network, and when the network increases, so does (usually) the number of nodes and possible paths as well. Table \vref{tabel:runTimeMumford} shows that the average run times of Mumford0 and Mumford1. It is worth mentioning, that the run times are approximately the same as the ones for the experiments who uses the Mandl Network, even though these experiments have a smaller network and fewer $NRS$, due to the decreased number of $i$ and the fact that Method 1 is used instead of Method 2. 

%Independent of whether it is the route set size, network size, the number of iterations or the number of ants in the colony that becomes bigger, the number of reads and writes to the Neo4j database increases, which again increases the run time. %\emph{\color{blue} Kanskje vi må si noe mer her, finne noe data på hvorfor dette øker run timen eller noe. Kanskje vi må si noe om at vi har observert at minnebruken aldri går over 30\%?}

\begin{table}[H]
    \centering
    \hspace*{-1.0cm}
    \begin{tabular}{|c|c|}
        \hline
        \textbf{Instance} & \textbf{Run Time$^1$ (secs)} \\
        \hline
        Mumford0 & 2368.0\\
        \hline
        Mumford1 & 5862.4\\
        \hline
    \end{tabular}
    \caption{Average Runtime in seconds of 10 Runs Using the Mumford Network}
    \label{tabel:runTimeMumford}
    \begin{itemize}[noitemsep]
    \item[$^1$:] $s$ = 50, $i$ = 50
    \end{itemize} 
\end{table}

%\textbf{Did the program demonstrate good performance?}

%\textbf{Is the programs performance different from predictions?}

%\textbf{How efficient is the program?}

%\textbf{Can you define the programs limitations?}
%When does it break, and what components contributes to its successful operations?

%\textbf{Do you understand why it works / doesn't work?}
%What is the impact of changing the program? How does the program respond to the new input? 

%\textbf{Did you learn what you wanted from the programs experiments?}

