\subsection{Evaluating Scalability Results}

Comparing the results from the Mumford networks (Table \vref{table:results_mumford}) to the results from the Mandl network (Tables \vref{table:performanceComparison_4}, \vref{table:performanceComparison_6}, \vref{table:performanceComparison_7}, and \vref{table:performanceComparison_bestRouteSet8}) one can see that the average number of direct travelers ($d_0$) decreases, while the number of unsatisfied passengers ($d_unsat$) increases in the networks \textit{mumford0} and \textit{mumford1}. Given the maximal number of routes in a route set, the maximal number of nodes in a route and the edge and node size, the Mumford networks should have a better probability of covering all edges compared to the Mandl network with four routes, but based on the results it seems like it does not. 

One reason for this is that, as mentioned in Section \vref{subsec:scalabilityExperiments_setup}, Method 1 is used when deciding which route a passenger should use, and not Method 2 as used when testing the Mandl Network. In Method 1 the transfer penalties are not considered when choosing a path, which leads to possibly choosing paths with many transfers. This again leads to a higher number of unsatisfied travelers, and a lower number of direct travelers compared to using Method 2. Further it leads to that the average travel time becomes higher, because the transfer transfer penalties, which is sat to 5 minutes per transfer, afterwards is added to the travel time. 

Another reason is that when testing the Mumford Networks 50 iterations is used compared to 125 when testing the Mandl Network. This results in that the Mumford networks, which are in fact twice or more the size of the Mandl network, are explored by 3750 less ants in total and thus have a smaller probability of finding the best route sets.  

A third reason, and maybe most important reason, may be that when calculating the $TOTFIT$ as described in Section \vref{sec:algoEvaluation}, the $F_1(r)$ are emphasized to much compared both $F_2(r)$ and $F_3(r)$. As described in \emph{\color{blue} Section XX}, $F_1(r)$ is already emphasized when running the tests on the Mandl Network, and this results in the best average travel time of all the published methods compared against. However, when the networks becomes significantly larger, such as the ones provided by \citet{mumford13}, the value of $F_1(r)$ becomes much larger than the values of both $F_2(r)$ and $F_3(r)$. This results in that the effect of $F_2(r)$ and $F_3(r)$ boarders against zero. In Table \vref{tabel:averageTotfitAllTestCases} the results of the average $TOTFIT$ is shown for each of the test cases ran. As one can observe the average $TOTFIT$ is much larger for the Mumford0 and Mumford1 cases compared to the Mandl-cases. The reader recalls from Section \vref{sec:totfit}, that the value of $F_2(r)$ is between $-300$ and $0$, and that the value of $F_3(r)$ is between $0$ and $1000$. The value of $F_1(r)$ is dependent on network size and the total demand, and as Table \vref{tabel:averageTotfitAllTestCases} shows dividing $F_1(r)$ on the $nodeSize^2$ is not sufficient enough because $F_1(r)$ becomes to big. 

\begin{table}[H]
    \centering
    \hspace*{-1.0cm}
    \begin{tabular}{|c|c|c|}
        \hline
        \textbf{Number} & \textbf{Instance} & \textbf{Average $TOTFIT$}\\
        \hline
        1 & Mandl (4 routes) & 98.0\\
        \hline
        2 & Mandl (6 routes) & 89.5\\
        \hline
        3 & Mandl (7 routes) & 87.1\\
        \hline
        4 & Mandl (8 routes) & 82.6\\
        \hline
        5 & Mumford0 & 3886.4\\
        \hline
        6 & Mumford1 & 14459.8\\
        \hline
    \end{tabular}
    \caption{Average Total Fitness for All Tests Cases. 1-4 are the results of 50 runs, 5-6 are the results of 10 runs. \emph{\color{blue} TODO: Sett inn nytt tall for 8 routes}}
    \label{tabel:averageTotfitAllTestCases}
\end{table}

The run time for the proposed method is dependent on the number of routes in a route set ($NRS$), the number of ants ($s$), the number of iterations ($i$), the size of the network and whether Method 1 or Method 2 is used. As one can observe from Table \vref{tabel:runTimeMandl} the average run time increases drastically when the number of routes increases. This may be because the number of relationship types, $NRT$, in the Neo4j database becomes significantly larger when increasing the number of routes. Table \vref{tabel:numberOfRelationshipTypes} shows the $NRT$ for each test case using the Mandl Network.  

\begin{table}[H]
    \centering
    \hspace*{-1.0cm}
    \begin{tabular}{|c|c|}
        \hline
        \textbf{Instance} & \textbf{Average Run Time (seconds)$^1$}\\
        \hline
        Mandl (4 routes) & 673.0\\
        \hline
        Mandl (6 routes) & 2442.1\\
        \hline
        Mandl (7 routes) & 3891.1\\
        \hline
        Mandl (8 routes) & 8830.2\\
        \hline
    \end{tabular}
    \caption{Average Runtime of 50 Runs for Experiments Using the Mandl Network \emph{\color{blue} TODO: Sett inn nytt tall for 8 routes}}
    \begin{itemize}[noitemsep]
    \item[$^1$:] $s$ = 50, $i$ = 125)
    \end{itemize} 
    \label{tabel:runTimeMandl}
\end{table}

Independent of whether it is the route set size, network size, the number of iterations or the number of ants in the colony that becomes bigger, the number of reads and writes to the Neo4j database increases, which again increases the run time. \emph{\color{blue} Kanskje vi må si noe mer her, finne noe data på hvorfor dette øker run timen eller noe. Kanskje vi må si noe om at vi har observert at minnebruken aldri går over 30\%?}

\begin{table}[H]
    \centering
    \hspace*{-1.0cm}
    \begin{tabular}{|c|c|}
        \hline
        \textbf{$NRS$} & \textbf{$NRT^1$}\\
        \hline
        4 & 25000\\
        \hline
        6 & 37500\\
        \hline
        7 & 43750\\
        \hline
        8 & 50000\\
        \hline
    \end{tabular}
    \caption{Number of Relationship Types Generated using the Mandl Network with Different Route Set Sizes}
    \begin{itemize}[noitemsep]
    \item[$^1$:] 50 ($s$) * 125 ($i$) * $NRS$
    \end{itemize} 
    \label{tabel:numberOfRelationshipTypes}
\end{table}

As stated above may the results suffer from using Method 1 compared to Method 2, but as shown in Table \vref{table:results_mumford} the run time is more than 10 times greater using using Method 2 compared to Method 1. When using Method 1 it is sufficient to only use the built in Dijkstra algorithm in Neo4j to find the shortest path between two nodes given a route set. When Method 2 is used, the proposed method finds all possible routes between two nodes given a route set, and further evaluates which route to choose based on the total travel time (including transfer penalties). The run time difference of using Method 1 compared to Method 2 increases as the network becomes bigger. This is because the algorithm finds a path between every two nodes in the network, and when the network increases, so does (usually) the number of nodes and possible paths as well. Table \vref{tabel:runTimeMumford} shows that the average run times of Mumford0 and Mumford1. As one can observer the run times are approximately the same as the ones for the experiments who uses the Mandl Network (Table \vref{tabel:runTimeMandl}), even though these experiments have a smaller network and fewer $NRS$. This is partly because of the decreased number of $i$ and partly because Method 1 is used instead of Method 2.  

\begin{table}[H]
    \centering
    \hspace*{-1.0cm}
    \begin{tabular}{|c|c|}
        \hline
        \textbf{Instance} & \textbf{Average Run Time (seconds)$^1$} \\
        \hline
        Mumford0 & 2368.0\\
        \hline
        Mumford1 & 5862.4\\
        \hline
    \end{tabular}
    \caption{Average Runtime of 10 Runs for Tests Using the Mumford Network}
    \label{tabel:runTimeMumford}
    \begin{itemize}[noitemsep]
    \item[$^1$:] $s$ = 50, $i$ = 50)
    \end{itemize} 
\end{table}

%\textbf{Did the program demonstrate good performance?}

%\textbf{Is the programs performance different from predictions?}

%\textbf{How efficient is the program?}

%\textbf{Can you define the programs limitations?}
%When does it break, and what components contributes to its successful operations?

%\textbf{Do you understand why it works / doesn't work?}
%What is the impact of changing the program? How does the program respond to the new input? 

%\textbf{Did you learn what you wanted from the programs experiments?}

