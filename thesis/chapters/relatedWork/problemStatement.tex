\section{Problem Statement}

Ant Colony Optimization has several advantages, such as natural parallelism and continuous positive feedback, which allows good solutions to be identified fast. However, ACO also has a few draw backs including getting stuck at local optima. We hope to overcome this by declaring some of the ants as ``crazy''. A crazy ant acts randomly and do not consider the pheromone level when creating routes. The probability that an ant is declared crazy decreases during run of the algorithm according to an implemented inertia weight. The inertia weight is an adapted feature from PSO that favors more exploring in the beginning of the algorithm, and more exploiting in the end. Further we hope to boost our results by declaring some ants as ``followers''. The followers unconditionally follows one of the best ants from the previous iteration, and by doing so boosts that ant's choice. This is highly inspired by how some bees in BCO chooses to follow another bee that has found a good ``food source''. As stated in Section \vref{sec:relatedWork}, the only attempt to combine different methods from swarm intelligence is to add a notion of the global best solution found to ACO and BCO. For this reason, we will in this thesis investigate the potential advantages and possible disadvantages of solving the UTRP with an ACO algorithm inspired by both BCO and PSO. This will help us establish Research Question \vref{itm:2a}, which is concerned about whether or not it is efficient to combine attributes from different swarm intelligence methods in order to improve the computational results.

We are highly motivated in the development of an ant colony inspired optimization algorithm for the urban transit network design problem (UTNDP). For this problem is travel demand a key variable. Accurate estimates of travel demand is an important factor for the algorithm, because a good route network will ensure that routes having the most traveling demands are satisfied with short paths and few vehicle transfers. AtB does not possess accurate data about the travel demand, and detailed investigations into measuring and predicting travel demand is an complex research problem, and beyond the scope of this thesis. 

Demand values are all provided for Mandl's benchmark problem. For the urban transit network design problem (UTNDP), Mandl's network seems to be the only benchmark instance used and acknowledged by researchers, including \citep{fan09}, \citep{kechagiopoulos14}, \citep{nikolic14}. Christoph Mandl \citep{mandl79} developed a heuristic algorithm for the urban transit network problem, and the method was applied and based on a real network in Switzerland, the Swiss transit network\citep{mandl80}. The data includes a small and dense network of 15 nodes and 21 edges, in addition to the travel times and travel demand for each edge. The total demand is 15570 trips per day, which is a relatively high demand for a small network. The travel time between the to farthest nodes in the network is 22 minutes along the shortest path. Mandl developed a solution in two phases, where a feasible set of routes were created in the first phase and in the second phase he tried to minimize the the total travel time, including in-vehicle time and waiting time, by reducing the number of transfers. Some important measures of Mandl's solution network includes, for example, 100\% service coverage, 69.4\% of the trips involving no transfers, 29.93\% of the trips involving one transfer, and only 0.13\% of the trips needing more than one transfer. We will in this thesis use Mandl's benchmark problem \citep{mandl79} as the input data, validate the methodology developed in this study, demonstrate the efficiency and effectiveness of our algorithm, and evaluate and compare our results. 

Through our structure literature review, we did not manage to find any previous research that used graph databases in combination with the vehicle routing problem and swarm intelligence. For this reason, and because we believe the graph database Neo4j \citep{website:neo4j} has several benefits we can take advantage of in solving our problem, we will investigate the possibilities, and potential advantages and disadvantages, Neo4J will have for our algorithm and the optimization process, giving us research question \vref{itm:3a} (What are the potential advantages and disadvantages of using a graph database in our implementation? )

We will focus our research on making the routes more convenient both concerning customers and operators. The objectives we want to focus on is satisfying the customers and keeping the operator costs at minimum. 

The current solution of AtB \citep{website:atb} consist of an experience based route network, meaning the routes has not been properly, computationally optimized concerning the travel demand and travel time. When a route network is not properly optimized, it can lead to a large number of transfers for passengers when they are traveling from their origin to their destination, resulting in a long total travel time. A minimum number of transfers and a minimum in-vehicle time are important factors for determining the customers satisfaction, and the objective in this thesis will therefore be to minimize the total travel time in the public transportation system, by optimizing the routes concerning the travel time and minimizing number of transfers. %TODO: Denne må omformuleres

The algorithm will be evaluated by the following performance criteria, inspired and used by including \citep{kechagiopoulos14}, \citep{mandl80}, \citep{nikolic14} and \citep{fan09}: 
\begin{itemize}
\item $d_0 (\%)$ - the percentage of demand satisfied without any transfers.
\item $d_1 (\%)$ - percentage of total transfer demands where the number of transfers are 1. 
\item $d_2 (\%)$ - percentage of total transfer demands where the number of transfers are 2. 
\item $d_{unsat}$ (\%) - the percentage of unsatisfied travelers. An unsatisfied traveler is described as a traveler with 3 or more transfers.
\item $ATT$  - the average travel time in minutes per transit user (mpu). The travel times incorporates a transfer penalty, which is sat to be 5 minutes per transfer for comparison reasons.
\end{itemize}
%\begin{itemize}
%\item Total travel time (should be as low as possible)
%\item Number of transfers (should be as low as possible)
%\begin{itemize}
%\item Number of direct travelers
%\item Travelers with one transfer
%\item Travelers with two transfers
%\end{itemize}
%\item Number of unsatisfied customers - more than two transfers (should be zero)
%\end{itemize}

Our goal is to develop an ant colony inspired algorithm, with additional features from the discussed SI methods, that creates a general solution with effective urban transit routes. We hope that this in the future can be used to optimize AtBs transit network in order to increase the number of public transportation passengers. Research Question \vref{itm:2c} is concerned about whether or not it is possible to apply our algorithm to optimize bus routes in urban cities. This question cannot be fully answered until it is applied to an urban city, but we will strive to create a method that is easily adaptable with the concerns of public transportation in cities in mind.
