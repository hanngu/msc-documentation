\section{Structured Literature Review}
To answer Research Question \vref{itm:1}, we conducted a Structured Literature Review based on the model suggested by \citet{kofod2014}. As stated in \citep{kofod2014} synthesizing available information before tackling an area of research can, for example, help mapping out existing solutions, avoid duplicating the effort, and highlight areas where additional research is required. This section is a summary of the conducted literature review, and the complete description can be found in Appendix \vref{appendixA} and Appendix \vref{appendixB}. 

\subsection{Problem Definition}
\emph{\color{blue} TODO: Må omformuleres}. 
As mentioned in the motivation, Section \vref{sec:motivation}, can efficient public transportation systems reduce negative effects of the private transportation networks. AtB's current solution consists of an experience based route network, and these manual attempts to provide acceptable solutions to the vehicle routing problems are not optimal.These issues formed the basis of our problem. We also believe that a graph database can have benefits we can take advantage of in solving our problem, and therefore wanted to investigate what might have been done before on this research area. The problem and the following research questions can be found in Appendix \ref{appendixA}, Section \vref{sec:researchQSLR}. 

\subsection{Process Description}
The list of sources for retrieving relevant literature to our defined research questions, were selected based on \citep[p.3]{kofod2014}, and can be found in Appendix \vref{appendixB}. 

Key search terms was decided based on our problem, formed into groups of synonyms, and assembled into a complete search term. The key search terms and the complete search term can be found in appendix \ref{appendixA}, section \vref{searchterm} and section \vref{sec:searchTerms}, respectively.

To exclude irrelevant literature, some Inclusion Criteria was decided to ensure a level of relevance. The Inclusion Criteria can be found in Appendix \ref{appendixA}, Section \vref{sec:inclusionCriteria}. After the Inclusion Criteria filtering, we had 42 sources for the related literature, including scientific papers and master theses. 

Quality Criteria was determined to ensure quality in the final papers and to filter away studies not theoretically relevant for our thesis. Each of the studies under consideration was classified according to the Quality Criteria, by scoring the papers on each Quality Criteria. Each point was given a score, where 1 point denote relevant, $\frac{1}{2}$ point denote partly relevant and 0 points denote not relevant. For retrieving the most relevant papers based on the content and not just the quality of the paper, Quality Criteria \ref{itm:qa1a} and \vref{itm:qa1b} was multiplied by 3. Quality Criteria \ref{itm:qa1a} is concerned about whether or not the research in fact is a Vehicle Routing Problem and Quality Criteria \ref{itm:qa1b} addresses whether or not swarm intelligence is the main optimization method. 

Table \vref{table:literature} shows the papers that were selected based on the Inclusion Criteria and the scores given based on Quality Criteria.

The average score of the read literature was 13.01 $\approx$ 13, and literature given a score $\geq{1.5}$ above average were selected. This resulted in 12 final researches, which can be seen in Table \vref{table:finalliterature}. These researches will form the foundation of our thesis, and Section \vref{sec:relatedWork} describes and summarizes the selected literature, referred to as Related Work. 
