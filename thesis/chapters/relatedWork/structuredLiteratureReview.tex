\section{Structured Literature Review}
\label{sec:structuredLiteratureReview}

\citet{cohen88} introduced a five-stage model for evaluating research in terms of a five-stage cycle. The development process should be run in iterative cycles, and the suggested model is used for evaluation throughout the project. The first stage in \citet{cohen88}'s model involves refining the research topic to a task, and identifying a view of how to accomplish the task. %The rest of the stages will be evaluated in Section \vref{subsec:evaluationCriteraCohen}. 
As stated in the motivation, Section \vref{sec:motivation}, congestion and environmental problems in Trondheim is increasing every year due to the population growth\citep{website:miljopakken}, and efficient public transportation systems can reduce the negative effects of these issues. Since AtB's\citep{website:atb} current solution consist of an experience based route network, may it not be the optimal solution. The task in this research is to optimize urban transit networks using swarm intelligence methods. Further, is the goal to make urban transit networks more convenient for passengers, and thus increase the number of public transportation passengers in Trondheim.% with Trondheim as a motive

The task p is tackled using swarm intelligene c to develop an algorithm s. What the existing solutions are available, state-of-the-art. what changes have been done, and has si been combined.  
To establish whether the research topic had been studied before and what the state-of-the art on these problems are, is a Structured Literature Review\citep{kofod2014} conducted. % As stated in \citep{kofod2014} synthesizing available information before tackling an area of research can, for example, help mapping out existing solutions, avoid duplicating the effort, and highlight areas where additional research is required.
Section \vref{sec:SLRprocessDescribtion} is a summary of the conducted literature review, and the complete review can be found in Appendix \vref{appendixA} and Appendix \vref{appendixB}. 
%We also believe that a graph database can have benefits we can take advantage of in solving our problem, and therefore wanted to investigate what might have been done before on this research area.

\subsection{Process Description}
\label{sec:SLRprocessDescribtion}

The list of sources for retrieving relevant literature to our defined research questions, were selected based on \citep[p.3]{kofod2014}, and can be found in Appendix \vref{appendixB}. 

Key search terms was decided based on our problem, formed into groups of synonyms, and assembled into a complete search term. The key search terms and the complete search term can be found in appendix \ref{appendixA}, section \vref{searchterm} and section \vref{sec:searchTerms}, respectively.

To exclude irrelevant literature, some Inclusion Criteria was decided to ensure a level of relevance. The Inclusion Criteria can be found in Appendix \ref{appendixA}, Section \ref{sec:inclusionCriteria}. After the Inclusion Criteria filtering, we had 42 sources for the related literature, including scientific papers and master theses. 

Quality Criteria was determined to ensure quality in the final papers and to filter away studies not theoretically relevant for our thesis. Each of the studies under consideration was classified according to the Quality Criteria, by scoring the papers on each Quality Criteria. Each point was given a score, where 1 point denote relevant, $\frac{1}{2}$ point denote partly relevant and 0 points denote not relevant. For retrieving the most relevant papers based on the content and not just the quality of the paper, Quality Criteria \ref{itm:qa1a} and \vref{itm:qa1b} was multiplied by 3. Quality Criteria \ref{itm:qa1a} is concerned about whether or not the research in fact is a Vehicle Routing Problem and Quality Criteria \ref{itm:qa1b} addresses whether or not swarm intelligence is the main optimization method. 

Table \vref{table:literature} shows the papers that were selected based on the Inclusion Criteria and the scores given based on Quality Criteria.

The average score of the read literature was 13.01 $\approx$ 13, and literature given a score $\geq{1.5}$ above average were selected. This resulted in 12 final researches, which can be seen in Table \vref{table:finalliterature}. This research will form the foundation of the thesis, and Section \vref{sec:relatedWork} describes and summarizes the final selected literature, referred to as the related work. 
