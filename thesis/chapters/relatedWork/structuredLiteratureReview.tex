\section{Structured Literature Review}

\emph{\color{red} HER MÅ VALG AV STUFF BEGRUNNES. Hvorfor har vi valgt de søkeordene vi har valgt! }

The process of conducting a systematic literature review (synthesizing available information) before tackling an area of research helps mapping out existing solutions, avoids bias in the work and duplicating the effort, identifies gaps of knowledge, and highlights areas where additional research is required\citep{kofod2014}. The process is as following, and is entirely explained in appendix \ref{appendixA} and \ref{appendixB}:
%related work needs to be performed by thoroughly considering what we believe are the most important key words for collecting the most relevant literature, concerning our goal and research questions, and is inspired by [kofod]
\begin{itemize}
\item[Step 1] To conduct a structured literature review it is vital to decide the problem to be solved, referred to as $P$, and the constraints used to guide the search, referred to as $C$\citep{kofod2014}. These points and the following research questions can be found in appendix \ref{appendixA}, section A.3. 

\item[Step 2] To retrieve all relevant literature to the defined research questions, a search strategy must be defined. The selection of which sources to be searched is selected based on \citep[p.3]{kofod2014}, and can be found in appendix \ref{appendixB}. \emph{\color{red}For consistency}, some key terms must be decided and formed into groups of synonyms, and assembled into a complete search term.  The key terms and the complete search term can be found in appendix \ref{appendixA}, section A.1 and section A.2, respectively.

\item[Step 3] To exclude irrelevant literature, some inclusion criteria must be decided to ensure a level of relevance to the very first pool. The inclusion criteria can be found in appendix \ref{appendixA}, section A.4. 

\item[Step 4] Quality criteria must be decided to ensure quality in the final papers and to filter away studies that are not theoretically relevant for our thesis. Each of the studies under consideration should be classified according to the quality criteria, by scoring the papers on each quality criteria. The threshold for studies to be accepted must also be specified.
The quality criteria can be found in appendix \ref{appendixA}, section A.5. Each point will be given a score, where 1 point means relevant, $\frac{1}{2}$ point means partly relevant and 0 points means not relevant. For retrieving the most relevant papers based on the content and not just the quality of the paper, quality criteria 1(a) and 1(b) will be multiplied with 3.

\item[Step 6] The selection of the final literature will be based on the quality criteria scores, and the literature given a score $\geq{1.5}$ above average will be selected, and finally, the final literature based on the threshold for the studies can be extracted. 
\end{itemize}

After the inclusion criteria filtering, we had 42 sources for the related literature, including scientific papers and master theses. Table \ref{table:literature} on page \pageref{table:literature} shows the papers that were selected based on the inclusion criteria, and scored according to the quality criteria. When selecting the final literature we decided to do this solely based on the quality criteria scores. The average score of the read literature was 13.01 $\approx$ 13, and literature given a score $\geq{1.5}$ above average were selected. After this sorting we ended up with 14 final literatures. These 12 literatures are going to create the foundation of our thesis. Table \ref{table:finalliterature} on page \pageref{table:finalliterature} shows the final selected literature, \emph{\color{red} and in the section below is a summary of the final selected literature which will form the basis of our related work.}
