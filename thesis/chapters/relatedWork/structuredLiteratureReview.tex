\section{Structured Literature Review}
To establish whether swarm intelligence methods is suitable for the vehicle routing problem (research question \ref{itm:1a}) and what the state-of-the-art in solving vehicle routing problems using swarm intelligence methods is (research question \ref{itm:1b}), we conducted a structured literature review[\ref{appendixA}][\ref{appendixB}] based on the model suggested by \citet{kofod2014}. Synthesizing available information before tackling an area of research can, for example, help mapping out existing solutions, avoid duplicating the effort, and highlight areas where additional research is required\citep{kofod2014}. 
%The process of conducting a systematic literature review (synthesizing available information) before tackling an area of research helps mapping out existing solutions, avoids bias in the work and duplicating the effort, identifies gaps of knowledge, and highlights areas where additional research is required\citep{kofod2014}. The process is as following, and is entirely explained in appendix \ref{appendixA} and \ref{appendixB}:
%related work needs to be performed by thoroughly considering what we believe are the most important key words for collecting the most relevant literature, concerning our goal and research questions, and is inspired by [kofod]

\subsection{Problem Definition}
As mentioned in the motivation, section \vref{sec:motivation}, can efficient public transportation systems reduce negative effects of the private transportation networks, such as traffic jams, increased travel times, air pollution, noise and accidents. AtB's current solution consists of an experience based route network, and these manual attempts to provide acceptable solutions to the vehicle routing problems are not optimal. In order to overcome this problem, the number of journal publications on vehicle routing problems have increased in recent decades. These issues formed the basis of our problem. The problem and the following research questions can be found in appendix \ref{appendixA}, section \vref{sec:researchQSLR}. 
\emph{\color{red} TODO: Neo4j}

\subsection{Process Description}
The list of sources for retrieving all relevant literature to our defined research questions, were selected based on \citep[p.3]{kofod2014}, and can be found in appendix \vref{appendixB}. 

Key terms was decided based on our problem, formed into groups of synonyms, and assembled into a complete search term. The key terms and the complete search term can be found in appendix \ref{appendixA}, section \vref{searchterm} and section \vref{sec:searchTerms}, respectively.

To exclude the irrelevant literature, some inclusion criteria was decided to ensure a level of relevance to the very first pool. The inclusion criteria can be found in appendix \ref{appendixA}, section \vref{sec:inclusionCriteria}. Quality criteria was decided to ensure quality in the final papers and to filter away studies not theoretically relevant for our thesis. Each of the studies under consideration was classified according to the quality criteria, by scoring the papers on each quality criteria. Each point was given a score, where 1 point means relevant, $\frac{1}{2}$ point means partly relevant and 0 points means not relevant. For retrieving the most relevant papers based on the content and not just the quality of the paper, quality criteria 1(a) and 1(b) was be multiplied by 3. After the inclusion criteria filtering, we had 42 sources for the related literature, including scientific papers and master theses. Table \vref{table:literature} shows the papers that were selected based on the inclusion criteria, and scored according to the quality criteria. 

The average score of the read literature was 13.01 $\approx$ 13, and literature given a score $\geq{1.5}$ above average were selected. After this sorting we ended up with 12 final literatures. Table \vref{table:finalliterature} shows the final selected literature, which will form the foundation of our thesis. Next section describes and summarize the final selected literature, the related work.
