\section{Motivation} 

Trondheim and neighboring municipalities are among the areas with the greatest population growth in Norway \citep{website:miljopakken}. More people means more traffic, and without action will congestion and environmental problems in these urban areas continue to increase every year. 

Private transportation has many advantages for the citizens compared to the public ones. The citizens do not have to wait for a vehicle or change vehicles during a trip, and it is often more convenient. But private transportation has a lot of negative issues, such as traffic jams and increased travel times, which leads to increased air pollution, noise and accidents. 
 
Having efficient public transportation systems can substantially reduce negative effects of the private transportation networks. The public transportation systems is better suited for urban needs, when they can transport more people per time unit than cars, needing much less space. The environment package \citep{website:miljopakken} for transport in Trondheim aims to provide better road networks and public transportation. With this they hope to achieve lower emissions, shorter traffic jams and less traffic noise. An inadequately designed transit network can cause very long waiting times and increase uncertainty in bus arriving time, resulting in less people using the service. Therefore, public transportation systems should be improved by providing better travel services, minimizing the total travel time, and inform the public about them, in order to convince more people to travel with it instead of using their own car.

\citet{website:atb} is responsible for planning the public transport throughout Sør-Trøndelag County, and bus services comprise the major component of the public transportation system here. Bus services also has specific attractive features, such as flexible routes, medium capacity, low cost, easy implementation, flexible fleet size, and use of existing facilities. 
From a meeting with AtB we learned that the current solution of AtB consists of an experience based route network, where transport planners has constructed reasonable bus route networks and schedules entirely manually, following guidelines and exploiting local knowledge. As a result, the efficiency of the resulting networks is dependent of the designers experience and his / hers existing knowledge of constraints and transportation demands. 

The problem of designing the optimal set of routes for a fleet of vehicles, in order to serve a given set of customers, is referred to the vehicle routing problem (VRP). The urban transit network problem(UTNDP) is an example of this broader optimization problem, and is the problem of designing urban transit routes and schedules. The two major components of the UTNDP are the urban transit routing problem(UTRP) and the urban transit scheduling problem (UTSP). UTRP involves the development of efficient transit routes on an existing transit network, with predefined pick-up / drop-off point (e.g bus routes), and UTSP is assigning the schedules for the vehicles. In practice, the two phases are usually implemented sequentially, with the routes determined in advance of the schedules.%Skal denne være med her?

Satisfying all customer needs and in addition keeping all operator costs in check, can be difficult to achieve. Operator costs can include the total number of buses, total bus running distance and the total operating hours. A minimum trip time, amount of transitions, and a not too crowded bus (customers can tolerate standing in 15 minutes ) are among the most important factors that determine the passengers choice of public transit, AtB experiences. 

In the past, transit planners have done reasonable job designing transit networks and schedules without the assistance of scientific tools or systematic procedures, by using their experience and judgment and following established guidelines. Nevertheless, for a really large network it is almost impossible to design an efficient transit route network and bus schedules relying only on past experience and guidelines, because in a large urban area the number of bus routes and bus stops are extremely large. The manual attempts to provide an acceptable solution to the vehicle routing problems are not able to solve these large network problems efficient. In order to overcome this problem, the number of journal publications on vehicle routing problems have increased in recent decades, with the researchers efforts to contribute to the problem.  The increase in researching on these areas is also due to the progress in computational resources, and this has opened new possibilities for modeling more complex routing problems. %These new arising real-world applications provide inspiration for developing new approaches for coordinating complex transportation processes. %Skrives om litt
%In recent years network optimization has become an important field in operational research. Shortest paths, network flows, traffic equilibrium, tsp, vrp, as well as design of an optimal network. Reasons for great interest in network optimization: broad applicability to problemts in pricate entrerprise as well as those in public systems. Networks: changing demands: leads to regurlarly adapted or expanded to meet changing nemands; leads to decision problems than can be supported by network optimization. \citep{mandl79}
%TODO: skrive litt til
After a structure literature review found in appendix[\ref{appendixA}][\ref{appendixB}], we see that swarm intelligence algorithms has been proven to be useful in many vehicle routing problems, and many studies have been done to improve the quality of such problems.

We have some experience with the graph database Neo4j \citep{website:neo4j}. Since Neo4j is based on graph theory, we believe it has several benefits that we can take advantage of when solving our problem / goal[\ref{itm:goal}], including a natural node-edge-structure and the possibility of saving information to both the nodes and edges. We envision that the nodes will represent bus stops, and the edges will represent the connectivity between the stops.

%Må skrives om
This chapter describes the preparatory research done in order to develop the proposed method. Section \vref{sec:definingResearchTopic} defines the research topic used in order to guide the Structured Literature Review\citep{kofod2014}, which further is described in Section \vref{sec:structuredLiteratureReview}. Section \vref{sec:relatedWork} reviews the relevant studied and will answer Research Question \vref{itm:1} as a whole. Finally, in Section \vref{subsec:problemStatement}, Research Question \vref{itm:2} and \vref{itm:3} are defined based on the answers of Research Question \vref{itm:1}.  
\subsection{Mandls Benchmark Problem}
%TODO: kjøpe/leie Mandl boka
%TODO: skrive om 
%Since there is not done any op
In the current research on the VRPs and SI, there are some benchmark data that are available for researchers to use for comparison.... However (as mentioned in RW), for the UTNDP problem, Mandls network seems to be the only benchmark instance used by researchers. The road network is a real Swiss road network comprising of 15 nodes and 21 connections between them. And has been thoroughly examined by many optimization approaches such as \citep{kechagiopoulos14}, \citep{fan09}, \citep{nikolic14}. %This makes it easy to measure the performance of your solution. 

%TODO: skrive om 
Christoph Mandl concentrated mainly on the UTRP, and developed a solution in two stages. First, a feasible set of

Christoph Mandl[] approaches the UTNDP problem in a generic form. Mandl concentrated on the UTRP, and developed a solution in two stages. First, a feasible set of routes was generated, and then heuristics were applied to improve the quality of the initial route set / tour plan. The route generation phase involved computing the shortest path between all sets of vertices's using Dijstras[], and then producing the route set with the shortest paths that contained the most nodes, respecting the positions of any nodes selected as terminals. Nodes not selected, were iteratively included into routes, or new routes were created with unserved nodes as route terminals. In the first phase, Mandl only considered in-vehicle travel costs when accessing route quality. He then suggested an initial route set, and used these in the second phase. The second phase included obtaining the new routes at an intersection node, including a node that was close to a route (if travel demand between this node and the node on the route was high), and / or excluded a node from a route set that was already served by another rote (if the travel demand between this node and the other nodes in the route was low). Waiting costs were also considered in the second phase, in addition to in-vehicle travel costs. Waiting times were fixed as constant values, according to specified vehicle frequencies. 

%IFylles inn i results: in order to demonstrate the efficiency and the effectiveness of our algorithm... the Swiss road network introduced by Mandl is used. This road network has been already used as a benchmark problem by many researchers in the literature. \citep{kechagiopoulos14}


\section{Problem Statement}
\label{sec:problemStatement}

\emph{\color{blue} TODO: referansene må oppdateres.}

%\begin{itemize}
%\item \citet{cohen88} \textbf{asks, how is your reformulation / method an improvement? And underlying assumptions.}

ACO has several advantages for the VRP, such as natural parallelism and continuous positive feedback, which allows good solutions to be identified fast. However, ACO also has a few drawbacks including the weakness of getting stuck at a local optima. We will investigate the possibility of overcoming this drawback by improving ACO, and including features from other swarm inspired methods. As stated in Section \vref{sec:relatedWork}, we only found one attempt to combine different methods from swarm intelligence.% is to add a notion of the global best solution found to ACO and BCO. 

We did not manage to find any previous research that used graph databases in combination with the vehicle routing problem and swarm intelligence. As mentioned in \vref{subsubsec:neo4j}, does the graph database Neo4j \citep{website:neo4j} have several advantageous features for managing graphs, and we will therefore determine potential advantages and disadvantages of using Neo4j in our implementation and in the optimization process, giving us research question \vref{itm:3a}.

%\item \citet{cohen88} \textbf{asks if any aspects have been abstracted away?}

%As mentioned, does the UTRP comprise the design of physical transportation routes needed to solve the UTNDP. UTSP involves the development of schedules. We will in this thesis focus on UTRP. Geographical issues.
%and the objective in this thesis will therefore be to minimize the total travel time in the public transportation system, by optimizing the routes concerning the travel time and minimizing number of transfers.
The current solution of AtB consist of an experience based route network, and therefore not properly, computationally optimized concerning the travel demand and travel time. When a route network is not properly optimized, it can lead to a large number of transfers for passengers when they are traveling from their origin to their destination, resulting in a long total travel time. A good route network will ensure that routes having the most traveling demands are satisfied with short paths and few vehicle transfers, making travel demand a key variable for the algorithm. AtB\citep{website:atb} does not possess accurate data about the travel demand, and detailed investigations into measuring and predicting travel demand is a complex research problem, and beyond the scope of (and abstracted away\citep{cohen88} from) this thesis. 
%\item \citet{cohen88} \textbf{Does it rely on other methods?}

Demand values are all provided for Mandl's benchmark problem\citep{mandl79}. This benchmark problem seems to be the only one widely used for the UTRP and is acknowledged by researchers\citep{fan09},\citep{kechagiopoulos14},\citep{nikolic14}. Christoph Mandl\citep{mandl79} developed a heuristic algorithm for the UTRP, and the method was applied and based on a real network in Switzerland, the Swiss transit network\citep{mandl80}. The data includes a small and dense network with 15 nodes and 21 edges, in addition to the travel times and travel demand for each edge, and will in this thesis be used as the input data. %The total demand is 15570 trips per day, which is a relatively high demand for a small network. The travel time between the to farthest nodes in the network is 22 minutes along the shortest path. 

%Mandl developed a solution in two phases, where a feasible set of routes were created in the first phase and in the second phase he tried to minimize the the total travel time, including in-vehicle time and waiting time, by reducing the number of transfers. Some important measures of Mandl's solution network includes, for example, 100\% service coverage, 69.4\% of the trips involving no transfers, 29.93\% of the trips involving one transfer, and only 0.13\% of the trips needing more than one transfer. 

%\item \citet{cohen88} \textbf{asks when have you successfully demonstrated a solution? Is there a recognized metric for evaluating the performance?}

In order to demonstrate a solution we will evaluate the performance of our algorithm against the performance criteria used in similar research for comparison[\citep{kechagiopoulos14},\citep{mandl80},\citep{nikolic14},\citep{fan09}], and are the following:
\begin{itemize}
\item The percentage of demand satisfied without any transfers, which should be as high as possible.
\item The percentage of total transfer demands where the number of transfers are 1, which should be as low as possible.
\item The percentage of total transfer demands where the number of transfers are 2, which should be as low as possible.
\item The percentage of unsatisfied travelers, which should be equal to zero. An unsatisfied traveler is described as a traveler with 3 or more transfers.
\item The average travel time in minutes per transit user, which should be as low as possible. %The travel times incorporates a transfer penalty, which is sat to be 5 minutes per transfer for comparison reasons. 
\end{itemize}

%\item \citet{cohen88} \textbf{asks is the research is representative for a class of tasks? What is the scope of the method?}
%Does it transfer to more complicated problems?
For the reasons stated above, (the scope of the research) we will in this thesis focus on the UTRP, creating effective urban transit routes, with an ACO algorithm inspired by both BCO and PSO. (Comparing our results with Mandl etc.) This will help us establish Research Question \vref{itm:2a}, which is concerned about whether or not it is efficient to combine attributes from different swarm intelligence methods in order to improve the computational results.

Our goal with our implementation is that it further can be used to optimize AtBs transit network in order to increase the number of public transportation passengers. Research Question \vref{itm:2c} is concerned about whether or not it is possible to apply our algorithm to optimize bus routes in urban cities. This question cannot be fully answered until it is applied to an urban city, but we will strive to create a method that is easily adaptable with the concerns of public transportation in cities in mind. The implementation will be easily changed with new demand values, and tested on other larger networks.
%\end{itemize}



  

