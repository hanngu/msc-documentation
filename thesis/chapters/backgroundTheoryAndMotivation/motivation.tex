\section{Motivation} 
\label{sec:motivation}

Trondheim and neighboring municipalities are among the areas with the greatest population growth in Norway \citep{website:miljopakken}. More people means more traffic, and without action will congestion and environmental problems in these urban areas continue to increase every year.

Private transportation has many advantages for the citizens compared to the public ones. This includes a decreased travel time for the citizens, because this enables them to travel directly to their destination without detours, making it (often) more convenient. But private transportation has a lot of negative issues, such as traffic jams and increased travel times, which leads to increased air pollution, noise and accidents. 

Having efficient public transportation systems can substantially reduce negative effects of the private transportation networks. The public transportation systems is better suited for urban needs, because they can transport more people per time unit than cars, needing much less space. The environment package \citep{website:miljopakken} for transport in Trondheim aims to provide better road networks and public transportation. With this they hope to achieve lower emissions, less traffic jams and less traffic noise. An inadequately designed transit network can cause very long waiting times and increase uncertainty in bus arriving time, resulting in less people using the service. Therefore, public transportation systems should be improved by providing better travel services, minimizing the total travel time, and inform the public about them, in order to convince more people to travel with it instead of using their own car.

\citet{website:atb} is responsible for planning the public transport throughout Sør-Trøndelag County in Norway, which includes the city of Trondheim. Bus services comprise the major component of the public transportation system in the county, and in Norway generally. Bus services has specific attractive features, such as flexible routes, low cost, easy implementation, flexible fleet size, and use of existing facilities. AtB's current solution consists of an experience based route network, where transport planners has constructed reasonable bus route networks and schedules entirely manually, following guidelines and exploiting local knowledge. As a result, the efficiency of the resulting networks is dependent of the designers experience and his / hers existing knowledge of constraints and transportation demands.

%The problem of designing the optimal set of routes for a fleet of vehicles, in order to serve a given set of customers, is referred to the vehicle routing problem (VRP). The urban transit network problem(UTNDP) is an example of this broader optimization problem, and is the problem of designing urban transit routes and schedules. The two major components of the UTNDP are the urban transit routing problem(UTRP) and the urban transit scheduling problem (UTSP). UTRP involves the development of efficient transit routes on an existing transit network, with predefined pick-up / drop-off point (e.g bus routes), and UTSP is assigning the schedules for the vehicles. In practice, the two phases are usually implemented sequentially, with the routes determined in advance of the schedules.%Skal denne være med her?
%Satisfying all customer needs and in addition keeping all operator costs in check, can be difficult to achieve. Operator costs can include the total number of buses, total bus running distance and the total operating hours. AtB experiences that a minimum trip time, amount of transitions, and a not too crowded bus, are among the most important factors that determine the passengers choice of public transit.

In the past, transit planners have done reasonable job designing transit networks and schedules without the assistance of scientific tools or systematic procedures. Nevertheless, for a large network it is almost impossible to design an efficient transit route network and bus schedules relying only on past experience and guidelines, because in a large urban area the number of bus routes and bus stops are extremely large. The manual attempts to provide an acceptable solution to the vehicle routing problems are not able to solve these large network problems optimally. In order to overcome this problem, the number of journal publications on vehicle routing problems have increased in recent decades, with the researchers efforts to contribute to the problem.  The increase in researching on these areas is also due to the progress in computational resources, and this has opened new possibilities for modeling more complex routing problems. Swarm Intelligence, described in Section \vref{sec:swarmIntelligence}, has proven to solve a great number of NP-hard problems. This includes \citet{dorigo97} and \citet{lucic03} who solved the Traveling Salesman Problem with respectively an ant system and a bee system. The employment of an automatized method, like a SI method, for generating bus routes, would not only release planning time for AtB, but hopefully also increase the quality of the routes. 
