\section{Motivation}


 \citet{website:atb} is responsible for planning the public transport throughout Sør-Trøndelag County. Bus services also has specific attractive features, such as flexible routes, medium capacity, low cost (of capital and running), easy implementation, flexible fleet size (easy to expand or contract this size), and use of existing facilities (roads and streets)., Bus services also comprise the major component of the public transportation system in Trondheim.

 \subsection{Manual approaches to the UTNDP}

 For many years transport planners devised reasonable bus route networks and schedules entirely manually, relying on past experience, following practical guidelines and utilizing local knowledge.

 As a result, the efficiency of the resulting networks is highly dependent of the designers expedience and his / hers knowledge of existing resource constraints and transportation demands. 

 Historically, transit planners have done a reasonable job without the aid of scientific tools or systematic procedures, just using their experience and professional judgment, while adhering to planners guidelines. However, for a really large network it is almost impossible to design an efficient transit route network configuration and bus schedules relying only on past experience and guidelines: in a large urban area the number of bus routes and bus stops are extremely large. In order to overcome this problem, research efforts have increased in recent decades, coinciding with developments in information and computer technology. 

 Satisfying all customer needs, and keeping all operator costs in check, is really difficult to achieve. Operator costs mainly refer to the total number of buses, total bus running distance and the total operation hours. A minimum trip time, amount of transitions, and a not too crowded bus (customers can tolerate standing in 15 minutes ) are among the most important factors that determine the passengers choice of public transit, AtB experiences.The main concern of bus companies is maximizing its profits, while the main concern of the government is to fulfill all needs of traveling in public. 

 The manual attempts to provide an acceptable solution this problem are not able to solve these large network problems efficiently.

\textit{Skal denne være med her ?}
 The problem of designing the optimal set of routes for a fleet of vehicles, in order to serve a given set of customers, is referred to the vehicle routing problem (VRP). The urban transit network problem(UTNDP) is an example of this broader optimization problem VRP, and is the problem of designing urban transit routes and schedules. The two major components of the UTNDP are the urban transit routing problem(UTRP) and the urban transit scheduling problem (UTSP). \citep{fan09}. UTRP involves the development of efficient transit routes on an existing transit network, with predefined pick-up/drop-off point (e.g bus routes), and UTSP is assigning the schedules for the passengers carrying vehicles. In practice, the two phases are usually implemented sequentially, with the routes determined in advance of the schedules. 

 The number of journal publications on Vehicle Routing Problems has steadily increased over the years. This is because of the progress in computational resources has opened new possibilities for modeling more complex routing problems. New arising real-world applications provide inspiration for developing new approaches for coordinating complex transportation processes.  \citep{vehiclerouting}

%Må skrives om
This chapter describes the preparatory research done in order to develop the proposed method. Section \vref{sec:definingResearchTopic} defines the research topic used in order to guide the Structured Literature Review\citep{kofod2014}, which further is described in Section \vref{sec:structuredLiteratureReview}. Section \vref{sec:relatedWork} reviews the relevant studied and will answer Research Question \vref{itm:1} as a whole. Finally, in Section \vref{subsec:problemStatement}, Research Question \vref{itm:2} and \vref{itm:3} are defined based on the answers of Research Question \vref{itm:1}.  

\subsection{Mandls benchmark problem}
In the current research on the VRPs, some benchmark data are available for researchers, these can be seen in some websites such as TSLIPB and OR-Library. However, for the UTNDPm Mandls network seems the only benchmark instanse popularity used by researchers. 

In order to demonstrate the efficiency and the effectiveness of our algorithm... the Swiss road network introduced by Mandl is used. This road network has been already used as a benchmark problem by many researchers in the literature. \citep{kechagiopoulos14}

This road netowrk is a real Swiss road network which comprised of 15 nodes and 21 connections between them. This road network har been widely examined by many optimization approaches such as \citep{kechagiopoulos14}

The results will be compared with the results published by Mandl, og flere her....