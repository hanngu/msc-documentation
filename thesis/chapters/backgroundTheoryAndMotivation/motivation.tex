\section{Motivation} 
\label{sec:motivation}

Trondheim and neighboring municipalities are among the areas with the greatest population growth in Norway \citep{website:miljopakken}. More people means more traffic, and without action will congestion and environmental problems in these urban areas continue to increase every year.\par
Private transportation has many advantages for the citizens compared to the public ones, such as the citizens not needing to wait for a vehicle or change vehicles during a trip, making it (often) more convenient. But private transportation has a lot of negative issues, such as traffic jams and increased travel times, which leads to increased air pollution, noise and accidents. 

Having efficient public transportation systems can substantially reduce negative effects of the private transportation networks. The public transportation systems is better suited for urban needs, because they can transport more people per time unit than cars, needing much less space. The environment package \citep{website:miljopakken} for transport in Trondheim aims to provide better road networks and public transportation. With this they hope to achieve lower emissions, shorter traffic jams and less traffic noise. An inadequately designed transit network can cause very long waiting times and increase uncertainty in bus arriving time, resulting in less people using the service. Therefore, public transportation systems should be improved by providing better travel services, minimizing the total travel time, and inform the public about them, in order to convince more people to travel with it instead of using their own car.

\citet{website:atb} is responsible for planning the public transport throughout Sør-Trøndelag County, and bus services comprise the major component of the public transportation system here. Bus services has specific attractive features, such as flexible routes, low cost, easy implementation, flexible fleet size, and use of existing facilities. 
AtB's current solution consists of an experience based route network, where transport planners has constructed reasonable bus route networks and schedules entirely manually, following guidelines and exploiting local knowledge. As a result, the efficiency of the resulting networks is dependent of the designers experience and his / hers existing knowledge of constraints and transportation demands.

%The problem of designing the optimal set of routes for a fleet of vehicles, in order to serve a given set of customers, is referred to the vehicle routing problem (VRP). The urban transit network problem(UTNDP) is an example of this broader optimization problem, and is the problem of designing urban transit routes and schedules. The two major components of the UTNDP are the urban transit routing problem(UTRP) and the urban transit scheduling problem (UTSP). UTRP involves the development of efficient transit routes on an existing transit network, with predefined pick-up / drop-off point (e.g bus routes), and UTSP is assigning the schedules for the vehicles. In practice, the two phases are usually implemented sequentially, with the routes determined in advance of the schedules.%Skal denne være med her?
Satisfying all customer needs and in addition keeping all operator costs in check, can be difficult to achieve. Operator costs can include the total number of buses, total bus running distance and the total operating hours. AtB experiences that a minimum trip time, amount of transitions, and a not too crowded bus, are among the most important factors that determine the passengers choice of public transit.

In the past, transit planners have done reasonable job designing transit networks and schedules without the assistance of scientific tools or systematic procedures, using their experience and judgment, and following established guidelines. Nevertheless, for a large network it is almost impossible to design an efficient transit route network and bus schedules relying only on past experience and guidelines, because in a large urban area the number of bus routes and bus stops are extremely large. The manual attempts to provide an acceptable solution to the vehicle routing problems are not able to solve these large network problems optimally. In order to overcome this problem, the number of journal publications on vehicle routing problems have increased in recent decades, with the researchers efforts to contribute to the problem.  The increase in researching on these areas is also due to the progress in computational resources, and this has opened new possibilities for modeling more complex routing problems. %These new arising real-world applications provide inspiration for developing new approaches for coordinating complex transportation processes. %Skrives om litt
%In recent years network optimization has become an important field in operational research. Shortest paths, network flows, traffic equilibrium, tsp, vrp, as well as design of an optimal network. Reasons for great interest in network optimization: broad applicability to problemts in pricate entrerprise as well as those in public systems. Networks: changing demands: leads to regurlarly adapted or expanded to meet changing nemands; leads to decision problems than can be supported by network optimization. \citep{mandl79}
%TODO: skrive litt til

%Based on the paper \citet{dorigo97}, we see that swarm intelligence algorithms is applicable in solving vehicle routing problems. \citet{dorigo97} and \citet{lucic03} are two of the first published papers that describes methods using swarm intelligence to solve highly complex route optimization problems. 

%To represent our data, we will investigate the usefulness of the graph database Neo4j \citep{website:neo4j}. Since Neo4j is based on graph theory, we believe it has several benefits that we can take advantage of when solving our problem / goal [\ref{itm:goal}], including a natural node-edge-structure and the possibility of saving information to both the nodes and edges. We envision that the nodes will represent bus stops, and the edges will represent the connectivity between the stops.
%\subsection{Related work}

The related work for this thesis is gathered through a structure literature review. This process is conducted by thoroughly considering what we believe is the most important key words for collecting the most relevant literature concerning our goal, and is properly described in the Experiments and Results-chapter[\ref{experimentsAndResults}], and is entirely explained in the appendix[\ref{appendixA}][\ref{appendixB}]. The literature in this section are the results of the final data synthesis of the structured literature review. \newline

Swarm intelligence algorithms has proven to be useful in many vehicle routing problems, and many studies have been done to improve the quality of such problems. One of the research questions in this thesis is to establish whether it is possible to  solve vehicle routing problems using swarm intelligence and the results of the conducted structured literature will help us establish that. \citet{dorigo97} and \citet{lucic03} are two of the first published papers that confirms this research question by describing methods using swarm intelligence to solve highly complex route optimization problems. The use respectively an ant colony system and a bee system to solve the Traveling Salesman Problem (TSP), which is a subproblem of VRP. 

\citet{hsiao04} presented an optima approach to search for the best path of a map considering the traffic loading conditions. To do this, they proposed an algorithm based on ACO techniques to search for the shortest path from a desired origin to a desired destination. They random-generated a map consisting of 100-500 nodes, and compared their algorithm to a brute method emphasizing on the time used to generate the route. Their results states that if the map consists of more than 200 nodes, the ACO performs better than a brute method. In fact, their results shows that the more nodes the map contains, the higher the benefit of using the ACO algorithm.  

\citet{yang07} presented an optimization model for a bus network design based on the coarse-grain parallel ant colony algorithm (CPACA). CPACA is an optimization algorithm the develops a strategy to update the increased pheromone, called Ant Weight, where the path searching activities of the ants are adjusted based on the objective function. The model aims to minimize the average trip time by maximizing the number of direct travelers per unit length. Their results are compared with the classical MAX-MIN ant system (MMAS)\citep{stutzle99} and with ACA with Ant-weight strategy (ACA+). The comparison shows that CPACA performs better regarding both average direct traveler density and run time. They concluded that suitable future research will be to improve the searching efficiency, because their simulation shows that their algorithms is more stable and the runtime is more satisfactory when the number of nodes are less than 1000. 

\citet{salehi-nezhad07} presented an algorithm to search for the best direction between two desired origin and destination intersections in cities, called Ant-based Vehicle Navigation algorithm. The algorithm was applied on a part of the city of \textit{Kerman}, and the results are encouraging. The algorithm provides a fast access, low cost and easy method for vehicle navigation in cities without assisting GPS.

\citet{tripathi09} solved the vehicle routing problem with stochastic demand, in which the customer demand is modeled as a stochastic variable. They performed this using an improved version of the ACO approach, called ns-AAA SO, which oriented the search progressively towards favoring the global optimal solution. The characteristics and search capability of ns-AAA SO was then compared with both a standard ACO algorithm and a genetic algorithm. The results shows that ns-AAA SO outperforms the other two algorithms in every problem instance described by the authors. 

\citet{salehinejad10} introduced a route selection system which uses an ant colony system to detect an optimum multiparameter direction between two desired points in urban areas. The system employs fuzzy logic for local pheromone updating, and the proposed algorithm is called Fuzzy Logic-Ant Colony System (FLACS). The algorithm is applied to a part of London, United Kingdom, consisting of 42 junctions (nodes). The FLACS algorithm is compared to a standard ACS-algorithm and a $A^*$-ACS-algorithm emphasizing on the parameters ``distance'', ``traffic'' and ``incident risk'', which is said to be important to travelers. The presented results are the average of 10 randomly selected O/D pairs, and their result graphs states that FLACS performs better at average than both the standard ACS and the $A^*$-ACS regarding operational cost. The performance is independent of the importance rate of the parameters mentioned above. It is, however, worth mentioning that the estimation of further traffic data is done by ANNs, and therefore the traffic data used for each algorithm is not exactly the same. FLACS has less running time than $A^*$-ACS, but more than the standard ACS due to its Fuzzy Logic system component. 

\citet{jiang10} describes an improved ACO (IACO) algorithm to solve the Urban Transit Network Optimization which is described as a typical nonlinear combinatorial optimization problem. Improvements to the algorithm includes a stagnation counter to determine if an ant had stagnated and adding of extra pheromone intensity to newly discovered paths. This is done to compensate for the classical ACOs shortcomings of easily falling into stagnation and therefore obtain a local optimal solution. The IACO algorithm is, like the algorithm described by \citet{yang07}, compared to the classical MMAS algorithm. The results shows significant improvement to the convergence speed compared to MMAS. The average number of iterations to find the optimal solution was reduced from 1060 using MMAS to 548 using IACO. The average path distance was shortened from 15554.74 using MMAS to 15509.02 using IACO.  

\citet{poorzahedy11} proposed an Ant System application for solving the bus network design problem. The bus network design problem is a problem that is usually decomposed into two problems, respectively network configuration and bus frequency determination. Some of the characteristics of the authors algorithm is that it works with a decision graph, rather than on the bus network it self. The algorithm is only concerned about one objective; a combination of travel time for the users and the bus fleet size for the operator. The application was used to design the bus network of Mashhad, a city with a population of over 2 millions, and further compared with a genetic algorithm (GA). Their results shows that their algorithm performs better than the GA in both the number of routes, fleet size, in-vehicle travel time and waiting time. GA performs better than the Ant System algorithm with respect to travelers walking time. 

\citet{dias14} introduced an inverted ACO (IACO) algorithm. The idea was that the IACO algorithm inverts the classical ACO logic by converting the attraction of ants towards pheromones into a repulsion effect. The IACO was then used in a decentralized traffic management system, where the drivers acts as the inverted ants. The ideas was that drivers are repelled by the scent of pheromones (other drivers), and the system thus avoids congested roads. The IACO algorithm described, was compared to a shortest-time algorithm (ST). The IACO algorithm performs better than the ST algorithm, with the respect to trip time, travel length, fuel consumption and $CO_2$ emissions. This is as long as a considerable amount (25-50\%, depending on whether it was tested on a radial and ring network or a lattice network) of the vehicles uses an algorithm to decide which road to choose. 

\citet{nikolic14} proposed a model for the transit network design problem. To do this they used an approach based on BCO metaheuristic. They considered the network design problem in a way that the algorithm decided the links that was included in the transit network, and further creating designed bus routes based on the links. They tested two orders of importance of the objective functions (an order that is best for passengers and an order that is best for the transit operator). The algorithm was tested on Mandl's benchmark problem of a Swiss bus network\citep{mandl80}. This is probably the only widely investigated and accepted benchmark problem in the relevant literature\citep{kechagiopoulos14}. Their algorithm was compared to other competitive approaches (\citet{mandl80}; \citet{shih94}; \citet{baaj95}; \citet{bagloee11}), and the results may be considered as ambiguous. The algorithm proposed by \citet{nikolic14} performed best regarding travel time and number for transfers, but was sometimes outperformed regarding in-vehicle time and out-of-vehicle time. 

\citet{sedighpour14} introduced a hybrid ACO (HACO) algorithm where they used a new state transition rule and a candidate list, as well as several local search techniques and a new pheromone update rule. The hybrid was designed to overcome some of the original ACOs shortcomings, such as slow computing speed and local convergence. The HACO algorithm was applied to the open vehicle routing problem, a variant of the vehicle routing problem in which vehicles are not required to return to the depot after completing a service. The HACO algorithm was compared with three versions of PSO (standard PSO, PSO without one-point move (PSOWO) and PSO without neighbors (PSOWN). The algorithms were tested on fifteen different sets, consisting of 19 to 72 nodes with 2 to 7 vehicles fixed at the minimum possible. Their result table showed that HACO found the optimal solution in 14 out of 15 set compared to PSO with 13/15, PSOWO with 0/15 and PSOWN with 9/15. In the one set HACO did not find an optimal solution it still performed better than the other algorithms. As mentioned above, HACO was designed to overcome some of the original ACOs shortcomings and we therefore believe it is worth mentioning that the algorithm was not actually compared to the original ACO algorithm. 

\citet{kechagiopoulos14} designed and presented an optimization algorithm based on PSO. Their goal was to find an efficient solution to the urban transit routing problem (UTRP), which is a NP-hard problem that deals with the construction of route networks for public transportation. The target problem was Mandl's benchmark problem. Their algorithm is compared with competitive approaches (including genetic algorithms and other metaheuristic approaches) mentioned in literature (\citet{baaj91}; \citet{chakroborty02}; \citet{kidwai98}; \citet{fan10}; \citet{fan09-2}; \citet{zhang10}; \citet{chew12}). The algorithms are compared with regard to the percentage of total transfer demands satisfied directly ($d_0$), with one transfer ($d_1$), two transfers ($d_2$), or with more than two transfers or not satisfied at all ($d_{unsat}$). The algorithms are also compared regarding average in-vehicle travel time ($ATT$). The experiments are conducted on route set designs with four, six, seven and eight routes. The proposed algorithm performs better than the competitors regarding $ATT$ independent the route size, and achieves a better percentage of direct travelers ($d_0$) except from when the route size is four. The percentage of passengers with more than two transfers or not satisfied at all ($d_{unsat}$), is $0.00$ for all the described algorithms independent of route size. \newline

Based on the described literature we can conclude that swarm intelligence-algorithms shows great promise in solving different vehicle routing problems. We specifically notice that SI-algorithms performs equal and sometimes better than genetic algorithms and  metaheurstic approaches like Simulated Annealing. We also notice that 2 out of the 12 described papers uses Mandl's benchmark problem as a test case for their proposed algorithm, and that 2 out of 12 papers compare their algorithm to \citet{stutzle99}s MAX-MIN ant system (MMAS). It is worth mentioning that none of the proposed algorithms performs optimally in every situation and that there is no consensus in which algorithm is the overall best. It is therefore difficult to draw a singular conclusion to what is the state of the art regarding swarm intelligence and vehicle routing problems. We suggest that a general state of the art regarding the algorithm may be described as being inspired by the way different swarms act in nature, and to further add artificial features to the individuals to optimize for certain objectives. Mandl's benchmark problem was used by 2 out of the 4 papers in our literature review published in 2014, and we therefore believe that using this benchmark problem as a test case may be considered as the state of the art.


%Skrive om at det "To the best of our knowledge" ikke eksisterer en supersverm algoritme. 

%\subsection{Mandls Benchmark Problem}
%TODO: kjøpe/leie Mandl boka
%TODO: skrive om 
%Since there is not done any op
In the current research on the VRPs and SI, there are some benchmark data that are available for researchers to use for comparison.... However (as mentioned in RW), for the UTNDP problem, Mandls network seems to be the only benchmark instance used by researchers. The road network is a real Swiss road network comprising of 15 nodes and 21 connections between them. And has been thoroughly examined by many optimization approaches such as \citep{kechagiopoulos14}, \citep{fan09}, \citep{nikolic14}. %This makes it easy to measure the performance of your solution. 

%TODO: skrive om 
Christoph Mandl concentrated mainly on the UTRP, and developed a solution in two stages. First, a feasible set of

Christoph Mandl[] approaches the UTNDP problem in a generic form. Mandl concentrated on the UTRP, and developed a solution in two stages. First, a feasible set of routes was generated, and then heuristics were applied to improve the quality of the initial route set / tour plan. The route generation phase involved computing the shortest path between all sets of vertices's using Dijstras[], and then producing the route set with the shortest paths that contained the most nodes, respecting the positions of any nodes selected as terminals. Nodes not selected, were iteratively included into routes, or new routes were created with unserved nodes as route terminals. In the first phase, Mandl only considered in-vehicle travel costs when accessing route quality. He then suggested an initial route set, and used these in the second phase. The second phase included obtaining the new routes at an intersection node, including a node that was close to a route (if travel demand between this node and the node on the route was high), and / or excluded a node from a route set that was already served by another rote (if the travel demand between this node and the other nodes in the route was low). Waiting costs were also considered in the second phase, in addition to in-vehicle travel costs. Waiting times were fixed as constant values, according to specified vehicle frequencies. 

%IFylles inn i results: in order to demonstrate the efficiency and the effectiveness of our algorithm... the Swiss road network introduced by Mandl is used. This road network has been already used as a benchmark problem by many researchers in the literature. \citep{kechagiopoulos14}


\emph{\color{blue} TODO: Motivation behind swarm intelligence}
  

