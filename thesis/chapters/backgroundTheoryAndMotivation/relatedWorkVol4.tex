\section{Related Work} 

Our first research question can be summed up to what has been done before regarding solving VRP's using swarm intelligence. To answer this research question and its subquestions we therefor conducted a structured literature review, which we believe will help us answer these questions. The process of the structured literature review is performed by thoroughly considering what we believe are the most important key words for collecting the most relevant literature, concerning our goal and research questions and using these keywords to search for articles in different search engines. The conducted process is described in the Experiments and Results-chapter[\ref{experimentsAndResults}], and is entirely explained in the appendix[\ref{appendixA}][\ref{appendixB}]. The literature in this section are the results of the final data synthesis of the structured literature review. They all describes a vehicle routing problem solved by a swarm inspired method. \newline

Swarm intelligence algorithms has proven to be useful in many vehicle routing problems, and many studies have been conducted to improve the solution, both regarding time complexity and quality, of such problems. \citet{dorigo97} and \citet{lucic03} were two of the first published papers that shows that methods from swarm intelligence is suitable for solving VRP's. They use respectively an ant colony system and a bee system to solve the Traveling Salesman Problem (TSP), which is a subproblem of VRP. 

Many computer scientists has since then studied the possibility of solving VRPs using swarm intelligence. \citet{hsiao04}, \citet{salehi-nezhad07}, \citet{tripathi09}, \citet{dias14}, and \citet{sedighpour14} all studied the possibility of using swarm intelligence to solve vehicle routing problems involving cars transporting either persons or goods. 

%Mangler informasjon om algoritmene som sammenlignes med. Parametersettingen er beskrevet, men ikke forklart. 
\citet{hsiao04} presented an approach to search for the best path of a map considering the traffic loading conditions. To do this, they proposed an ACO algorithm to search for the shortest path from a desired origin to a desired destination. The presented algorithm is a classic ACO algorithm without changes. To test their algorithm, the random-generated a map consisting of 100-500 nodes, and compared their algorithm to a brute method emphasizing on the time used to generate the route. Their results states that if the map consists of more than 200 nodes, the ACO performs better than a brute method. In fact, their results shows that the more nodes the map contains, the higher the benefit of using the ACO algorithm. 

%Sammenligner egentlig ikke mot en dritt? Algoritmen er bare testet på et veldig lite testsett. Eksperimentene for å finne parameterne er ikke nevnt, men de brukte parameterne er
\citet{salehi-nezhad07} presented an ant algorithm to search for the best path between two desired origin and destination intersections in cities, called Ant-based Vehicle Navigation algorithm. To get more accurate results than the classic ant algorithms they employed an \textit{awarding/punishment}-method. In order to find the best path, the presented algorithm is concerned about the parameters \textit{distance}, \textit{width}, \textit{traffic load}, \textit{road risk}, \textit{road quality}, and \textit{number of intersections}. The importance of each of the parameters are weighted from 0 to 1 by the user. 

%Mangler informasjon om algoritmene som sammenlignes med
\citet{tripathi09} solved the vehicle routing problem with stochastic demand, in which the customer demand is modeled as a stochastic variable. They performed this using an improved version of the ACO approach, called ns-AAA SO. The proposed algorithm orients the search progressively towards favoring the global optimal solution. To do this they defines that a complete iteration consists of two tours: The first tour is a social tour that corresponds to a standard ACO iteration. The second tour is a neighborhood tour where the ants are allowed to communicate important information found in the social tour and change their solutions. If the fitness value of the new solution is better, the old solution found in the social tour is replaced. To favor the optimal solution, the path of the global best ant is given more pheromones. Further, to prevent the search for from entrapment into a local optima, a minimum quantity of pheromone on any edge, $t_{min}$, is always maintained. The performance of ns-AAA SO was compared with both a standard ACO algorithm and a genetic algorithm. Their results shows that ns-AAA SO outperforms the other two algorithms in every problem instance described by the authors.

%Mandler informasjon om parametersetting
\citet{dias14} introduced an inverted ACO (IACO) algorithm. The idea was that the IACO algorithm inverts the classical ACO logic by converting the attraction of ants towards pheromones into a repulsion effect. Their presented research combines Dijkstra and IACO, to address both the static (edge distance) and dynamic (car density) natures of a traffic network. The proposed approach was used in a decentralized traffic management system, where the drivers acted as the inverted ants. The idea was that drivers are repelled by the scent of pheromones (other drivers), and the system thus avoids congested roads. The described approach was compared to a shortest-time algorithm (ST). The IACO algorithm performs better than the ST algorithm, with the respect to trip time, travel length, fuel consumption and $CO_2$ emissions. This is as long as a considerable amount (25-50\%, depending on whether it was tested on a radial and ring network or a lattice network) of the vehicles uses an algorithm to decide which road to choose. 

%Sammenligner ikke algoritmen mot original ACO, selvom de sier at de ønsker å forbedre ACO i introduksjonen
\citet{sedighpour14} introduced a hybrid ACO (HACO) algorithm where they used a new state transition rule and a candidate list, as well as several local search techniques and a new pheromone update rule. The hybrid was designed to overcome some of the original ACOs shortcomings, such as slow computing speed and local convergence. The HACO algorithm was applied to the open vehicle routing problem, a variant of the vehicle routing problem in which vehicles are not required to return to the depot after completing a service. The HACO algorithm was compared with three versions of PSO (standard PSO, PSO without one-point move (PSOWO) and PSO without neighbors (PSOWN). The algorithms were tested on fifteen different sets, consisting of 19 to 72 nodes with 2 to 7 vehicles fixed at the minimum possible. Their result table showed that HACO found the optimal solution in 14 out of 15 set compared to PSO with 13/15, PSOWO with 0/15 and PSOWN with 9/15. In the one set HACO did not find an optimal solution it still performed better than the other algorithms.\newline

Urban Transit Network Design Problem (UTRP), a subproblem of VRP, considers other objectives and requires other methods for solving than classical VRP problems. \citet{yang07}, \citet{salehinejad10}, \citet{jiang10}, \citet{poorzahedy11}, \citet{nikolic14}, and \citet{kechagiopoulos14} all addressed problems related to urban transit networks. \newline

%Parameterne er nevnt og den sammenligner med relevante algoritmer
\citet{yang07} presented an optimization model for a urban bus network design (UBND) based on the coarse-grain parallel ant colony algorithm (CPACA). CPACA is an optimization algorithm that develops a strategy to update the increased pheromone, called Ant Weight, where the path searching activities of the ants are adjusted based on the objective function. The model aims to minimize the average trip time by maximizing the number of direct travelers per unit length. Their results are compared with the classical MAX-MIN ant system (MMAS)\citep{stutzle99} and with ACA with Ant-weight strategy (ACA+). The comparison shows that CPACA performs better regarding both average direct traveler density and run time. They concluded that suitable future research will be to improve the searching efficiency, because their simulation shows that their algorithms is more stable and the runtime is more satisfactory when the number of nodes are less than 1000. 

%Parameterne er nevnt og den sammenligner med relevante algoritmer.
\citet{salehinejad10} introduced a route selection system which uses an ant colony system to detect an optimum multiparameter direction between two desired points in urban areas. The system employs fuzzy logic for local pheromone updating, and the proposed algorithm is called Fuzzy Logic-Ant Colony System (FLACS). The algorithm is applied to a part of London, United Kingdom, consisting of 42 junctions (nodes). The FLACS algorithm is compared to a standard ACS-algorithm and a $A^*$-ACS-algorithm emphasizing on the parameters ``distance'', ``traffic'' and ``incident risk'', which is said to be important to travelers. The presented results are the average of 10 randomly selected O/D pairs, and their result graphs states that FLACS performs better at average than both the standard ACS and the $A^*$-ACS regarding operational cost, regardless of the importance rate of the parameters mentioned above. It is, however, worth mentioning that the estimation of further traffic data is done by ANNs, and therefore the traffic data used for each algorithm is not exactly the same. The research shows that FLACS has less running time than $A^*$-ACS, but more than the standard ACS due to its Fuzzy Logic system component. 

%Parameterne er nevnt, men ikke begrunnet. Sammenligner med relevante algoritmer. 
\citet{jiang10} describes an improved ACO (IACO) algorithm to solve the Urban Transit Network Optimization which is described as a typical nonlinear combinatorial optimization problem. Improvements to the algorithm includes a stagnation counter to determine if an ant had stagnated and adding of extra pheromone intensity to newly discovered paths. This is done to compensate for the classical ACOs shortcomings of easily falling into stagnation and therefore obtain a local optimal solution. The IACO algorithm is, like the algorithm described by \citet{yang07}, compared to the classical MMAS algorithm. The results shows significant improvement to the convergence speed compared to MMAS. The average number of iterations to find the optimal solution was reduced from 1060 using MMAS to 548 using IACO. The average path distance was shortened from 15554.74 using MMAS to 15509.02 using IACO. 

%Parameterne er nevnt, diskutert og begrunnet. Sammenligner for så vidt med relevante algoritme (GA).
\citet{poorzahedy11} proposed an Ant System application for solving the bus network design problem. The bus network design problem is a problem that is usually decomposed into two problems, respectively network configuration and bus frequency determination. Some of the characteristics of the authors algorithm is that it works with a decision graph, rather than on the bus network it self. The algorithm is only concerned about one objective; a combination of travel time for the users and the bus fleet size for the operator. The application was used to design the bus network of Mashhad, a city with a population of over 2 millions, and further compared with a genetic algorithm (GA). Their results shows that their algorithm performs better than the GA in both the number of routes, fleet size, in-vehicle travel time and waiting time. GA performs better than the Ant System algorithm with respect to travelers walking time. The walking time attained by AS is 3 \% greater than the walking time attained by the GA. We believe this is because both the fleet size and number of routes are attained by AS is smaller than the ones obtained by GA. Both the GA and the AS performs significantly better than the existing solution on all measures.  

%Parameterne er nevnt, men ikke begrunnet. Sammenligner med relevante algoritmer. 
\citet{nikolic14} proposed a model for solving the transit network design problem. To do this they used an approach based on BCO metaheuristic. They divided the network design problem in two parts; the first part designed the actual network (decided which links to be included) and the second part created bus routes based on the designed network. They tested two orders of importance of the objective functions; an order that is best for passengers and an order that is best for the transit operator. The algorithm was tested on Mandl's benchmark problem of a Swiss bus network\citep{mandl80}. Their algorithm was compared to other competitive approaches (\citet{mandl80}; \citet{shih94}; \citet{baaj95}; \citet{bagloee11}) regarding total travel time, in-vehicle time, out-of-vehicle time, fleet size and number of transfers. The results may be considered as somewhat ambiguous. The algorithm proposed by \citet{nikolic14} performed best regarding total travel time and number of transfers if the order of importance was set to favor what was best for passengers and the number of lines was greater than 4. If the order of importance was set to favor what was best for the operator, the algorithm created the solution with the smallest fleet size independent of number of lines, but then the algorithm performed mediocre regarding all the other measures. 

%Parameterne er nevnt, diskutert og begrunnet. Sammenligner med relevante algoritmer.
\citet{kechagiopoulos14} designed and presented an optimization algorithm based on PSO. Their goal was to find an efficient solution to the urban transit routing problem (UTRP), which is a NP-hard problem that deals with the construction of route networks for public transportation. The target problem was, like \citet{nikolic14}, Mandl's benchmark problem. Their algorithm is compared with competitive approaches (including genetic algorithms and other metaheuristic approaches) mentioned in literature (\citet{baaj91}; \citet{chakroborty02}; \citet{kidwai98}; \citet{fan10}; \citet{fan09-2}; \citet{zhang10}; \citet{chew12}). The algorithms are compared with regard to the percentage of total transfer demands satisfied directly ($d_0$), with one transfer ($d_1$), two transfers ($d_2$), or with more than two transfers or not satisfied at all ($d_{unsat}$). The algorithms are also compared regarding average in-vehicle travel time ($ATT$). The experiments are conducted on route set designs with four, six, seven and eight routes. The proposed algorithm performs better than the competitors regarding $ATT$ independent the route size, and achieves a better percentage of direct travelers ($d_0$) except from when the route size is four. The percentage of passengers with more than two transfers or not satisfied at all ($d_{unsat}$), is $0.00$ for all the described algorithms independent of route size.  \newline

Using methods from swarm intelligence to solve VRP's are still fairly unexplored, and there are many questions to be both asked and answered. However, there has in the passed decade been published a fare amount of research on the subject, and we believe that research which clearly compares the proposed SI-algorithm to other relevant algorithms increases the credibility of the research and also contributes valuable information to the field. We see that a lot of the newest published research addresses the weaknesses of classical SI-methods and makes changes to the original algorithms to overcome some of these. In these cases we believe that the conducted experiments should include comparison with other swarm inspired methods to indicate whether or not their solution improved the addressed weaknesses. 

\citet{tripathi09}, \citet{yang07}, \citet{salehinejad10}, and \citet{jiang10} all presented research where their swarm inspired algorithm was designed to overcome some of the known weaknesses, such as getting stuck at local optima. They also compared their solutions with other corresponding swarm methods, achieving promising results. We believe this increases the quality of their research, because they are able to concretely say whether or not the addressed weaknesses are improved. 

\citet{hsiao04} and \citet{sedighpour14} created ACO inspired algorithms, and they also wanted to overcome some of the known weaknesses of ACO. \citet{hsiao04} stated in the introduction that their method differs from other ACO methods because ``it can be implemented easily, it is flexible for many different problems' formulation, and the most of all, it can escape the local optima of the given problem''. However, neither \citet{hsiao04} nor \citet{sedighpour14} compared their proposed algorithms to other implementations of ACO. Because of this we believe their researches are not able to conclude whether or not it actually overcomes some of ACO's weaknesses. 

\citet{hsiao04}'s algorithm is in fact only tested against against a brute method which is not described in the research other than in the result table. 

\citet{nikolic14} and \citet{kechagiopoulos14} both uses Mandl's benchmark problem as input. This benchmark problem is a small network containing 15 nodes and 21 edges.  Mandl's network is widely used and acknowledged by multiple researchers, including \citet{nikolic14}, \citet{kechagiopoulos14}, and \citet{fan09}. We believe this is a strength of their researches, because this enables them to compare their results to a numerous of other solutions using the same benchmark problem. However, as mentioned, the Mandl network is quite small. We believe that the robustness of their algorithms regarding both time and space complexity could have been verified by also applying their algorithm to a larger test case. 

\citet{salehi-nezhad07} also applied their solution to a small test set containing only 27 intersections and requiring only 5 ants and, like the authors that used the Mandl Network, we believe the robustness could have been verified by also applying the algorithm to a larger test set. \citet{salehi-nezhad07} did not compare their algorithm against any other algorithm, and we therefor believe it is difficult to verify their results. 

Neither \citet{dias14} nor \citet{poorzahedy11} tests their solution against other swarm intelligence methods, but against other reasonable algorithms, respectively a Shortest Time-algorithm and a GA. In \citet{dias14}'s research we believe that it makes sense to only test against an ST-algorithm, because they have inverted a core factor of the original ACO algorithm. An comparison to, for example, the original ACO would not only be non-descriptive, but possibly not feasible. We also believe that the comparison against a GA in the research of \citet{poorzahedy11} is descriptive, because they did not add additional features to the AS algorithm. \newline

The performance of metaheuristic methods, such as swarm inspired methods, are highly dependent on their parameter settings. \citet{hsiao04}, \citet{salehi-nezhad07}, \citet{tripathi09}, \citet{sedighpour14}, \citet{yang07}, \citet{salehinejad10}, \citet{jiang10}, \citet{poorzahedy11}, \citet{nikolic14}, and \citet{kechagiopoulos14} all describes the parameters used by their algorithm. This makes their experiments feasible to replicate and compare against. In addition, \citet{sedighpour14}, \citet{poorzahedy11}, and \citet{kechagiopoulos14} discussed and justified their parameter settings by conducting their experiments in two parts; one for parameter setting and one for performance. We believe this increases the credibility of their research. We also believe the process of parameter tuning is an important contribution to the field of swarm intelligence in general. Other authors, like \citet{salehi-nezhad07}, \citet{yang07}, describes their parameter settings as a product of ``trial and error''.  The research of \citet{dias14} lacks information about parameter settings all together, and we considers this to be a weakness of their research. \newline

Considering the research questions mentioned above, we believe it is safe to say that 


Based on the described literature we can conclude that swarm intelligence-algorithms shows great promise in solving different vehicle routing problems. We specifically notice that SI-algorithms performs equal and sometimes better than genetic algorithms and classical metaheurstic approaches such as Simulated Annealing. We also notice that 2 out of the 12 described papers uses Mandl's benchmark problem as a test case for their proposed algorithm, and that 2 out of 12 papers compare their algorithm to \citet{stutzle99}s MAX-MIN ant system (MMAS). It is worth mentioning that none of the proposed algorithms performs optimally in every situation and that there is no consensus in which algorithm is the overall best. It is therefore difficult to draw a singular conclusion to what is the state of the art regarding swarm intelligence and vehicle routing problems. We suggest that a general state of the art regarding the algorithm may be described as being inspired by the way different swarms act in nature, and to further add artificial features to the individuals to optimize for certain objectives. Mandl's benchmark problem was used by 2 out of the 4 papers in our literature review published in 2014, and we therefore believe that using this benchmark problem as a test case may be considered as the state of the art.



%Changes done: forklare endringene