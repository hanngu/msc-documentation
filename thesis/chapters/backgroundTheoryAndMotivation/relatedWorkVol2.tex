Swarm intelligence algorithms has proven to be useful in many vehicle routing problems, and many studies have been done to improve the quality of such problems. One of the research questions in this thesis is to establish whether it is possible to  solve vehicle routing problems using swarm intelligence and the results of the conducted structured literature will help us establish that. \citet{dorigo97} and \citet{lucic03} are two of the first published papers that confirms this research question by describing methods using swarm intelligence to solve highly complex route optimization problems. The use respectively an ant colony system and a bee system to solve the Traveling Salesman Problem (TSP), which is a subproblem of VRP.

\subsection{Related work}
The related work for this thesis is gathered through a structure literature review. This process is conducted by thoroughly considering what we believe is the most important key words for collecting the most relevant literature concerning our goal and research questions. It is properly described in the Experiments and Results-chapter[\ref{experimentsAndResults}], and is entirely explained in the appendix[\ref{appendixA}][\ref{appendixB}]. The literature in this section are the results of the final data synthesis of the structured literature review. 

\subsubsection{State of the art}
There are also other researchers that has joined in for solving the vehicle routing problem using swarm intelligence. 
Presenting algorithms based on ACO techniques, \citet{hsiao04} searched for the best path of a map considering the traffic loading conditions, \citet{yang07} develops a strategy to update the increased pheromone to optimize a bus network design, \citet{salehi-nezhad07} searched for the best direction between two desired origin and destination intersections in cities, \citet{salehinejad10} introduced a route selection system to detect an optimum multiparameter direction between two desired points in urban areas, \citet{tripathi09} tries to solve the vehicle routing problem with stochastic demand, in which the customer demand is modeled as a stochastic variable, \citet{jiang10} tries to solbve the Urban Transit Network Optimization which is described as a typical nonlinear combinatorial optimization problem, \citet{poorzahedy11} tries to solve the bus network design problem, respectively network configuration and bus frequency determination, \citet{dias14} used it in a decentralized traffic management system, and \citet{sedighpour14} used a new state transition rule and a candidate list, as well as several local search techniques and a new pheromone update rule, to overcome some of the original ACOs shortcomings.  \citet{nikolic14} proposed a model for the transit network design problem, using an approach based on BCO metaheuristic. \citet{kechagiopoulos14} designed and presented an optimization algorithm based on PSO, to find an efficient solution to the urban transit routing problem (UTRP). 

\subsubsection{Changes done to the classical swarm intelligence methods}
%Changes to SI
\citet{yang07}, \citet{salehi-nezhad07}, \citet{tripathi09}, \citet{salehinejad10}. \citet{jiang10}, \citet{dias14}, \citet{sedighpour14} have all done changes to the classical swarm intelligence methods in order to improve them.
 
\citet{yang07} presented a coarse-grain parallel ant colony algorithm (CPACA). CPACA is an optimization algorithm the develops a strategy to update the increased pheromone, called Ant Weight, where the path searching activities of the ants are adjusted based on the objective function. The model aims to minimize the average trip time by maximizing the number of direct travelers per unit length. Their results are compared with the classical MAX-MIN ant system (MMAS)\citep{stutzle99} and with ACA with Ant-weight strategy (ACA+). The comparison shows that CPACA performs better regarding both average direct traveler density and run time. They concluded that suitable future research will be to improve the searching efficiency, because their simulation shows that their algorithms is more stable and the runtime is more satisfactory when the number of nodes are less than 1000. 

\citet{salehi-nezhad07} presented an algorithm called Ant-based Vehicle Navigation algorithm. The algorithm was applied on a part of the city of \textit{Kerman}, and the results are encouraging. The algorithm provides a fast access, low cost and easy method for vehicle navigation in cities without assisting GPS.

\citet{tripathi09} used an improved version of the ACO approach, called ns-AAA SO, which oriented the search progressively towards favoring the global optimal solution. The characteristics and search capability of ns-AAA SO was then compared with both a standard ACO algorithm and a genetic algorithm. The results shows that ns-AAA SO outperforms the other two algorithms in every problem instance described by the authors. 

\citet{salehinejad10} uses an ant colony system, employing fuzzy logic for local pheromone updating. The proposed algorithm is called Fuzzy Logic-Ant Colony System (FLACS). The algorithm is applied to a part of London, United Kingdom, consisting of 42 junctions (nodes). The FLACS algorithm is compared to a standard ACS-algorithm and a $A^*$-ACS-algorithm emphasizing on the parameters ``distance'', ``traffic'' and ``incident risk'', which is said to be important to travelers. The presented results are the average of 10 randomly selected O/D pairs, and their result graphs states that FLACS performs better at average than both the standard ACS and the $A^*$-ACS regarding operational cost. The performance is independent of the importance rate of the parameters mentioned above. It is, however, worth mentioning that the estimation of further traffic data is done by ANNs, and therefore the traffic data used for each algorithm is not exactly the same. FLACS has less running time than $A^*$-ACS, but more than the standard ACS due to its Fuzzy Logic system component. 

\citet{jiang10} describes an improved ACO (IACO). Improvements to the ACO algorithm includes a stagnation counter to determine if an ant had stagnated and adding of extra pheromone intensity to newly discovered paths. This is done to compensate for the classical ACOs shortcomings of easily falling into stagnation and therefore obtain a local optimal solution. The IACO algorithm is, like the algorithm described by \citet{yang07}, compared to the classical MMAS algorithm. The results shows significant improvement to the convergence speed compared to MMAS. The average number of iterations to find the optimal solution was reduced from 1060 using MMAS to 548 using IACO. The average path distance was shortened from 15554.74 using MMAS to 15509.02 using IACO. 

\citet{dias14} introduced an inverted ACO (IACO) algorithm. The idea was that the IACO algorithm inverts the classical ACO logic by converting the attraction of ants towards pheromones into a repulsion effect. The IACO was then used in a decentralized traffic management system, where the drivers acts as the inverted ants. The ideas was that drivers are repelled by the scent of pheromones (other drivers), and the system thus avoids congested roads. The IACO algorithm described, was compared to a shortest-time algorithm (ST). The IACO algorithm performs better than the ST algorithm, with the respect to trip time, travel length, fuel consumption and $CO_2$ emissions. This is as long as a considerable amount (25-50\%, depending on whether it was tested on a radial and ring network or a lattice network) of the vehicles uses an algorithm to decide which road to choose. 

\citet{sedighpour14} introduced a hybrid ACO (HACO) algorithm where they used a new state transition rule and a candidate list, as well as several local search techniques and a new pheromone update rule. The hybrid was designed to overcome some of the original ACOs shortcomings, such as slow computing speed and local convergence. The HACO algorithm was applied to the open vehicle routing problem, a variant of the vehicle routing problem in which vehicles are not required to return to the depot after completing a service. The HACO algorithm was compared with three versions of PSO (standard PSO, PSO without one-point move (PSOWO) and PSO without neighbors (PSOWN). The algorithms were tested on fifteen different sets, consisting of 19 to 72 nodes with 2 to 7 vehicles fixed at the minimum possible. Their result table showed that HACO found the optimal solution in 14 out of 15 set compared to PSO with 13/15, PSOWO with 0/15 and PSOWN with 9/15. In the one set HACO did not find an optimal solution it still performed better than the other algorithms. As mentioned above, HACO was designed to overcome some of the original ACOs shortcomings and we therefore believe it is worth mentioning that the algorithm was not actually compared to the original ACO algorithm. 

\textbf{Er disse forandret?}\newline
%Not changed??
\citet{hsiao04} proposed an algorithm based on ACO techniques to search for the shortest path from a desired origin to a desired destination. They random-generated a map consisting of 100-500 nodes, and compared their algorithm to a brute method emphasizing on the time used to generate the route. Their results states that if the map consists of more than 200 nodes, the ACO performs better than a brute method. In fact, their results shows that the more nodes the map contains, the higher the benefit of using the ACO algorithm. 

%Not changed??
\citet{poorzahedy11} proposed an Ant System application for solving the bus network design problem. The bus network design problem is a problem that is usually decomposed into two problems, respectively network configuration and bus frequency determination. Some of the characteristics of the authors algorithm is that it works with a decision graph, rather than on the bus network it self. The algorithm is only concerned about one objective; a combination of travel time for the users and the bus fleet size for the operator. The application was used to design the bus network of Mashhad, a city with a population of over 2 millions, and further compared with a genetic algorithm (GA). Their results shows that their algorithm performs better than the GA in both the number of routes, fleet size, in-vehicle travel time and waiting time. GA performs better than the Ant System algorithm with respect to travelers walking time. 

%Not changed?
\citet{nikolic14} proposed approach based on BCO metaheuristic. They considered the network design problem in a way that the algorithm decided the links that was included in the transit network, and further creating designed bus routes based on the links. They tested two orders of importance of the objective functions (an order that is best for passengers and an order that is best for the transit operator). The algorithm was tested on Mandl's benchmark problem of a Swiss bus network\citep{mandl80}. This is probably the only widely investigated and accepted benchmark problem in the relevant literature\citep{kechagiopoulos14}. Their algorithm was compared to other competitive approaches (\citet{mandl80}; \citet{shih94}; \citet{baaj95}; \citet{bagloee11}), and the results may be considered as ambiguous. The algorithm proposed by \citet{nikolic14} performed best regarding travel time and number for transfers, but was sometimes outperformed regarding in-vehicle time and out-of-vehicle time. 

%Not changed?
\citet{kechagiopoulos14} designed and presented an optimization algorithm based on PSO. Their goal was to find an efficient solution to the urban transit routing problem (UTRP), which is a NP-hard problem that deals with the construction of route networks for public transportation. The target problem was Mandl's benchmark problem. Their algorithm is compared with competitive approaches (including genetic algorithms and other metaheuristic approaches) mentioned in literature (\citet{baaj91}; \citet{chakroborty02}; \citet{kidwai98}; \citet{fan10}; \citet{fan09-2}; \citet{zhang10}; \citet{chew12}). The algorithms are compared with regard to the percentage of total transfer demands satisfied directly ($d_0$), with one transfer ($d_1$), two transfers ($d_2$), or with more than two transfers or not satisfied at all ($d_{unsat}$). The algorithms are also compared regarding average in-vehicle travel time ($ATT$). The experiments are conducted on route set designs with four, six, seven and eight routes. The proposed algorithm performs better than the competitors regarding $ATT$ independent the route size, and achieves a better percentage of direct travelers ($d_0$) except from when the route size is four. The percentage of passengers with more than two transfers or not satisfied at all ($d_{unsat}$), is $0.00$ for all the described algorithms independent of route size.

\textbf{Sammenlikne resultater fra changed og ikke changed algorithms og se om det er lurt å gjøre forandringer?}
\subsubsection{Mandl's Benchmark Problem}
\subsection{Mandls Benchmark Problem}
%TODO: kjøpe/leie Mandl boka
%TODO: skrive om 
%Since there is not done any op
In the current research on the VRPs and SI, there are some benchmark data that are available for researchers to use for comparison.... However (as mentioned in RW), for the UTNDP problem, Mandls network seems to be the only benchmark instance used by researchers. The road network is a real Swiss road network comprising of 15 nodes and 21 connections between them. And has been thoroughly examined by many optimization approaches such as \citep{kechagiopoulos14}, \citep{fan09}, \citep{nikolic14}. %This makes it easy to measure the performance of your solution. 

%TODO: skrive om 
Christoph Mandl concentrated mainly on the UTRP, and developed a solution in two stages. First, a feasible set of

Christoph Mandl[] approaches the UTNDP problem in a generic form. Mandl concentrated on the UTRP, and developed a solution in two stages. First, a feasible set of routes was generated, and then heuristics were applied to improve the quality of the initial route set / tour plan. The route generation phase involved computing the shortest path between all sets of vertices's using Dijstras[], and then producing the route set with the shortest paths that contained the most nodes, respecting the positions of any nodes selected as terminals. Nodes not selected, were iteratively included into routes, or new routes were created with unserved nodes as route terminals. In the first phase, Mandl only considered in-vehicle travel costs when accessing route quality. He then suggested an initial route set, and used these in the second phase. The second phase included obtaining the new routes at an intersection node, including a node that was close to a route (if travel demand between this node and the node on the route was high), and / or excluded a node from a route set that was already served by another rote (if the travel demand between this node and the other nodes in the route was low). Waiting costs were also considered in the second phase, in addition to in-vehicle travel costs. Waiting times were fixed as constant values, according to specified vehicle frequencies. 

%IFylles inn i results: in order to demonstrate the efficiency and the effectiveness of our algorithm... the Swiss road network introduced by Mandl is used. This road network has been already used as a benchmark problem by many researchers in the literature. \citep{kechagiopoulos14}



Based on the described literature we can conclude that swarm intelligence-algorithms shows great promise in solving different vehicle routing problems. We specifically notice that SI-algorithms performs equal and sometimes better than genetic algorithms and  metaheurstic approaches like Simulated Annealing. We also notice that 2 out of the 12 described papers uses Mandl's benchmark problem as a test case for their proposed algorithm, and that 2 out of 12 papers compare their algorithm to \citet{stutzle99}s MAX-MIN ant system (MMAS). It is worth mentioning that none of the proposed algorithms performs optimally in every situation and that there is no consensus in which algorithm is the overall best. It is therefore difficult to draw a singular conclusion to what is the state of the art regarding swarm intelligence and vehicle routing problems. We suggest that a general state of the art regarding the algorithm may be described as being inspired by the way different swarms act in nature, and to further add artificial features to the individuals to optimize for certain objectives. Mandl's benchmark problem was used by 2 out of the 4 papers in our literature review published in 2014, and we therefore believe that using this benchmark problem as a test case may be considered as the state of the art.

%Changes done: forklare endringene