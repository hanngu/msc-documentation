\section{Vehicle Routing Problem (VRP)}
The Vehicle Routing Problem (VRP) was first introduced by \citet{dantzig59} and is a generic name given to a broad class of optimization problems. It can be described as the task of designing the optimal set of routes for a fleet of vehicles, in order to serve a given set of customers. The problem involves making deliveries to a set of customers with known demands on routes originating and terminating at one or more depots, and the objective is to minimize the total route cost. The VRP is NP-hard and many variations exists:
\begin{itemize}
\item The capacitated vehicle routing problem is like VRP with the additional constraint that every vehicles must have uniform capacity of a single commodity.
\item Multiple Depot VRP is when the vendor uses many depots to supply the customers.
\item Periodic VRP is when the classical VRP is generalized by extending the planning period to M days.
\item Split Delivery VRP is a relaxation of the VRP wherein it is allowed that the same customer can be served by different vehicles if it reduces overall costs. 
\item Stochastic VRP are VRPs where one or several components of the problem are random.
\item VRP with Backhauls is a VRP in which customers can demand or return some commodities.
\item VRP with Pick-Up and Delivering is a VRP in which the possibility that customers return some commodities is contemplated.
\item VRP with Satellite Facilities is the use of satellite facilities to replenish vehicles during a route.
\item VRP with Time Windows is the same problem that VRP with the additional restriction that in VRPTW a time window is associated with each customer
\end{itemize}
\citep{website:neo}

\subsection{Urban Transit Network Design Problem (UTNDP)}

%TODO: Skrive mer detalj om UTRP og UTSP side 6-10 i fan09

The urban transit network problem(UTNDP) is an example of the broader optimization problem VRP, and is the problem of designing urban transit routes and schedules. The two major components of the UTNDP are the urban transit routing problem(UTRP) and the urban transit scheduling problem (UTSP). \citep{fan09}. UTRP involves the development of efficient transit routes on an existing transit network, with predefined pick-up/drop-off point (e.g bus routes), and UTSP is assigning the schedules for the passengers carrying vehicles. In practice, the two phases are usually implemented sequentially, with the routes determined in advance of the schedules. 

The urban transit network design problem is an underclass of the vehicle routing problem, and is the problem of designing urban transit routes and schedules which deal with practical constraints.

UTNDP is concerned with the determination of a set of routes with corresponding schedules for an urban public transport system. Its two main components are the Urban Transit Routing Problem (UTRP) and the Urban Transit Scheduling Problem(UTSP). 
\begin{itemize}
\item Urban Transit Routing Problem: development of efficient and effective transit routes taking into account existing road networks and predefined pick-up / drop off points. (UTRP aims to develop a set of vehicle routes for an existing urban transit network while satisfying specific constraints. )
\item Urban Transit Scheduling Problem: Involves the assignment of schedules to all vehicles that are used to carry passengers on a road network. 
\end{itemize}
UTRP and UTSP are solved sequentially because the development of routes should be completed before the development of schedules starts.
\citep{kechagiopoulos14}



