\section{Problem Statement} 

We note that in the related work there has not been done any research that aims to combine attributes from different swarm intelligence methods in order to improve the optimization process. For this reason, we will in this thesis investigate the possibility to combine some of the best attributes from the discussed swarm intelligence algorithms in order to achieve better results regarding route optimization. 
\begin{itemize}
\item Is if efficient to combine attributes from different swarm intelligence methods in order to improve the computational results?
\end{itemize}

We are highly motivated in the development of an ant colony inspired optimization algorithm for the urban transit network design problem (UTNDP). For this is travel demand a key variable, because accurate estimates of travel demand is an important factor for the algorithm. AtB does not possess accurate data about the travel demand, and detailed investigations into measuring and predicting travel demand is an complex research problem, and beyond the scope of this thesis. 

In the current research on the vehicle routing problems and swarm intelligence, there are some benchmark data available for researchers to use for comparison. However, for the urban transit network design problem (UTNDP), Mandl's network seems to be the only benchmark instance used and acknowledged by researchers, and his work has been widely cited.%Referanser , including \citep{fan09}, \citep{kechagiopoulos14}, \citep{nikolic14}. 

Christoph Mandl \citep{mandl79} developed a heuristic algorithm for the urban transit network problem. This method was applied and based on a real network in Switzerland, the Swiss transit network\citep{mandl80}. The data includes a small and dense network of 15 nodes and 21 edges, in addition to the travel times and travel demand for each edge. The total demand is 15570 trips per day, which is a relatively high demand for a small network. The travel time between the to farthest nodes in the network is 22 minutes along the shortest path. Mandl developed a solution in two phases, where a feasible set of routes were created in the first phase and in the second phase he tried to minimize the the total travel time, including in-vehicle time and waiting time, by reducing the number of transfers. Some important measures of Mandl's solution network includes, for example, 100\% service coverage, 69.4\% of the trips involving no transfers, 29.93\% of the trips involving one transfer, and only 0.13\% of the trips needing more than one transfer.

Demand values are all provided for Mandl's instance, we will use Mandl's benchmark problem \citep{mandl79} as the input data. 
As mentioned are these benchmark problems real, practical problems, and with some simplification assumptions, will we make them appropriate to use to validate the methodology developed in this study, demonstrate efficiency and effectiveness of our algorithm, and evaluate and compare our results. 

Through our structure literature review, we did not manage to find any previous research that used graph databases in combination with the vehicle routing problem and swarm intelligence. For this reason, and because we believe the graph database Neo4j \citep{website:neo4j} has several benefits we can take advantage of in solving our problem, we will investigate the possibilities, and potential advantages and usefulness, Neo4J will have for our algorithm and the optimization process.
\begin{itemize}
\item What are the possibilities and potential advantages of using a graph database in our implementation?
\end{itemize}

We will focus our research on making the routes more convenient both concerning customers and operators. The objectives we want to focus on is satisfying the customers and keeping the operator costs at minimum. 

The current solution of AtB \citep{website:atb} consist of, as mentioned, an experience based route network, meaning the routes has not been properly, computationally optimized concerning the travel demand and travel time. When a route network is not properly optimized, it can lead to a large number of transfers for customers when they are traveling from a to b, resulting in a long total travel time. A minimum number of transfers and a minimum in-vehicle time are important factors for determining the customers satisfaction. The objective in this thesis will therefore be to minimize the total travel time in the public transportation system, by decreasing the number of transfers. %TODO: Denne må omformuleres

The algorithm will be evaluated by the following evaluation criteria, which is inspired by \citep{kechagiopoulos14}, \citep{mandl80}, \citep{nikolic14} and \citep{fan09}. TODO: remove this section?
\begin{itemize}
\item Total travel time (should be as low as possible)
\item Number of transfers (should be as low as possible)
\begin{itemize}
\item Number of direct travelers
\item Travelers with one transfer
\item Travelers with two transfers
\end{itemize}
\item Number of unsatisfied customers - more than two transfers (should be zero)
\end{itemize}

 %\citep{mandl79}
Our goal is to develop an ant colony inspired algorithm that creates a general solution with effective urban transit routes and schedules, and that this in the future can be used to optimize AtBs transit network in order to increase the number of public transportation passengers. 
\begin{itemize}
\item Is it possible to apply our algorithm to optimize bus routes in urban cities?
\end{itemize}
