\subsection{Mandls Benchmark Problem}
%TODO: kjøpe/leie Mandl boka
%TODO: skrive om 
%Since there is not done any op
In the current research on the VRPs and SI, there are some benchmark data that are available for researchers to use for comparison.... However (as mentioned in RW), for the UTNDP problem, Mandls network seems to be the only benchmark instance used by researchers. The road network is a real Swiss road network comprising of 15 nodes and 21 connections between them. And has been thoroughly examined by many optimization approaches such as \citep{kechagiopoulos14}, \citep{fan09}, \citep{nikolic14}. %This makes it easy to measure the performance of your solution. 

%TODO: skrive om 
Christoph Mandl concentrated mainly on the UTRP, and developed a solution in two stages. First, a feasible set of

Christoph Mandl[] approaches the UTNDP problem in a generic form. Mandl concentrated on the UTRP, and developed a solution in two stages. First, a feasible set of routes was generated, and then heuristics were applied to improve the quality of the initial route set / tour plan. The route generation phase involved computing the shortest path between all sets of vertices's using Dijstras[], and then producing the route set with the shortest paths that contained the most nodes, respecting the positions of any nodes selected as terminals. Nodes not selected, were iteratively included into routes, or new routes were created with unserved nodes as route terminals. In the first phase, Mandl only considered in-vehicle travel costs when accessing route quality. He then suggested an initial route set, and used these in the second phase. The second phase included obtaining the new routes at an intersection node, including a node that was close to a route (if travel demand between this node and the node on the route was high), and / or excluded a node from a route set that was already served by another rote (if the travel demand between this node and the other nodes in the route was low). Waiting costs were also considered in the second phase, in addition to in-vehicle travel costs. Waiting times were fixed as constant values, according to specified vehicle frequencies. 

%IFylles inn i results: in order to demonstrate the efficiency and the effectiveness of our algorithm... the Swiss road network introduced by Mandl is used. This road network has been already used as a benchmark problem by many researchers in the literature. \citep{kechagiopoulos14}

