\subsection{Mandl's Benchmark Problem}
%TODO: skrive om 

In the current research on the vehicle routing problems and swarm intelligence, there are some benchmark data available for researchers to use for comparison[referanser]. However, for the urban transit routing problem, Mandl's network seems to be the only benchmark instance used and acknowledged by researchers, and his work has also widely cited by researchers.  
 
\citet{mandl79} developed a heuristic algorithm for the urban transit network problem. This method was applied and based on a real network in Switzerland, the Swiss road network. The data includes a small and dense network of 15 nodes and 21 edges, in addition to the travel times and travel demand for each edge. The total demand is 15,570 trips per day, which is a relatively high demand for a small network. The travel time between the to farthest nodes in the network is 22 minutes along the shortest path. 

Mandl developed a solution in two phases. 
%&(His algorithm had certain advantages, as it proceeded to find a feasible set of lines and then iteratively changing whil reducing the total transportation time in each step. Therefore the algorithm produces a number of feasible set compared by the transportation planner.)

\begin{enumerate}
\item Firstly, a feasible set of routes was created. 
\begin{enumerate}
\item Generating routes; finding the shortest paths between all the connected nodes, to form a possible solution set. Mandl used Dijkstras algorithm to first find the shortest distances and routes between all pairs of nodes.  
\item For each transit node pairs, the shortest path from all possible shortest paths, with the highest number of nodes, is selected, following heuristic guidelines. 
\item Nodes not selected, were iteratively included into appropriate existing routes, or new routes were created with the unserved nodes as route terminals. 
In this first stage, only the in-vehicle travel cost were considered when accessing route quality. He then suggested an initial route set, by connecting selected transit node or terminal pairs following heuristic guidelines, and used these in the second stage. 
\end{enumerate}
\item The second stage tries to minimize an objective function of total travel time, including in-vehicle time and waiting time. This stage included obtaining new routes at an intersection route, including a node that was close to a route (if the travel demand was high), and excluded a node from a route set that was already served by another route (if travel demand was low). phase included obtaining the new routes at an intersection node, including a node that was close to a route (if travel demand between this node and the node on the route was high), and excluded a node from a route set that was already served by another rote (if the travel demand between this node and the other nodes in the route was low). Waiting times were fixed as constant values, according to specific vehicle frequencies.  
\end{enumerate}

Some important measures of Mandl's solution network includes, for example, 100\% service coverage, 69,4\% of the trips involving no transfers, 29,93\% of the trips involving one transfer, and only 0,13\% of the trips needing more than one transfer. The limitation of Mandl's method is the lack of consideration of the demand pattern during the route construction stage.\citep{zhao03}


%Traffic assignment: Finding number of people traveling along a specific line or arc. As mentioned, minimizing changes and some behave according to a weighted sum of transportation and waiting time. The flow of each arc is computes as the sum of all people usng the arc by one og the two paths. The flow on each arc og each line is not completely defined, because if more than one line proceeds along the same arcs, people might use both of them along these arcs, thus the flow assignment to lines is not unique. 
%Route planning: The problem can be formulated as follows: Transportation network: N = (X,A), where X represents stops that have to be served and A represents streets that can be used.
%Traffic assigments, assuming that the now how many will use a certain arc (total flow between each pair of nodes is called the traffic assignment provlem. and supplies the planner of a rad network with mportant information about the flow density. Kleins alg for minimum cost flow. )


%One of the main handicaps of urban transportation systems is estimating t is rather expensive and therefore not not frequently. In reality the demand is different each time of day and to deal with this problem average demand has to be used or the max demand. It is no use finding optimal sets of lines for different hours of the day, because the organization problems would become enormous and also no assenger would be interested in hacing differnt lines at different times. does not deal with varying travel and waiting times, these also change during the day. transportation times are supposed to be avreage transportation times. 




