%TODO: skrive om 

In the current research on the vehicle routing problems and swarm intelligence, there are some benchmark data available for researchers to use for comparison. However, for the urban transit network design problem (UTNDP), Mandl's network seems to be the only benchmark instance used and acknowledged by researchers, and his work has been widely cited, including \citep{fan09}, \citep{kechagiopoulos14}, \citep{nikolic14}. %Referanser
\par
Christoph Mandl \citep{mandl79} developed a heuristic algorithm for the urban transit network problem. This method was applied and based on a real network in Switzerland, the Swiss transit network\citep{mandl80}. The data includes a small and dense network of 15 nodes and 21 edges, in addition to the travel times and travel demand for each edge. The total demand is 15,570 trips per day, which is a relatively high demand for a small network. The travel time between the to farthest nodes in the network is 22 minutes along the shortest path. Mandl developed a solution in two phases, where a feasible set of routes were created in the first phase and in the second phase he tried to minimize the the total travel time, including in-vehicle time and waiting time, by reducing the number of transfers. Some important measures of Mandl's solution network includes, for example, 100\% service coverage, 69.4\% of the trips involving no transfers, 29.93\% of the trips involving one transfer, and only 0.13\% of the trips needing more than one transfer. %The limitation of Mandl's method is the lack of consideration of the demand pattern during the route construction stage \citep{zhao03}.
\par
\citet{kechagiopoulos14}'s target problem was Mandl's benchmark problem, and they designed and presented an optimization algorithm based on PSO. Their goal was to find an efficient solution to the urban transit routing problem (UTRP), which is a NP-hard problem that deals with the construction of route networks for public transportation. Their algorithm is compared with competitive approaches (including genetic algorithms and other metaheuristic approaches) mentioned in literature (\citet{baaj91}; \citet{chakroborty02}; \citet{kidwai98}; \citet{fan10}; \citet{fan09-2}; \citet{zhang10}; \citet{chew12}). The algorithms are compared with regard to the percentage of total transfer demands satisfied directly ($d_0$), with one transfer ($d_1$), two transfers ($d_2$), or with more than two transfers or not satisfied at all ($d_{unsat}$). The algorithms are also compared regarding average in-vehicle travel time ($ATT$). The experiments are conducted on route set designs with four, six, seven and eight routes. The proposed algorithm performs better than the competitors regarding $ATT$ independent the route size, and achieves a better percentage of direct travelers ($d_0$) except from when the route size is four. The percentage of passengers with more than two transfers or not satisfied at all ($d_{unsat}$), is $0.00$ for all the described algorithms independent of route size.\par
\citet{nikolic14} proposed approach based on BCO metaheuristic, and the algorithm was also tested on Mandl's benchmark problem. They considered the network design problem in a way that the algorithm decided the links that was included in the transit network, and further creating designed bus routes based on the links. They tested two orders of importance of the objective functions (an order that is best for passengers and an order that is best for the transit operator). Their algorithm was compared to other competitive approaches (\citet{mandl80}; \citet{shih94}; \citet{baaj95}; \citet{bagloee11}), and the results may be considered as ambiguous. The algorithm proposed by \citet{nikolic14} performed best regarding travel time and number for transfers, but was sometimes outperformed regarding in-vehicle time and out-of-vehicle time. 






