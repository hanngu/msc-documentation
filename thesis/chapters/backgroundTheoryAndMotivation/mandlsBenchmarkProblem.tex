\subsection{Mandl's Benchmark Problem}
%TODO: kjøpe/leie Mandl boka
%TODO: skrive om 
In the current research on the vehicle routing problems and swarm intelligence, there are some benchmark data available for researchers to use for comparison[referanser]. However, for the urban transit routing problem, Mandl's network seems to be the only benchmark instance used and acknowledged by researchers, and his work has also widely cited by researchers.  
 
Christoph Mandl developed a heuristic algorithm for the urban transit network problem. This method was applied and based on a real network in Switzerland, the Swiss road network[referanse]. The data includes a small and dense network of 15 nodes and 21 edges, in addition to the travel times and travel demand for each edge. The total demand is 15,570 trips per day, which is a relatively high demand for a small network. The travel time between the to farthest nodes in the network is 22 minutes along the shortest path. 

Mandl developed a solution in two stages / phases. 
\begin{enumerate}
\item First, a feasible initial route network / set of routes was generated. The solution procedures:
\begin{enumerate}
 \item Route generation phase: find the shortest path between all connected transit nodes, using Dijkstras, to form a possible solution set. 
 \item For each transit node pairs, the shortest path from all possible shortest paths, with the highest number of nodes, is selected, following heuristic guidelines. 

 \item Checking service coverage, nodes not selected, were iteratively included into appropriate existing routes, or new routes were created with the unserved nodes as route terminals. 
In this first stage, only the in-vehicle travel cost were considered when accessing route quality. He then suggested an initial route set, by connecting selected transit node or terminal pairs following heuristic guidelines, and used these in the second stage. 
 \end{enumerate}
\item The second stage tries to minimize an objective function of total travel time, including in-vehicle time and waiting time. This stage included obtaining new routes at an intersection route, including a node that was close to a route (if the travel demand was high), and excluded a node from a route set that was already served by another route (if travel demand was low). phase included obtaining the new routes at an intersection node, including a node that was close to a route (if travel demand between this node and the node on the route was high), and excluded a node from a route set that was already served by another rote (if the travel demand between this node and the other nodes in the route was low). Waiting times were fixed as constant values, according to specific vehicle frequencies.  The 
\end{enumerate}

Some important measures of Mandl's solution network includes, for example, 100\% service coverage, 69,4\% of the trips involving no transfers, 29,93\% of the trips involving one transfer, and only 0,13\% of the trips needing more than one transfer. The limitation of Mandl's method is the lack of consideration of the demand pattern during the route construction stage.\citep{zhao03}

\subsubsection {Input Data}
\begin{itemize}
\item \textbf{MandlCoords.txt} - This file includes 15 lines, with the (x,y) coordinated of the 15 nodes. These coordinated was not supplied in Mandl's literature, so these are copied from \citet{fan09} and are approximate for the picture to be drawn.

\begingroup
\obeyspaces\obeylines
\input{assets/instances/MandlCoords.txt}%
\endgroup%

\item \textbf{MandlTravelTimes.txt} - The travel times matrix gives the travel times in takes in minutes between the nodes. This matrix is symmetrical, travel times between each node and itself are zero, and ``Inf'' indicates that there is no direct link between the nodes. 

\begingroup
\obeyspaces\obeylines
\input{assets/Instances/MandlTravelTimes.txt}%
\endgroup%

\item \textbf{MandlDemand.txt} - The demand matrix shows the travel demand between each node pair, which is the average number of passenger trips per day.. This matrix is also symmetrical. (These in no demand either to or from node 15, but it is always included in the route sets because the rules that are used insist on every node being included(?)) For coding reasons, we changed the numbers in the main diagonal (from to left corner to bottom right corner) from all zero to all ’a’s. 

\begingroup
\obeyspaces\obeylines
\input{assets/Instances/MandlDemand.txt}%
\endgroup%

\end{itemize}

