\section{Graph Database}


A graph database is a database based on graph theory. Graphs are one of the most efficient and natural way of working with data. Graph databases consists of nodes, edges, and properties to represent and store data. Nodes represent entities, such as people, accounts and bus stops, properties are relevant information that relate to the nodes, and edges are the lines that connect nodes to other nodes, and they represent the relationship between the node couple. Most of the information is stored in the edges, for instance the travel time or travel demand between to nodes. It and uses graph structures for semantic queries. It is a storage system where every element contains a direct pointer to its neighbor elements and no index lookups are necessary. 

\par %Skrives om litt
Compared to relational databases, are graph databases often faster for associative data sets[Wikipedia:P], and map more directly to the structure of object-oriented applications. They can scale more naturally to large data sets as they do not typically require extensive join operations. As they depend less on a inflexible schema, they are more suitable to manage changing data with evolving schema's. 

\par
Graph databases are a powerful tool for graph-like queries, for instance computing the shortest path between to nodes in the graph!!! Other graph-like queries can be performed over a graph database in a natural way (for example graph's diameter computations or community detection).

\par

\textit{Denne er foreløpig kopiert fra Wikipedia: }%Skrives om 
Compared with relational databases, graph databases are often faster for associative data sets[citation needed], and map more directly to the structure of object-oriented applications. They can scale more naturally to large data sets as they do not typically require expensive join operations. As they depend less on a rigid schema, they are more suitable to manage ad hoc and changing data with evolving schema's. Conversely, relational databases are typically faster at performing the same operation on large numbers of data elements.
Graph databases are a powerful tool for graph-like queries, for example computing the shortest path between two nodes in the graph. Other graph-like queries can be performed over a graph database in a natural way (for example graph's diameter computations or community detection).

\subsection{Neo4j}
Neo4J \citep{website:neo4j} is a native graph database, and is a success way to store a transport graph. It is known for extremely fast traversals of relationships. Neo4J can be easily constructed to a transit network. It is fast finding shortest path (by Dijkstra's algorithm), and it is easy to change the database via the REST API. 
Neo4J is a highly scalable open source graph database that supports ACID, has high-availability clustering for enterprise deployments, and comes with a web-based administration tool that includes full transaction support and visual node-link graph explorer. Neo4j is accessible from most programming languages using its built-in REST web API interface. Neo4j is the most popular graph database in use today.[10]

Advantages:
\begin{itemize} 
\item Flexibility: model, develop and visualize the world as you experience it. Its simply nodes and relationships. 
\item Performance: Hyper-connectivity at speed. 
\item Scalability: Scales up and out, supporting tens of billions of nodes and relationships, and hundred of thousands of ACID transactions per seconds. 
\item Speed: Able to search trough millions of connections per second, with real time queries that stay fast even as your database grows. 
\end{itemize}


%\begin{itemize}
%\item Native graph database. 
%\item Property graph. 
%\item Made for real time queries. 
%\item Really fast traversals of relations.
%\item Neo4J has an API that supports traversing - finding shortest path - can weight edges, nodes -
%\end{itemize}
%//en korteste vei mellom n og m men max lengde 3
%match p = shortestPath ( (n) - [*...3]--(m)) return p
%(Vekte kanter: Neo44J har et API som støtter traversering, Dijkstras er innebyd ++ som lar deg vekte kanter - hva er raskeste vei ++)

%Neo4J can be used to evaluate routes after the ants have created route sets.