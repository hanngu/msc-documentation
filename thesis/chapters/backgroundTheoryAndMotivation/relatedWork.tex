\subsection{Related work}

To gather related work we conducted a structured literature review. This process is described in the Experiments and Results-chapter[\ref{experimentsAndResults}], and is entirely explained in the appendix[\ref{appendixA}][\ref{appendixB}]. The literature in this section is the result of the final data synthesis. \newline

Swarm intelligence algorithms has proven to be useful in many vehicle routing problems, and many studies have been done to improve the quality of such problems. One of this thesis research questions is to establish what the state of the art are for solving vehicle routing problems using swarm intelligence inspired methods. \citet{dorigo97} and \citet{lucic03} showed that, using respectively an ant colony system and a bee systems, swarm intelligence can solve highly complex problems such as the Traveling Salesman Problem (TSP), which is a subproblem of VRP. 

\citet{hsiao04} presented an optima approach to search for the best path of a map considering the traffic loading conditions. To do this, they proposed an algorithm based on ACO techniques to search for the shortest path from a desired original to a desired destination. They random generated a map consisting of 100-500 nodes, and compared their algorithm to a brute method emphasizing on the time used to generate the route. Their results states that if the map consists of $>200$ nodes, the ACO performs better than a brute method. In fact, their results shows that the more nodes the map contains, the higher is the benefit of using the ACO algorithm. 

\citet{yang07} presented an optimization model for a bus network design based on the coarse-grain parallel ant colony algorithm (CPACA). CPACA is an optimization algorithm the develops a strategy to update the increased pheromone, called Ant Weight, where the path searching activities of the ants are adjusted based on the objective function. The model aims to minimize the average trip time by maximizing the number of direct travelers per unit length. Their results are compared with the classical MAX-MIN ant system (MMAS)\citep{stutzle99} and with ACA+ (ACA with Ant-weight strategy). The comparison shows that CPACA performs better regarding both average direct traveler density and run time. They conclude that future research will be improving the searching efficiency, because their simulation shows that their algorithms is more stable and the runtime is more satisfactory when the number of nodes are $<1000$. 

\citet{salehi-nezhad07} presented an algorithm to search for the best direction between two desired origin and destination intersections in cities, called Ant-based Vehicle Navigation algorithm. The algorithm was applied on a part of the city of \textit{Kerman}, and the results are encouraging. The algorithm provides a fast access, low cost and easy method for vehicle navigation in cities without assisting GPS.

\citet{tripathi09} solved a vehicle routing problem with stochastic demand, in which the customer demand has been modeled as a stochastic variable. They performed this using an improved version of the ACO approach, called ns-AAA SO, which oriented the search progressively towards favoring the global optimal solution. The characteristics and search capability of ns-AAA SO was compared with both a standard ACO algorithm and a genetic algorithm. The results shows that ns-AAA SO outperforms the other two algorithms in every problem instance described by the authors. 

\citet{salehinejad10} introduced a route selection system which uses an ant colony system to detect an optimum multiparameter direction between two desired points in urban areas. The system employs fuzzy logic for local pheromone updating, and the proposed algorithm is called Fuzzy Logic-Ant Colony System (FLACS). The algorithm is applied to a part of London, United Kingdom, consisting of 42 junctions (nodes). The FLACS algorithm is compared to a standard ACS-algorithm and a $A^*$-ACS-algorithm emphasizing on the parameters ``distance'', ``traffic'' and ``incident risk'', which is said to be important to travelers. The presented results are the average of 10 randomly selected O/D pairs, and their result graphs states that FLACS performs better at average than both the standard ACS and the $A^*$-ACS regarding operational cost. The performance is independent of the importance rate of the parameters mentioned above. It is, however, worth mentioning that the estimation of further traffic data is done by ANNs, and therefor the traffic data used for each algorithm is not exactly the same. FLACS has less running time than $A^*$-ACS, but more than the standard ACS due to its Fuzzy Logic system component. 

\citet{jiang10} describes an improved ACO (IACO) algorithm to solve the Urban Transit Network Optimization which is described a typical nonlinear combinatorial optimization problem. Improvements to the algorithm included a stagnation counter to determine if an ant had stagnated and extra pheromone intensity to newly discovered paths. This is done to compensate for the classical ACOs shortcomings of easily falling into stagnation and therefor obtain a local optimal solution. The IACO algorithm is, like the algorithm described by \citet{yang07}, compared with the classical MMAS algorithm. The results shows significant improvement to the convergence speed compared to MMAS. The average number of iterations to find the optimal solution was reduced from 1060 using MMAS to 548 using IACO. The average path distance was shortened from 15554.74 using MMAS to 15509.02 using IACO.  

\citet{poorzahedy11} proposed an Ant System application for solving the bus network design problem. Among the characteristics of their algorithm is that it works at a decision graph, rather than on the bus network it self. The algorithm cares for only one objective; a combination of travel time for the users and bus fleet size for the operator. The application was applied to design the bus network of the City of Mashhad with over 2 million in population, and compared with a genetic algorithm (GA). Their results shows that their algorithm performs better than the GA in both number of routes, fleet size, in-vehicle travel time and waiting time. GA performs better the Ant System algorithm in the users walking time. 

\citet{dias14} introduced an inverted ACO (IACO) algorithm. The idea is that the IACO algorithm inverts the classical ACO logic by converting the attraction of ants towards pheromones into a repulsion effect. The IACO is then used in a decentralized traffic management system, where the drivers acts as the inverted ants. The drivers are repelled by the scent of pheromones (other drivers), and thus the system avoids congested roads. The IACO algorithm described was compared to a shortest-time algorithm (ST). The IACO algorithm performs better than the ST algorithm, with the respect to trip time, travel length, fuel consumption and $CO_2$ emissions, as long as a considerable amount (25-50\%, depending on whether it was tested on a radial and ring network or a lattice network) of the vehicles uses an algorithm to decide which road to choose. 

\citet{nikolic14} proposed a model for the transit network design problem. To do this they used an approach based on BCO metaheuristic. They considered the network design problem in a way that the algorithm decided the links that was included in the transit network, and further creating designed bus routes based on the links. They tested two order of importance of the objective functions (order that is best for passengers and order that is best for the transit operator). The algorithm was tested on Mandl's benchmark problem of a Swiss bus network\citep{mandl80}. This is probably the only widely investigated and accepted benchmark problem in the relevant literature\citep{kechagiopoulos14}. Their algorithm was compared to other competitive approaches (\citet{mandl80}; \citet{shih94}; \citet{baaj95}; \citet{bagloee11}), and the results may be considered as ambiguous. The algorithm proposed by \citet{nikolic14} performed best regarding travel time and number for transfers, but was sometimes outperformed regarding in-vehicle time and out-of-vehicle time. 

\citet{sedighpour14} introduced a hybrid ACO (HACO) algorithm where they used a new state transition rule and a candidate list, as well as several local search techniques and a new pheromone update rule. The hybrid was designed to overcome some of the original ACOs shortcomings, such as slow computing speed and local convergence. The HACO algorithm was applied to the open vehicle routing problem, a variant of the vehicle routing problem, in which vehicles are not required to return to the depot after completing a service. The HACO algorithm was compared with three versions of PSO (standard PSO, PSO without one-point move (PSOWO) and PSO without neighbors (PSOWN). The algorithms were tested on fifteen different sets consisting of 19 to 72 nodes with 2 to 7 vehicles fixed at the minimum possible. Their result table showed that HACO found the optimal solution in 14 out of 15 set compared to PSO with 13/15, PSOWO with 0/15 and PSOWN with 9/15. In the one set HACO did not find an optimal solution it still performed better than the other algorithms. As mentioned above, HACO was designed to overcome some of the original ACOs shortcomings and we therefor believe it is worth mentioning that the algorithm was not in fact compared to an original ACO algorithm. 

\citet{kechagiopoulos14} designed and presented an optimization algorithm based on PSO. Their goal was to find an efficient solution to the urban transit routing problem (UTRP), which is a NP-hard problem that deals with the construction of route networks for public transportation. The target problem was Mandl's benchmark problem. Their algorithm is compared with competitive approaches (including genetic algorithms and other metaheuristic approaches) mentioned in literature (\citet{baaj91}; \citet{chakroborty02}; \citet{kidwai98}; \citet{fan10}; \citet{fan09-2}; \citet{zhang10}; \citet{chew12}). The algorithms are compared regarding the percentage of total transfer demands satisfied directly ($d_0$), with one transfer ($d_1$), two transfers ($d_2$), or with more than two transfers or not satisfied at all ($d_{unsat}$). The algorithms are also compared regarding average in-vehicle travel time ($ATT$). The experiments are conducted on route set designs with four, six, seven and eight routes. The proposed algorithm performs better than the competitors regarding $ATT$ independent the route size, and achieves a better percentage of direct travelers ($d_0$) except from when the route size is four. The percentage of passengers with more than two transfers or not satisfied at all ($d_{unsat}$), is $0.00$ for all the described algorithms independent of route size. \newline

Based on the described literature we conclude that swarm intelligence-algorithms shows great promise in solving different vehicle routing problems. We specifically notice that SI-algorithms performs equal and sometimes better than genetic algorithms. We also notice that 2 out of the 12 described papers uses Mandl's benchmark problem as a test case for their proposed algorithm. It is worth mentioning that none of the proposed algorithms performs optimally in every situation and that there is no consensus in which algorithm is the overall preferred. 

%Skrive om at det "To the best of our knowledge" ikke eksisterer en supersverm algoritme. 
