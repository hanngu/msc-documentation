\subsection{Performance Comparison}
\label{subsec:performanceComparison_setup}
%We will in this stage test the system's performance, comparing the results with the results published by xxx, and with the basic ACO implementation. 

The value for each parameters is determined in the parameter settings experiments, and can be seen in Table \vref{table:parameterSettings2}. For the ACO comparison, will the value of parameter $AF$, and $CA$ be 0\% and $p_b$ will be 0.0. The memory feature and notion of the global best solution will also be removed from the ants.

The performance criteria used to evaluate the effectiveness of the algorithm are the following:
\begin{itemize}
\item $d_0 (\%)$ - the percentage of demand satisfied without any transfers. 
\item $d_1 (\%)$ - percentage of total transfer demands where the number of transfers are 1. 
\item $d_2 (\%)$ - percentage of total transfer demands where the number of transfers are 2. 
\item $d_{unsat}$ (\%) - the percentage of unsatisfied travelers. An unsatisfied traveler is described as a traveler with 3 or more transfers. 
\item $ATT$  - the average travel time in minutes per transit user (mpu). In all approaches published in the literature, the transfer penalty of 5 min is applied to each route for each required transfer, and the same transfer penalty will be used by the proposed SSO algorithm.
%$ATT$ is calculated as follows:
%$$ATT = \frac{\sum\limits^{p}_{p=1}TT}{p_{size}}$$
%where $p_{size}$ is the number of passengers and $TT$ is the travel time used by passenger $p$. 
\end{itemize}
The criteria for good performance includes the percentage of demand satisfied without any transfers is high, and that the average travel time is as low.

There will be carried out 50 runs per experiment, recording the algorithm's average, best and worst produced result. The median solution, standard deviation, and margin of error (confidence interval) will also be presented. 

In order for our experimental results to be comparable with the results published by other researchers will four, six, seven and eight routes in the route set be examined. The execution time (in seconds) of the proposed algorithm with each route set will be presented.

