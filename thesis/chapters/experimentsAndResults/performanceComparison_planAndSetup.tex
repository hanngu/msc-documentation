\section{Performance Comparison}

\subsection{Experimental Plan}
When the optimal parameters are found will the comparison studies determine how the proposed algorithm performs compared to the methods published by \citet{mandl79}, \citet{kechagiopoulos14}, \citet{nikolic14}, \citet{kidwai98}, \citet{fan10}, \citet{chakroborty02}, \citet{zhang10}, \citet{chew12}, and \citet{baaj91}. This will help us answer Research Question \vref{itm:2b1}, which is concerned about how the results of proposed algorithm compare to the results of other researchers. Four different cases will be studied, each having different number of routes. In order for our experimental results to be comparable with the results published by the researches mentioned above, will cases with four, six, seven and eight routes in the route set be examined.

In order to determine Research Question \vref{itm:2a}, whether the additional attributes from PSO and BSO has improved the basic ACO algorithm, the proposed algorithm will also be compared against a generic ACO implementation.


\subsection{Experimental Setup}
\label{subsec:performanceComparison_setup}
%We will in this stage test the system's performance, comparing the results with the results published by xxx, and with the basic ACO implementation. 

In order to evaluate the performance of the proposed algorithm, \textit{five} performance criteria are used:
\begin{itemize}
\item $d_0 (\%)$ - the percentage of passengers satisfied without any transfers. 
\item $d_1 (\%)$ - percentage passengers transferring once. 
\item $d_2 (\%)$ - percentage of passengers transferring twice. 
\item $d_{unsat}$ (\%) - the percentage of unsatisfied passengers. An unsatisfied passenger is described as a passenger with 3 or more transfers. 
\item $ATT$  - the average travel time in minutes per transit user (mpu). In all approaches published in the literature, the transfer penalty of 5 min is applied to each route for each required transfer, and the same transfer penalty will be used by the proposed SSO algorithm.
%$ATT$ is calculated as follows:
%$$ATT = \frac{\sum\limits^{p}_{p=1}TT}{p_{size}}$$
%where $p_{size}$ is the number of passengers and $TT$ is the travel time used by passenger $p$. 
\end{itemize}
These criteria are for comparison reasons the same as the ones presented by \citet{mandl79}, \citet{kechagiopoulos14}, \citet{kidwai98}, \citet{fan10}, \citet{chakroborty02}, \citet{zhang10}, and \citet{chew12}. The criteria for good performance includes that the percentage of passengers satisfied without any transfers is high, and that the average travel time and the percentage of unsatisfied customers is as low. 

As mentioned in Section \vref{sec:expPlan} will the algorithm be compared against a generic ACO algorithm, in addition to comparisons against the proposed methods of the authors mentioned above. The generic ACO is identical to the proposed SSO algorithm, but without the additional features such as Following Ants, Crazy Ants, a notion of the global best found solution so far and memory. 

In order for our results to be comparable with results published, experiments with four, six, seven and eight routes in each route set will be examined. 

There will be carried out 50 runs per experiment, recording the values of $d_0$, $d_1$, $d_2$, $d_unsat$, and $ATT$ of the best found solution, i.e. the one who had the lowest $TOTFIT$. After 50 runs the best and worst solution will be be presented, along with the average, median and standard deviation for the values described above. The minimum, maximum, average, median and standard deviation for the execution time (in seconds) of the algorithm will also be presented.  


