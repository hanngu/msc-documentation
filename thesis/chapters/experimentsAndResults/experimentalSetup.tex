\section{Experimental Setup}

%The experimental setup should include all data - parameters etc, that would allow a person to repeat your experiments. 
 
\subsection{Parameter Settings}
\label{subsec:parameterSettings_setup}
The parameters tested with different values are described in Table \vref{table:parameters}. Each of these parameters is assigned a default value that will be used when other parameters are tested. The default value of $s$ and $i$ are inspired by the corresponding values described in \citet{salehi-nezhad07}, \citet{poorzahedy11}, \citet{sedighpour14}, and \citet{kechagiopoulos14}. 

$E$, $BR$, $AF$, and $CA$ are parameters unique for the proposed solution, and the default value for these are chosen based on preliminary testing not included in this thesis. 

The default value of $p_b$ is also partly determined based on preliminary testing, but it is also important to acknowledge that the value of $p_b$ is strongly dependent on the value of $p_v$. $p_v$ is, as mentioned in Section \vref{sec:algoInitialization}, the pheromone constant used to determine how much pheromone to be added to all visited edges. Different values of $p_v$ will not be tested because the ants are only concerned about the ratio of the pheromone level on edge $e$ and the summed pheromone level on all the possible edges when choosing which edge to walk. $p_v$ could in fact be any real number, as long as the constant are the same in all cases. $p_v$ is in this thesis sat to 0.1. 

The value of the inertia weight, $IW$, is neither tested with different values partly because the value of $IW$ directly affects $CA$ (as described in Section \vref{sec:algoGeneratingSuperSwarm}) and because implementation of different inertia weight strategies is beyond the scope of this theses. $IW$ is in this thesis sat to 1.0. 

\begin{table}[H]
	\small
	\begin{tabular}{|l|l|l|}
    	\hline
    	Parameters & Description & Default\\
    	\hline
    	$s$ & The SuperSwarm Colony Size & 50\\
    	$i$ & The numbers of iterations (which is the stop criteria) & 50\\
    	$E$ & The percentage of pheromones to evaporate at each iteration & 10\%\\
    	$p_b$ & The pheromone constant added to edges in the best route sets & 0.5\\
    	$BR$ & The percentage of route sets to be granted extra pheromone & 10\%\\
    	$AF$ & The percentage of ants to be followed & 10\%\\
    	$CA$ & The probability that a given ant is declared ``crazy'' & 10\%\\
   	    \hline
    \end{tabular}
    \caption {Parameters to be Tested and their Default Value}.
    \label{table:parameters}
\end{table}

For determining the most fitted parameters, the algorithm will be run 10 time for each candidate value, except for the candidate values of $AF$ and $CA$, which will be run 50 times. $AF$ and $CA$ are directly linked to features from respectively BCO and PSO, and these parameters will therefor be tested more thoroughly.

Minimum \textit{five} different candidate values are tested for each parameter. The candidate values of $s$ and $i$ are inspired by the corresponding values of the literature described in Section \vref{relatedWork}. The parameters $E$, $BR$, $AF$, and $CA$ are all stated as a percentage and their candidate values will therefor be in the range 0 to 100. $AF$ and $CA$ are the only two tested with 0, because that will help determine the effect of BCO and PSO. As stated above, $p_b$ is dependent on the $p_v$ value, and because $p_v$ is sat to 0.1, $p_b$ will be tested in the range 0.1 to 0.9. The idea of $p_b$ is to ``reward'' the edges in the best route sets, and we therefor chose to keep $p_b$ equal or greater than $p_v$. However, $p_b$ is always lesser than 1.0 to avoid the ants to get stuck at local optima by over appreciating the edges in the best route sets. 

The parameter values selected will be based on the results concerning the average travel time $ATT$ and the total fitness $TOTFIT$. The selected values will be the ones that produced the lowest average $ATT$ and the lowest $TOTFIT$. The reader recalls from Section \vref{sec:algoEvaluation} that the better the total fitness, the lower the $TOTFIT$ value. If the lowest $ATT$ and the lowest $TOTFIT$ are not achieved by the same parameter value $PV$, the selected value $SV$ is determined by the following formula:
\[
    SV= 
\begin{cases}
    PV_{bestATT},& \text{if } \frac{ATT_{PV_{bestTOTFIT}}}{ATT_{PV_{bestATT}}}\geq \frac{TOTFIT_{PV_{bestATT}}}{TOTFIT_{PV_{bestTOTFIT}}}\\
    PV_{bestTOTFIT},& \text{otherwise}
\end{cases}
\]


\subsection{Performance Comparison}

To demonstrate reliability, we will carry out \emph{\color{blue} TODO: XX } replicate runs per experiment, recording the average (AVG), best (MAX) and worst (MIN) ant.  

\emph{\color{blue} TODO: Because of reasons }we believe that ATT and $d_0$ is the most important parameters for determining best and worst ants, therefore, the selection of the MAX and MIN ant will be selected based on these parameters. 

\subsubsection{Selecting MAX ant}
\begin{algorithm}[H]
$Ant_{i}$ = ant with highest $d_0$\;
$Ant_{j}$ = ant with lowest ATT\;
\eIf{($Ant_{i}$ = $Ant_{j}$)}{
	Select this ant
}
{
	$d_0(\%)$ = $(d_0(lowest) / d_0(highest))*100$\;
	$ATT(\%)$ = $(ATT(lowest) / ATT(highest))*100$\;
	\eIf{ ($ d_0(\%) $ $ \geq $ ATT(\%)) }{
		select $Ant_{j}$
	}
	{
		select $Ant_{i}$
	}
}
 \caption{Selecting MAX Ant}
\end{algorithm}


The comparison is done by calculating the percentage of the difference between the highest and lowest $d_0$ and $ATT$ values. If the difference in the two $d_0$'s is higher than the difference in the two $ATT$'s, the ant with lowest ATT is selected, or else the ant with the highest $d_0$ is selected. As mentioned, a low as possible $ATT$ value and a high as possible $d_0$ value is what we want to achieve.

\subsubsection{Selecting MIN ant}
\begin{algorithm}[H]
$Ant_{i}$ = ant with lowest $d_0$\;
$Ant_{j}$ = ant with highest ATT\;
\eIf{($Ant_{i}$ = $Ant_{j}$)}{
	Select this ant
}
{
	$d_0(\%)$ = $(d_0(lowest) / d_0(highest))*100$\;
	$ATT(\%)$ = $(ATT(lowest) / ATT(highest))*100$\;
	\eIf{ ($ d_0(\%) $ $ \leq $ ATT(\%)) }{
		select $Ant_{j}$
	}
	{
		select $Ant_{i}$
	}
}
 \caption{Selecting MIN Ant}

\end{algorithm}

The comparison is also done by calculating the percentage of the difference between the highest and lowest $d_0$ and $ATT$ values. If the difference in the two $ATT$'s is higher than the difference in the two $d_0$'s, the the ant with highest ATT is selected, or else the ant with the lowest $d_0$ is selected. A high $ATT$ and a low $d_0$ is not considered good. 

\subsubsection{Algorithm Parameters}
After running test stage 1, (finding the optimal algorithm parameters), these are the final parameters which will be used in stage 2.

\begin{table}[H]
	\centering
    \begin{tabular}{|l|l|l|l|l|l|l|l|}
 	\hline
 	$s$ & $i$ & $p_{e}$ & $p_{v}$ & $p_{b}$ & $BR$ & $FA$ & $CA$  \\
 	\hline
    ~ & ~ & ~ & ~ & ~ & ~ & ~ & ~ \\
	\hline
    \end{tabular}
    \caption {Final selected parameters}
    Final Parameters from the parameter settings experienced, found in Table \vref{table:parameterSettings2}
    \label{table:finalParameters}
	\end{table}

\subsubsection{Route Set Sizes}
4,6,7 and 8 routes per set.

\subsection{Scalability Experiments}
Time and Space Complexity
