\section{Experimental Setup}

%The experimental setup should include all data - parameters etc, that would allow a person to repeat your experiments. 
 
\subsection{Parameter Settings}
\label{subsec:parameterSettings_setup}

Testing different parameter values is a time-consuming task. Because of this a set of candidate values for each parameter is determined based on the experience from previous research described in Section \vref{realtedWork}, our own  and qualified guesses made by the authors of this thesis. Minimum \textit{four} values for each parameter are tested while the other are held constant. The parameter values selected will be based on the ones that produce best computational results concerning the average travel time $ATT$ and the total fitness $TOTFIT$, where best result means the lowest average $ATT$ and the lowest $TOTFIT$ values produced. The reader recalls from Section \vref{sec:algoEvaluation} that the better the total fitness, the lower the $TOTFIT$ value. If the lowest $ATT$ and the lowest $TOTFIT$ are not achieved by the same parameter value, the selected value is determined by the following formula:
\[
    SelectedValue= 
\begin{cases}
    Value_{bestATT},& \text{if } x\geq 1\\
    Value_{bestTOTFIT},              & \text{otherwise}
\end{cases}
\]
The parameters that will be tested are as follows:
\emph{\color{blue} TODO: her må vi skrive at vi velger verdier på bakgrunn av hva vi mener er best også, med tanke på hva som gir mening og sånn. }  

\begin{itemize}
\item The SuperSwarm Colony Size, $s$. 
\item The numbers of iterations (which is the stop criteria), $i$.
\item The percentage of pheromones to evaporate at each iteration, $E$.
\item The pheromone constant, $p_v$, to determine how much pheromone to be added to each edge as it is visited by ants.
\item The pheromone constant, $p_b$, to determine how much extra pheromone to be granted to edges included in the \textit{n} best route sets.
\item The percentage of complete route sets to be granted extra pheromone, $BR$.
\item The percentage of ants to be followed, $FA$.
\item The probability that a given ant is declared ``crazy'', $CA$.
\end{itemize}

The default values of each parameter, which will be held constant when other parameters are tested, are selected based on qualified guesses. These default values are shown in Table \vref{tbl:defaultValues}. For determining the most fitted parameters, the algorithm will be run 10 time for each candidate value, except for the candidate values of $FA$ and $CA$, which will be ran 100 times. The candidate values of $FA$ and $CA$ are run 10 times more than the rest, because the best values of these will help us answer Research Question \vref{itm:2a}, which concerns whether or not it is efficient to combine attributes from different swarm intelligence methods. By running the algorithm more times, we reduce the potential randomness in the results. To further establish the effect of implementing features from PSO and BCO, $CA$ and $FA$ will be tested with more parameter values, including 0\% and 100\%. $p_{v}$ and $p_{b}$ will be tested together because their values directly affects the effect of the other. 

\begin{table}[H]
\label{tbl:defaultValues}
	\centering
    \begin{tabular}{|l|l|l|l|l|l|l|l|}
 	\hline
 	$s$ & $i$ & $p_{e}$ & $p_{v}$ & $p_{b}$ & $BR$  & $FA$ & $CA$  \\
 	\hline
    50 & 50 & 10\% & 0.1 & 0.5 & 10\% & 10\%  & 10\%  \\
	\hline
    \end{tabular}
    \caption {Default Values for The Parameter Setting Experiment} \emph{\color{blue} TODO: Reasons for these values:}
    \label{table:parameter_startvalues}
	\end{table}

\subsection{Performance Comparison}

To demonstrate reliability, we will carry out \emph{\color{blue} TODO: XX } replicate runs per experiment, recording the average (AVG), best (MAX) and worst (MIN) ant.  

\emph{\color{blue} TODO: Because of reasons }we believe that ATT and $d_0$ is the most important parameters for determining best and worst ants, therefore, the selection of the MAX and MIN ant will be selected based on these parameters. 

\subsubsection{Selecting MAX ant}
\begin{algorithm}[H]
$Ant_{i}$ = ant with highest $d_0$\;
$Ant_{j}$ = ant with lowest ATT\;
\eIf{($Ant_{i}$ = $Ant_{j}$)}{
	Select this ant
}
{
	$d_0(\%)$ = $(d_0(lowest) / d_0(highest))*100$\;
	$ATT(\%)$ = $(ATT(lowest) / ATT(highest))*100$\;
	\eIf{ ($ d_0(\%) $ $ \geq $ ATT(\%)) }{
		select $Ant_{j}$
	}
	{
		select $Ant_{i}$
	}
}
 \caption{Selecting MAX Ant}
\end{algorithm}


The comparison is done by calculating the percentage of the difference between the highest and lowest $d_0$ and $ATT$ values. If the difference in the two $d_0$'s is higher than the difference in the two $ATT$'s, the ant with lowest ATT is selected, or else the ant with the highest $d_0$ is selected. As mentioned, a low as possible $ATT$ value and a high as possible $d_0$ value is what we want to achieve.

\subsubsection{Selecting MIN ant}
\begin{algorithm}[H]
$Ant_{i}$ = ant with lowest $d_0$\;
$Ant_{j}$ = ant with highest ATT\;
\eIf{($Ant_{i}$ = $Ant_{j}$)}{
	Select this ant
}
{
	$d_0(\%)$ = $(d_0(lowest) / d_0(highest))*100$\;
	$ATT(\%)$ = $(ATT(lowest) / ATT(highest))*100$\;
	\eIf{ ($ d_0(\%) $ $ \leq $ ATT(\%)) }{
		select $Ant_{j}$
	}
	{
		select $Ant_{i}$
	}
}
 \caption{Selecting MIN Ant}

\end{algorithm}

The comparison is also done by calculating the percentage of the difference between the highest and lowest $d_0$ and $ATT$ values. If the difference in the two $ATT$'s is higher than the difference in the two $d_0$'s, the the ant with highest ATT is selected, or else the ant with the lowest $d_0$ is selected. A high $ATT$ and a low $d_0$ is not considered good. 

\subsubsection{Algorithm Parameters}
After running test stage 1, (finding the optimal algorithm parameters), these are the final parameters which will be used in stage 2.

\begin{table}[H]
	\centering
    \begin{tabular}{|l|l|l|l|l|l|l|l|}
 	\hline
 	$s$ & $i$ & $p_{e}$ & $p_{v}$ & $p_{b}$ & $BR$ & $FA$ & $CA$  \\
 	\hline
    ~ & ~ & ~ & ~ & ~ & ~ & ~ & ~ \\
	\hline
    \end{tabular}
    \caption {Final selected parameters}
    Final Parameters from the parameter settings experienced, found in Table \vref{table:parameterSettings2}
    \label{table:finalParameters}
	\end{table}

\subsubsection{Route Set Sizes}
4,6,7 and 8 routes per set.

\subsection{Scalability Experiments}
Time and Space Complexity
