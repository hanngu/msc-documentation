\section{Experimental Setup}

The experimental setup should include all data - parameters etc, that would allow a person to repeat your experiments. 
 
\subsection{Parameter Settings}
\label{subsec:parameterSettings_setup}

Because there is no way of defining the most effective values of the parameters, a selection of some of the best parameters is tested. For achieving this goal, minimum \textit{four} values for each parameter are tested while the other are held constant. For determining these parameters each one of the instances will run 10 times. The most optimal amount of iterations would have been to run it 100 times. It takes approximately 20 minutes to run \textit{one} test case 10 times on the machine we have been assigned. Because of a limited amount of time on our thesis, we do not have the available time to test the parameters 100 times. The parameter values selected is the ones that produce best computational results concerning the $ATT$ and $TOTFIT$, where best result means the lowest average $ATT$ and $TOTFIT$ values produced. $p_{v}$ and $p_{b}$ will be tested together \emph{\color{red}because ... }. The parameters that will be tested are the following:

\begin{itemize}
\item The Ant colony size, $s$.
\item The total iterations, $i$, which is the the termination criteria - and it will be tested from when it converges. 
\item The evaporation constant, $p_{e}$, to determine how much pheromone to be removed at each edge at each iteration. 
\item The pheromone constant, $p_{v}$, to determine how much pheromone to be added to each edge as it is visited by ants. 
\item The pheromone constant, $p_{b}$, to determine how much extra pheromone to be granted to edges included in the \textit{n} best route sets.
\item The percentage of complete route sets to be granted extra pheromone, the best route sets, $BR$.
\item The percentage of ants to be followed, $FA$.
\item The percentage of Crazy Ants, $CA$.The Inertia Weight value, $IW$, is 1.
\end{itemize}

When the best parameter value concerning $ATT$ and $TOTFIT$ is determined, this parameter will be selected and held constant for the rest parameter settings. The tests are performed in the following order; $p_{e}$, $p_{v}$ and $p_{b}$, $BR$, $FA$, $s$ and $i$.

\subsection{Performance Comparison}

\subsubsection{Algorithm Parameters}
After running test stage 1, (finding the optimal algorithm parameters), these are the final parameters which will be used in stage 2.

\begin{table}[H]
	\centering
    \begin{tabular}{|l|l|l|l|l|l|l|l|}
 	\hline
 	$s$ & $i$ & $p_{e}$ & $p_{v}$ & $p_{b}$ & $BR$ & $FA$ & $CA$  \\
 	\hline
    ~ & ~ & ~ & ~ & ~ & ~ & ~ & ~  \\
	\hline
    \end{tabular}
    \caption {Final selected parameters}
    Final Parameters from the parameter settings experienced, found in Table \ref{table:parameterSettings2}
    \label{table:finalParameters}
	\end{table}

\subsubsection{Route Set Sizes}
4,6,7 and 8 routes per set.

\subsection{Scalability Experiments}
Time and Space Complexity
