This chapter will in detail describe the experiments conducted. The development and test environment is informed in Section xx.. The network chosen to test the performance of the proposed algorithm is, as mentioned, the Mandl network\citep{mandl79}, and details about the data is found in Appendix \vref{sec:inputData}. The performance criteria used to evaluate and compare the algorithm's performance is described in Section \vref{sec:performanceCriteria}. The experimental setup and results respectively Section \vref{sec:expSetup}, and \vref{sec:expResults} are divided in to three parts. Where the first part is the parameter setting experiment, the second part is the performance comparison experiments, and part three is the scalability experiments. The scalability experiments will test the algorithm on larger networks, where the input data is retrieved from \citet{mumford13}. 


%Trying and failing is a major part of research. However, to have a chance of success you need a plan driving the experimental research, just as you need a plan for your literature search. Further, plans are made to be revised and this revision ensures that any further decisions made are in line with the work already completed.  
 
%Such questions should be connected to your research questions so that in the evaluation of your results you can discuss the results wrt to the research questions.

%CCohen: The purpose of evaluation at this stage is to convince the researcher and the community that studies of a program - independent of their results - are well-designed and complete. Experiment schema's would offer researchers a shorthand to describe their studies.