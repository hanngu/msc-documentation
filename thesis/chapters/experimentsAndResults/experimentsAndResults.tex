In this chapter will the experiments of the proposed algorithm to the UTRP be presented. Section \vref{sec:expPlan} will present what series of experiments are planned. Section xx describes the environment used in order to develop the proposed algorithm, in addition to the environment used to run the experiments. Section \vref{sec:expSetup} goes into detail about values used to perform the experiments, and presents the criteria for good performance. Finally, Section \vref{sec:expResults} will present the computational results. 

The series of experiments planned are divided in to three parts: parameter settings, performance comparison, and scalability.

The parameter settings and performance comparison experiments will test the proposed algorithm on the Swiss road network introduced by Mandl\citep{mandl79}. The scalability will be tested using the algorithm on the transit networks found in the files accompanying \citet{mumford13}. The exact input data for the Mandl Network is included in Appendix \vref{sec:inputData}, and details about the transit networks used when testing scalability can be found in \citet{mumford13}.

%This chapter will in detail describe the experiments conducted. 

%Trying and failing is a major part of research. However, to have a chance of success you need a plan driving the experimental research, just as you need a plan for your literature search. Further, plans are made to be revised and this revision ensures that any further decisions made are in line with the work already completed.  
 
%Such questions should be connected to your research questions so that in the evaluation of your results you can discuss the results wrt to the research questions.

%CCohen: The purpose of evaluation at this stage is to convince the researcher and the community that studies of a program - independent of their results - are well-designed and complete. Experiment schema's would offer researchers a shorthand to describe their studies.