\section{Experimental Results}

%Results should be clearly displayed and should provide a suitable representation of your results for the points you wish to make. Graphs should be labeled in a legible font and if more than one result is displayed on the same graph then these should be clearly marked.   Please choose carefully rather than presenting every results. Too much information is hard to read and often hides the key information you wish to present. Make use of statistical methods when presenting results, where possible to strengthen the results.  Further, the format of the presentation of results should be chosen based on what issues in the results you wish to highlight. You may wish to present a subset in the experimental section and provide additional results in the appendix.

\subsection{Parameter Settings}
\label{subsec:parameterSettings_results}

Table \vref{table:parameterSettings2} shows the parameters tested, their candidate values and the selected value. The complete experimental steps with the corresponding results can be found in, Appendix \ref{appendixC}, Table \vref{table:pm1} and Table \vref{table:pm2}. 
    \begin{table}[H]
    \centering
    \begin{tabular}{|c|c||c|}
    \hline
    Parameters & Candidate values & Selected value\\
    \hline
    $s$ & 10, 25, 50, 100, 150 & 100$^*$ \\
    $i$ & 1, 10 , 50, 100, 125 & 100 \\
    $E$ & 1\%, 10\%, 25\% 50\%, 75\%, 90\%, 99\% & 50\% \\
    $p_{b}$ & 0.0, 0.1, 0.3, 0.5, 0.7, 0.9 & 0.9 \\
    $BR$ & 1\%, 10\%, 25\% 50\%, 75\%, 90\%, 99\% & 25\% \\
    $AF$ & 0\%, 1\%, 5\%, 10\%, 50\%, 75\%, 100\% & 0\% \\
    $CA$ & 0\%, 1\%, 5\%, 10\%, 50\%, 75\%, 100\% & 5\% \\
    \hline
    \end{tabular}
    \caption {Results from the parameter settings experiment}
    The complete experimental steps with the corresponding results can be found in, Appendix \ref{appendixC}, Table \vref{table:pm1} and Table \vref{table:pm2}.
    \begin{itemize}[noitemsep]
    \item[$^*$:] \emph{\color{red} ikke den beste, men på grunnlag av..}
    \end{itemize}
    \label{table:parameterSettings2}
    \end{table}


\subsection{Performance Comparison}

\begin{table}[H]
	\centering
    \begin{tabular}{|l|l l l l l l l l|}
    \hline
    Route 1: & ~ & ~ & ~ & ~ & ~ & ~ & ~ & ~ \\
    Route 2: & ~ & ~ & ~ & ~ & ~ & ~ & ~ & ~ \\
    Route 3: & ~ & ~ & ~ & ~ & ~ & ~ & ~ & ~ \\
    Route 4: & ~ & ~ & ~ & ~ & ~ & ~ & ~ & ~ \\
	\hline
    \end{tabular}
    \caption {The best route sets, having four routes}
    Results constructed by the proposed algorithm, compared with results constructed by other approaches.
    \label{table:performanceComparison_bestRouteSet4}
	\end{table}

\begin{table}[H]
	\centering
    \begin{tabular}{|l||l|l|l|l|l|}
 	\hline
 	Algorithm & $d_0(\%)$ & $d_1(\%)$ & $d_2(\%)$ & $d_{unsat}(\%)$ & $ATT$ \\
 	\hline
    Mandl[ref] & 69.94 & 29.93 & 0.13 & 0.00 & 12.90 \\
    Kechapocholous AVG [ref] & 90.52 & 8.75 & 0.73 & 0.00 & 10.71 \\
    Kechapocholous BEST [ref] & 91.84 & 7.64 & 0.51 & 0.00 & 10.64 \\
    Fan and Mumford BEST [ref] & 93.26 & 6.74 & 0.00 & 0.00 & 11.37 \\
	\hline
    \hline
    The Proposed SSO Algorithm & ~ & ~ & ~ & ~ & ~ \\
    \hline
    \end{tabular}
    \caption {Comparing the best route sets, having four routes}
    Results constructed by the proposed algorithm with route sets constructed by other approaches.
    \label{table:performanceComparison_4}
	\end{table}

\subsection{Scalability Experiments}
Time and Space Complexity