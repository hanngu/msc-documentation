\section{Scalability}

\subsection{Experimental Plan}
%The programs are rarely informative if they are designed to run on a single example - Therefore we will test algorithm on other, larger, networks, to check whether it is general and not just optimized for Mandl. (Mumford)
To answer Research Question \vref{itm:2c}, which is concerned about whether or not it is possible to apply the proposed algorithm to transit networks in real urban cities, the proposed algorithm will be tested against larger networks than the Mandl Network. This is because the majority of real cities have larger transit networks than the relatively small Mandl Network, which only contains 15 nodes (bus stops) and 21 edges (roads). The scalability experiments will help us establish if the proposed algorithm supports larger network as input and if so, how a larger network affects the run time and the result quality.
%The exact shape of any normal curve is totally determined by its mean and standard deviation. %Therefore, if we know the mean and standard deviation of a statistic, we can find the mean and standard deviation of the sampling distribution of the statistic


\subsection{Experimental Setup}
\label{subsec:scalabilityExperiments_setup}

In order to test scalability the generated instances added as supplementary material to \citet{mumford13} will be used.
The mentioned instances consists of transport networks with bus stops (nodes) and roads (edges). Coordinates for each node, travel time for each edge and demand between each two nodes are all provided. Constraint \vref{itm:constraintRouteSize} specifies that the maximum number of nodes, $Max(n)$ and minimum number of nodes, $Min(n)$, in a route should be predefined, along with the size of the route set, $RS_{size}$. These values are all specified in the supplementary material to \citet{mumford13}. Table \vref{table:dataSet_mumford} defines the properties of the instances, along with $Max(n)$, $Min(n)$ and $RS_{size}$ for each instance. The tests will be performed with both values of parameter $s$ and $i$ sat to 50 for all instances of Mumford, and the algorithm will be ran 10 times for each instance. The Mumford Networks are tested with a smaller value of $s$ than the Mandl Network due to the fact that the limit of relationships types is sat to $2^{16} \approx 66 000$. As stated in Section \vref{subsec:networkGeneration} is each route for each ant for each iteration created as a relationship type. If we were to use the initial  

\begin{table}[H]
	\centering
	\begin{tabular}{|l|l|l|l|l|l|}
    	\hline
    	Instance & Nodes&Edges & $Min(n)$ & $Max(n)$ & $RS_{size}$\\
    	\hline
   	    Mumford0 & 30&90 & 2&15 & 12 \\
   	    Mumford1 & 70&210 & 10&30 & 15 \\
   	    Mumford2 & 110&385 & 10&22 & 56 \\
   	    Mumford3 & 127&425 & 12&25 & 60 \\
   	    \hline
    \end{tabular}
    \caption{Data set Mumford}
    \label{table:dataSet_mumford}
\end{table}

When testing the Mumford networks, \textit{Method 1} will be used to find the routes for the passenger and not \textit{Method 2} as used when testing the Mandl network. The reader recalls from Section \vref{sec:f1} that Method 1 is where the path with the shortest traveling time, not considering any transitions, is chosen. In the second method, Method 2, the transitions is considered, and chooses the path with the shortest traveling time, including transfer penalties. As Table \vref{table:results_mumford} shows, Method 2 produces a better $TOTFIT$, but due to the excessive difference in run time, Method 1 is chosen for this experiment. 

\begin{table}[H]
    \centering
    \begin{tabular}{|l|l|l|l|l|l|}
        \hline
        Method & TOTFIT & Run time(sec) \\
        \hline
        1 & -61.42 & 6764 \\
        2 & -167.12 & 68018 \\
        \hline
    \end{tabular}
    \caption{Mumford0 execution time in seconds, Method1 vs. Method2 \emph{\color{blue} gammel algorithme}}
    \label{table:results_mumford}
\end{table}

