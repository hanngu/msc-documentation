\section{Experimental Plan}
\label{sec:expPlan}
The series of experiments planned are divided in to three parts: parameter settings, performance comparison, and scalability.

The parameter settings and performance comparison experiments will test the proposed algorithm on the Swiss road network introduced by Mandl\citep{mandl79}. The scalability will be tested using the algorithm on the transit networks found in the files accompanying\citet{mumford13}. The exact input data for the Mandl Network is included in Appendix \vref{sec:inputData}, and details about the transit networks used when testing scalability can be found in \citet{mumford13}.

\subsection{Parameter Settings}
Metaheuristics, like ACO, requires good initial parameters to solve concrete problems optimally. The parameter settings experiment will study the effect of the variation of the parameters, and will be conducted to find the most optimal algorithm parameters. As mentioned in Section \vref{sec:relatedWork}, refers several authors to their parameter settings experiments as a product of ``trial and error'', without presenting the parameter values tested. For contributing to the field and providing a starting point for future research, will this thesis include a complete review of the conducted experiment. %The selected values for the parameters will be used as the default values in the performance comparison experiments.

\subsection{Performance Comparison}
When the optimal parameters are found will the comparison studies determine if the proposed algorithm's computational results performs better than the result published in the literature, and with this establish Research Question \ref{itm:2b1}. Four different cases will be studied, each having different number of routes. In order for our experimental results to be comparable with the results published by other researchers, will cases with four, six, seven and eight routes in the route set be examined. For each case, the best route will be presented and the results will be compared with the results published by \citet{mandl79}, \citet{kechagiopoulos14}, \citet{nikolic14}, \citet{kidwai98}, \citet{fan10}, \citet{chakroborty02}, \citet{zhang10}, \citet{chew12}, and \citet{baaj91}.

In order to determine Research Question \vref{itm:2a}, whether the additional attributes from PSO and BSO improves the ACO algorithm, will the proposed algorithm's results be compared to results produced by a plain ACO implementation. To do this, will the algorithm be run without the additional features, which are ants declared crazy, ants to be followed and ants to follow, and the ant's notion of the global best solution.

\subsection{Scalability}
%The programs are rarely informative if they are designed to run on a single example - Therefore we will test algorithm on other, larger, networks, to check whether it is general and not just optimized for Mandl. (Mumford)
To answer Research Question \vref{itm:2c}, which is concerned about whether or not it is possible to apply the proposed algorithm to transit networks in real urban cities, the proposed algorithm will be tested against larger networks than the Mandl Network. This is because the majority of real cities have larger transit networks than the relatively small Mandl Network, which only contains 15 nodes (bus stops) and 21 edges (roads). The scalability experiments will help us establish if the proposed algorithm supports larger network as input and if so, how a larger network affects the run time and the result quality. 

%The exact shape of any normal curve is totally determined by its mean and standard deviation. %Therefore, if we know the mean and standard deviation of a statistic, we can find the mean and standard deviation of the sampling distribution of the statistic

%For most of the parameters the different parameter values will only be tested a limited number of times, and a statistical analysis will not be provided. This is because a complete analysis of the parameter settings is beyond the scope of this thesis, and the results of the tests without a statistical analysis should only be considered as indicative. The parameters directly linked to features of PSO and BCO will be tested more thoroughly and a statistical analysis will be provided, because this will help us establish Research Question \vref{itm:2a}. Research Question \vref{itm:2b1} is concerned about how the proposed algorithm's computational results compare to results published in the literature. 


%The exact shape of any normal curve is totally determined by its mean and standard deviation. %Therefore, if we know the mean and standard deviation of a statistic, we can find the mean and standard deviation of the sampling distribution of the statistic

%For most of the parameters the different parameter values will only be tested a limited number of times, and a statistical analysis will not be provided. This is because a complete analysis of the parameter settings is beyond the scope of this thesis, and the results of the tests without a statistical analysis should only be considered as indicative. The parameters directly linked to features of PSO and BCO will be tested more thoroughly and a statistical analysis will be provided, because this will help us establish Research Question \vref{itm:2a}. 