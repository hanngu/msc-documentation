\section{Experimental Plan}

\subsubsection{Measures}
The experimental results generated by the proposed algorithm(s) will be compared with the respective results (Mandl, Mumford etc), based on the evaluation criteria in section 3.1.2: 
These measures were adopted from \citet{kechagiopoulos14}.

\begin{itemize}
\item $d_0 (\%)$ - percentage of total transfer demands satisfied directly
\item $d_1 (\%)$ - percentage of total transfer demands where the number of transfers are 1
\item $d_2 (\%)$ - percentage of total transfer demands where the number of transfers are 2
\item $d_unsat$ (\%) - number of unsatisfied travelers (more than 2 transfers)
\item $ATT$  - average travel time between each node couples, including the time penalties for possible transfers, experienced by each passenger of the transit network.
\end{itemize}

\subsubsection{Criteria for good performance}
Who defines the criteria?\emph{\color{red} TODO}
The criteria for good performance:
$$TOTFIT_{r} = F_{1} + F_{2} + F_{3}$$
\begin{itemize}
\item $F_{1}(r)$ should ideally be 0 -  the total difference between an ants route and the shortest possible route. i.e. each route found by the ant is equal to the shortest possible route.
\item $F_{2}(r)$ ideally should every route be direct: (-3) * node couple size - reflects how many of the routes (node couples) are direct, with one transfer and with two transfers.  
\item $F—{3}(r)$ ideally should be 0 - reflects how many of the routes that are unsatisfied (not possible to travel direct, with 1 transfer or with 2 transfers). 
\end{itemize}

The optimal solution, meaning all routes are direct, which is not possible to achieve): -315.
  
\subsubsection{Experimental plan}
What experiments or series of experiments are planned. Test the program with many different examples. 

\begin{enumerate}
\item First we will test the results from the ACO implementation on Mandl's network, and compare with Mandl. 
The aim with this is to test the solution quality and check if we need to change the algorithm / add features from other SI algorithms in order to improve the results, in order to answer research question 2.

\item If we need to change the algorithm and add features from PSO / BSO, we will check whether it is efficient to combine different swarm intelligence methods' attributes to get better results concerning the vehicle routing problem, in order to answer the research question 2 a.

\item Record our run times - test the efficiency of our algorithm - The aim is to find the what takes time - And to test whether Neo4J is suited, and if Neo4J Dijkstra's or A* is best concerning run times? The aim is to see if the potential advantages for using a graph database in our implementation, and if this is useful in the optimization process, in order to answer research question 3 and 3 a.

\item Time and space  \emph{\color{red} TODO}

\item The programs are rarely informative if they are designed to run on a single example - Therefore we will test algorithm on other networks, to check whether it is general and not just optimized for Mandl. (Mumford)

\end{enumerate}

We will also test the robustness of our algorithm(s), to demonstrate reliability, we will carry out 20(?) replicate runs per experiment, recording average, best and standard deviation. 


