\section{Experimental Plan}
\label{sec:expPlan}
The series of experiments planned are divided in to three parts: parameter settings; performance comparison; and scalability experiments.

\subsection{Input Data}
The parameter settings and performance comparison experiments will test the proposed algorithm on the Swiss road network introduced by Mandl\citep{mandl79}. Appendix \vref{sec:inputData} includes the details about the input data used.% As mentioned, this benchmark problem is acknowledged by many researchers in the literature\citep{kechagiopoulos14},\citep{fan09},\citep{niccolik14}. 

\subsection{Performance Criteria}
\label{sec:performanceCriteria}
The performance criteria used to evaluate the effectiveness of the algorithm are the following:
\begin{itemize}
\item $d_0 (\%)$ - the percentage of demand satisfied without any transfers. $d_0$ is calculated as follows:
$$ d_0 = \frac{DirectTravelers}{TotalTravelers}*100$$
\item $d_1 (\%)$ - percentage of total transfer demands where the number of transfers are 1. $d_1$ is calculated as follows:
$$ d_1 = \frac{OneTransferTravelers}{TotalTravelers}*100$$
\item $d_2 (\%)$ - percentage of total transfer demands where the number of transfers are 2. $d_2$ is calculated as follows:
$$ d_2 = \frac{TwoTransfersTravelers}{TotalTravelers}*100$$
\item $d_{unsat}$ (\%) - the percentage of unsatisfied travelers. An unsatisfied traveler is described as a traveler with 3 or more transfers and $d_{unsat}$ is calculated as follows:
$$ d_{unsat} = 100 - d_0 - d_1 - d_2$$
\item $ATT$  - the average travel time in minutes per transit user (mpu). $ATT$ is calculated as follows:
$$ATT = \frac{\sum\limits^{p}_{p=1}TT}{p_{size}}$$
where $p_{size}$ is the number of passengers and $TT$ is the travel time used by passenger $p$. 
\end{itemize}

The criteria for good performance includes that the percentage of demand satisfied without any transfers is as high as possible, and that the average travel time is as low as possible.

\subsection{Parameter Settings}
Metaheuristics, like ACO, requires good initial parameters to solve concrete problems optimally. The parameter settings experiment will study the effect of the variation of the parameters, and will be conducted to find the most optimal algorithm parameters. As mentioned in Section \vref{sec:relatedWork}, refers several authors to their parameter settings experiments as a product of ``trial and error'', without presenting the parameter values tested. For contributing to the field and providing a starting point for future research, will this thesis include a complete review of the conducted experiment. %The selected values for the parameters will be used as the default values in the performance comparison experiments.

\subsection{Performance Comparison}
When the optimal parameters are found will the comparison studies determine if the proposed algorithm's computational results performs better than the result published in the literature, and with this establish Research Question \ref{itm:2b1}. Four different instances will be studied, each having different number of routes. In order for our experimental results to be straight comparable with the results published by other researchers, will four different cases with four, six, seven and eight routes in the route set be examined. For each case, the best route will be presented and the respective experimental results are compared with the results published by \citet{kechagiopoulos14}, \citet{mandl79}, \citet{fan09} mer her. 

In order to determine Research Question \vref{itm:2a}, whether the additional attributes from PSO and BSO improves the ACO algorithm, will the proposed algorithm's results be compared to results produced by a plain ACO implementation. To do this, will the algorithm  be run without the additional features, which are ants declared crazy, ants to be followed and ants to follow, and the ant's notion of the global best solution. Since the crazy ants explore new routes by selecting edges randomly and the ants have a notion of the global best solution, may this help the ACO's limitation of getting stuck at a local optima. The following ants feature may boost the algorithm's performance, by rewarding edges in the best routes with more pheromone. %The performance criteria will also be used to evaluate the effectiveness of the additional SI methods compared to the basic ACO. 
%Research Question \vref{itm:2b1} is concerned about how the proposed algorithm's computational results compare to results published in the literature. 

\subsection{Scalability Experiments}
%The programs are rarely informative if they are designed to run on a single example - Therefore we will test algorithm on other, larger, networks, to check whether it is general and not just optimized for Mandl. (Mumford)
The Mandl network used for comparison is a relatively small network, and will not be a comparable basis considering Research Question \vref{itm:2c}, which is concerned about if it is possible to apply the proposed algorithm to transit networks in real urban cities. The majority of real cities have larger transit networks, and the scalability experiments will therefore determine if the algorithm supports input from larger networks. The input data for this experiment is retrieved from \citet{mumford13}.

%The exact shape of any normal curve is totally determined by its mean and standard deviation. %Therefore, if we know the mean and standard deviation of a statistic, we can find the mean and standard deviation of the sampling distribution of the statistic

%For most of the parameters the different parameter values will only be tested a limited number of times, and a statistical analysis will not be provided. This is because a complete analysis of the parameter settings is beyond the scope of this thesis, and the results of the tests without a statistical analysis should only be considered as indicative. The parameters directly linked to features of PSO and BCO will be tested more thoroughly and a statistical analysis will be provided, because this will help us establish Research Question \vref{itm:2a}. 