\section{Experimental Plan}

What experiments or series of experiments are planned. Test the program with many different examples.


\subsection{Plan}
The robustness of our algorithm(s) will be tested. To demonstrate reliability, we will carry out 10 replicate runs per experiment, recording average, max and min.  
%\emph{\color{red} TODO: }

\emph{\color{red} Because of reasons }we believe that ATT and $d_0$ is the most important parameters, therefore the selection of MAX and MIN will be compared to these parameters.

\subsubsection{Selecting MAX Ant}

\begin{algorithm}[H]
$Ant_{highestd_0}$ = ant with highest $d_0$\;
$Ant_{lowestATT}$ = ant with lowest ATT\;
\eIf{($Ant_{highestd_0}$ = $Ant_{lowestATT}$)}{
	Select this ant
}
{
	$d_0(\%)$ = $(d_0(lowest) / d_0(highest))*100$\;
	$ATT(\%)$ = $(ATT(lowest) / ATT(highest))*100$\;
	\eIf{ ($ d_0(\%) $ $ \geq $ ATT(\%)) }{
		select $Ant_{lowestATT}$
	}
	{
		select $Ant_{highestd_0}$
	}
}
 \caption{Selecting MAX Ant}
\end{algorithm}

\subsubsection{Selecting MIN Ant}
\begin{algorithm}[H]
$Ant_{lowestd_0}$ = ant with lowest $d_0$\;
$Ant_{highestATT}$ = ant with highest ATT\;
\eIf{($Ant_{lowestd_0}$ = $Ant_{highestATT}$)}{
	Select this ant
}
{
	$d_0(\%)$ = $(d_0(lowest) / d_0(highest))*100$\;
	$ATT(\%)$ = $(ATT(lowest) / ATT(highest))*100$\;
	\eIf{ ($ d_0(\%) $ $ \leq $ ATT(\%)) }{
		select $Ant_{highestATT}$
	}
	{
		select $Ant_{lowestd_0}$
	}
}
 \caption{Selecting MIN Ant}
\end{algorithm}


\subsection{Stage 1 - Parameter Settings}

In order to study the effect of the variation of the parameters on the objective function values / measures, we conducted a series of experiments to find the most optimal algorithm parameters. The values of these parameters affect directly or indirectly on the final solution quality. The goal is to find some robust parameters which allow the algorithm to find high quality solutions for a wide range of problem instances with different features. 

\subsection{Stage 2 - Performance Comparison}

First we will evaluate results with only the ACO implementation, on Mandl's network and comparing it with the results from other ACO implementations. The aim with this is to test solution quality.

After we have found the optimal algorithm parameters, we will add features from PSO and test the system's performance, comparing it with the respective results. \emph{\color{red}ACO has a known limitation of being stuck on local optima, and the time/space(?) complexity is high.} We will therefore 
add features from PSO - inertial weight, more random in the start and accepting more solutions, knowledge about the best global solution for every iteration, keep the best current solution. After adding SI features from PSO / BSO, we will check whether it is efficient to combine different swarm intelligence methods' attributes to get better results concerning the vehicle routing problem, in order to answer the research question 2 a.

We will test the results comparing it with the performance criteria from section 4.3.

To test whether Neo4J is suited, and if Neo4J Dijkstra's or A* is best concerning run times? The aim is to see if the potential advantages for using a graph database in our implementation, and if this is useful in the optimization process, in order to answer research question 3 and 3 a.


%\begin{enumerate}

%\item Add features from PSO - inertial weight, more random in the start and accepting more solutions, knowledge about the best global solution for every iteration, keep the best current solution.

%\item Evaluate results with only the ACO implementation, on Mandl's network and comparing it with the results from other ACO implementations. The aim with this is to test solution quality and check if we need to change the algorithm / add features from other SI algorithms in order to improve the results, in order to answer research question 2. \emph{\color{red} TODO: Known limitations of ACO. Stuck on local optima, time complexity.}

%\item After adding SI features from PSO / BSO, we will check whether it is efficient to combine different swarm intelligence methods' attributes to get better results concerning the vehicle routing problem, in order to answer the research question 2 a.

%\item And to test whether Neo4J is suited, and if Neo4J Dijkstra's or A* is best concerning run times? The aim is to see if the potential advantages for using a graph database in our implementation, and if this is useful in the optimization process, in order to answer research question 3 and 3 a.

%\end{enumerate}

\subsection{Stage 3 - Time and Space Complexity / Scalability Experiments}

In order to test whether the method is general and not tuned in to run on a single example, we will run the algorithm on other, larger, networks. Mumford included larger networks. Here we will test the time and space complexity.
%The programs are rarely informative if they are designed to run on a single example - Therefore we will test algorithm on other networks, to check whether it is general and not just optimized for Mandl. (Mumford)
Record our run times - test the efficiency of our algorithm.


