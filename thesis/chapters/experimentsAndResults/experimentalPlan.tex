\section{Experimental Plan}

Trying and failing is a major part of research. However, to have a chance of success you need a plan driving the experimental research, just as you need a plan for your literature search. Further, plans are made to be revised and this revision ensures that any further decisions made are in line with the work already completed.  

The plan should include what experiments or series of experiments are planned and what question the individual or set of experiments aim to answer. Such questions should be connected to your research questions so that in the evaluation of your results you can discuss the results wrt to the research questions.  

We will evaluate our work against the following criteria: 
\begin{itemize}
	\item Solution quality - comparing our results with Mandl.
	\item Efficiency -  Record our run times
	\item Robustness - To demonstrate reliability, we will carry out 20 replicate runs per experiment, recording average, best and standard deviation. 
\end{itemize}
