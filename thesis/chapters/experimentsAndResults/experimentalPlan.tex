\section{Experimental Plan}
\label{sec:expPlan}
In this chapter will the computational results of the proposed algorithm to the UTRP be presented. The algorithm will be tested on the Swiss road network introduced by Mandl,\citep{mandl79}, and details about the input data is found in Appendix \vref{sec:inputData}. As mentioned has this network already been used as a benchmark problem by many researchers in the literature[refs], in addition is a recognized metric to evaluate the performance published by the researchers. These performance criteria used to evaluate and compare the algorithm's performance is presented in Section \vref{sec:performanceCriteria}. The experimental setup and results, Section \vref{sec:expSetup}, and \vref{sec:expResults} respectively, are divided in to three parts. 

The first part will is the parameter settings experiment. Metaheuristics, like ACO, requires good initial parameters to solve concrete problems optimally. In order to study the effect of the variation of the parameters, will we conduct a series of experiments to find the most optimal algorithm parameters concerning the performance criteria. As mentioned in Section \vref{sec:relatedWork}, refers several authors to their parameter settings experiments as a product of ``trial and error'', without presenting the parameter values tested. For contributing to the field and providing a starting point for future research, will this thesis include a complete review of the conducted experiment.  

%The exact shape of any normal curve is totally determined by its mean and standard deviation. %Therefore, if we know the mean and standard deviation of a statistic, we can find the mean and standard deviation of the sampling distribution of the statistic

%For most of the parameters the different parameter values will only be tested a limited number of times, and a statistical analysis will not be provided. This is because a complete analysis of the parameter settings is beyond the scope of this thesis, and the results of the tests without a statistical analysis should only be considered as indicative. The parameters directly linked to features of PSO and BCO will be tested more thoroughly and a statistical analysis will be provided, because this will help us establish Research Question \vref{itm:2a}. 

The second part is the performance criteria experiments, Section xx. Four different instances of the will be studied, each having different number of routes, in order for our experimental results to be straight comparable with the results published by other researchers. More precisely will we examine four different cases with four, six, seven and eight routes in the route set, respectively. For each case, the best route will be presented and the respective experimental results are compared with the results published by [refs]. In all these approaches a transfer penalty of 5 min is applies to each route for each required tranfer. The same transfer penalty is also used by the proposed SSO algorithm. 

\emph{\color{blue} TODO: } The scalability experiments will test the algorithm on larger networks. Since Mandl's network only consist of 15 nodes and 21 edges, will the input data for this experiment be other networks, and there are retrieved from \citet{mumford13}. 

The parameter experiments will the experiment help establish Research Question \vref{itm:2a}, by studying the effect of the variation on the additional PSO and BCO parameters. By comparing the results will we completely establish Research Question \vref{itm:2a}, and it will also help establish Research Question \vref{itm:2b}. The scalability experiments will completely establish Research Question \vref{itm:2b}. \emph{\color{blue} her kan RQ nevnes.  }

%The programs are rarely informative if they are designed to run on a single example - Therefore we will test algorithm on other, larger, networks, to check whether it is general and not just optimized for Mandl. (Mumford)

%In order to test whether the method is general and not tuned in to run on a single example, we will run the algorithm on other, larger, networks. (Mumford included larger networks.) Aim: Here we will test the time and space complexity.

%Record our run times - test the efficiency of our algorithm.