\section{Experimental Plan}

%Trying and failing is a major part of research. However, to have a chance of success you need a plan driving the experimental research, just as you need a plan for your literature search. Further, plans are made to be revised and this revision ensures that any further decisions made are in line with the work already completed.  

%The plan should include what experiments or series of experiments are planned and what question the individual or set of experiments aim to answer. Such questions should be connected to your research questions so that in the evaluation of your results you can discuss the results wrt to the research questions.  

What experiments or series of experiments are planned

\begin{enumerate}
\item First we will test the results from the ACO implementation on Mandl's network, and compare with Mandl. 
The aim with this is to test the solution quality and check if we need to change the algorithm / add features from other SI algorithms in order to improve the results, in order to answer research question 2.

\item Experimental results generated by the proposed ACO algorithm are compared with the respective results published in the literature based on the evaluation criteria in section, which are the following:
\begin{itemize}
\item $d_0 (\%)$ - percentage of total transfer demands satisfied directly
\item $d_1 (\%)$ - percentage of total transfer demands where the number of transfers are 1
\item $d_2 (\%)$ - percentage of total transfer demands where the number of transfers are 2
\item $d_unsat$ (\%) - number of unsatisfied travelers (more than 2 transfers)
\item $ATT$  - average travel time between each node couples, including the time penalties for possible transfers, experienced by each passenger of the transit network.
\end{itemize}

\item If we need to change the algorithm and add features from PSO / BSO, we will check whether it is efficient to combine different swarm intelligence methods' attributes to get better results concerning the vehicle routing problem, in order to answer the research question 2 a.

\item Record our run times - test the efficiency of our algorithm - The aim is to find the what takes time - And to test whether Neo4J is suited, and if Neo4J Dijkstra's or A* is best concerning run times? The aim is to see if the potential advantages for using a graph database in our implementation, and if this is useful in the optimization process, in order to answer research question 3 and 3 a.

\item Time and space complexity \emph{\color{red} TODO}

\item Test algorithm on other network, to check whether it is general and not just optimized for Mandl.

\end{enumerate}



We will also test the robustness of our algorithm(s), to demonstrate reliability, we will carry out 20(?) replicate runs per experiment, recording average, best and standard deviation. 


