\section{Experimental Plan}
\label{sec:expPlan}

The experiments planned are divided in to three parts: parameter settings; performance comparison; and scalability experiments.

The proposed algorithm will be tested on the Swiss road network introduced by Mandl,\citep{mandl79}, which has already been used as a benchmark problem by many researchers in the literature\citep{kechagiopoulos14}, \citep{mandl79}, \citep{fan09}. Details about the input data is found in Appendix \vref{sec:inputData}.  

The performance criteria used to evaluate the effectiveness of the algorithm, have been adopted by many researchers in the literature\citep{kechagiopoulos14}, \citep{mandl79}, \citep{fan09}, and are the following:
\begin{itemize}
\item $d_0 (\%)$ - the percentage of demand satisfied without any transfers. $d_0$ is calculated as follows:
$$ d_0 = \frac{DirectTravelers}{TotalTravelers}*100$$
\item $d_1 (\%)$ - percentage of total transfer demands where the number of transfers are 1. $d_1$ is calculated as follows:
$$ d_1 = \frac{OneTransferTravelers}{TotalTravelers}*100$$
\item $d_2 (\%)$ - percentage of total transfer demands where the number of transfers are 2. $d_2$ is calculated as follows:
$$ d_2 = \frac{TwoTransfersTravelers}{TotalTravelers}*100$$
\item $d_{unsat}$ (\%) - the percentage of unsatisfied travelers. An unsatisfied traveler is described as a traveler with 3 or more transfers and $d_{unsat}$ is calculated as follows:
$$ d_{unsat} = 100 - d_0 - d_1 - d_2$$
\item $ATT$  - the average travel time in minutes per transit user (mpu). The travel times incorporates a transfer penalty, which is sat to be 5 minutes per transfer for comparison reasons. $ATT$ is calculated as follows:
$$ATT = \frac{\sum\limits^{p}_{p=1}TT}{p_{size}}$$
where $p_{size}$ is the number of passengers and $TT$ is the travel time used by passenger $p$. 
\end{itemize}

The criteria for good performance includes that the $d_0 (\%)$ is higher, and $ATT$ is lower. 

The parameter settings experiment is conducted in order to study the effect of the variation of the parameters, and a series of experiments will be conducted to find the most optimal algorithm parameters concerning the performance criteria. Metaheuristics, like ACO, requires good initial parameters to solve concrete problems optimally. As mentioned in Section \vref{sec:relatedWork}, refers several authors to their parameter settings experiments as a product of ``trial and error'', without presenting the parameter values tested. For contributing to the field and providing a starting point for future research, will this thesis include a complete review of the conducted experiment. The selected values for the parameters will be used as the default values for the second part of the experiments, which is the performance comparison experiments.

In the performance comparison experiments, will four different instances be studied, each having different number of routes, in order for our experimental results to be straight comparable with the results published by other researchers. More precisely will we examine four different cases with four, six, seven and eight routes in the route set, respectively. For each case, the best route will be presented and the respective experimental results are compared with the results published by [refs]. In all these approaches a transfer penalty of 5 min is applied to each route for each required transfer, and the same transfer penalty will also be used by the proposed SSO algorithm. 

The scalability experiments will test the algorithm on larger networks. Since Mandl's network only consist of 15 nodes and 21 edges, will the input data for this experiment be other networks, and these are retrieved from \citet{mumford13}. %The programs are rarely informative if they are designed to run on a single example - Therefore we will test algorithm on other, larger, networks, to check whether it is general and not just optimized for Mandl. (Mumford)

Apart from finding the optimal parameters to the proposed algorithm, is the parameter experiment conducted in order to study the effect of the variation on the additional PSO and BCO parameters. The algorithm will first be tested without the additional PSO and BSO attributes, and the additional features will be compared against the results from the basic ACO. This will establish Research Question \vref{itm:2a}, which is concerned whether it is efficient to combine attributes from different swarm-intelligence method in order to improve the basic ACO algorithm. Assumptions: The features from PSO will result in the ants being more exploring in the beginning, exploring routes randomly - which may help the ACO's limitation at getting stuck at a local optima. 

When the optimal parameters and Research Question \vref{itm:2a} are determined, will the comparison studies determine if the proposed algorithm's computational results performs better than the result published in the literature, with regard to the performance criteria, and with this establish Research Question \vref{itm:2b1}. %Research Question \vref{itm:2b1} is concerned about how the proposed algorithm's computational results compare to results published in the literature. 

The Mandl network used for comparison is a relatively small network, and will not be a comparable basis considering Research Question \vref{itm:2b}, which is concerned about if it is possible to apply the proposed algorithm to transit networks in real urban cities. The majority of real cities have larger transit networks, and the scalability experiments will therefore determine if the algorithm supports input from larger networks, and with this determine Research Question \vref{itm:2b}.

%The exact shape of any normal curve is totally determined by its mean and standard deviation. %Therefore, if we know the mean and standard deviation of a statistic, we can find the mean and standard deviation of the sampling distribution of the statistic

%For most of the parameters the different parameter values will only be tested a limited number of times, and a statistical analysis will not be provided. This is because a complete analysis of the parameter settings is beyond the scope of this thesis, and the results of the tests without a statistical analysis should only be considered as indicative. The parameters directly linked to features of PSO and BCO will be tested more thoroughly and a statistical analysis will be provided, because this will help us establish Research Question \vref{itm:2a}. 