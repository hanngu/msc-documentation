\section{Performance Criteria}
\label{sec:performanceCriteria}

The criteria used in order to evaluate the performance of the proposed algorithm, which have been adopted by many researchers in the literature \emph{color{red}(Referanser)}.

To measure the efficiency and the quality of our algorithm, the experimental results generated by the proposed algorithm(s) will be compared with the respective results (Mandl, Mumford etc), based on the evaluation criteria in section 3.1.2: 
%These measures were adopted from \citet{kechagiopoulos14}.

\begin{itemize}
\item $d_0 (\%)$ - the percentage of demand satisfied without any transfers. $d_0$ is calculated as follows:
$$ d_0 = \frac{DirectTravelers}{TotalTravelers}*100$$
\item $d_1 (\%)$ - percentage of total transfer demands where the number of transfers are 1. $d_1$ is calculated as follows:
$$ d_1 = \frac{OneTransferTravelers}{TotalTravelers}*100$$
\item $d_2 (\%)$ - percentage of total transfer demands where the number of transfers are 2. $d_2$ is calculated as follows:
$$ d_2 = \frac{TwoTransfersTravelers}{TotalTravelers}*100$$
\item $d_{unsat}$ (\%) - the percentage of unsatisfied travelers. An unsatisfied traveler is described as a traveler with 3 or more transfers and $d_{unsat}$ is calculated as follows:
$$ d_{unsat} = 100 - d_0 - d_1 - d_2$$
\item $ATT$  - the average travel time in minutes per transit user (mpu). The travel times incorporates a transfer penalty, which is sat to be 5 minutes per transfer for comparison reasons. $ATT$ is calculated as follows:
$$ATT = \frac{\sum\limits^{p}_{p=1}TT}{p_{size}}$$
where $p_{size}$ is the number of passengers and $TT$ is the travel time used by passenger $p$. 
\end{itemize}

