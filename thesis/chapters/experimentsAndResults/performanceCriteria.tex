\section{Performance Criteria}
\label{sec:performanceCriteria}

The criteria used in order to evaluate the performance of the proposed algorithm, which have been adopted by many researchers in the literature \emph{color{red}(Referanser)}.

To measure the efficiency and the quality of our algorithm, the experimental results generated by the proposed algorithm(s) will be compared with the respective results (Mandl, Mumford etc), based on the evaluation criteria in section 3.1.2: 
%These measures were adopted from \citet{kechagiopoulos14}.

\begin{itemize}
\item $d_0 (\%)$ - the percentage of demand satisfied without any transfers, \emph{\color{red}and is calculated as followed:}
\item $d_1 (\%)$ - percentage of total transfer demands where the number of transfers are 1, \emph{\color{red}and is calculated as followed:}
\item $d_2 (\%)$ - percentage of total transfer demands where the number of transfers are 2, \emph{\color{red}and is calculated as followed:}
\item $d_{unsat}$ (\%) - the percentage of demand unsatisfied, which should be equal to zero, \emph{\color{red}and is calculated as followed:}
\item $ATT$  - the average travel times in minutes per transit user (mpu). This incorporates transfer waiting times, \emph{\color{red}and is calculated as followed:}
\end{itemize}

To demonstrate reliability, we will carry out \emph{\color{red} XX } replicate runs per experiment, recording the average (AVG), best (MAX) and worst (MIN) ant.  
\emph{\color{red} TODO:}

\emph{\color{red} Because of reasons }we believe that ATT and $d_0$ is the most important parameters for determining best and worst, therefore, in the performance comparison experiments, the selection of MAX and MIN will be selected based on these parameters.

\subsection{Selecting MAX Ant}

\begin{algorithm}[H]
$Ant_{i}$ = ant with highest $d_0$\;
$Ant_{j}$ = ant with lowest ATT\;
\eIf{($Ant_{i}$ = $Ant_{j}$)}{
	Select this ant
}
{
	$d_0(\%)$ = $(d_0(lowest) / d_0(highest))*100$\;
	$ATT(\%)$ = $(ATT(lowest) / ATT(highest))*100$\;
	\eIf{ ($ d_0(\%) $ $ \geq $ ATT(\%)) }{
		select $Ant_{j}$
	}
	{
		select $Ant_{i}$
	}
}
 \caption{Selecting MAX Ant}
\end{algorithm}

The comparison is done by calculating the percentage of the difference between the highest and lowest $d_0$ and $ATT$ values. If the difference in the two $d_0$'s is higher than the difference in the two $ATT$'s, the ant with lowest ATT is selected, or else the ant with the highest $d_0$ is selected. As mentioned, a low as possible $ATT$ value and a high as possible $d_0$ value is what we want to achieve.

\subsection{Selecting MIN Ant}
\begin{algorithm}[H]
$Ant_{i}$ = ant with lowest $d_0$\;
$Ant_{j}$ = ant with highest ATT\;
\eIf{($Ant_{i}$ = $Ant_{j}$)}{
	Select this ant
}
{
	$d_0(\%)$ = $(d_0(lowest) / d_0(highest))*100$\;
	$ATT(\%)$ = $(ATT(lowest) / ATT(highest))*100$\;
	\eIf{ ($ d_0(\%) $ $ \leq $ ATT(\%)) }{
		select $Ant_{j}$
	}
	{
		select $Ant_{i}$
	}
}
 \caption{Selecting MIN Ant}
\end{algorithm}
The comparison is also done by calculating the percentage of the difference between the highest and lowest $d_0$ and $ATT$ values. If the difference in the two $ATT$'s is higher than the difference in the two $d_0$'s, the the ant with highest ATT is selected, or else the ant with the lowest $d_0$ is selected. A high $ATT$ and a low $d_0$ is not considered good. 
