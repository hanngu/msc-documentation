\section{Performance Criteria}

The aim of designing a route network is to optimize specific criteria that define its efficiency and quality.
The criteria used in order to evaluate the performance of the proposed algorithm, which have been adopted by many researchers in the literature (Referanser).

To measure the efficiency and the quality of our algorithm, the experimental results generated by the proposed algorithm(s) will be compared with the respective results (Mandl, Mumford etc), based on the evaluation criteria in section 3.1.2: 
These measures were adopted from \citet{kechagiopoulos14}.

\begin{itemize}
\item $d_0 (\%)$ - the percentage of demand satisfied without any transfers, which should be as high as possible
\item $d_1 (\%)$ - percentage of total transfer demands where the number of transfers are 1
\item $d_2 (\%)$ - percentage of total transfer demands where the number of transfers are 2
\item $d_unsat$ (\%) - the percentage of demand unsatisfied, which should be equal to zero.
\item $ATT$  - the average travel times in minutes per transit user, which should be as low as possible
\end{itemize}




