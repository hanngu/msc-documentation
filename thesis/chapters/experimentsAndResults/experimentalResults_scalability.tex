\subsection{Scalability Experiments}
\label{subsec:scalabilityExperiments_results}

\begin{table}[H]
    \centering
    \begin{tabular}{|l|l|l|l|l|l|}
        \hline
        Test case & Method & TOTFIT & ATT & Running time(sec) \\
        \hline
        1 & 1 & -61.42 & 17.54 & 6764 \\
        1 & 2 & -167.12 & 17.36 & 68018 \\
        
        \hline
    \end{tabular}
    \caption{Mumford0 execution time in seconds, method1 vs method2 \emph{\color{blue} gammel algorithme}}
    \label{table:results_mumford}
\end{table}

The reader recalls from Section \vref{sec:f1},  \textit{Method 1} is where the path with the shortest traveling time, not considering any transitions, is chosen. In the second method, \textit{Method 2}, the transitions is considered, and chooses the path with the shortest traveling time, including transfer penalties.


%-------Mumford3--------
%Exception in thread "main" org.neo4j.graphdb.TransactionFailureException: Could not create token

%Caused by: org.neo4j.kernel.impl.store.UnderlyingStorageException: Id capacity exceeded: 65536 is not within bounds
 %[0; 65535] for Neo4jTest1/neostore.relationshiptypestore.db.id


\begin{table}[H]
    \centering
    \hspace*{-1.0cm}
    \begin{tabular}{|l|l|l|l|l|l|l|l|l|}
        \hline
        Test case & Nodes & Edges & Min/max nodes  & $d_0$ & $d_1$ & $d_2$ & $d_{unsat}$ & $ATT$\\
        \hline
        Mumford0 & 30 & 90 & 2/15 & & & & &\\
        Mumford1 & 70 & 210 & 10/30 & & & & &\\
        Mumford2 & 110 & 385 & 10/22 & & & & &\\
        Mumford3$^a$ & 127 & 425 & 12/25 & & & & &\\
        \hline
    \end{tabular}
    \caption{Data set Mumford}
    \begin{itemize}[noitemsep]
    \item[$^a$:] Could not create token - Id capacity exceeded: 65536 is not within bounds
    \end{itemize}
    \label{table:dataSet_mumford}
\end{table}
It is worth mentioning, that the amount of iterations and swarm size is sat to 50, and the number of runs is 10. \emph{\color{blue} TODO.}