\subsection{Scalability Experiments}
\label{subsec:scalabilityExperiments_results}

\begin{table}[H]
    \centering
    \begin{tabular}{|l|l|l|l|l|l|}
        \hline
        Test case & Method & TOTFIT & ATT & Running time(sec) \\
        \hline
        1 & 1 & -61.42 & 17.54 & 6764 \\
        1 & 2 & -167.12 & 17.36 & 68018 \\
        
        \hline
    \end{tabular}
    \caption{Mumford0 execution time in seconds, method1 vs method2 \emph{\color{blue} gammel algorithme}}
    \label{table:results_mumford}
\end{table}

The reader recalls from Section \vref{sec:f1},  \textit{Method 1} is where the path with the shortest traveling time, not considering any transitions, is chosen. In the second method, \textit{Method 2}, the transitions is considered, and chooses the path with the shortest traveling time, including transfer penalties.


%-------Mumford2--------
%http://neo4j.com/docs/2.1.7/capabilities-capacity.html

%Caused by: org.neo4j.kernel.impl.store.UnderlyingStorageException: Id capacity exceeded: 65536 is not within bounds [0; 65535] for Neo4jTest3/neostore.relationshiptypestore.db.id