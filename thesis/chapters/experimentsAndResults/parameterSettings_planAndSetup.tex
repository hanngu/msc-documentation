\section{Parameter settings}
\label{sec:parametersettings}

\subsection{Experimental plan}
Metaheuristics, like the ant colony optimization (ACO), requires good initial parameters to solve concrete problems optimally. The parameter settings experiment will study the effect of the variation of the parameters, and will be conducted in the attempt of finding the most optimal parameters for the system. As mentioned in Section \vref{sec:relatedWork}, do several authors refers to their parameter settings experiments as a product of ``trial and error'', without presenting the parameter values tested. For contributing to the field and providing a starting point for future research, this thesis will include a complete review of the conducted experiment. %The selected values for the parameters will be used as the default values in the performance comparison experiments. 
Also, studying the effect of the additional parameters inspired by particle swarm optimization (PSO) and bee colony optimization (BCO) will help establish whether these attributes improves the standard ACO implementation. This is will further help answer Research Question \vref{itm:RQ2}.

\subsection{Experimental setup}
\label{subsec:parameterSettings_setup}
The parameters used in the proposed system (CSS) are described in Section \vref{sec:algoInitialization}, and the parameters to be tested are presented in Table \ref{table:parameters}. As one can see, each parameter is assigned a default value to be held constant while the other parameters are tested. Each parameter are run with minimum \textit{four} different candidate values, presented in Table \vref{table:parameterSettings2}. The selected value will be the one that produce the lowest Total Fitness ($TOTFIT$). As stated in Section \vref{sec:algoEvaluation}, the lower the $TOTFIT$, the better the solution. 

The default- and candidate values for parameters $s$ and $i$ are both inspired by the corresponding values described in related research \citep{salehi-nezhad07, poorzahedy11, sedighpour14, kechagiopoulos14}. The parameters $E$, $AF$, and $CA$ are considered unique for the proposed system, and their default values are chosen based on preliminary testing not included in this thesis. Even though the standard ACO implementations include an evaporation parameter, $E$ is unique for this research because $E$ is stated as a percentage. $E$, $AF$, and $CA$ will be tested with values in the range from 0\% to 100\%. %To determine the effect of the additional BCO and PSO features by removing them from the implementation, will parameters $AF$ and $CA$ also be tested with value 0 and 1.0.

The idea of the parameter $p_b$ is to test whether rewarding edges in the best route sets, by adding more pheromone to edges walked by the ``Following Ants'', will boost the system's performance. The default value of $p_b$ is 0.0, which implies that no extra pheromone is granted these edges. Because the amount of extra pheromone granted each edge is dependent on both $p_b$ and $AF$, the candidate values of $p_b$ will be tested with the selected value of $AF$. It is worth mentioning that the value of $p_b$ must be seen in context with the value of parameter $p_v$. The value of $p_v$ is the constant added to each edge each time it is visited by an ant. The values for $p_v$ will not be tested with different values because the value could, in fact, be any real number, as long as it is constant. $p_v$ is, as mentioned in Section \vref{sec:algoInitialization}, sat to 0.1. Due to this, the candidate values of $p_b$ will be in the range from 0.0 to 1.3. 

The value of the inertia weight, $IW$, will not be tested with different values and is sat to 1.0. This is because, as described in Section \vref{sec:algoGeneratingSuperSwarm}, $IW$ directly affects $CA$, and because an implementation of different inertia weight strategies is beyond the scope of this thesis. 

\begin{table}[H]
    \centering
    %\hspace*{-0.5cm}
    \begin{tabular}{|l|m{7cm}|l|}
        \hline
        \textbf{Parameter} & \textbf{Description} & \textbf{Default Value}\\
        \hline
        $s$ & The Colony Size & 50\\
        
        $i$ & Number of iterations (the stop criteria) & 50\\
        
        $E$ & Percentage of pheromones to evaporate at each iteration & 10\%\\
        
        $CA$ & The probability of a given ant to be declared ``crazy'' & 10\%\\
        
        $AF$ & Percentage of ants to be followed & 10\%\\
        
        $p_b$ & Pheromone constant added to reward edges walked by ``Following Ants'' & 0.0\\
           \hline
    \end{tabular}
    \caption {Parameters to be tested, and their default value to be held constant while the other parameters are tested}
    \label{table:parameters}
\end{table}

For each candidate value of parameters $s$, $i$, $E$, and $p_b$, 30 runs will be carried out. For each candidate value of parameters $AF$ and $CA$, 50 runs will be carried out. The reason for the increased amount of runs is because the results of $AF$ and $CA$ will contribute in establishing \vref{itm:RQ2}. By running the system with additional runs, the margin of error will most likely decrease and will thus increase the validity of the results.

For all candidate values, the margin of error of the Confidence Interval with a confidence level of 95\% will be presented. In addition will the best and worst produced $TOTFIT$ value and the standard deviation be presented. For each parameter, the value that resulted in the lowest average $TOTFIT$ will be selected. 

All runs will be executed on Ubuntu instances provided by the Google Cloud Platform \citep{website:google}. The instances are of type ``n1-highcpu-2'', containing two 2.6GHz Intel Xeon E5 (Sandy Bridge) virtual CPUs with 1.8 GB memory. 
%In addition will $p_b$ run with the new value of $AF$.

%This is due to the the central limit theorem[Ref - Statistiske Metoder A.Hald p.139], which states that the sampling distribution of any statistic will be normal or nearly normal, if the sample size is large enough, and many statisticians say that a sample size of 30 is large enough.
