\subsection{Algorithm Parameters}

\begin{itemize}
\item Ants colony size: 100.
\item Transfer penalty: 5 min
\item Routes in the route set: 4
\item Maximum number of nodes in routes: 8
\item Total iterations: 100
\item Minimum number of nodes: 3 TODO
\item Maximum route length at the initialization stage: TODO
\item Pheromone increase value on edge when selecting next node: 0.1
\item Pheromone increase value on all edges in best route set each iteration: 0.5
\item Pheromone decrease value: 1.0
\end{itemize}
Reasons for these numbers: TODO

\subsection{Objective Function Values}
$$TOTFIT_{r} = F_{1} + F_{2} + F_{3}$$
\begin{itemize}
\item $F_{1}(r)$ equals the total difference between an ants route and the shortest possible route. Ideally it should be 0, i.e. each route found by the ant is equal to the shortest possible route.
\item $F_{2}(r)$ reflects how many of the routes (node couples) are direct, with one transfer and with two transfers. Ideally every route should be direct: (-3) * node Couple Size. 
\item $F—{3}(r)$ reflects how many of the routes that are unsatisfied (not possible to travel direct, with 1 transfer or with 2 transfers). Ideally this should be 0.
\end{itemize}

Experimental results generated by the proposed ACO algorithm are compared with the respective results published in the literature based on the evaluation criteria in section, which are the following:
\begin{itemize}
\item $d_0 (\%)$ - percentage of total transfer demands satisfied directly
\item $d_1 (\%)$ - percentage of total transfer demands where the number of transfers are 1
\item $d_2 (\%)$ - percentage of total transfer demands where the number of transfers are 2
\item $d_unsat$ (\%) - number of unsatisfied travelers (more than 2 transfers)
\item $ATT$  - average travel time between each node couples, including the time penalties for possible transfers, experienced by each passenger of the transit network.
\end{itemize}

Optimal solution (if all routes are direct - not possible to achieve): -315.