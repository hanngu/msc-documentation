
Swarm behavoior is found in many different species in nature, including fish schools and flocks of birds. Many of the species that practice swarm behavior does this because of a biological need to stay togheter. An example of this is that predators usually attacks one individual, and not an entire flock. This swarm behavior is also found in social insects like ants, wasps, bees and termites. They collaborate on tasks including building nests, gather food and organizing production. These social insect colonies have shown us that simple organisms can perform complex tasks by interacting with each other. The colonies are highly distributed and self-organized, and they adapts well to changes in the environment. Swarm intelligence \citep{swarmintelligence} is a branch of artificial intelligence that is strongly influenced by the swarm behavior found i nature, and it tries to adapt these characteristics in intelligent computer systems. 

Swarm intelligence algorithms has proven to be usefull in many vehicle routing problems. \citet{acs:tsp1} and \citet{bs:tsp1} shows that, using respectively an ant colony system and a bee systems, swarm intelligence can solve highly complex problems such as the Traveling Salesman Problem (TSP). \citet{ant-vehicleNavigation} presents an algorithm to search for the best direction between two desired origin and destination intersections in cities, called Ant-based Vehicle Navigation algorithm.