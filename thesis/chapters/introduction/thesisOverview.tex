\section{Thesis overview}

%\begin{itemize}
%\item What does this thesis contain
%\item Give results in a general way
%\end{itemize}

%This section provides the reader with an overview of what is coming in the next
%chapters. You want to say more than what is explicit in the chapter name, if
%possible, but still keep the description short and to the point.

Chapter \ref{backgroundAndMotivation} sets the context for this thesis. It starts by discussing the domain swarm intelligence in Section \ref{sec:swarmIntelligence}, and includes theory about ant colony optimization, bee colony optimization, and particle swarm optimization. In Section \ref{sec:VRP} the Vehicle Routing Problem is described, along with the Urban Transit Network Design Problem, which is a subproblem of the VRP. Section \ref{sec:graphdb} describes the term graph theory and puts it in context with graph databases. The graph database Neo4j is also presented. 

Chapter \ref{relatedWork} describes the preparatory research conducted in order to define the problem to be solved in this thesis. Section \ref{sec:definingResearchTopic} defines the research topic used in order to guide the structured literature review \citep{kofod2014}, which further is described in Section \ref{sec:structuredLiteratureReview}. Section \ref{sec:relatedWork} discusses and analyzes the related work and will eventually answer \ref{itm:RQ1}. Finally, in Section \ref{subsec:problemStatement}, the problem statement is proposed.  

Chapter \ref{theModel} describes the implemented system in detail. It starts with an overall description in Section \ref{section:methodDescription}, followed by a description of the chosen development environment in Section \ref{sec:developmentEnvironment}. Further, the assessments used in order to construct solutions are explained in Section \ref{sec:systemAssessments}. Sections \ref{sec:algoInitialization} - \ref{sec:evaporation} are detailed descriptions of each of the system's steps.

Chapter \ref{experimentsAndResults} documents the experiments conducted in this thesis. The experiments are divided into three parts, where each part contains the experimental plan, setup, and results. Section \ref{sec:parametersettings} documents the parameter setting experiment, and will also document a justification of the selected parameters. Section \ref{sec:performanceComparison} documents the performance comparison experiments with the ACO implementation and other published methods. Section \ref{sec:networkExpansion} documents the network expansion experiments, which are conducted on larger networks. 

Chapter \ref{discussionAndConclusion} concludes this thesis. Section \ref{sec:discussion} evaluates and discusses the results obtained by the experiments in Chapter \ref{experimentsAndResults}. Section \ref{sec:conclusion} is the overall conclusion which also answers the goal and research questions sat for this thesis. Section \ref{sec:contributions} presents the contributions, and finally, Section \ref{sec:futureWork} ends this thesis by looking at possible directions in which to take in the future. 

