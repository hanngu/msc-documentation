\section{Thesis Overview}

%\begin{itemize}
%\item What does this thesis contain
%\item Give results in a general way
%\end{itemize}

%This section provides the reader with an overview of what is coming in the next
%chapters. You want to say more than what is explicit in the chapter name, if
%possible, but still keep the description short and to the point.

Chapter 2 sets the context for this thesis. It starts by discussing the domain swarm intelligence in Section \ref{sec:swarmIntelligence}, and includes theory about ant colony optimization (ACO), bee colony optimization (BCO) and particle swarm optimization (PSO). In Section \ref{sec:VRP} the Vehicle Routing Problem (VRP) is described, along with the Urban Transit Network Design Problem, which is a subproblem of the VRP. Section \ref{sec:graphdb} is describes graph theory and puts it in context with graph databases. The graph database ``Neo4j'' is also presented. 

Chapter 3 describes the preparatory research done in order to develop the proposed method. Section \ref{sec:definingResearchTopic} defines the research topic used in order to guide the Structured Literature Review\citep{kofod2014}, which further is described in Section \ref{sec:structuredLiteratureReview}. Section \ref{sec:relatedWork} describes and discusses related work and will at the end answer \ref{itm:RQ1}. Finally, in Section \ref{subsec:problemStatement} the problem statement is proposed.  

Chapter 4 describes the implemented system in detail. It starts by an overall description, followed by a description of the chosen development environment. Further, the assessments used in order to construct solutions by the system is explained. Section \ref{sec:algoInitialization} - \ref{sec:evaporation} is a detailed description of each step conducted by the system.

Chapter 5 documents the experiments conducted for this thesis. The series of experiments are divided in three parts, where each part contains the experimental plan, setup and results. Section \ref{sec:performanceComparison} documents the parameter setting experiment, \ref{sec:performanceComparison} documents the performance comparison experiments and \ref{sec:networkExpansion} documents the network expansion experiments.

Chapter 6 concludes this thesis. Section \ref{sec:discussion} evaluates and discuss the results obtained by the experiments in the previous chapter. Section \ref{sec:conclusion} concludes this thesis, and answer the goal and research questions sat for this thesis. \ref{sec:futureWork} ends this thesis by looking at what could have been done differently, and possible directions in which to take in the future. 

