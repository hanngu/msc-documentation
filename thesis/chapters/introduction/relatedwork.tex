\section{Related Work}

Swarm intelligence algorithms has proven to be useful in many vehicle routing problems. \citet{dorigo97} and \citet{lucic03} showed that, using respectively an ant colony system and a bee systems, swarm intelligence can solve highly complex problems such as the Traveling Salesman Problem (TSP). \citet{jiang10} presented an optima approach to search for the best path of a map considering the traffic loading conditions. To do this, they proposed an algorithm based on ACO techniques to search for the shortest path from a desired original to a desired destination. \citet{salehi-nezhad07} presented an algorithm to search for the best direction between two desired origin and destination intersections in cities, called Ant-based Vehicle Navigation algorithm. \citet{tripathi09} solved a vehicle routing problem with stochastic demand, in which the customer demand has been modeled as a stochastic variable. They performed this using an improved version of the ACO approach, which oriented the search progressively towards favoring the global optimal solution. \citet{jiang10} developed an improved ACO algorithm to solve the Urban Transit Network Optimization which is a typical nonlinear combinatorial optimization problem. Improvements to the algorithm included a stagnation counter to determine if an ant had stagnated and extra pheromone intensity to newly discovered paths. \citet{dias14} introduced an inverted ACO (IACO) algorithm. The idea is that the IACO algorithm inverts the classical ACO logic by converting the attraction of ants towards pheromones into a repulsion effect. The IACO is then used in a decentralized traffic management system, where the drivers acts as the inverted ants. The drivers are repelled by the scent of pheromones (other drivers), and thus the system avoids congested roads. The result is distributed traffic through the network. 