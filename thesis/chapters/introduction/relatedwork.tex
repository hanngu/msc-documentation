\section{Related Work}

%Kanskje skrive litt om hvordan vi fant related work? Evt. henvise til appendix?

Swarm intelligence algorithms has proven to be useful in many vehicle routing problems, and many studies have been done to improve the quality of such problems. \citet{dorigo97} and \citet{lucic03} showed that, using respectively an ant colony system and a bee systems, swarm intelligence can solve highly complex problems such as the Traveling Salesman Problem (TSP). \citet{hsiao04} presented an optima approach to search for the best path of a map considering the traffic loading conditions. To do this, they proposed an algorithm based on ACO techniques to search for the shortest path from a desired original to a desired destination. \citet{yang07} presented an optimization model for a bus network design based on the coarse-grain parallel ant colony algorithm (CPACA). CPACA is an optimization algorithm the develops a strategy to update the increased pheromone, called Ant Weight, where the path searching activities of the ants are adjusted based on the objective function. The algorithm also benefits from parallelization strategies of the classical ant colony algorithm. The model aims to minimize the average trip time by maximizing the number of direct travelers per unit length. \citet{salehi-nezhad07} presented an algorithm to search for the best direction between two desired origin and destination intersections in cities, called Ant-based Vehicle Navigation algorithm. \citet{tripathi09} solved a vehicle routing problem with stochastic demand, in which the customer demand has been modeled as a stochastic variable. They performed this using an improved version of the ACO approach, which oriented the search progressively towards favoring the global optimal solution. \citet{salehinejad10} introduced a route selection system which uses an ant colony system to detect an optimum multiparameter direction between two desired points in urban areas. The system employs fuzzy logic for local pheromone updating. They emphasize on the parameters ``distance'', ``traffic'' and ``incident risk'', which is said to be important to city travelers. \citet{jiang10} developed an improved ACO algorithm to solve the Urban Transit Network Optimization which is a typical nonlinear combinatorial optimization problem. Improvements to the algorithm included a stagnation counter to determine if an ant had stagnated and extra pheromone intensity to newly discovered paths. \citet{dias14} introduced an inverted ACO (IACO) algorithm. The idea is that the IACO algorithm inverts the classical ACO logic by converting the attraction of ants towards pheromones into a repulsion effect. The IACO is then used in a decentralized traffic management system, where the drivers acts as the inverted ants. The drivers are repelled by the scent of pheromones (other drivers), and thus the system avoids congested roads. The result is distributed traffic through the network. \citet{nikolic14} proposed a model for the transit network design problem. To do this they used an approach based on BCO metaheuristic. They considered the network design problem in a way that the algorithm decided the links that was included in the transit network, and further creating designed bus routes based on the links. \citet{sedighpour14} introduced a hybrid ACO (HACO) algorithm where they used a new state transition rule and a candidate list, as well as several local search techniques and a new pheromone update rule. The hybrid was designed to overcome some of the original ACOs shortcomings, such as slow computing speed and local convergence. The HACO algorithm was applied to the open vehicle routing problem, a variant of the vehicle routing problem, in which vehicles are not required to return to the depot after completing a service. \citet{kechagiopoulos14} designed and presented an optimization algorithm based on PSO. Their goal was to find an efficient solution to the urban transit routing problem. The target problem was Mandl's benchmark problem of a Swiss bus network, which is probably the only widely investigated and accepted benchmark problem in the relevant literature. 

%Kanskje ha en eller annen konklusjon basert på related work?  

%Lese litt andre related work-stuff. Kanskje resultater skal være med? 