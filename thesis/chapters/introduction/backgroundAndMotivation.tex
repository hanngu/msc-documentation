\section{Motivation}

Trondheim and neighboring municipalities are among the areas with the greatest population growth in Norway \citep{website:miljopakken}. More people means more traffic, and without action will congestion and environmental problems increase every year. 

Private transportation have many advantages for the citizens compared to the public ones. They do not have to wait for a vehicle at the beginning of a trip, change vehicles during the trip, and it is often more convenient. The negative issues of private transportation, such as traffic jams and increased travel times, means increased air pollution, noise and accidents. 

Having efficient public transportation systems can substantially reduce negative effects of private transportation networks 
\citep{kechagiopoulos14} . The environment package \citep{website:miljopakken} for transport in Trondheim involves providing better road networks and public transport. With this they hope to achieve lower emissions, shorter traffic jams and less traffic noise. Inadequately designed transit network can cause very long waiting times and increase uncertainty in bus arriving time \citep{nikolic14}. Therefore, public transportation systems should be improved by providing better travel services, and inform the public about them, in order to convince more people to travel with it instead of using their own car. 

 \citet{website:atb} is responsible for planning the public transport throughout Sør-Trøndelag County, and bus services comprise the major component of the public transportation system. Bus services also has specific attractive features, such as flexible routes, medium capacity, low cost (of capital and running), easy implementation, flexible fleet size (easy to expand or contract this size), and use of existing facilities (roads and streets). From a meeting with AtB we learned that the current solution of AtB consists of an experience based route network, where no common methodology is used, and where all bus route networks and schedules are designed manually. As a result, the efficiency of the resulting networks is highly dependent of the designers expedience and his / hers knowledge of existing resource constraints and transportation demands.

 Satisfying all customer needs, and keeping all operator costs in check, is really difficult to achieve \citep{kechagiopoulos14}. Operator costs mainly refer to the total number of buses, total bus running distance and the total operation hours. A minimum trip time, amount of transitions, and a not too crowded bus (customers can tolerate standing in 15 minutes ) are among the most important factors that determine the passengers choice of public transit, AtB experiences.  
 \textit{The main concern of bus companies is maximizing its profits, while the main concern of the government is to fulfill all needs of traveling in public} \citep{kechagiopoulos14}. The manual attempts to provide an acceptable solution this problem are not able to solve these large network problems efficiently \citep{kechagiopoulos14}. 




















