\section{Background and motivation}
\label{sec:backgroundAndMotivation}

%---------------MOTIVATION FRA BACKGROUND---------------%
Trondheim and neighboring municipalities are among the areas with the greatest population growth in Norway \citep{website:miljopakken}. More people lead to more traffic, and without action will congestion and environmental problems in these urban areas continue to increase every year. 

Private transportation has many advantages for the citizens compared to the public ones including a decreased travel time and a direct travel from the origin to the destination. However, private transportation has some negative issues. An increased number of vehicles on the roads, which further leads to increased traffic jams, air pollution, noise, and accidents are some of these negative issues. 

Having efficient public transportation systems can substantially reduce the negative effects of private transportation. Public transportation systems are better suited for urban needs because they can transport more people per time unit than cars, needing much less space. The environment package for transportation in Trondheim\citep{website:miljopakken}, aims to provide an improved public transportation system. With this effort, they hope to increase the number of public transportation passengers, and with this achieve lower emissions, decreased traffic jams, and less traffic noise. An inadequately designed transit network can cause few route opportunities and high travel times for routes with high demand, resulting in less people using the service. Therefore, public transportation systems should be improved by providing a more suitable transit network, in order to convince more people to use public transportation services instead of their private vehicles.

AtB\citep{website:atb} is responsible for planning and operating the transit network throughout Sør-Trøndelag County in Norway, which includes the city of Trondheim.  Bus services comprise the major component of the public transportation system in the county, and in Norway generally. Moreover, bus services have specific attractive features, such as flexible routes, low cost, easy implementation, flexible fleet size, and use of existing facilities. AtB's current solution consists of an experience based route network, where transport planners has constructed reasonable transit networks and schedules entirely manually, exploiting local knowledge. As a result, the efficiency of the resulting networks is dependent on the designers experience and their existing knowledge of constraints and transportation demands.

In the past, transit planners have done a reasonable job designing transit networks and schedules without the assistance of scientific tools or systematic procedures. Nevertheless, for a large network it is almost impossible to develop an efficient transit route network and bus schedules relying only on experience and guidelines. This is because, in large urban areas the number of bus routes and bus stops is extremely large.

The problem of designing the optimal set of routes for a fleet of vehicles, to serve a given set of customers, is referred to the Vehicle Routing Problem (VRP). The Urban Transit Network Problem (UTNDP) is an example of this broad NP-hard optimization problem and involves designing urban transit routes and schedules. Because routing problems are represented as a road network by relevant locations in a graph, a graph database can be a natural way to represent the data. VRP and graph databases are described in depth in Section \vref{sec:VRP} and Section \vref{sec:graphdb}, respectively. 

The manual attempts in providing acceptable solutions to VRPs are not able to solve these large network problems optimally. In order to overcome and contribute to this problem, the number of journal publications on VRPs has increased in recent decades. The increase in research on these areas is also due to the progress in computational resources, and this has opened new possibilities for modeling more complex routing problems. Swarm Intelligence (SI) has proven to solve a great number of NP-hard problems\citep{dorigo97, lucic03}. SI based algorithms are metaheuristic optimization algorithms and are typically used to find optimal solutions to combinatorial optimization problems. SI is described in depth in Section \vref{sec:swarmIntelligence}. The employment of such an automatized method  for generating bus routes, would not only release planning time for AtB, but could also increase the quality of the network.

%---------------- Gammel INTRODUCTION ------------------%
%The task of solving vehicle routing problems, more precisely the urban transit routing problem (UTRP), using swarm intelligence (SI) methods is attempted solved by several researchers in the community, described in Section \vref{sec:relatedWork}. However, the attempt of combining attributes from different SI-method seems to be an innovative approach. The proposed algorithm was implemented to determine if combining attributes from different SI-methods was effective. 

%The goal was initially to optimize bus routes in Trondheim to increase the number of public transportation passengers. However, to determine the algorithm's performance, the results will have to be compared to a standard. For the UTRP, Mandl's benchmark problem is used by several researchers in the literature, and a recognized metric is established for evaluating the performance. Not only are these metric used by other researchers and therefore easy to compare with the algorithm's results, but they also corresponds to our goal, described in Section \vref{itm:goal}, which is increasing the number of public transportation passengers. A good solution for passengers and thus for the algorithm is, as mentioned, one that provides a low average travel time, a high percentage of passengers traveling directly or with one transfer form the origin to their destination and a low percentage of both passengers transferring twice and the amount of \textit{unsatisfied} passengers. An unsatisfied passenger is one that needs to transfer more than two times. 

%Mandl's network is, as mentioned, a small network, and only testing the algorithm's performance on Mandl's network will not determine if the proposed algorithm works for a broad number of transit networks. The implementation is made adaptable in such a way that different networks (number of nodes, route sets) and parameters (demand values, travel times) can be tested - and the algorithm is tested on larger networks to establish if the algorithm produce viable results regarding Trondheim's transit network.

%--------------- GAMMEL MOTIVATION --------------------%
%Trondheim and neighboring municipalities are among the areas with the greatest population growth in Norway \citep{website:miljopakken}. More people means more traffic, and without action will congestion and environmental problems in these urban areas continue to increase every year. 

%Private transportation has many advantages for the citizens compared to the public ones. The citizens do not have to wait for a vehicle or change vehicles during a trip, and it is often more convenient. But private transportation has a lot of negative issues, such as traffic jams and increased travel times, which leads to increased air pollution, noise and accidents. 

%Having efficient public transportation systems can substantially reduce negative effects of the private transportation networks. The environment package \citep{website:miljopakken} for transport in Trondheim aims to provide better road networks and public transportation. With this they hope to achieve lower emissions, shorter traffic jams and less traffic noise. An inadequately designed transit network can cause very long waiting times and increase uncertainty in bus arriving time, resulting in less people taking use of the service. Therefore, public transportation systems should be improved by providing better travel services, and inform the public about them, in order to convince more people to travel with it instead of using their own car. 

%\citet{website:atb} is responsible for planning the public transport throughout Sør-Trøndelag County, and bus services comprise the major component of the public transportation system. From a meeting with AtB we learned that the current solution of AtB consists of an experience based route network, where no common methodology is used, and where all bus route networks and schedules are designed manually. As a result, the efficiency of the resulting networks is highly dependent of the designers expedience and his / hers knowledge of existing resource constraints and transportation demands. 

%TODO: skrive om.
%Satisfying all customer needs in addition to keeping all operator costs in check is really difficult to achieve. The manual attempts to provide an acceptable solution to the urban routing and scheduling problems are not able to solve these large network problems efficient. The problem of designing the optimal set of routes for a fleet of vehicles, in order to serve a given set of customers, is referred to the vehicle routing problem (VRP). The urban transit network problem(UTNDP) is an example of this broader optimization problem VRP, and is the problem of designing urban transit routes and schedules. The two major components of the UTNDP are the urban transit routing problem(UTRP) and the urban transit scheduling problem (UTSP). \citep{fan09}. UTRP involves the development of efficient transit routes on an existing transit network, with predefined pick-up/drop-off point (e.g bus routes), and UTSP is assigning the schedules for the passengers carrying vehicles. In practice, the two phases are usually implemented sequentially, with the routes determined in advance of the schedules. 

%TODO: Hanne: fortelle kort om hvorfor swarm intelligence er interessant. 





