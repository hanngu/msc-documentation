\section{Background and Motivation}

\section{COHEN}

In the first stage a topic is refined to a task and a view of how to accomplish the task; in the second, the view is refined to a specific method; in the third, a program is developed to implement the method; in the fourth, experiments are designed to test the program; in the fifth, the experiments are run\citep{cohen88}

\subsection{Stage 1 - Criteria for evaluating research problems}
Refining the research topic to a task, and identifying a view.
Task: what we want a computer to do, view: a rough idea how to do it. Can you justify the research task, and is your view how to solve the task viable? 

\subsubsection{Is the task worthy of attention? Why?}
\emph{\color{orange} Motivation}
Why is is interesting? AtB, pollution ++ 

\subsubsection{How is your reformulation an improvement?}
Has it been studied before. Yes, the conducted SLR helped us find these results.\emph{\color{orange} Structured Literature Review}
Why do you expect this new perspective to be an improvement? (Cannot afford to spend months implementing a system unless were pretty sure something interesting will happen.)\emph{\color{orange} Problem Statement}
\begin{itemize}
\item Trondheim has not been computationally optimized.
\item ACO with additional features. ACO limitations.ss
\item Neo4j.
\end{itemize}

\subsubsection{Is the research representative of a class of tasks?}
\emph{\color{orange} Problem Statement, Evaluation}
Developing a general solution.

\subsubsection{Have any aspects been abstracted away?}
\emph{\color{orange} Problem Statement}
Accurate estimates of travel demand is an important factor for the algorithm, AtB does not possess accurate data about the travel demand, and detailed investigations into measuring and predicting travel demand is an complex research problem, and beyond the scope of this thesis. 

\subsubsection{When have you successfully demonstrated a solution?}
\emph{\color{orange} Problem Statement, Experiments and Results, Evaluation}
Mandl's benchmark problem, performance criteria.

\subsection{Stage 2 - Criteria for evaluating methods}
View refined to a method for solving the task
\subsubsection{How is the method an improvement?}
\emph{\color{orange} Related Work, Problem Statement}
ACO limitations: ACO with additional features from PSO and BSO. 
\emph{\color{blue} Mer spørsmål her.}

\subsubsection{Is it a recognized metric for evaluating the performance?}
\emph{\color{orange} Problem Statement, Experiments and Results}
Mandl's benchmark problem, performance criteria.

\subsubsection{Does it rely on other methods?}
\emph{\color{orange} Model?}
Input data from Mumford and Fan. Dijkstras algorithm.

\subsubsection{Underlying assumptions}
\emph{\color{orange} Problem Statement, Model, Discussion?}
ACO limitations. The type \textit{Following Ant} is inspired by the way BCO initialize some bees to be followers in the search for the best food source. The type \textit{Crazy Ant} is created in order to compensate for the original ACO algorithm weakness of sometimes getting stuck at a local optima. In the early iterations of the algorithm, the particles tends to explore more, and becoming more organized and coordinated in the late iterations.

\subsubsection{What is the scope of the method?}
\emph{\color{orange} Evaluation}
How extendible is it? - Scalability experiments
Does it address exactly the task?
Could it be applied to other problems? Trondheim's bus network?
Does it transfer to more complicated problems?

\subsubsection{When it cannot provide a good solution, why?}
\emph{\color{orange} Evaluation and Discussion}

\subsubsection{How well is the method understood?}
\begin{itemize}
\item Why does it work? \emph{\color{orange} Evaluation, Discussion}
\item Under what circumstances wont it work? \emph{\color{orange} Evaluation, Discussion}
\item Have the design decision been justified? \emph{\color{orange} Model}
\end{itemize}

\subsubsection{What is the relationship between the problem and the method?}
Why does it work for this task? \emph{\color{orange} Evaluation, Discussion}

\subsection{Stage 3 - Criteria for evaluating method implementation}
A program is developed to implement the method. Ask whether the program is informative. This is primary for the individual researcher, not for the community at large. 

\subsubsection{How demonstrative is the program?}
\emph{\color{orange} Experiments and results}
\begin{itemize}
\item Can we evaluate its external and internal behavior, and if this behavior clearly demonstrate the method. 
\item Can the class of capabilities necessary for the task be demonstrated by a well-defined set of test cases?
\item How many test cases does it demonstrate?
\end{itemize}
\subsubsection{Is it specially tuned for a particular example?}
\emph{\color{orange} Experiments, Evaluation, Discussion}
Parameter settings - optimized for the Mandl network - but will test on other, larger network: scalability experiments will be executed so this wont happen.

\subsubsection{How well does the program implement the method?}
Assess how well the program implements the method.
\emph{\color{orange} Experiments and results?}
\begin{itemize}
\item Can you determine the programs limitations?
\item Have parts been left out? why and to what extent?
\item Has the implementation forced a detailed definition or even a reevaluation of the method? How was this reevaluation accomplished?
\end{itemize}

\subsubsection{Is the programs performance predictable?}
\emph{\color{orange} Experimental plan?} Aim with test. What do we think will happen.
Because purpose of building a program is to answer something we didn't already know - can we predict performance in advance?

\section{Stage 4 - Criteria for evaluation experiments design}
\emph{\color{orange} Experiments and results} - Her kan man skrive at testene er designet med tanke på cohen, referer til cohen.
Design experiments with the newly implemented system - evaluate whether experiments with these systemt will be informative. 

\subsection{How many examples can be demonstrated?}
\emph{\color{orange} Experimental plan}
\begin{itemize}
\item Are they qualitatively different?
\item Does these examples illustrate the capabilities that are claimed? Do they illustrate limitations?
(Må huske å claime capabilities.)
\item Is the number of examples sufficient to justify the inductive generalizations?
\end{itemize}
\subsection{Should Program performance be compared to a standard? }
\emph{\color{orange} Experiments - performance criteria}
What are criteria for good performance? Who defines the criteria.

\subsection{Does the program purport to be general?}
\emph{\color{orange} Scalability Experiments}
Asks whether our experiments claim that a program is general.

\subsection{Is a series of related programs being evaluated?}
\emph{\color{orange} Evaluation and Discussion}
Suggests criteria for comparing programs. Are programs are evaluated in the context of other research projects.
(Her må vi skrive at resultatene kan være annerledes pga metoder brukt for å finne korteste vei. Dijkstras vs. metonene mandl og kechapocholus har brukt.)
\begin{itemize}
\item Can you determine how differences in the programs are manifested as differences in behavior?
\item If the method was implemented different, how does these differences affect the generalization?
\item Were difficulties encountered in implementing method in other programs? (Dette har vi ikke gjort)
\end{itemize}

\section{Stage 5 - Analyze the experiments' results}
Ask whether the system works, and why it works. Convince the research community that the methods are viable, and suggest further research.

\subsection{How did the programs performance compare to its selected standard?}
\emph{\color{orange} Results, Evaluation and Discussion}

\subsection{Is the programs performance different from predictions?}
\emph{\color{orange}Evaluation and Discussion}. Husk å ha predictions i experimental plan!

\subsection{How efficient is the program in terms of space and knowledge requirements?}
\emph{\color{orange} Results, Evaluation and Discussion - space complexity}
\subsection{Did the program demonstrate good performance?}
\emph{\color{orange} Results, Evaluation and Discussion - performance comparison}
\subsection{Did you learn what you wanted from the programs and experiments?}
\emph{\color{orange} Evaluation and Discussion, Conclusion}
\subsection{Is it easy for the intended users to understand?}
\subsection{Can you define the programs performance limitations?}
\emph{\color{orange} Evaluation and Discussion, Conclusion}
Identify frailties and strengths. when does it break, and what components contributes to its successfull operations?
\subsection{Do you understand why the program works and doesn't work?}
\emph{\color{orange} Evaluation and Discussion, Conclusion}
VITKTIG PUNKT! NÅR FUNKER DEN IKKE? HVORFOR DET? 
\begin{itemize}
\item What is the impact of changing the program even slightly?
\item Can the effect of different control strategies be determined?
\item hos does the program respond if the input is rearranged, noisy, or missing?
\item What is the relationship between characteristics of the test problems and performance?
\item Can the understanding of the program be generalized to the method? To a larger task?
\end{itemize}


%\textbf{Denne skal skrives om når rapporten er ferdig}
%\par
%Trondheim and neighboring municipalities are among the areas with the greatest population growth in Norway \citep{website:miljopakken}. More people means more traffic, and without action will congestion and environmental problems in these urban areas continue to increase every year. 

%Private transportation has many advantages for the citizens compared to the public ones. The citizens do not have to wait for a vehicle or change vehicles during a trip, and it is often more convenient. But private transportation has a lot of negative issues, such as traffic jams and increased travel times, which leads to increased air pollution, noise and accidents. 

%Having efficient public transportation systems can substantially reduce negative effects of the private transportation networks. The environment package \citep{website:miljopakken} for transport in Trondheim aims to provide better road networks and public transportation. With this they hope to achieve lower emissions, shorter traffic jams and less traffic noise. An inadequately designed transit network can cause very long waiting times and increase uncertainty in bus arriving time, resulting in less people taking use of the service. Therefore, public transportation systems should be improved by providing better travel services, and inform the public about them, in order to convince more people to travel with it instead of using their own car. 

%\citet{website:atb} is responsible for planning the public transport throughout Sør-Trøndelag County, and bus services comprise the major component of the public transportation system. From a meeting with AtB we learned that the current solution of AtB consists of an experience based route network, where no common methodology is used, and where all bus route networks and schedules are designed manually. As a result, the efficiency of the resulting networks is highly dependent of the designers expedience and his / hers knowledge of existing resource constraints and transportation demands. 

%TODO: skrive om.
%Satisfying all customer needs in addition to keeping all operator costs in check is really difficult to achieve. The manual attempts to provide an acceptable solution to the urban routing and scheduling problems are not able to solve these large network problems efficient. The problem of designing the optimal set of routes for a fleet of vehicles, in order to serve a given set of customers, is referred to the vehicle routing problem (VRP). The urban transit network problem(UTNDP) is an example of this broader optimization problem VRP, and is the problem of designing urban transit routes and schedules. The two major components of the UTNDP are the urban transit routing problem(UTRP) and the urban transit scheduling problem (UTSP). \citep{fan09}. UTRP involves the development of efficient transit routes on an existing transit network, with predefined pick-up/drop-off point (e.g bus routes), and UTSP is assigning the schedules for the passengers carrying vehicles. In practice, the two phases are usually implemented sequentially, with the routes determined in advance of the schedules. 

%TODO: Hanne: fortelle kort om hvorfor swarm intelligence er interessant. 











