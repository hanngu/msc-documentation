\section{Background and Motivation}

Trondheim and neighboring municipalities are among the areas with the greatest population growth in Norway \citep{website:miljopakken}. More people means more traffic, and without action will congestion and environmental problems in these urban areas continue to increase every year. 

Private transportation has many advantages for the citizens compared to the public ones. The citizens do not have to wait for a vehicle or change vehicles during a trip, and it is often more convenient. But private transportation has a lot of negative issues, such as traffic jams and increased travel times, which leads to increased air pollution, noise and accidents. 

Having efficient public transportation systems can substantially reduce negative effects of the private transportation networks. The environment package \citep{website:miljopakken} for transport in Trondheim aims to provide better road networks and public transportation. With this they hope to achieve lower emissions, shorter traffic jams and less traffic noise. An inadequately designed transit network can cause very long waiting times and increase uncertainty in bus arriving time, resulting in less people taking use of the service. Therefore, public transportation systems should be improved by providing better travel services, and inform the public about them, in order to convince more people to travel with it instead of using their own car. 

 \citet{website:atb} is responsible for planning the public transport throughout Sør-Trøndelag County, and bus services comprise the major component of the public transportation system. From a meeting with AtB we learned that the current solution of AtB consists of an experience based route network, where no common methodology is used, and where all bus route networks and schedules are designed manually. As a result, the efficiency of the resulting networks is highly dependent of the designers expedience and his / hers knowledge of existing resource constraints and transportation demands.

To satisfy all customer needs in addition to keeping all operator costs in check is really difficult to achieve. 

TODO: Kort om the vehicle routing problem og swarm intelligence, og hvorfor dette er interessant. 

The problem of designing urban transit routes and schedules, is often referred to as the urban transit network problem. The two major components of the UTNDP are the urban transit routing problem and the urban transit scheduling problem. UTNDP is also an example of a broader class of optimization problems, called the vehicle routing problem discussed in chapter 2. 
\citep{fan09}

The manual attempts to provide an acceptable solution to these urban routing and scheduling problems are not able to solve these large network problems efficient.
















