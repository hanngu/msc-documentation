\section{Research methods}

%What methodology will you apply to address the goals: theoretic/analytic, model/abstraction
%or design/experiment? This section will describe the research methodology applied
%and the reason for this choice of research methodology.

In this thesis, two research methods are applied. The first research method used is a structured literature review, introduced by \citep{kofod2014}. This research is conducted in order to establish the state-of-the-art of using swarm intelligence methods and graph databases to solve Vehicle Routing Problems. The results of the final synthesis are presented as the related work that further forms the basis for the proposed problem statement. 

The second research method is the design and development of the proposed system. Experiments comparing the proposed system to a generic ACO algorithm and several published methods are conducted. To ensure a sufficient comparison basis, the proposed system use a well-known benchmark problem.  

The proposed system also attempts to solve larger problems than the benchmark problem described above. By larger problems, we mean a network with a realistic number of bus stops, roads and routes in the route set. This will establish whether the proposed system supports larger networks, which further allow us to discuss the possible usability in a real urban city. 

For all the experiments, numeric values of established performance criteria, including average travel time, is presented. These values are further used to discuss the performance of the proposed system

%The comparison will be based on the numeric values of well-established performance criteria, such as average travel time experienced by each passenger.

%Experiments comparing the proposed system to other methods described in literature will be conducted. This is done

%In order to answer \vref{itm:RQ1} a Structured Literature Review, as described by \citep{kofod2014}, is conducted. This is done to ensure that we are able to draw valid conclusions about what the state-of-the-art is. The results of the final synthesis is described in Related Work, and this is the basis for the problem statement.

%The research method that is used in order to answer \vref{itm:RQ2} and \vref{itm:RQ3}, is designing the proposed system and conducting relevant experiments. 

%In order to test whether or not the added attributes from other swarm intelligence algorithms is efficient, the performance of the proposed algorithm and generic ACO algorithm is compared. 

%To establish how the proposed system performs compared to other methods described in literature, experiments solving a well-known benchmark problem with the proposed system is conducted. 

%In order to test if the proposed system is applicable for optimizing the transit routes in large urban cities, the proposed system will be tested with larger networks as input. 


