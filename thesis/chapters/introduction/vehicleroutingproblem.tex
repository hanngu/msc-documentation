\section{Vehicle Routing Problem}

The number of journal publications on Vehicle Routing Problems has steadily increased over the years. This is because of the progress in computational resources has opened new possibilities for modeling more complex routing problems. New arising real-world applications provide inspiration for developing new approaches for coordinating complex transportation processes.  \citep{vehiclerouting}

The vehicle routing problem was first introduced by Dantzig and Ramser(1959). 
It is the most challenging combinatorial optimization task. It consists of designing the optimal set of routes for fleet of vehicles in order to serve a set of customers. The vehicle routing problem is a generic name given to a whole class of problems in which a set of routes for a fleet of vehicles based at one or several depots must be determined for a number of geographically dispersed cities. The objective is to deliver a set of customers with known demands on minimum-cost vehicle routes originating and terminating at a depot. 

\begin{itemize}
\item Capacitated VRP: Minimize the vehicle fleet and the sum of travel time, and the total demand of commodities for each route may not exceed the capacity of the vehicle which serves that route
\item VRP with time windows: Minimize the vehicle fleet and the sum of travel time and waiting time to supply all customers in their required hours.
\item Multiple Depot VRP
\item Periodic VRP ++ 
\end{itemize}

\subsection{The Urban Transit Network Design Problem(UTNDP)}
The Urban Transit Network Design Problem(UTNDP) belongs to a broader class of optimization problems (VRP). It is concerned with the determination of a set of routes with corresponding schedules for an urban public transport system. 
Its two main components are the Urban Transit Routing Problem (UTRP) and the Urban Transit Scheduling Problem(UTSP).
\begin{itemize}
\item UTRP: development of efficient and effective transit routes taking into account existing road networks and predefined pick-up / drop off points. UTRP aims to develop a set of vehicle routes for an existing urban transit network while satisfying specific constraints. 
\item UTSP: Involves the assignment of schedules to all vehicles that are used to carry passengers on a road network. 
\end{itemize}
UTRP and UTSP are solved sequentially because the development of routes should be completed before the development of schedules starts.
\citep{kechagiopoulos13}





