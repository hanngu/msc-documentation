The task of solving vehicle routing problems, more precisely the urban transit routing problem (UTRP), using swarm intelligence (SI) methods is attempted solved by several researchers in the community, described in Section \vref{sec:relatedWork}. However, the attempt of combining attributes from different SI-method seems to be an innovative approach. The proposed algorithm was implemented to determine if combining attributes from different SI-methods was effective. 

The goal was initially to optimize bus routes in Trondheim to increase the number of public transportation passengers. However, to determine the algorithm's performance, the results will have to be compared to a standard. For the UTRP, Mandl's benchmark problem is used by several researchers in the literature, and a recognized metric is established for evaluating the performance. Not only are these metric used by other researchers and therefore easy to compare with the algorithm's results, but they also corresponds to our goal, described in Section \vref{itm:goal}, which is increasing the number of public transportation passengers. A good solution for passengers and thus for the algorithm is, as mentioned, one that provides a low average travel time, a high percentage of passengers traveling directly or with one transfer form the origin to their destination and a low percentage of both passengers transferring twice and the amount of \textit{unsatisfied} passengers. An unsatisfied passenger is one that needs to transfer more than two times. 

Mandl's network is, as mentioned, a small network, and only testing the algorithm's performance on Mandl's network will not determine if the proposed algorithm works for a broad number of transit networks. The implementation is made adaptable in such a way that different networks (number of nodes, route sets) and parameters (demand values, travel times) can be tested - and the algorithm is tested on larger networks to establish if the algorithm produce viable results regarding Trondheim's transit network.
