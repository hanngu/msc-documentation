\section{Goal and research questions}
\label{sec:goalAndResearchQuestions}
%TODO: skrive et mer utfyllende avsnitt
This thesis is focused on creating a method that optimizes the bus routes in urban transit transit networks. By doing so we hope that in the future more travelers will choose public transportation over private, because the designed routes hopefully will be more convenient. This thesis will focus on create beneficial routes with a low average travel time, as well as keeping the number of transfer per passenger at a minimum. By doing so we hope to contribute to the following goal: 

\begin{itemize}
\item \label{itm:goal} \textbf{Increase the number of public transportation passengers by optimizing urban transit networks}
\end{itemize}

Based on the proposed goal, three research questions have been formulated: 

\begin{itemize}

    \item[\textbf{\namedlabel{itm:RQ1}{RQ 1}:}] \textbf{What is the state-of-the-art in solving vehicle routing problems using swarm intelligence methods and graph databases?}
    \newline
    This research question will establish what the state-of-the-art is regarding solving vehicle routing problems using swarm intelligence, and whether graph databases is used in combination with this. Because we are motivated by creating a solution that combines different attributes from different swarm intelligence-methods(\ref{itm:RQ2}), this question will help us establish if there have been conducted research that have attempted this. Further it will help us establish if graph databases have been used in combination with swarm intelligence and vehicle routing problems in the past.  
    %Research Questions 2 and 3 are both created based on the answers to Research Question 1. 

    \item[\textbf{\namedlabel{itm:RQ2}{RQ 2}:}]\label{itm:RQ2} \textbf{Is it efficient to add attributes from other swarm intelligence-methods in order to improve the ant colony optimization algorithm?}
    \newline
    ACO algorithms has proven to solve NP-hard problems in the past, but it also have some well known limitations. By answering this question we will help establish if an ACO algorithm can benefit from attributes form other swarm intelligence methods regarding solution quality. 

    \item[\textbf{\namedlabel{itm:RQ3}{RQ 3}:}]\label{itm:RQ3}\textbf{Is it possible to apply the proposed algorithm to optimize urban transit routes in large urban cities?}
    \newline
    If the proposed system is to be used to optimize the route sets in large urban cities, it must be able to support large networks with many bus stops, roads and allowed routes. This question is asked in order to validate whether the proposed system is possible to use when a given transit network becomes large. 
\end{itemize}



%\begin{enumerate}[label=\textbf{\arabic*})]
%\item \label{itm:1}
 %   \begin{enumerate}
  %  \item \label{itm:1a} \textbf{Is swarm intelligence methods suitable for the vehicle routing problem?}\newline
   % This research question 
    %\item \label{itm:1b} \textbf{What is the state-of-the-art in solving vehicle routing problems using swarm intelligence methods?}\newline
   % This research question is dependent on the question above. If swarm intelligence methods in fact are suitable for solving vehicle routing problems we want to establish what the state-of-the-art is. 
   % \item \label{itm:1c} \textbf{What changes have been done to the classical swarm intelligence-methods to improve them?}\newline
   % After we have established the state-of-the art, we want to study what changes that have been done to the classical swarm intelligence methods to improve them. This will help us identify what changes that have proven to be successful / unsuccessful in the past, and allow us to get inspiration from this when designing our own model.  
   % \item \label{itm:1d}\textbf{Have there been any attempts to combine different swarm intelligence-methods?} \newline
    
   % \item \label{itm:1e} \textbf{Have graph databases been employed in combination with the vehicle routing problem and swarm intelligence?}
   % \newline
   % \emph{\color{blue} alal }
	%\end{enumerate}
   
%\item \label{itm:2}
  %  \begin{enumerate}
  %  \item \label{itm:2a} \textbf{Is it efficient to add attributes from other swarm intelligence-methods in order to improve the ant colony optimization algorithm?}
  %  \item \label{itm:2b1} \textbf{How does the proposed method perform compared to methods published in literature?}
   % \item \label{itm:2c} \textbf{Is it possible to apply the proposed algorithm to optimize urban transit routes in large urban cities?}
   % \end{enumerate}
%\item \label{itm:3}
	%\begin{enumerate}
	%\item \label{itm:3a} \textbf{What are the potential advantages and disadvantages of using a graph database in our implementation?}
   % \end{enumerate}
%\end{enumerate}