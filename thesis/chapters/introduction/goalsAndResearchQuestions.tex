\section{Goal and research questions}
\label{sec:goalAndResearchQuestions}
%TODO: skrive et mer utfyllende avsnitt
This thesis will focus on the development of a swarm intelligence system to optimize bus routes in urban transit networks. We hope that this can result in an increased number of public transportation passengers, when the designed network hopefully becomes more convenient for the passengers. This thesis will focus on creating beneficial routes with a low average travel time, as well as keeping the number of transfer per passenger at a minimum. This motivation is drawn to the following goal: 

%The goal of this thesis is to improve urban transit networks. The overall goal is that these transit routes can further increase the number of public transportation passengers. With a sufficient transit network, public transportation will be more attractive to urban travelers.

\begin{itemize}
\item[\textbf{Goal:}] \label{itm:goal} \textbf{Develop a system to improve urban transit networks.}
\end{itemize}

Based on the proposed goal, the following research questions are formulated: 

\begin{itemize}

    \item[\textbf{\namedlabel{itm:RQ1}{RQ 1}:}] \textbf{What is the state-of-the-art in solving Vehicle Routing Problems using swarm intelligence methods and graph databases?}
    \newline
    By answering this question, we will establish a theoretical foundation for the thesis, as well as identify methods proposed in published literature. We are in this thesis motivated in creating a solution that combines attributes from different swarm intelligence methods, and this question will help us establish if there have been any previous research attempting this. It will also help us establish if graph databases have been used in combination with vehicle routing problems and swarm intelligence methods in the past. 
    %This research question will establish what the state-of-the-art is regarding solving vehicle routing problems using swarm intelligence, and whether graph databases is used in combination with this. 

    %Research Questions 2 and 3 are both created based on the answers to Research Question 1. 

    \item[\textbf{\namedlabel{itm:RQ2}{RQ 2}:}]\label{itm:RQ2} \textbf{Is it efficient to add attributes from other swarm intelligence methods in order to improve a standard ant colony optimization implementation?}
    \newline
    Ant colony optimization (ACO) algorithms have proven to solve NP-hard problems in the past. However, the algorithm has some well-known limitations, such as local convergence. By answering this question, we will establish if an ACO algorithm can benefit in additional attributes from other swarm intelligence methods. 

    \item[\textbf{\namedlabel{itm:RQ3}{RQ 3}:}]\label{itm:RQ3}\textbf{Is it possible to apply the proposed system to optimize urban transit routes in large urban cities?}
    \newline
    If the proposed system is to be used to optimize route networks in large urban cities, it must be able to support large networks with many bus stops, roads, and transit routes. This question aims to establish whether the proposed system is possible to use when a given transit network becomes significantly large. 
\end{itemize}



%\begin{enumerate}[label=\textbf{\arabic*})]
%\item \label{itm:1}
 %   \begin{enumerate}
  %  \item \label{itm:1a} \textbf{Is swarm intelligence methods suitable for the vehicle routing problem?}\newline
   % This research question 
    %\item \label{itm:1b} \textbf{What is the state-of-the-art in solving vehicle routing problems using swarm intelligence methods?}\newline
   % This research question is dependent on the question above. If swarm intelligence methods in fact are suitable for solving vehicle routing problems we want to establish what the state-of-the-art is. 
   % \item \label{itm:1c} \textbf{What changes have been done to the classical swarm intelligence-methods to improve them?}\newline
   % After we have established the state-of-the art, we want to study what changes that have been done to the classical swarm intelligence methods to improve them. This will help us identify what changes that have proven to be successful / unsuccessful in the past, and allow us to get inspiration from this when designing our own model.  
   % \item \label{itm:1d}\textbf{Have there been any attempts to combine different swarm intelligence-methods?} \newline
    
   % \item \label{itm:1e} \textbf{Have graph databases been employed in combination with the vehicle routing problem and swarm intelligence?}
   % \newline
   % \emph{\color{blue} alal }
	%\end{enumerate}
   
%\item \label{itm:2}
  %  \begin{enumerate}
  %  \item \label{itm:2a} \textbf{Is it efficient to add attributes from other swarm intelligence-methods in order to improve the ant colony optimization algorithm?}
  %  \item \label{itm:2b1} \textbf{How does the proposed method perform compared to methods published in literature?}
   % \item \label{itm:2c} \textbf{Is it possible to apply the proposed algorithm to optimize urban transit routes in large urban cities?}
   % \end{enumerate}
%\item \label{itm:3}
	%\begin{enumerate}
	%\item \label{itm:3a} \textbf{What are the potential advantages and disadvantages of using a graph database in our implementation?}
   % \end{enumerate}
%\end{enumerate}