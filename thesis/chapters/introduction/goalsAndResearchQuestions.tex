\section{Goal and Research Questions}
%TODO: skrive et mer utfyllende avsnitt

\begin{itemize}
\item[\textbf{Goal:}] \label{itm:goal} \textbf{Increase the number of public transportation passengers by optimizing urban transit networks.}
\end{itemize}

\begin{itemize}

    \item[\textbf{RQ 1:}]\label{itm:RQ1} \textbf{What is the state-of-the-art in solving vehicle routing problems using swarm intelligence methods and graph databases?}
    \newline
    This research question is dependent on the question above. If swarm intelligence methods in fact are suitable for solving vehicle routing problems we want to establish what the state-of-the-art is.
    We will more precisely like to find research that concerns whether or not swarm intelligence methods are suitable for solving vehicle routing problems. Because we are motivated by creating a solution that combines different attributes from different swarm intelligence-methods, this question will help us establish if there have been conducted research that tries exactly that. 
    %
    Research Question 1 will be answered after a Structured Literature Review\citep{kofod2014}. Research Questions 2 and 3 are both created based on the answers to Research Question 1. 

    \item[\textbf{RQ 2:}]\label{itm:RQ2} \textbf{Is it efficient to add attributes from other swarm intelligence-methods in order to improve the ant colony optimization algorithm?}

    \item[\textbf{RQ 3:}]\label{itm:RQ3}\textbf{Is it possible to apply the proposed algorithm to optimize urban transit routes in large urban cities?}

\end{itemize}



%\begin{enumerate}[label=\textbf{\arabic*})]
%\item \label{itm:1}
 %   \begin{enumerate}
  %  \item \label{itm:1a} \textbf{Is swarm intelligence methods suitable for the vehicle routing problem?}\newline
   % This research question 
    %\item \label{itm:1b} \textbf{What is the state-of-the-art in solving vehicle routing problems using swarm intelligence methods?}\newline
   % This research question is dependent on the question above. If swarm intelligence methods in fact are suitable for solving vehicle routing problems we want to establish what the state-of-the-art is. 
   % \item \label{itm:1c} \textbf{What changes have been done to the classical swarm intelligence-methods to improve them?}\newline
   % After we have established the state-of-the art, we want to study what changes that have been done to the classical swarm intelligence methods to improve them. This will help us identify what changes that have proven to be successful / unsuccessful in the past, and allow us to get inspiration from this when designing our own model.  
   % \item \label{itm:1d}\textbf{Have there been any attempts to combine different swarm intelligence-methods?} \newline
    
   % \item \label{itm:1e} \textbf{Have graph databases been employed in combination with the vehicle routing problem and swarm intelligence?}
   % \newline
   % \emph{\color{blue} alal }
	%\end{enumerate}
   
%\item \label{itm:2}
  %  \begin{enumerate}
  %  \item \label{itm:2a} \textbf{Is it efficient to add attributes from other swarm intelligence-methods in order to improve the ant colony optimization algorithm?}
  %  \item \label{itm:2b1} \textbf{How does the proposed method perform compared to methods published in literature?}
   % \item \label{itm:2c} \textbf{Is it possible to apply the proposed algorithm to optimize urban transit routes in large urban cities?}
   % \end{enumerate}
%\item \label{itm:3}
	%\begin{enumerate}
	%\item \label{itm:3a} \textbf{What are the potential advantages and disadvantages of using a graph database in our implementation?}
   % \end{enumerate}
%\end{enumerate}