\section{Motivation}
\begin{itemize}
\item Why do we want to do this?
\item What makes this interesting?
\end{itemize}

AtB goes with deficit. AtB går med underskudd. \par
A known problem is rush traffic (takes time to get to work, and not good for the environemt - global warming etc). Reducing traffic demands helps reduce traffic jams and gasoline consumptiion, thereby alleviating the air pollution problems (fra Path-Planning Alforithms for Public Transportation Systems) To reduse traffic demands with making more
users take public transport, the users must be satisfied. 
\begin{itemize}
\item Full busses makes users less satisfied. The passengers can allow to stand on the bus for a maximum of 15 minutes (AtB surveys). 
\item The new initiatives they have tried to get more passengers is cost-ineffective (mobile apps etc.) 
\item New lines has high costs and it takes time for these busses to fill up. 
\end{itemize}
\par
So how can we satisfy more users and in addition to decreasing the costs without decreasing/increasing the number of busses. ==> optimize. 
\par
The current solution of AtB consist of an experience based route network. There has never been done any analysis and it has never been optimized. Optimalisering = nyhetsverdi.




