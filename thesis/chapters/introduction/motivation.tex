Trondheim and neighboring municipalities are among the areas with the greatest population growth. More people means more traffic, and without action will congestion and environmental problems increase every year. The environment package (“Miljøpakken”) for transport in Trondheim involves providing better public transport. They hope to achieve lower emissions, shorter traffic jams and less traffic noise.

Private transportation have many advantages for the citizens compared to the public ones. They do not have to wait for a vehicle at the beginning og the trips, change vehicles trough their trip, and it is more convenient. Bus negative issues, such as traffic james and increased travel times, means increased air pollution, noise and accidents. 

Having efficient public transportation systems can substantially reduce negative effects of private transportation networks. [Pangiotis].  Inedquately designed transit network can cause very long waiting times and increase uncertaint in bus arriving time[Milos Nikolic]. A minimum trip time, amount of transitions and not too crowded vehicles are among the most important factors that determne the passengers choise of public transit[ref møtet med AtB?]. As a result public transportation systems should be improved by providing better travel sercives and inform the public about them, in order to convince more people to travel with it instead of using their own car. 

AtB is responsible for planning public transport throughout Sør-Trøndelag County. The current solution of AtB consists of an experience based route network, and has never been optimized with other methods than human experience.  No common methodology is used. All bus route netorks anc schedules were designed manually. As a result, the efficiency of the resulting networks was highly dependent of the designes expeciende and his knowledge of existing resource constraints and transportation demans. These manual attempts to provide an acceptable solution to the UTRP problem were not able to solve large network probles efficiently. 

It is certain that bus services comprise the major component of a public transportation system. Among public transportation alternatives, “bus” has specific attractive features. These are flexible routes, medium capacity, low cost (of capital and running) easy implementation, flexible fleet size (easy to expand or contract this size), and use existing facilities (roads and streets). Bus services comprise the major component of public transportation system in Trondheim.

To satisfy all customer needs, and keep all operator costs in check, is really difficult to achieve[Pangiotis]. Operator costs mainly refer to the total number of busses, total bus rinning distance and the total operation hours. Main concern of bus company is macimizing its profits while the main concern of the government is to fulfill all needs of travelling in public[P]. 

As a result of all of this, in this thesis, we would like to optimize the bus routes. Try to make it concinient for the passengers but also minimize fleet size? etc.

What is done before mothafucka!!















