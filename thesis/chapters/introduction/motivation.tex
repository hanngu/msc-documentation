\section{Motivation}

Trondheim and neighboring municipalities are among the areas with the greatest population growth in Norway. More people means more traffic, and without action will congestion and environmental problems increase every year. 

Private transportation have many advantages for the citizens compared to the public ones. People do not have to wait for a vehicle at the beginning of a trip, change vehicles during the trip, and it is often more convenient. The negative issues of private transportation, such as traffic jams and increased travel times, means increased air pollution, noise and accidents. 

Having efficient public transportation systems can substantially reduce negative effects of private transportation networks. 
[Pangiotis]. The environment package (“Miljøpakken”) for transport in Trondheim involves providing better road networks and public transport. They hope to achieve lower emissions, shorter traffic jams and less traffic noise.


  Inadequately designed transit network can cause very long waiting times and increase uncertainty in bus arriving time[Milos Nikolic]. (A minimum trip time, amount of transitions and not too crowded vehicles are among the most important factors that determine the passengers choice of public transit[ref møtet med AtB?].) Therefore, public transportation systems should be improved by providing better travel services and inform the public about them, in order to convince more people to travel with it instead of using their own car. 

AtB is responsible for planning the public transport throughout Sør-Trøndelag County. The current solution of AtB consists of an experience based route network, where no common methodology is used, and all bus route networks and schedules are designed manually. As a result, the efficiency of the resulting networks is highly dependent of the designers expedience and his/hers knowledge of existing resource constraints and transportation demands. These manual attempts to provide an acceptable solution to the UTRP problem is not able to solve large network problems efficiently. 

It is certain that bus services comprise the major component of a public transportation system. Among public transportation alternatives, “bus” has specific attractive features. These are flexible routes, medium capacity, low cost (of capital and running) easy implementation, flexible fleet size (easy to expand or contract this size), and use existing facilities (roads and streets). Bus services comprise the major component of public transportation system in Trondheim.

To satisfy all customer needs, and keep all operator costs in check, is really difficult to achieve[Pangiotis]. Operator costs mainly refer to the total number of buses, total bus running distance and the total operation hours. Main concern of bus company is maximizing its profits while the main concern of the government is to fulfill all needs of traveling in public[P]. 

As a result of all of this, in this thesis, we would like to optimize the bus routes. Try to make it convenient for the passengers but also minimize fleet size? etc.

What is done before mothafucka!!















