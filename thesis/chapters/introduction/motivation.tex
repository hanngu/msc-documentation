\section{Motivation}
\begin{itemize}
\item Why do we want to do this?
\item What makes this interesting?
\end{itemize}

Trondheim and neighboring municipalities are among the areas with the greatest population growth. More people means more traffic, and without action will congestion and environmental problems increase every year. The environment package (“Miljøpakken”) for transport in Trondheim involves providing better main road network, improving public transport and providing better conditions for those who walk and cycle. They hope to achieve lower emissions, shorter traffic jams and less traffic noise.

Atb is responsible for planning public transport throughout sør-trøndelag county. The current solution of AtB consist of an experience based route network. There has never been done any analysis and it has never been optimized. The new initiatives they have tried to get more passengers is cost-ineffective (mobile apps etc.) New routes has high costs and it takes time for these busses to fill up.

To increase public transportation the public transportation offer must be improved. Satisfy customers, so they want to take public transportation. Factors for making customers satisfied includes overall trip time, amount of transitions, not too crowded busses (customers can tolerate standing in 15 minutes [ref AtB meeting]).

Public transportation is also a tool to increase mobility, create modal options and choice for the citizens. Among public transportation alternatives, “bus” has specific attractive features. These are flexible routes, medium capacity, low cost (of capital and running) easy implementation, flexible fleet size (easy to expand or contract this size), and use existing facilities (roads and streets). 

So how can we satisfy more users and in addition to decreasing the costs without decreasing/increasing the number of busses. ==> optimize. 





