\section{Motivation}
\begin{itemize}
\item Why do we want to do this?
\item What makes this interesting?
\end{itemize}
Trondheim and neighboring municipalities are among the areas with the greatest population growth. More people means more traffic, and without action will congestion and environmental problems increase every year. The environment package (“Miljøpakken”) for transport in Trondheim involves providing better main road network, improving public transport and providing better conditions for those who walk and cycle. They hope to achieve lower emissions, shorter traffic jams and less traffic noise.

AtB is responsible for planning public transport throughout Sør-Trøndelag County. The current solution of AtB consists of an experience based route network, and has never been optimized with other methods than human experience. The new initiatives they have tried to increase the number of passengers, such as mobile apps, have turned out to be cost-ineffective. 

To increase the number of passengers, the public transportation offer must be improved. Satisfied customers mean more passengers, and factors for making customers satisfied includes a minimum trip time, amount of transitions, and a not too crowded bus (customers can tolerate standing in 15 minutes [ref AtB meeting]). 

Public transportation is also a tool to increase mobility, create modal options and choice for the citizens. Among public transportation alternatives, “bus” has specific attractive features. These are flexible routes, medium capacity, low cost (of capital and running) easy implementation, flexible fleet size (easy to expand or contract this size), and use existing facilities (roads and streets).

So to satisfy users and in addition to decreasing the costs without decreasing/increasing the number of busses, the public transportation network must be optimized. 







